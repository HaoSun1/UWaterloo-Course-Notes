% header -----------------------------------------------------------------------
% Template created by texnew (author: Alex Rutar); info can be found at 'https://github.com/alexrutar/texnew'.
% version (1.13)


% doctype ----------------------------------------------------------------------
\documentclass[11pt, a4paper]{memoir}
\usepackage[utf8]{inputenc}
\usepackage[left=3cm,right=3cm,top=3cm,bottom=4cm]{geometry}
\usepackage[protrusion=true,expansion=true]{microtype}


% packages ---------------------------------------------------------------------
\usepackage{amsmath,amssymb,amsfonts}
\usepackage{graphicx}
\usepackage{etoolbox}

% Set enimitem
\usepackage{enumitem}
\SetEnumitemKey{nl}{nolistsep}
\SetEnumitemKey{r}{label=(\roman*)}

% Set tikz
\usepackage{tikz, pgfplots}
\pgfplotsset{compat=1.15}
\usetikzlibrary{intersections,positioning,cd}
\usetikzlibrary{arrows,arrows.meta}
\tikzcdset{arrow style=tikz,diagrams={>=stealth}}

% Set hyperref
\usepackage[hidelinks]{hyperref}
\usepackage{xcolor}
\newcommand\myshade{85}
\colorlet{mylinkcolor}{violet}
\colorlet{mycitecolor}{orange!50!yellow}
\colorlet{myurlcolor}{green!50!blue}

\hypersetup{
  linkcolor  = mylinkcolor!\myshade!black,
  citecolor  = mycitecolor!\myshade!black,
  urlcolor   = myurlcolor!\myshade!black,
  colorlinks = true,
}


% macros -----------------------------------------------------------------------
\DeclareMathOperator{\N}{{\mathbb{N}}}
\DeclareMathOperator{\Q}{{\mathbb{Q}}}
\DeclareMathOperator{\Z}{{\mathbb{Z}}}
\DeclareMathOperator{\R}{{\mathbb{R}}}
\DeclareMathOperator{\C}{{\mathbb{C}}}
\DeclareMathOperator{\F}{{\mathbb{F}}}

% Boldface includes math
\newcommand{\mbf}[1]{{\boldmath\bfseries #1}}

% proof implications
\newcommand{\imp}[2]{($#1\Rightarrow#2$)\hspace{0.2cm}}
\newcommand{\impe}[2]{($#1\Leftrightarrow#2$)\hspace{0.2cm}}
\newcommand{\impr}{{($\Rightarrow$)\hspace{0.2cm}}}
\newcommand{\impl}{{($\Leftarrow$)\hspace{0.2cm}}}

% align macros
\newcommand{\agspace}{\ensuremath{\phantom{--}}}
\newcommand{\agvdots}{\ensuremath{\hspace{0.16cm}\vdots}}

% convenient brackets
\newcommand{\brac}[1]{\ensuremath{\left\langle #1 \right\rangle}}
\newcommand{\norm}[1]{\ensuremath{\left\lVert#1\right\rVert}}
\newcommand{\abs}[1]{\ensuremath{\left\lvert#1\right\rvert}}

% arrows
\newcommand{\lto}[0]{\ensuremath{\longrightarrow}}
\newcommand{\fto}[1]{\ensuremath{\xrightarrow{\scriptstyle{#1}}}}
\newcommand{\hto}[0]{\ensuremath{\hookrightarrow}}
\newcommand{\mapsfrom}[0]{\mathrel{\reflectbox{\ensuremath{\mapsto}}}}
 
% Divides, Not Divides
\renewcommand{\div}{\bigm|}
\newcommand{\ndiv}{%
    \mathrel{\mkern.5mu % small adjustment
        % superimpose \nmid to \big|
        \ooalign{\hidewidth$\big|$\hidewidth\cr$/$\cr}%
    }%
}

% Convenient overline
\newcommand{\ol}[1]{\ensuremath{\overline{#1}}}

% Big \cdot
\makeatletter
\newcommand*\bigcdot{\mathpalette\bigcdot@{.5}}
\newcommand*\bigcdot@[2]{\mathbin{\vcenter{\hbox{\scalebox{#2}{$\m@th#1\bullet$}}}}}
\makeatother

% Big and small Disjoint union
\makeatletter
\providecommand*{\cupdot}{%
  \mathbin{%
    \mathpalette\@cupdot{}%
  }%
}
\newcommand*{\@cupdot}[2]{%
  \ooalign{%
    $\m@th#1\cup$\cr
    \sbox0{$#1\cup$}%
    \dimen@=\ht0 %
    \sbox0{$\m@th#1\cdot$}%
    \advance\dimen@ by -\ht0 %
    \dimen@=.5\dimen@
    \hidewidth\raise\dimen@\box0\hidewidth
  }%
}

\providecommand*{\bigcupdot}{%
  \mathop{%
    \vphantom{\bigcup}%
    \mathpalette\@bigcupdot{}%
  }%
}
\newcommand*{\@bigcupdot}[2]{%
  \ooalign{%
    $\m@th#1\bigcup$\cr
    \sbox0{$#1\bigcup$}%
    \dimen@=\ht0 %
    \advance\dimen@ by -\dp0 %
    \sbox0{\scalebox{2}{$\m@th#1\cdot$}}%
    \advance\dimen@ by -\ht0 %
    \dimen@=.5\dimen@
    \hidewidth\raise\dimen@\box0\hidewidth
  }%
}
\makeatother


% macros (theorem) -------------------------------------------------------------
\usepackage[thmmarks,amsmath,hyperref]{ntheorem}
\usepackage[capitalise,nameinlink]{cleveref}

% Numbered Statements
\theoremstyle{change}
\theoremindent\parindent
\theorembodyfont{\itshape}
\theoremheaderfont{\bfseries\boldmath}
\newtheorem{theorem}{Theorem.}[section]
\newtheorem{lemma}[theorem]{Lemma.}
\newtheorem{corollary}[theorem]{Corollary.}
\newtheorem{proposition}[theorem]{Proposition.}

% Claim environment
\theoremstyle{plain}
\theorempreskip{0.2cm}
\theorempostskip{0.2cm}
\theoremheaderfont{\scshape}
\newtheorem{claim}{Claim}
\renewcommand\theclaim{\Roman{claim}}
\AtBeginEnvironment{theorem}{\setcounter{claim}{0}}

% Un-numbered Statements
\theorempreskip{0.1cm}
\theorempostskip{0.1cm}
\theoremindent0.0cm
\theoremstyle{nonumberplain}
\theorembodyfont{\upshape}
\theoremheaderfont{\bfseries\itshape}
\newtheorem{definition}{Definition.}
\theoremheaderfont{\itshape}
\newtheorem{example}{Example.}
\newtheorem{remark}{Remark.}

% Proof / solution environments
\theoremseparator{}
\theoremheaderfont{\hspace*{\parindent}\scshape}
\theoremsymbol{$//$}
\newtheorem{solution}{Sol'n}
\theoremsymbol{$\blacksquare$}
\theorempostskip{0.4cm}
\newtheorem{proof}{Proof}
\theoremsymbol{}
\newtheorem{nmproof}{Proof}

% Format references
\crefformat{equation}{(#2#1#3)}
\Crefformat{theorem}{#2Thm. #1#3}
\Crefformat{lemma}{#2Lem. #1#3}
\Crefformat{proposition}{#2Prop. #1#3}
\Crefformat{corollary}{#2Cor. #1#3}
\crefformat{theorem}{#2Theorem #1#3}
\crefformat{lemma}{#2Lemma #1#3}
\crefformat{proposition}{#2Proposition #1#3}
\crefformat{corollary}{#2Corollary #1#3}


% macros (algebra) -------------------------------------------------------------
\DeclareMathOperator{\Ann}{Ann}
\DeclareMathOperator{\Aut}{Aut}
\DeclareMathOperator{\chr}{char}
\DeclareMathOperator{\coker}{coker}
\DeclareMathOperator{\disc}{disc}
\DeclareMathOperator{\End}{End}
\DeclareMathOperator{\Fix}{Fix}
\DeclareMathOperator{\Frac}{Frac}
\DeclareMathOperator{\Gal}{Gal}
\DeclareMathOperator{\GL}{GL}
\DeclareMathOperator{\Hom}{Hom}
\DeclareMathOperator{\id}{id}
\DeclareMathOperator{\im}{im}
\DeclareMathOperator{\lcm}{lcm}
\DeclareMathOperator{\Nil}{Nil}
\DeclareMathOperator{\rank}{rank}
\DeclareMathOperator{\Res}{Res}
\DeclareMathOperator{\Spec}{Spec}
\DeclareMathOperator{\spn}{span}
\DeclareMathOperator{\Stab}{Stab}
\DeclareMathOperator{\Tor}{Tor}

% Lagrange symbol
\newcommand{\lgs}[2]{\ensuremath{\left(\frac{#1}{#2}\right)}}

% Quotient (larger in display mode)
\newcommand{\quot}[2]{\mathchoice{\left.\raisebox{0.14em}{$#1$}\middle/\raisebox{-0.14em}{$#2$}\right.}
                                 {\left.\raisebox{0.08em}{$#1$}\middle/\raisebox{-0.08em}{$#2$}\right.}
                                 {\left.\raisebox{0.03em}{$#1$}\middle/\raisebox{-0.03em}{$#2$}\right.}
                                 {\left.\raisebox{0em}{$#1$}\middle/\raisebox{0em}{$#2$}\right.}}


% macros (analysis) ------------------------------------------------------------
\DeclareMathOperator{\M}{{\mathcal{M}}}
\DeclareMathOperator{\B}{{\mathcal{B}}}
\DeclareMathOperator{\ps}{{\mathcal{P}}}
\DeclareMathOperator{\pr}{{\mathbb{P}}}
\DeclareMathOperator{\E}{{\mathbb{E}}}
\DeclareMathOperator{\supp}{supp}
\DeclareMathOperator{\sgn}{sgn}

\renewcommand{\Re}{\ensuremath{\operatorname{Re}}}
\renewcommand{\Im}{\ensuremath{\operatorname{Im}}}
\renewcommand{\d}[1]{\ensuremath{\operatorname{d}\!{#1}}}


% file-specific preamble -------------------------------------------------------
\DeclareMathOperator{\Inv}{Inv}
\DeclareMathOperator{\inv}{inv}
\newcommand{\sqbinom}[2]{\begin{bmatrix}#1\\#2\end{bmatrix}}


% constants --------------------------------------------------------------------
\newcommand{\subject}{REPLACE}
\newcommand{\semester}{REPLACE}


% formatting -------------------------------------------------------------------
% Fonts
\usepackage{kpfonts}
\usepackage{dsfont}

% Adjust numbering
\numberwithin{equation}{section}
\counterwithin{figure}{section}
\counterwithout{section}{chapter}
\counterwithin*{chapter}{part}

% Footnote
\setfootins{0.5cm}{0.5cm} % footer space above
\renewcommand*{\thefootnote}{\fnsymbol{footnote}} % footnote symbol

% Table of Contents
\renewcommand{\thechapter}{\Roman{chapter}}
\renewcommand*{\cftchaptername}{Chapter } % Place 'Chapter' before roman
\setlength\cftchapternumwidth{4em} % Add space before chapter name
\cftpagenumbersoff{chapter} % Turn off page numbers for chapter
\maxtocdepth{section} % table of contents up to section

% Section / Subsection headers
\setsecnumdepth{section} % numbering up to and including "section"
\newcommand*{\shortcenter}[1]{%
    \sethangfrom{\noindent ##1}%
    \Large\boldmath\scshape\bfseries
    \centering
\parbox{5in}{\centering #1}\par}
\setsecheadstyle{\shortcenter}
\setsubsecheadstyle{\large\scshape\boldmath\bfseries\raggedright}

% Chapter Headers
\chapterstyle{verville}

% Page Headers / Footers
\copypagestyle{myruled}{ruled} % Draw formatting from existing 'ruled' style
\makeoddhead{myruled}{}{}{\scshape\subject}
\makeevenfoot{myruled}{}{\thepage}{}
\makeoddfoot{myruled}{}{\thepage}{}
\pagestyle{myruled}
\setfootins{0.5cm}{0.5cm}
\renewcommand*{\thefootnote}{\fnsymbol{footnote}}

% Titlepage
\title{\subject}
\author{Alex Rutar\thanks{\itshape arutar@uwaterloo.ca}\\ University of Waterloo}
\date{\semester\thanks{Last updated: \today}}

\begin{document}
\pagenumbering{gobble}
\hypersetup{pageanchor=false}
\maketitle
\newpage
\frontmatter
\hypersetup{pageanchor=true}
\tableofcontents*
\newpage
\mainmatter


% main document ----------------------------------------------------------------
\chapter{REPLACE}
\begin{enumerate}
    \item For $a,b,k\in\N$,
        \begin{equation}\label{e:van}
            \binom{a+b}{k} = \sum_{j=1}^k\binom{a}{j}\cdot\binom{b}{k-j}
        \end{equation}
        We prove this with a bijection:
        \begin{equation*}
            \mathcal{B}(a+b,k) \leftrightharpoons \bigcup_{j=0}^k\mathcal{B}(a,j)\times\mathcal{B}(b,k-j)
        \end{equation*}
        given by $S\mapsto (S\cap\{1,\ldots,a\},(S\cap\{a+1,\ldots,a+b\})^{(-a)}$ and $(P,Q)\mapsto P\cup Q^{(a)}$, where $\mathcal{B}(n,i)$ is the set of $i-$element subsets of $\{1,2,\ldots,n\}$ and for $C\subseteq\Z$ and $q\in\Z$, $C^{(q)}=\{c+q:c\in C\}$.
        Note that the equation in fact gives the polynomial identity
        \begin{equation*}
            \binom{x+y}{k}=\sum_{j=0}^k\binom{x}{j}\binom{y}{k-j}
        \end{equation*}
        in $\Q[x,y]$.
        We denote the falling factorial $(x)_i=x(x-1)(x-2)\cdots(x-i+1)$, which has degree $i$ for each $i\in\N$.
        In particular, $(x)_i=i!\binom{x}{i}$, so multiplying our identity by $k!$, we get
        \begin{equation*}
            (x+y)_k=\sum_{j=0}^k\binom{k}{j}(x)_j(y)_{k-j}
        \end{equation*}
        Compare this with the standard binomial theorem
        \begin{equation*}
            (x+y)^k=\sum_{j=0}^k\binom{k}{j}x^jy^{k-j}
        \end{equation*}
        These are called sequences of binomial type.
    \item Here's another identity.
        For $n\geq 0$ and $s,t\geq 1$,
        \begin{equation*}
            \binom{n+s+t-1}{s+t-1}=\sum_{k=0}^n\binom{k+s-1}{s-1}\binom{n-k+t-1}{t-1}
        \end{equation*}
        Let $\mathcal{M}(m,r)$ denote a multiset of size $m$ with elements of $r$ types, so that $|\mathcal{M}(m,r)|=\binom{m+r-1}{r-1}$.
        Let's define a bijection
        \begin{equation}\label{e:po1}
            \mathcal{M}(n,s+t)\rightleftharpoons\bigcup_{k=1}^n\mathcal{M}(k,s)\times\mathcal{M}(n-k,t)
        \end{equation}
        $\mu=(m_1,\ldots,m_{s+t})\mapsto ((m_1,\ldots,m_s),(m_{s+1},\ldots,m_{s+t}))$ and $(\nu,\theta)\mapsto \nu\theta$.
        Note that if $f,g$ are polynomials of degree $d$ and $e$ respectively, then $\sum_{k=0}^nf(k)g(n-k)$ is a polynomial in $n$ of degree $d+e-1$.

        Is there some way to understand  \cref{e:po1}?
        It is unclear, with our known techniques, that this corresponds to a polynomial identity since there is a variable $n$ in the exponent.
        However, we can use generating functions.
        Define
        \begin{align*}
            \sum_{n=0}^\infty\binom{n+s+t-1}{s+t-1}z^n &= \sum_{n=0}^\infty|\mathcal{M}(n,s+t)|z^n= \sum_{(m_1,\ldots,m_{s+t})}z^{m_1+\cdots+m_{s+t}}\\
                                                       &= \left(\sum_{m=0}^\infty z^m\right)^{s+t}\\
                                                       &= \frac{1}{(1-z)^{s+t}}=\frac{1}{(1-z)^s}\frac{1}{(1-z)^t}\\
                                                       &= \sum_{k=0}^\infty\binom{k+s-1}{s-1}z^k\sum_{\ell=0}^\infty\binom{\ell+t-1}{t-1}z^\ell\\
                                                       &= \sum_{n=0}^\infty z^n\left(\sum_{k=0}^n\binom{k+s-1}{s-1}\binom{n-k+t-1}{t-1}
        \end{align*}
        Similarly, \cref{e:van} is equivalent to saying $(1+z)^{a+b}=(1+z)^a(1+z)^b$.
        Note that $(1+z)^n=\sum_{k=0}^n\binom{n}{k}z^k=\sum_{k=0}^\infty\binom{n}{k}z^k$ for $n\in\N$.

        Can we substitute $\frac{1}{(1-q)^t}=(1+z)^n$ where $z=-q$ and $n=-t$?
    \item Consider
        \begin{equation*}
            (x_1+x_2)^n = \sum_{i=0}^n\binom{n}{i}x_1^ix_2^{n-i}
        \end{equation*}
        and
        \begin{equation*}
            (x_1+x_2)^n = \sum_{f:N_n\to\{1,2\}}\prod_{j=1}^n x_{f(j)}
        \end{equation*}
        More generally, we can consider
        \begin{equation*}
            (x_1+\cdots+x_k)^n = \sum_{f:N_n\to N_k}\prod_{j\in N_n}x_{f(j)}
        \end{equation*}
        If we set all $x_1=\cdots=x_k=1$, then $k^n$ gives the number of functions from $N_n$ to $N_k$.
        If we set $x_i=q^i$ for all $i\in N_k$, then we get
        \begin{equation*}
            \left(\frac{q-q^{k+1}}{1-q}\right)^n =(q+q^2+\cdots+q^k)^n = \sum_{f:N_n\to N_k}q^{f(1)+\cdots+f(k)}
        \end{equation*}
        Collect all the terms in $(x_1+\cdots+x_k)^n$ that produce the same monomial.
        Given a multiset $\mu$ with $m_1+\cdots+m_k=n$, write $x_1^{m_1}\cdots x_k^{m_k}=\underline{x}^\mu$.
        Then
        \begin{equation*}
            (x_1+\cdots+x_k)^n =\frac{n!}{m_1!\cdots m_k!}\underline{x}^\mu= \sum_{\mu\in \mathcal{M}(n,t)}\binom{n}{\mu}\underline{x}^\mu
        \end{equation*}

    \item How can we interpret
        \begin{equation*}
            P_n(q)=\prod_{i=1}^n(1+q+q^2+\cdots+q^{i-1})
        \end{equation*}
        In general, if we set $q=1$, we see that $P_n(1)=n!$.
        We might hope that there is some weight function on permutations $w:\mathcal{S}_n\to\N$ such that $P_n(q)=\sum_{\sigma\in\mathcal{S}_n}q^{w(\sigma)}$.
        Recall the bijection $I_n:\mathcal{S}_n\to\mathcal{Q}_n$ from chapter 1.
        Let's find some weight function $\nu:\mathcal{Q}_n\to\N$ such that $\sum_{\rho\in\mathcal{Q}_n}x^{\nu(\rho)}=P_n(q)$, then ``pull back'' the definition of $\nu:\mathcal{Q}_n\to\N$ to get a definition for $\omega:\mathcal{S}_n\to\N$.
        Note that $\sum_{h\in N_r}q^{h-1}=1+q+\cdots+q^{r-1}$.
        Thus
        \begin{equation*}
            \sum_{\rho=(h_1,\ldots,h_n)\in\mathcal{Q}_n}q^{(h_1-1)+(h_2-1)+\cdots+(h_n-1)} = \prod_{i=1}^n(1+q+\cdots+q^{i-1})=P_n(q)
        \end{equation*}
        so we can define $\nu(\rho)=|\rho|-n$ and $\sum_{q\in\mathcal{Q}_n}q^{|\rho|-n}=P_n(q)$.
        We also have
        \begin{equation*}
            \sum_{\rho\in\mathcal{Q}_n}q^{(h_1-1)+\cdots+(h_n-1)} = (1+q+\cdots+q^{n-1})(1+q+\cdots+q^{n-2})\cdots(1+q)(1)
        \end{equation*}
        For notation, define $[m]_q=1+q+\cdots+q^{m-1}=\frac{1-q^m}{1-q}$.
        Then $[m]_q!=[m]_q[m-1]_q\cdots[1]_q$.
        \begin{equation*}
            \begin{array}{c|cccccc}
                &1&q&q^2&q^3&q^4\\
                \hline
                q[3]_q & 0&1&1&1\\
                \hline
                [2]_q[3]_q & 1&2&2&1\\
                -q[2]_q[3]_q & 0&-1&-2&-2&-1\\
                q^2[2]_q[3]_q & 0&0&1&2&2&1\\
                \hline
                [6]_q & 1&1&1&1&1&1
            \end{array}
        \end{equation*}
        so that $[6]_q=(1-q+q^2)[2]_q[3]_q$.
        An \textbf{inversion} in $\sigma=a_1\ldots a_n\in\mathcal{S}_n$ is a pair $(i,j)$ of indices $1\leq i<j\leq n$ with $a_i>a)j$.
        Define $\Inv(\sigma)$ as the set of inversions of $\sigma$, and $\inv(\sigma)=|\Inv(\sigma)|$.
        Notice that if $\sigma=a_1\ldots a_n\mapsto\rho=(h_1,\ldots,h_n)$, then for each $1\leq i\leq n$, $h_i-1$ is the number of inversions of $\sigma$ with $i$ in the first coordinate.
        Recall
        \begin{align*}
            \mathcal{S}_n&\leftrightharpoons\mathcal{B}(n,k)\times\mathcal{S}_k\times\mathcal{S}_{n-k}\\
            \sigma=a_1\ldots a_n&\leftrightarrow (A,\beta,\gamma)\\
            \inv(\sigma) &= w(A)+\inv(\beta)+\inv(\gamma)
        \end{align*}
        Assuming such a weight fuction $w(A)$ exists, then
        \begin{align*}
            [n]!_q&=\sum_{\sigma\in\mathcal{S}_n}q^{\inv(\sigma)} =\sum_{(A,\beta,\gamma)}q^{w(A)+\inv(\beta)+\inv(\gamma)}\\
                  &= [k]!_q\cdot[n-k]!_q\cdot\sum_{A\in\mathcal{B}(n,k)}q^{w(A)}
        \end{align*}
        so that
        \begin{equation*}
            \sum_{A\in\mathcal{B}(n,k)}q^{w(A)}=\frac{[n]!_q}{[k]!_q\cdot[n-k]!_q}=\sqbinom{n}{k}_q
        \end{equation*}
\end{enumerate}
\begin{equation*}
    \sum_{S\in\mathcal{B}(n,k)} q^{\text{sum}(S)} = q^{\binom{k+1}{2}}\sqbinom{n}{k}_q
\end{equation*}
\begin{theorem}
    Let $V$ be an $n-$dimensional vector space over a finite field $\F_q$.
    Then for $0\leq k\leq n$, the number of $k-$dimensional subspaces of $V$ is $\sqbinom{n}{k}_q$.
\end{theorem}
\begin{lemma}
    Let $L:V\to W$ be a linear transformation that is surjective.
    Then $\dim V=\dim W+\dim(\ker L)$.
    So if this is over a finite field $\R_q$, every $w\in W$ is the image of exactly $q^{\dim(\ker(L)}$ vectors $v\in V$.
\end{lemma}
\end{document}


