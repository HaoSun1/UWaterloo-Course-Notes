% header -----------------------------------------------------------------------
% Template created by texnew (author: Alex Rutar); info can be found at 'https://github.com/alexrutar/texnew'.
% version (1.13)


% doctype ----------------------------------------------------------------------
\documentclass[11pt, a4paper]{memoir}
\usepackage[utf8]{inputenc}
\usepackage[left=3cm,right=3cm,top=3cm,bottom=4cm]{geometry}
\usepackage[protrusion=true,expansion=true]{microtype}


% packages ---------------------------------------------------------------------
\usepackage{amsmath,amssymb,amsfonts}
\usepackage{graphicx}
\usepackage{etoolbox}

% Set enimitem
\usepackage{enumitem}
\SetEnumitemKey{nl}{nolistsep}
\SetEnumitemKey{r}{label=(\roman*)}

% Set tikz
\usepackage{tikz, pgfplots}
\pgfplotsset{compat=1.15}
\usetikzlibrary{intersections,positioning,cd}
\usetikzlibrary{arrows,arrows.meta}
\tikzcdset{arrow style=tikz,diagrams={>=stealth}}

% Set hyperref
\usepackage[hidelinks]{hyperref}
\usepackage{xcolor}
\newcommand\myshade{85}
\colorlet{mylinkcolor}{violet}
\colorlet{mycitecolor}{orange!50!yellow}
\colorlet{myurlcolor}{green!50!blue}

\hypersetup{
  linkcolor  = mylinkcolor!\myshade!black,
  citecolor  = mycitecolor!\myshade!black,
  urlcolor   = myurlcolor!\myshade!black,
  colorlinks = true,
}


% macros -----------------------------------------------------------------------
\DeclareMathOperator{\N}{{\mathbb{N}}}
\DeclareMathOperator{\Q}{{\mathbb{Q}}}
\DeclareMathOperator{\Z}{{\mathbb{Z}}}
\DeclareMathOperator{\R}{{\mathbb{R}}}
\DeclareMathOperator{\C}{{\mathbb{C}}}
\DeclareMathOperator{\F}{{\mathbb{F}}}

% Boldface includes math
\newcommand{\mbf}[1]{{\boldmath\bfseries #1}}

% proof implications
\newcommand{\imp}[2]{($#1\Rightarrow#2$)\hspace{0.2cm}}
\newcommand{\impe}[2]{($#1\Leftrightarrow#2$)\hspace{0.2cm}}
\newcommand{\impr}{{($\Rightarrow$)\hspace{0.2cm}}}
\newcommand{\impl}{{($\Leftarrow$)\hspace{0.2cm}}}

% align macros
\newcommand{\agspace}{\ensuremath{\phantom{--}}}
\newcommand{\agvdots}{\ensuremath{\hspace{0.16cm}\vdots}}

% convenient brackets
\newcommand{\brac}[1]{\ensuremath{\left\langle #1 \right\rangle}}
\newcommand{\norm}[1]{\ensuremath{\left\lVert#1\right\rVert}}
\newcommand{\abs}[1]{\ensuremath{\left\lvert#1\right\rvert}}

% arrows
\newcommand{\lto}[0]{\ensuremath{\longrightarrow}}
\newcommand{\fto}[1]{\ensuremath{\xrightarrow{\scriptstyle{#1}}}}
\newcommand{\hto}[0]{\ensuremath{\hookrightarrow}}
\newcommand{\mapsfrom}[0]{\mathrel{\reflectbox{\ensuremath{\mapsto}}}}
 
% Divides, Not Divides
\renewcommand{\div}{\bigm|}
\newcommand{\ndiv}{%
    \mathrel{\mkern.5mu % small adjustment
        % superimpose \nmid to \big|
        \ooalign{\hidewidth$\big|$\hidewidth\cr$/$\cr}%
    }%
}

% Convenient overline
\newcommand{\ol}[1]{\ensuremath{\overline{#1}}}

% Big \cdot
\makeatletter
\newcommand*\bigcdot{\mathpalette\bigcdot@{.5}}
\newcommand*\bigcdot@[2]{\mathbin{\vcenter{\hbox{\scalebox{#2}{$\m@th#1\bullet$}}}}}
\makeatother

% Big and small Disjoint union
\makeatletter
\providecommand*{\cupdot}{%
  \mathbin{%
    \mathpalette\@cupdot{}%
  }%
}
\newcommand*{\@cupdot}[2]{%
  \ooalign{%
    $\m@th#1\cup$\cr
    \sbox0{$#1\cup$}%
    \dimen@=\ht0 %
    \sbox0{$\m@th#1\cdot$}%
    \advance\dimen@ by -\ht0 %
    \dimen@=.5\dimen@
    \hidewidth\raise\dimen@\box0\hidewidth
  }%
}

\providecommand*{\bigcupdot}{%
  \mathop{%
    \vphantom{\bigcup}%
    \mathpalette\@bigcupdot{}%
  }%
}
\newcommand*{\@bigcupdot}[2]{%
  \ooalign{%
    $\m@th#1\bigcup$\cr
    \sbox0{$#1\bigcup$}%
    \dimen@=\ht0 %
    \advance\dimen@ by -\dp0 %
    \sbox0{\scalebox{2}{$\m@th#1\cdot$}}%
    \advance\dimen@ by -\ht0 %
    \dimen@=.5\dimen@
    \hidewidth\raise\dimen@\box0\hidewidth
  }%
}
\makeatother


% macros (theorem) -------------------------------------------------------------
\usepackage[thmmarks,amsmath,hyperref]{ntheorem}
\usepackage[capitalise,nameinlink]{cleveref}

% Numbered Statements
\theoremstyle{change}
\theoremindent\parindent
\theorembodyfont{\itshape}
\theoremheaderfont{\bfseries\boldmath}
\newtheorem{theorem}{Theorem.}[section]
\newtheorem{lemma}[theorem]{Lemma.}
\newtheorem{corollary}[theorem]{Corollary.}
\newtheorem{proposition}[theorem]{Proposition.}

% Claim environment
\theoremstyle{plain}
\theorempreskip{0.2cm}
\theorempostskip{0.2cm}
\theoremheaderfont{\scshape}
\newtheorem{claim}{Claim}
\renewcommand\theclaim{\Roman{claim}}
\AtBeginEnvironment{theorem}{\setcounter{claim}{0}}

% Un-numbered Statements
\theorempreskip{0.1cm}
\theorempostskip{0.1cm}
\theoremindent0.0cm
\theoremstyle{nonumberplain}
\theorembodyfont{\upshape}
\theoremheaderfont{\bfseries\itshape}
\newtheorem{definition}{Definition.}
\theoremheaderfont{\itshape}
\newtheorem{example}{Example.}
\newtheorem{remark}{Remark.}

% Proof / solution environments
\theoremseparator{}
\theoremheaderfont{\hspace*{\parindent}\scshape}
\theoremsymbol{$//$}
\newtheorem{solution}{Sol'n}
\theoremsymbol{$\blacksquare$}
\theorempostskip{0.4cm}
\newtheorem{proof}{Proof}
\theoremsymbol{}
\newtheorem{nmproof}{Proof}

% Format references
\crefformat{equation}{(#2#1#3)}
\Crefformat{theorem}{#2Thm. #1#3}
\Crefformat{lemma}{#2Lem. #1#3}
\Crefformat{proposition}{#2Prop. #1#3}
\Crefformat{corollary}{#2Cor. #1#3}
\crefformat{theorem}{#2Theorem #1#3}
\crefformat{lemma}{#2Lemma #1#3}
\crefformat{proposition}{#2Proposition #1#3}
\crefformat{corollary}{#2Corollary #1#3}


% macros (algebra) -------------------------------------------------------------
\DeclareMathOperator{\Ann}{Ann}
\DeclareMathOperator{\Aut}{Aut}
\DeclareMathOperator{\chr}{char}
\DeclareMathOperator{\coker}{coker}
\DeclareMathOperator{\disc}{disc}
\DeclareMathOperator{\End}{End}
\DeclareMathOperator{\Fix}{Fix}
\DeclareMathOperator{\Frac}{Frac}
\DeclareMathOperator{\Gal}{Gal}
\DeclareMathOperator{\GL}{GL}
\DeclareMathOperator{\Hom}{Hom}
\DeclareMathOperator{\id}{id}
\DeclareMathOperator{\im}{im}
\DeclareMathOperator{\lcm}{lcm}
\DeclareMathOperator{\Nil}{Nil}
\DeclareMathOperator{\rank}{rank}
\DeclareMathOperator{\Res}{Res}
\DeclareMathOperator{\Spec}{Spec}
\DeclareMathOperator{\spn}{span}
\DeclareMathOperator{\Stab}{Stab}
\DeclareMathOperator{\Tor}{Tor}

% Lagrange symbol
\newcommand{\lgs}[2]{\ensuremath{\left(\frac{#1}{#2}\right)}}

% Quotient (larger in display mode)
\newcommand{\quot}[2]{\mathchoice{\left.\raisebox{0.14em}{$#1$}\middle/\raisebox{-0.14em}{$#2$}\right.}
                                 {\left.\raisebox{0.08em}{$#1$}\middle/\raisebox{-0.08em}{$#2$}\right.}
                                 {\left.\raisebox{0.03em}{$#1$}\middle/\raisebox{-0.03em}{$#2$}\right.}
                                 {\left.\raisebox{0em}{$#1$}\middle/\raisebox{0em}{$#2$}\right.}}


% macros (analysis) ------------------------------------------------------------
\DeclareMathOperator{\M}{{\mathcal{M}}}
\DeclareMathOperator{\B}{{\mathcal{B}}}
\DeclareMathOperator{\ps}{{\mathcal{P}}}
\DeclareMathOperator{\pr}{{\mathbb{P}}}
\DeclareMathOperator{\E}{{\mathbb{E}}}
\DeclareMathOperator{\supp}{supp}
\DeclareMathOperator{\sgn}{sgn}

\renewcommand{\Re}{\ensuremath{\operatorname{Re}}}
\renewcommand{\Im}{\ensuremath{\operatorname{Im}}}
\renewcommand{\d}[1]{\ensuremath{\operatorname{d}\!{#1}}}


% file-specific preamble -------------------------------------------------------
\DeclareMathOperator{\Ps}{\mathcal{P}}
\DeclareMathOperator{\im}{im}
\renewcommand{\Re}{\ensuremath{\operatorname{Re}}}
\renewcommand{\Im}{\ensuremath{\operatorname{Im}}}
\DeclareMathOperator{\proj}{proj}
\DeclareMathOperator{\Int}{Int}
\DeclareMathOperator{\Id}{Id}
\DeclareMathOperator{\diam}{diam}
\newcommand{\inner}[2]{\left\langle #1, #2 \right\rangle} % inner product


% constants --------------------------------------------------------------------
\newcommand{\subject}{REPLACE}
\newcommand{\semester}{REPLACE}


% formatting -------------------------------------------------------------------
% Fonts
\usepackage{kpfonts}
\usepackage{dsfont}

% Adjust numbering
\numberwithin{equation}{section}
\counterwithin{figure}{section}
\counterwithout{section}{chapter}
\counterwithin*{chapter}{part}

% Footnote
\setfootins{0.5cm}{0.5cm} % footer space above
\renewcommand*{\thefootnote}{\fnsymbol{footnote}} % footnote symbol

% Table of Contents
\renewcommand{\thechapter}{\Roman{chapter}}
\renewcommand*{\cftchaptername}{Chapter } % Place 'Chapter' before roman
\setlength\cftchapternumwidth{4em} % Add space before chapter name
\cftpagenumbersoff{chapter} % Turn off page numbers for chapter
\maxtocdepth{section} % table of contents up to section

% Section / Subsection headers
\setsecnumdepth{section} % numbering up to and including "section"
\newcommand*{\shortcenter}[1]{%
    \sethangfrom{\noindent ##1}%
    \Large\boldmath\scshape\bfseries
    \centering
\parbox{5in}{\centering #1}\par}
\setsecheadstyle{\shortcenter}
\setsubsecheadstyle{\large\scshape\boldmath\bfseries\raggedright}

% Chapter Headers
\chapterstyle{verville}

% Page Headers / Footers
\copypagestyle{myruled}{ruled} % Draw formatting from existing 'ruled' style
\makeoddhead{myruled}{}{}{\scshape\subject}
\makeevenfoot{myruled}{}{\thepage}{}
\makeoddfoot{myruled}{}{\thepage}{}
\pagestyle{myruled}
\setfootins{0.5cm}{0.5cm}
\renewcommand*{\thefootnote}{\fnsymbol{footnote}}

% Titlepage
\title{\subject}
\author{Alex Rutar\thanks{\itshape arutar@uwaterloo.ca}\\ University of Waterloo}
\date{\semester\thanks{Last updated: \today}}

\begin{document}
\pagenumbering{gobble}
\hypersetup{pageanchor=false}
\maketitle
\newpage
\frontmatter
\hypersetup{pageanchor=true}
\tableofcontents*
\newpage
\mainmatter


% main document ----------------------------------------------------------------
\chapter{Cardinality}
\section{Principles}
Cardinality is a way of thinking about the size of a set.
\begin{definition}
    Two sets $A$ and $B$ have the same \textbf{cardinality} if there is a bijection between the sets. If this is the case, we
    say that $|A|=|B|$. If there exists an injection, then we say $|A|\leq|B|$.
\end{definition}
In particular, cardinality is an equivalence relation.
\begin{enumerate}
    \item Reflexive: $|A|\sim|A|$ by the identity map.
    \item Symmetric: If $f:A\to B$ is a bijection, then $f^{-1}:B\to A$ is also a bijection.
    \item Transitive: If $f:A\to B$ and $g:B\to C$ are bijections, then $\phi=g\circ f:A\to C$ is a bijection.
\end{enumerate}
If $A\subseteq B$, then $|A|\leq|B|$ since the embedding maps are injective (the identity function restricted to $A$).
For example, we have $|\N|\leq|\Z|\leq|\Q|\leq|\R|$. We also have $|\N|=|\Z|$ from the bijection given, say, by $f:\Z\to\N$
defined by
\[f(n)=
\begin{cases}
    2n & n>0\\
    -2n+1 & n\leq 0
\end{cases}
\]
which is also listed below.
\begin{center}
    \begin{tabular}{cccccccc}
        $\Z$&0&1&-1&2&-2&3&$\cdots$\\
        $\N$&1&2&3 &4&5 &6&$\cdots$
    \end{tabular}
\end{center}
\begin{definition}
    A set $A$ is countable if $A$ is finite or countably infinite. $A$ is countably infinite if $|A|=|\N|$.
\end{definition}
Countable sets can be ``listed''. If $A$ is finite, we can write $A=\{a_1,\dots,a_n\}$ for some $n\in\N$. If $A$ is countably
infinite, then there exists a bijection $U:\N\to A$ that lets us write
\[A=\{U(i):i\in\N\}\]
and write $a_i=U(i)$. On the other hand, if $A=\{a_i:i\in\N\}$, we have our bijection $f:A\to\N$ given by $a_i\mapsto i$.

\section{Cardinality Examples}
\begin{enumerate}
    \item $\N\times\N=\{(a,b):a,b\in\N\}$. We have $|\N\times\N|=|\N|$.
    \item $|\Q|=|\N|$.
\end{enumerate}
\begin{proposition} The following hold:
    \begin{enumerate}[label=(\arabic*)]
        \item Every infinite subset of $\N$ is countably infinite.
        \item If $A$ is infinite and $|A|\leq|\N|$, then $|\N|=|A|$.
    \end{enumerate}
\end{proposition}
\begin{proof} Prove (1), (2) separately:
    \begin{enumerate}[label=(\arabic*)]
        \item We use the well-ordering property of $\N$: every non-empty subset of $\N$ has a least element. Let $B$
            be an infinite subset of $\N$, so it is non-empty. Thus $B$ has some least element $b_1$. But then, $B\setminus\{b_1\}$
            is also non-empty, so we can repeat this process to create an increasing sequence
            \[b_1<b_2<b_3<\cdots<\]
            I claim that every element of $B$ is in this set. Let $b\in B$ and consider $\{n\in B:n\leq b\}$. This set is
            finite with, say, $k$ elements, so $b=b_k$. We then get our bijection by the standard map $b_i\mapsto i$.
        \item Assume $j:A\to\N$ is an injection. Let $B=j(A)\subseteq\N$. Notice $j:A\to B$ is a bijection, so $|A|=|B|$
            and $B$ is infinite. By (1), $B$ is countably infinite, so $|B|=|\N|$, and the result follows by transitivity.
    \end{enumerate}
\end{proof}

\section{Uncountable Sets}
\begin{theorem}
    The set of real numbers $\{x:0\leq x<1\}=[0,1)$ is uncountable.
\end{theorem}
\begin{proof}[Cantor]
    Suppose it's countable, say $[0,1)=\{r_i:i\in\N\}$. Let $r_i=.r_{i1}r_{i2}\ldots$, with $r_{ij}\in\{0,\ldots,9\}$.
    Define $a$ by $a=.a_1a_2a_3\ldots$ were
    \[a_k=
    \begin{cases}
        1 &: r_{kk}\in\{5,6,7,8,9\}\\
        8 &: r_{kk}\in\{0,1,2,3,4\}
    \end{cases}
    \]
    and note that $a$ has a unique decimal representation. Since $a_k\neq r_{kk}$ for any $k$, $a\neq r_k$ for any $k$.
\end{proof}
\begin{remark}[Author's Remark]
    If you work with some topological properties, you can work with sets called \textit{perfect sets}. Perfect sets
    are closed sets that contain no isolated points: any element $a\in S$ can be written as a limit $\lim\{a_i\}$
    where $a_i\in S\setminus\{a\}$. In particular, the interval $[0,1]$ is a perfect set. We then have the following
    theorem:
\end{remark}
\begin{theorem}
    Non-empty perfect sets are uncountable.
\end{theorem}
\begin{proof}
    See~\ref{cantor} for a proof.
\end{proof}
Note that the proof is essentially the diagonalization argument described above!
\begin{corollary}
    $\R$ is uncountable.
\end{corollary}
\begin{proof}
    Suppose $\R$ is countable, say $g:\R\to\N$ is a bjection. Then
    \[ g:[0,1)\subseteq\R\to\N\]
    so
    \[g\circ j:[0,1)\to\N\]
    is a bijection, so $[0,1)$ is countable - a contradiction.
\end{proof}

\section{Cardinal Numbers}
We use the following notation: $|\N|=\aleph_0$, $|\R|=\aleph_1$. But does this notation make sense? This is the subject of
the Continuum Hypothesis: is there a set $A$ with $|\N|<|A|<|\R|$? This is undecidable; it is independent of the standard
axioms (ZFC axioms).
\begin{definition}
    Given a set $A$, the power set of $A$ denoted $\Ps(A)$ is defined as $\Ps(A)=\{x:x\subseteq A\}$.
\end{definition}
\begin{theorem}[Cantor]
    For any set $A$, $|A|<|\Ps(A)|$, where $|A|<|B|$ if $|A|\leq|B|$ and $|A|\neq|B|$.
\end{theorem}
\begin{proof}
    We certainly have an injection given by the map $a\mapsto\{a\}$, so $|A|\leq|\Ps(A)|$. Thus suppose we have
    some bijection $g:A\to\Ps(A)$. Define the set
    \[B=\{a\in A:a\notin g(a)\}\subseteq A\]
    Since $B\subseteq A$, we have $B\in\Ps(A)$. Hence there exists $x\in A$ such that $g(x)=B$. But now we have our contradiction
    in two cases! If $x\in B$, then $x\notin g(x)=B$. If $x\notin B=g(A)$, then $x\in B$. Thus no such $g$ exists.
\end{proof}
Using this we can construct an infinite list of cardinalities, since $|A|<|\Ps(A)|<|\Ps(\Ps(A))|<\cdots$.
\begin{definition}
    We define $2^A=\{f:A\to\{0,1\}\}$.
\end{definition}
For example, if $|A|=n$, then $|2^A|=2^n=|\Ps(A)|$.
\begin{theorem}
    $|2^A|=|\Ps(A)|$.
\end{theorem}
\begin{proof}
    Define $g:\Ps(A)\to 2^A$ by $B\mapsto \mathds{1}_B$ where $\mathds{1}_B$ is the indicator function defined as
    \[\mathds{1}_B=
    \begin{cases}
        0 &:x\notin B\\
        1 &:x\in B
    \end{cases}
    \]
    and $\mathds{1}_B\in 2^A$ certainly. $g$ is injective: if $B,C\subseteq A$ and $B\neq C$, then there exists
    some $x\in B$ but $x\notin C$ without loss of generality so $\mathds{1}_B(x)=1$ and $\mathds{1}_C(x)=0$. $g$
    is surjective: take $f\in 2^A$ and set $B=\{x\in A:f(x)=1\}$. Then $f=\mathds{1}_B$ so $g(B)=f$.
\end{proof}
\begin{corollary}
    $|A|<|2^A|$.
\end{corollary}
\begin{theorem}[Schroeder-Bernstein]
    If $|A|\leq|B|$ and $|B|\leq|A|$ then $|A|=|B|$.
\end{theorem}
\begin{proof}
    General idea: partition $A$ into two sections, $D$ and $D^c$ so that $D^c=g(f(D)^c$. If this holds, then we can define
    the bijection as
    \[\phi(x)=
    \begin{cases}
        f(x)&:x\in D\\
        g^{-1}(x)&:x\in D^c
    \end{cases}
    \]
    Define $Q:\Ps(A)\to\Ps(A)$ by the map
    \[E\mapsto\left[g(f(E)^c)\right]^c\subseteq A\]
    We wish to show that $Q$ has a fixed point, that is some $D\subseteq A$ such that $Q(D)=D$.

    We first show that if $E\subseteq F\subseteq A$, then $Q(E)\subseteq Q(F)$. This is simply a matter of following
    definitions.
    \begin{align*}
        f(E)\subseteq f(F)&\Rightarrow f(E)^c\supseteq f(F)^c\\
        &\Rightarrow g(f(E)^c)\subseteq g(f(F)^c)\\
        &\Rightarrow (g(f(E)^c))^c\subseteq (g(f(F)^c))^c\\
        &\Rightarrow Q(E)\subseteq Q(F)
    \end{align*}
    Now let $\mathcal{D}=\{E\subseteq A:E\subseteq Q(E)\}$. Set $D=\bigcup_{E\in\mathcal{D}}E\subseteq A$. If
    $E\in\mathcal{D}$, then $E\subseteq D$. By the claim, $Q(E)\subseteq Q(D)$. If $E\in\mathcal{D}$ then
    $E\subseteq Q(E)\subseteq Q(D)$, since $E\subseteq D$. So
    \begin{align*}
        \bigcup_{E\in\mathcal{D}}E\subseteq Q(D) &\Rightarrow Q(D)\subseteq Q(Q(D))\\
        &\Rightarrow Q(D)\in\mathcal{D}\\
        &\Rightarrow Q(D)\subseteq D
    \end{align*}
    Hence $D=Q(D)$.

\end{proof}
As discussed at the beginning, cardinality is an equivalence relation. The notation $|A|\leq|B|$ also makes sense as an
ordering by Schroeder-Bernstein. Finally by Cantor's argument, we have an infinite set of cardinalities.
\begin{corollary}\hspace{1cm}
    \begin{enumerate}
        \item If $A_1\subseteq A_2\subseteq A_3$, and $|A_1|=|A_3|$, then $|A_1|=|A_2|=|A_3|$.
        \item $|(0,1)|=|[0,1)|=|\R|$
        \item $|\R|=|2^{\N}|$.
    \end{enumerate}
\end{corollary}
\begin{proof}
    \begin{enumerate}
        \item We have injections $i,j$
            \[A_1\overset{i}{\hookrightarrow}A_2\overset{j}{\hookrightarrow}A_3\]
            given by the embedding maps, and a bijection $k:A_3\to A_1$. Then $k\circ j:A_2\to A_1$ is an injection, so by
            Schroeder-Bernstein, $|A_1|=|A_2|$ and $|A_2|=|A_3|$ by transitivity.
        \item It suffices to show $|(0,1)|=|\R|$. Consider $f(x)=\arctan x$ which is a bijection $f:\R\to\left( \frac{-\pi}{2},\frac{\pi}{2} \right)$.
            Thus
            \[\frac{1}{\pi}\arctan x+\frac{1}{2}:\R\to(0,1)\]
            is a bijection. There are many other examples of such functions! A good exercise is to find a rational function.

            \textit{Alternative Proof}:
            \begin{center}
                \begin{tikzpicture}[scale=2]
                    \draw (-3,0) -- (3,0);
                    \draw (0,-0.5) -- (0,1.5);
                    \draw[name path=semicirc] (1,1) arc (0:-180:1);
                    \draw[radius=0.05,fill=white] (1,1) node[above right]{$1$} circle;
                    \draw[radius=0.05,fill=white] (-1,1) node[above left]{$0$} circle;
                    \draw[radius=0.05,fill=black] (0,1);
                    \draw[dashed] (0,1) -- (2.5,0);
                    \draw[dashed, name path=line] (0,1) -- (-1.5,0)node[below]{$x$};
                    \path [name intersections={of=semicirc and line,by=E}];
                    \node [fill=red,inner sep=2pt,label=-90:$f(x)$] at (E) {};
                \end{tikzpicture}
            \end{center}
        \item $|\R|=|2^{\N}|$. Recall $2^{\N} =\{f:\N\to\{0,1\}\}$. Show $|[0,1)|=|2^{\N}|$. Take $r\in[0,1)$ and write as $r=.r_1r_2r_3\ldots$
            where $r_j\in\{0,1\}$ (binary representation of $r$. Define $f_r(n)=r_n,n\in\N$ so $f_r:\N\to\{0,1\}$ so $f_r\in 2^{\N}$.
            Define $i:[0,1)\to2^{\N}$ by the map $r\mapsto f_r$. This is injective since if $r\neq r'$, then the $k^{th}$ digits
            are different for some $k$ and that means $f_r\neq f_{r+1}$ and $|[0,1)|\leq|2^{\N}|$.

            Similarly, we have an injection $2^{\N}\to[0,1)$ given
            \[f\mapsto0.0f(1)0f(2)0f(3) \ldots \in [0,1)\]
            This is an injection because non-unique binary representation have to end with a tail of 1's (in one case) and
            a tail of 0's (in the other case). (A good exercise is to think about how to formalize this properly). Thus
            by Schroeder-Bernstein, the result follows.
    \end{enumerate}
\end{proof}

\chapter{Basic Topology}
\section{Characterizing Completeness in \texorpdfstring{$\R$}{R}}
\subsection{Basic Properties of $\R$}
\begin{definition}
    The set $\R$ is an ordered field, containing $\Q$, which satisfies the completeness axiom. The completeness axiom states
    that every bounded, increasing sequence converges.
\end{definition}
\begin{definition}
    Say $(x_n)$ converges if for any $\epsilon>0$, there exists $N$ such that $\forall n\geq N$, $|x_n-L|<\epsilon$.
    Write $\lim_{n\to\infty}x_n=L$, $(x_n)\to L$, $x_n\underset{n\to\infty}{\to}L$.
\end{definition}
\begin{theorem}[Archimedean Principle]
    Given any $r\in\R$, there exists $N\in\N$ such that $N\geq r$.
\end{theorem}
\begin{proof}
    Suppose not. Then there exists $r\in\R$ such that $N\leq r$ for all $N\in\N$. Set $x_n=n$, so $(x_n)$ is an increasing
    bounded sequence, so by completeness it must converge to, say, $L$. Take $\epsilon=1/3$. Then there exists $N$ such that
    $|x_n-L|<\frac{1}{3}$ for all $n\geq N$. Look at $|x_N-x_{N+1}|\leq|x_n-L|+|L-x_{n+1}|<\frac{2}{3}$ but $1=|N-(N+1)|$,
    a contradiction.
\end{proof}
\begin{exercise}
    Show that for every $x<y$, $x,y\in\R$, there exists $\frac{p}{q}\in\Q$ such that $x<\frac{p}{q}<y$ (in other words that
    every open interval contains a rational number - $\Q$ is dense in $\R$).
\end{exercise}
\begin{definition}
    Let $S\subseteq\R$. An \textbf{upper bound} for $S$ is some $r\in\R$ such that for any $x\in S$, $x\leq r$. If a set has an upper bound,
    then we say it is \textbf{bounded above}. We say $x'$ is a \textbf{least upper bound} of $S$ (denoted $LUB$, $\sup$)
    we mean an upper bound $x'$ for the set such that if $y<x'$, then $y$ is not an upper bound.
\end{definition}
We can similarly define lower bounds, and greatest lower bounds (denoted $GLB$, $\inf$).
\begin{remark}\hspace{1cm}
    \begin{enumerate}
        \item If $b\in S$ and $b$ is an upper bound, then $b=\sup S$.
        \item If $(x_n)$ is an increasing, bounded sequence, and $S=\{x_1,x_2,\ldots\}$, then $S$ is bounded above with
            least upper bound $\sup S=\lim x_n$.
            \begin{center}
                \begin{tikzpicture}[scale=0.7]
                    \draw (0,-1) -- (0,4);
                    \draw (-1,0) -- (15,0);
                    \draw[dashed] (-1,3) -- (15,3);
                    \foreach \x in {1,2,...,14} {
                    \draw (\x,0);
                    }
                \end{tikzpicture}
            \end{center}
        \item $B=\sup S$ iff $B$ is an upper bound for $S$ and $\forall\epsilon>0$, there exiss $x\in S$ such that $x>B-\epsilon$.
    \end{enumerate}
\end{remark}
\subsection{Four equivalent statements of completeness}
\begin{theorem}[Completeness]
    If $S$ is a non-empty subset of $\R$ that is bounded above, then $\sup S$ exists.
\end{theorem}
\begin{proof}
    We say $z\geq S$ if $z\geq x$ for all $x\in S$. Pick $y\in S$ (since $S\neq\emptyset$). Let $x_0=y-1$. Pick $N_0$ to
    be the least integer such that $x_0+N_0\geq S$ (since $S$ is bounded above). Note $N_0\geq 1$ since ``$x_0\geq S$''
    is false. Put $x_1=x_0+N_0-1$ ($\geq x_0$ as $N_0\geq 1$). We have $x_0+N_0-1\not\geq S$. So $\exists s_1\in S$ such
    that $s_1>x_0+N_0-1=x_1$. Also note that $x_1+1=x_0+N_0\geq S$. Now choose the least integer $N_1$ such that $x_1+\frac{N_1}{2}\geq S$.
    $N_1=0$ would be too small, so $N_1\geq 1$. Since $x_1+1\geq S$, $N_1\leq 2$. Set $x_2=x_1+\frac{N_1-1}{2}\geq x_1$.
    By definition of $N_1$, $\exists s_2\in S$ with $s_2>x_2$. But also $x_2+\frac{1}{2}=x_1+\frac{N_1}{2}\geq S$.
    Inductively, define $x_n=x_{n-1}+\frac{N_{n-1}-1}{n}$ where $N_{n-1}$ is the least integer such that $x_{n-1}+\frac{N_{n-1}}{n}\geq S$.
    Then $\exists s_n\in S$ such that $x_n>s_n$ and $x_n+\frac{1}{n}\geq S$. At the next step, $N_n\geq 1\Rightarrow x_n\leq x_{n+1}$.
    The sequence $(x_n)$ is increasing and bounded above by any upper bound for $S$. Hence by the completeness axiom,
    $(x_n)\to L\in\R$.

    We claim $L=\sup S$. Certainly $L$ is an upper bound for $S$. If $\exists s\in S$ such that $s>L$, then there exists
    $N$ such that $s>L+\frac{1}{N}\geq x_N+\frac{1}{N}$, a contradiction. Then to show that $L$ is a least upper bound,
    we see for any $\epsilon>0$, there exists $s\in S$ so that $L-\epsilon<s$.
\end{proof}
\begin{remark}[Author's Remark]
    \textit{Alternative Proof:}
\end{remark}
\begin{theorem}
    The following are equivalent:
    \begin{enumerate}
        \item Every bounded, increasing sequence converges
        \item Every Cauchy sequence has a unique limit
        \item Bolzano-Weierstrass holds.
        \item Every bounded set has a least upper bound
    \end{enumerate}
\end{theorem}
\begin{definition}
    Say $(x_n)$ is \textbf{Cauchy} if $\forall\epsilon>0$, $\exists N$ such that $\forall n,m\geq N$ we have
    $|x_n-x_m|<\epsilon$.
\end{definition}
\begin{remark}\hspace{1cm}
    \begin{enumerate}
        \item A Cauchy sequence is bounded
        \item Any convergent sequence is Cauchy
    \end{enumerate}
\end{remark}
\begin{theorem}[Completeness property]
    Cauchy sequences converge.
\end{theorem}
\begin{proof}
    Follows from the Heine-Borel Theorem, namely that all bounded sequences have a convergent subsequence.
\end{proof}
\subsection{Limit Superior and Limit Inferior}
Let $(x_n)$ be a bounded sequence, and let $A_n=\inf\{x_n,x_{n+1},x_{n+2},\ldots\}$. Note that $A_n$ is nondecreasing
and bounded, and hence converges.
\begin{definition}
    The \textbf{Limit Inferior}, denoted $\lim\inf$, is defined
    \[\lim\inf (x_n)=\lim_{n\to\infty}\left(\inf\{x_n,x_{n+1},x_{n+2},\ldots\}\right)\]
\end{definition}
\begin{remark}\hspace{1cm}
    \begin{enumerate}
        \item The Limit Superior can be defined similarly.
        \item With $A_n$ defined as above, we have $\lim\inf(x_n)=\sup_n A_n$.
    \end{enumerate}
\end{remark}
\begin{proposition}
    Every bounded sequence has a subsequence that converges to $\lim\sup x_n$ and a subsequence that converges to
    $\lim\inf x_n$.
\end{proposition}
\begin{proposition}
    $L=\lim\sup x_n$ if and only if $\forall\epsilon>0$, $x_n<L+\epsilon$ for all but finitely many $n$ and $x_n>L-\epsilon$
    for infinitely many $n$.
\end{proposition}
\begin{proof}
    Proof sketch: For any $k$ get $(n_k)$ such that $L-\frac{1}{k}<x_{n_k}<L+\frac{1}{k}$ where $L=\lim\sup x_n$. The terms $(x_{n_k})$
    form a subsequence converging to $L$.
\end{proof}
\begin{theorem}
    We have $\lim\inf(x_n)\leq\lim\sup(x_n)$, with equality if and only if $\lim(x_n)=L$ exists. In this case, equality
    holds at $L$.
\end{theorem}
\begin{proof}
    $(\Rightarrow)$ $\lim(x_n)=L$ implies that every subsequence converges to $L$.
    $(\Leftarrow)$ Since $L=\lim\sup x_n=\lim\inf x_n$, we have $N_1$ such that $x_n<L+\epsilon$ for $n\geq N_1$
    and $N_2$ such that $x_n>L-\epsilon$ for $n\geq N_2$. Choose $N=\max\{N_1,N_2\}$ and for all $n\geq N$,
    \[L-\epsilon<x_n<L+\epsilon\]
    This can be done for any $\epsilon>0$ so $\lim x_n=L$.
\end{proof}
\begin{theorem}[Bolzano-Weierstrass]
    Every bounded subsequence has a convergent subsequence.
\end{theorem}
\begin{proof}
    $\lim\sup x_n$ is the limit of a subsequence of $x_n$.
\end{proof}
\section{Metric Spaces}
\subsection{Basic Properties}
\begin{definition}
    The pair $(X,d)$ consisting of a set $X$ along with a function $d:X\times X\to[0,\infty)$ (called a distance function or
    metric) satisfying
    \begin{enumerate}
        \item Positive Definite: $d(x,y)=0$ if and only if $x=y$
        \item Symmetric: $d(x,y)=d(y,x)$
        \item Triangle Inequality: $d(x,y)\leq d(x,z)+d(z,y)$ for all $x,y,z$
    \end{enumerate}
    is called a metric space.
\end{definition}
\begin{example}\hspace{1cm}
    \begin{enumerate}
        \item $\R$, $d(x,y)=|x-y|$
        \item $\R^n$, $d(x,y)=\left( \sum\limits_{i=1}^n(x_i-y_i) \right)^{1/2}$
        \item $(\R^2,d_1)$ where $d_1(x,y)=|x_1-y_1|+|x_2-y_2|$
        \item $(\R^2,d_\infty)$ where $d_\infty(x,y)=\max(|x_1-y_1|,|x_2-y_2|)$.
            What do the unit balls of these metrics look like? We consider the sets $\{x\in\R^2:d(x,0)<1\}$.
            \begin{center}
                \begin{tikzpicture}[scale=3]
                    \draw[->] (0,-1.1) -- (0,1.1);
                    \draw[->] (-1.1,0) -- (1.1,0);
                    \draw[dashed] (0,0) circle[radius=1];
                    \draw[dashed] (-1,0) -- (0,-1) -- (1,0) -- (0,1) -- cycle;
                    \draw[dashed] (1,1) -- (1,-1) -- (-1,-1) -- (-1,1) -- cycle;

                    \node[above right] (A) at (0.707,0.707){$d_2$};
                    \node[above right] (A) at (1,1){$d_\infty$};
                    \node[above right] (A) at (0.5,0.5){$d_1$};
                \end{tikzpicture}
            \end{center}
        \item $(X,d)$ where $X$ is any set and
            \[d(x,y)=
            \begin{cases}
                0&:x=y\\
                1&:\text{else}
            \end{cases}
            \]
        \item $x=\{(x_n)_{n=1}^\infty\text{ bounded sequences}\}=l^\infty$, $d_\infty(x,y)=\sup_n|x_n-y_n|$.
            For example let $x_n=1-\frac{1}{n}$, $y_n=\frac{1}{n}$, then
            \[d_\infty(x,y)=\sup_n\left|\left( 1-\frac{1}{n} \right)-\frac{1}{n}\right|=1\]
            One can check that $d_\infty$ is a metric, and $l^\infty$ is a vector space. $c_0=\{(x_n)\in l^\infty:(x_n)\to 0\}$
            is a vector subspace.
        \item $X=\{(x_n)_{n=1}^\infty:\sum\limits_{n=1}^\infty |x_n|<\infty\}=l^1$, and $d_1(x,y)=\sum\limits_{n=1}^\infty|x_n-y_n|$.
        \item $X=\{(x_n)_{n=1}^\infty:\sum|x_n|^2<\infty\}=l^2$ and $d_2(x,y)=\left(\sum\limits_{n=1}^\infty|x_n-y_n|^2\right)^{1/2}$.
            Note that $l^2$ is also an inner product space, and the metric $d_2$ is the natural metric arising from the inner
            product $\inner{x}{y}=\sum\limits_{n=1}^\infty x_ny_n$. The norm is given by $\norm{x}=\inner{x}{x}^{1/2}$
            and $d(x,y)=\norm{x-y}$.
        \item Inner Product Space $x$ with metric $d(x,y)=\sqrt{\inner{x-y}{x-y}}$. The Cauchy-Schwarz inequality gives
            the triangle inequality. C-S states
            \[|\inner{x}{y}|\leq\norm{x}\norm{y}\]
            Then
            \begin{align*}
                \norm{x+y}^2&=\inner{x+y}{x+y}\\
                &=\inner{x}{x}+\inner{x}{y}+\inner{y}{x}+\inner{y}{y}\\
                &=\norm{x}^2+2\inner{x}{y}+\norm{y}^2\\
                &\leq\norm{x}^2+2\norm{x}\norm{y}+\norm{y}^2\\
                &=(\norm{x}+\norm{y})^2
            \end{align*}
            so that $\norm{x+y}\leq\norm{x}+\norm{y}$. Thus
            \begin{align*}
                d(x,y)&=\norm{x-y}\\
                &=\norm{(x-z)+(z-y)}\\
                &\leq\norm{x-z}+\norm{z-y}\\
                &=d(x,z)+d(z,y)
            \end{align*}
            Thus $(X,d)$ is a metric space.
    \end{enumerate}
\end{example}
\subsection{Topology of Metric Spaces}
For convergence, let $x_n,x_0\in X$. Say $x_n\to x_0$ if $d(x_n,x_0)\to 0$.
We can also define open balls $B(x_0,r)=\{x\in X:d(x,x_0)<r\}$. For example, this is the open
sphere in $\R^3$, open disc in $\R^2$, and open interval in $\R$. What are the balls in $X$ with the discrete metric?
We have $B(x_0,r)=X$ if $r>1$, and $B(x_0,r)=\{x_0\}$ if $r\leq 1$. Notice that if $x_n\to x_0$, then $x_n=x_0$ eventually.
\begin{definition}
    Let $U\subseteq X$. Say $x_0\in U$ is an interior point of $U$ if $\exists r>0$ such that $B(x_0,r)\subseteq U$.
    We will write $\Int U=U^\circ=\{\text{interior points of $U$}\}$. Say $U$ is open if every point of $U$ is an
    interior point (i.e. $U=\Int U$).
\end{definition}
\begin{example}\hspace{1cm}
    \begin{enumerate}
        \item $\R$, $U=[0,1)$. Then $\Int U=(0,1)$.
        \item $\R^2,d_1$, $U=(x,y):0<x<1$. Then $\Int U=U$ and $U$ is open. Note that the open sets of $\R^n$ with respect
            to the $d_1,d_2,d_\infty$ metrics are all the same.
        \item $\R$ with the usual metric: $\Int\Q=\emptyset$.
        \item $X$, discrete metric. Every set is open since all singletons are balls.
        \item $X,\emptyset$ are open in every metric space.
    \end{enumerate}
\end{example}
\begin{proposition}
    Balls are open sets.
\end{proposition}
\begin{proof}
    Take a ball $B(x_0,r)$ and let $y\in B(x_0,r)$. Take $\epsilon=r-d(y,x_0)$. We show that $B(y,\epsilon)\subseteq B(x_0,r)$.
    Let $z\in B(y,\epsilon)$. Note that $d(z,x_0)\leq d(z,y)+d(y,x_0)<\epsilon+d(y,x_0)=r$ so $z\in B(x_0,r)$.
\end{proof}
\begin{proposition}
    \begin{enumerate}
        \item If $U_1,U_2$ are open, then $U_1\cap U_2$ is open (this extends to finite intersections).
        \item If $U_\alpha:\alpha\in I$ are open sets, then so is $\bigcup_{\alpha\in I}U_\alpha$.
    \end{enumerate}
\end{proposition}
\begin{proof}
    \begin{enumerate}
        \item Let $x\in U_1\cap U_2$. Let $B(x,r_1)\subseteq U_1$ and $B(x,r_2)\subseteq U_2$. Then $B(x,\min\{r_1,r_2\})\subseteq U_1\cap U_2$.
        \item Let $x\in\bigcup_{\alpha\in I}U_\alpha$. Then $x\in U_i$ for some $i\in I$, so there exists some ball
            $B(x,r)\subseteq U_i\subseteq\bigcup_{\alpha\in I}U_\alpha$ as desired.
    \end{enumerate}
\end{proof}
\begin{proposition}
    $U$ is an open set iff $U$ is a union of balls.
\end{proposition}
\begin{proof}
    $(\Rightarrow)$ balls are open as are unions of balls by previous results.
    $(\Rightarrow)$ For any $x\in U$, there exists $B(x,r_x)\subseteq U$. Take $\bigcup_{x\in U}B(x,r_x)=U$.
\end{proof}
\begin{proposition}
    $\Int U$ is the union of all open subsets of $U$.
\end{proposition}
\begin{proof}
    $(\subseteq)$ Let $x\in\Int U$. Then by definition $x$ is in an open subset of $U$. Similarly, if $y\in\text{RHS}$,
    then $y$ is in some open subset contained in $U$, so it is in the interior.
\end{proof}
\begin{definition}
    A set $U$ is closed if $U^c=X\setminus U$ is open.
\end{definition}
\begin{example}
    \begin{enumerate}
        \item $[a,b]$ in $\R$ is closed
        \item $\Q$ is not closed as $\Q^c$ is not open
        \item $X$ with the discrete metric: every set is closed since every set is open.
        \item Any metric space, $X,\emptyset$ are closed.
        \item $\{x_0\}$ is a closed set in a metric space (in other words, metric spaces are Haussdorf).
            \begin{proof}
                We show that $X\setminus\{x_0\}$ is open. Take $y\in X$ with $y\neq x_0$. Then consider
                \[B(y,d(y,x_0))=\{x:d(y,x)<d(y,x_0)\}\]
                certainly does not contain $x_0$.
            \end{proof}
        \item $\Z$, usual metric. Let $n\in\Z$ and consider $B(n,1)=\{n\}$. Even though this is not the discrete metric,
            $\Z$ has the discrete topology.
    \end{enumerate}
\end{example}
\begin{proposition} \hspace{1cm}
    \begin{enumerate}
        \item Any finite union of closed sets is closed.
        \item Any intersection of closed sets is closed.
    \end{enumerate}
\end{proposition}
\begin{proof}
    Identical to the open set cases, with DeMorgan's laws.
\end{proof}
\begin{definition}
    Let $E\subseteq X$. A point $x\in X$ is an \textit{accumulation point} of $E$ if for any $r>0$, $B(x,r)$ contains a point
    of $E$ other than $x$. The points in $E$ that are not accumulation points of $E$ are called \textit{isolated points}.
\end{definition}
\begin{example}
    \begin{enumerate}
        \item $E=[0,1)$ in $\R$. The set of accumulation points of $E$ is $[0,1]$.
        \item $E=[0,1)\cup\{2\}$. $2$ is an isolated point.
        \item $\Q^c$. $\R$ is the set of accumulation points/
        \item $\Z$: all points are isolaged
        \item $E\subseteq X$ with the discrete metric: every point is isolated.
    \end{enumerate}
\end{example}
\begin{proposition}
    If $X$ is an accumulation point of $E$, then every ball centred at $x$ contains infinitely many points of $E$.
\end{proposition}
\begin{proof}
    Exercise.
\end{proof}
\begin{theorem}
    A set $E$ is closed if and only if $E$ contains all its accumulation points.
\end{theorem}
\begin{corollary}
    Every finite set is closed.
\end{corollary}
\begin{proof}
    First assume $E$ is closed. Let $x\notin E$ and show $x$ is not an accumulation point of $E$. Then $x\in E^c$ and
    $E^c$ is an open set. Then we have $r>0$ such that $B(x,r)\subseteq E^c$. Since $B(x,r)\cap E=\emptyset$, then $X$
    is not an accumulation point of $E$.

    Now assume $E$ contains all its accumulation points. We want to prove that $E^c$ is open. Take $x\in E^c$ and show
    $x$ is an interior point of $E^c$. We know that $x$ is not an accumulation point of $E$. That means $\exists r>0$ such that
    $B(x,r)\cap E=\emptyset$. Hence $B(x,r)\subseteq E^c$ and that proves $x$ is an interior point of $E^c$ so $E$ is closed.
\end{proof}
\begin{theorem}\hspace{1cm}
    \begin{enumerate}
        \item $\overline{E}$ is closed
        \item $\displaystyle\overline{E}=\bigcap_{\substack{B\supseteq E\\B\text{ is closed}}}B$
    \end{enumerate}
\end{theorem}
\begin{proof}\hspace{1cm}
    \begin{enumerate}
        \item Show that $(\overline{E})^c$ is open. Let $x\in(\overline{E})^c$. Then $x\notin E$ and $x$ is not an accumulation
            point of $E$. Hence $\exists r>0$ such that $B(x,r)\cap E=\emptyset$. Furthermore, $B(x,r)\cap\overline{E}=\emptyset$.
            Suppose $z\in B(x,r)\cap\overline{E}$ where $z$ is an accumulation point of $E$. But then every open set
            containing $z$ contains points in $E$ so $B(x,t)$ muct contain points of $E$ being open and containing $Z$,
            a contradiction
        \item TODO
    \end{enumerate}
\end{proof}
Recall that $A$ is dense in $X$ if $\overline{A}=X$. For example,
\[l_1=\{(x_n)_{n=1}^\infty:\sum\limits_{n=1}^\infty|x_n|<\infty\}\]
\[l_\infty=\{(x_n)_{n=1}^\infty:\sup_n|x_n|<\infty\}\]
\[d_\infty(x,y)=\sup_n|x_n-y_n|\]
\[c_0=\{(x_n)_{n=1}^\infty:x_n\to 0\}\subseteq l_\infty\]
We see that $l_1\subseteq c_0\subseteq l_\infty$. Prove that $l_1$ is dense in $c_0$ with the $c_\infty$ metric. Thus
suppose that $x\in c_0\setminus l_1$ and show that $x$ is an assumulation point of $l_1$. Take $B(x,\epsilon)$ and choose $N$
such that $|x_n|<\epsilon/2$ for all $n\geq N$. Let $X^{(N)}=(x_1,x_2,\ldots,x_{n-1},0,0,\ldots)\in l_1$ and $X^{(N)}\in l_1$.
Furthermore, $d(x,x^{(N)})=\sup_{n\geq N}|x_n-x_n^{(N)}|<\epsilon$ so $x$ is an assumulation point of $l_1$.

\begin{definition}
    We denote the boundary of $E$ by
    \[\partial E=\overline{E}\cap\overline{E^c}\]
\end{definition}
Note that $x\in\partial E$ iff every open set that contains $X$ contains points of both $E$ and $E^c$.
\begin{definition}
    Say $A\subseteq(X,d)$ is bounded if there exists some $x_0,M$ with $A\subseteq B(x_0,M)$.
\end{definition}
\subsection{Sequences in Metric Spaces}
\begin{definition}
    A sequence $(x_n)$ in $(X,d)$ is said to converge to $x\in X$ if for all $\epsilon>0$, there exists $N$ such that
    $d(x_n,x)<\epsilon$ for all $n\geq N$; that is $x_n\in (x,\epsilon)$. Equivalently, $(d(x,x_n))_{n=1}^\infty\to 0$ in
    $\R$ with the usual metric.
\end{definition}
\begin{proposition}
    $(x_n)\to x$ if and only if for every open set containing $x$ contains all but
    finitely many $x_n$.
\end{proposition}
\begin{proof}
    $(\Leftarrow)$ Get $B(x,\epsilon)\subseteq U$ and get $N$ such that $x_n\in B(x,\epsilon)$ for all $n\geq N$.
    $(\Rightarrow)$ $B(x,\epsilon)$ is an open set containing $x$ so get get $N$ such that $x_n\in B(x,\epsilon)$ for all $n\geq N$.
\end{proof}
We thus have that limits are unique. If $x_n\to x$ and $x_n\to y$, if $x\neq y$ then there exsts disjoint neighbourhoods
of $x$ and $y$, contradiction by the previous proposition.

Any convergent sequence $(x_n)$ is bounded, meaning $\{x_n:n=1,2,\ldots\}$ is a bounded set.
\begin{proof}
    Say $d(x_n,x)<1$ for all $n\geq N$. Then $B(x,\max(1,1+d(x,x_j):j=1,\ldots,N-1))\supseteq\{x_n:n=1,2,\ldots\}$.
\end{proof}
\begin{definition}
    Say $(x_n)$ is \textit{Cauchy} if for all $\epsilon>0$, there exists $N$ such that $d(x_n,x_m)<\epsilon$.
\end{definition}
Any Cauchy sequence is bounded, and every convergent sequence is Cauchy. If a Cauchy sequence has a convergent subsequence
with limit $x$, then the Cauchy sequence converges to $x$. All same proofs as before. However, there are metric spaces where Cauchy
sequences do not necessarily converge. For example, $\Q$ with the usual metric.
\begin{proposition}
    $x\in\overline{E}$ iff there is a sequence $(x_n)$ in $E$ such that $x_n\to x$.
\end{proposition}
\begin{proof}
    Suppose $x\in\overline{E}$. Then for all $n$, $B(x,1/n)\cap E\neq\emptyset$. Let $x_n\in B(x,1/n)\cap E$. Then $x_n\to x$.
    Conversely, let $x=\lim x_n,x_n\in E$. Then for any $\epsilon>0$, $B(x,\epsilon)\cap E\neq\emptyset$ so $x\in\overline{E}$.
\end{proof}
\begin{proposition}
    $E$ is closed iff whenever $x_n\in E$ and $x_n\to x$, then $x\in E$
\end{proposition}
\begin{proof}
    $E$ is closed iff $\overline{E}$ is closed.
\end{proof}

\section{Topological Spaces}
\subsection{Basic Notions}
\begin{definition}
    A \textit{topological space} $(X,\tau)$ where $\tau$ is a collection of subsets of $X$ satisfying the following axioms:
    \begin{enumerate}
        \item $X,\emptyset\in\tau$ (basic sets)
        \item $\displaystyle\bigcup\limits_{S\in T\subseteq\tau}S\in\tau$ (closure under union)
        \item $\displaystyle\bigcup\limits_{i=1}^n S_i\in\tau$ (closure under finite intersection)
    \end{enumerate}
\end{definition}
\begin{definition}
    A collection $C\subseteq\tau$ is called a \textit{base} for $X$ if every open set in $X$ is the union of sets from $C$.
\end{definition}
Given a metric space $(X,d)$, we have seen that the balls $B(x,r)$ are indeed open sets.
Furthermore, we know that every open set in $X$ can be written as a union of open balls.
Thus we can talk about the induced topological space $(X,\tau)$ consisting of unions of open balls with respect to $d$: we say that the open balls consist of a base for $X$.
\begin{definition}
    A topological space $(X,\tau)$ is called \textit{metrizable} if it is induced by some metric $d$ on $X$.
\end{definition}
Some names are convenient.
We can define the \textit{discrete topology} by setting $\tau=\mathcal{P}(X)$.
In fact, this is precisly the topological space induced by the discrete metric on $X$.
\begin{definition}
    We say that a topological space is Hausdorff if for any $x,y\in X$, there exist open sets $\tau_x\ni x$ and $\tau_y\ni y$ with $\tau_x\cap\tau_y=\emptyset$.
\end{definition}
\begin{theorem}
    The topological space induced by a metric is Hausdorff.
\end{theorem}
\begin{proof}
    Exercise.
\end{proof}
\begin{theorem}
    Every separable metric space has a countable base.
\end{theorem}
\begin{proof}
    Let $X$ be a separable metric space, and $S\subseteq X$ a countable dense subset of $S$. Then for each
    $s\in S$, define the collection $\{B(s,r):r\in\Q\}$ of open balls with rational radius around each point.
    Then the set $C:=\{B(s,r):(s,r)\in S\times\Q\}$ is countable, and I claim that it is a countable basis for
    $X$.

    Let $U\subseteq X$ be open, and I claim that $U=\{c\in C:c\subseteq U\}$. The reverse inclusion is satisfied
    by definition, so it suffices to show $U\subseteq \bigcup\{c\in C:c\subseteq U\}$. Thus let $x\in U$ be arbitrary,
    and let $\epsilon>0$ be such that $B(x,\epsilon)\subseteq U$. By density of $S$, there exists some $y$ such that
    $y\in B(x,\epsilon/3)$. Then for any $\epsilon/3<r<\epsilon/2$, the open ball $B(y,r)\subseteq B(x,\epsilon)$
    by the triangle inequality, and $x\in B(y,r)$ since $d(y,x)<\epsilon/3$. In particular, by density
    of $\Q$, choose $r_0\in\Q$ satisfying $\epsilon/3<r_0<\epsilon/2$; but then $B(y,r_0)\subseteq B(x,\epsilon)\subseteq U$
    and $B(y,r_0)\in C$. Thus $B(y,r_0)\in\{c\in C:c\subseteq U\}$ and $x\in B(y,r_0)$ so $x\in\bigcup\{c\in C:c\subseteq U\}$
    and equality holds, as desired.
\end{proof}
\begin{corollary}
    Every compact metric space $X$ has a countable base.
\end{corollary}
\subsection{Connectedness}
\begin{definition}
    A topological space $X$ is \textit{not connected} if $X=U\cup V$ where $U,V$ are open and non-empty and $U\cap V=\emptyset$.
    $E\subseteq X$ is connected if $E\neq (E\cap U)\cup (E\cap V)$ where $U,V$ are open in $X$, $E\cap U$, $E\cap V$ are both
    non-empty and $E\cap U\cap V=\emptyset$.
\end{definition}
\begin{example}
    \begin{enumerate}
        \item $E=(0,1)\cup[2,3]$ - not connected
        \item $\Q$ - not connected
        \item $S\subseteq\R$ is connected iff $S$ is an interval.
    \end{enumerate}
\end{example}
\section{Baire Category Theorem}
\subsection{Definitions and Proof}
\begin{definition}
    Say $A\subseteq X$ is nowhere dense if $(\overline{A})^\circ=\emptyset$.
\end{definition}
For example, $\Z$ in $\R$ is nowhere dense, and $\Q$ in $\R$ is not nowhere dense.
\begin{definition}
    $A\subseteq X$ is \textit{first category} if $A=\bigcup\limits_{n=1}^\infty V_n$ where each $V_n$ is nowhere dense.
    $A\subseteq X$ is \textit{second category} if $A$ is not first category.
\end{definition}
For example, $\Q$ is first category since $\Q=\bigcup_{n=1}^\infty\{r_n\}$.
In this way, every countable set is first category.
\begin{enumerate}
    \item $A$ is nowhere dense implies $A^c$ is dense.
        This follows since $S$ is dense iff $S$ intersects every open set.
    \item $A$ is closed and nowhere dense iff $A^c$ is open and dense.
\end{enumerate}
\begin{proposition}
    If $A,B$ are first category, then so is $A\cup B$.
\end{proposition}
\begin{proposition}
    $X$ is second category if and only if the intersection of every countable family of open, dense sets in $X$ is non-empty.
\end{proposition}
\begin{proof}
    This is a direct application of DeMorgan's identities.

    $(\Rightarrow)$ $X$ is first category iff $X=\bigcup_{n=1}^\infty X_n$ where $X_n$ are closed and nowhere dense.
    Let $\{G_n\}$ be a countable family of open dense sets.
    Suppose $\bigcap{n=1}^\infty G_n=\emptyset$.
    Thus $\bigcup_{n=1^\infty}(G_n)^c=X$ so $X$ is first category, a contradiction.

    $(\Leftarrow)$. Say $X$ is not second category.
    Then $X$ is first category, so $X=\bigcup_{n=1}^\infty X_n$ where $X_n$ are closed and nowhere dense.
    Then
    \[\emptyset=X^c=\left(\bigcup_{n=1}^\infty X_n\right)^c=\bigcap_{n=1}^\infty X_n^c\]
    where the $X_n^c$ are open and dense.
\end{proof}
\begin{theorem}[Baire Category]
    A non-empty complete metric space $X$ is second category.
\end{theorem}
\begin{proof}
    Let $\{A_n\}_{n=1}^\infty$ be open and dense, and show $\bigcap_{n=1}^\infty A_n\neq\emptyset$.
    Get $x_1\in A_1$, so get $B(x_1,r_1)=U_1\subseteq A_1$.
    $A_2$ is dense, so intersects $U_1$, and $A_2\cap U_1\neq\emptyset$.
    Say $x_2\in A_2\cap U_1\subseteq A_2\cap A_1$.
    Since finite intersections are open, get $V_2\subseteq A_2\cap U_1$, without loss of generality $r_2\leq r_1/2$.
    Set $U_2=B(x_2,r_2/2)$.
    Then $\overline{U}_2\subseteq V_2\subseteq A_2\cap U_1$, and $\diam\overline{U}_2\leq\frac{1}{2}\diam\overline{U}_1$.

    Proceed inductively to get $x_n$, open sets $U_n\ni x_n$ and $U_n\subseteq\bigcap_{k=1}^n A_k$, $\overline{U}_n\subseteq U_{n-1}$ and $\diam U_n\to 0$.
    Check $(x_n)$ is Caucy.
    Let $\epsilon>0$ and pick $N$ such that $\diam U_N<\epsilon$.
    If $n,m\geq N$, then $x_n,x_m\in U_n$ so $d(x_n,x_m)<\diam U_n<\epsilon$.
    Thus $(x_n)$ is Cauchy so by completeness, $\lim x_n=x\in X$.
    Furthermore, $x_n\in\overline{U}_n$ for all $n\geq N$, so $x\in\overline{U}_n\subseteq U_{n-1}\subseteq\bigcap_{j=1}^{n-1}A_j$ for all $n$.
    But then $x\in A_j$ for all $j$, so the intersection is non-empty.
\end{proof}
\subsection{Applications of the Baire Category Theorem}
\begin{corollary}
    $\R$ is uncountable.
\end{corollary}
\begin{proof}
    Suppose not, so $\R=\bigcup\limits_{n=1}^\infty\{x_n\}$, but then $\R$ would be first categpry, a contradiction since $\R$ is a non-empty complete metric space.
\end{proof}
\begin{corollary}
    Perfect sets are uncountable.
\end{corollary}
\begin{proof}
    Suppose $E$ is countable and write $E=\bigcup\limits_{n=1}^\infty\{x_n\}$.
    Since each $x_n$ is an accumulation point of $E$, any open set must contain infinitely many points of $E$.
    Consider $E$ as a metric space, and since $E$ is closed, $E$ is complete.
    Furthermore, $\{x_n\}$ is not open in $E$ for it not, then $\{x_n\}=V\cap E$ where $V$ is open in $X$, contradicting the fact that $V\cap E$ must contain infinitely many points in $E$.
    Thus $E$ is a countable union of closed, nowhere dense sets, a contradiction to the Baire Category Theorem.
\end{proof}
\chapter{Completeness and Compactness}
\section{Completeness}
\begin{definition}
    A metric space in which every Cauchy sequence converges (to an element in $X$) is called \textit{complete}.
\end{definition}
For example, $\R$ and $\R^n$ are complete, but $\Q$ is not complete. Discrete metric spaces are complete because the
only Cauchy sequences are sequences that are eventually constant.
\begin{proposition}
    If $X$ is complete and $E$ is closed, then $E$ is complete.
\end{proposition}
\begin{proof}
    Let $(x_n)$ be a Cauchy sequence in $E$. Then it is also a Cauchy sequence in $X$ so it has a limit in $X$. But since $E$
    is closed, this limit is in $E$, so $E$ is complete.
\end{proof}
\section{Compactness}
\begin{definition}
    An \textit{open cover} of $A$ is a family of open sets $\{G_\alpha\}$ such that $\bigcup_\alpha G_\alpha\supseteq A$.
    A subcover of an open cover is a subset of the open cover which still contains $A$.
\end{definition}
\begin{definition}
    A metric space $X$ is \textit{compact} if every open cover of $X$ has a finite subcover.
\end{definition}
\begin{example}
    $\R$ is not compact: $(-n,n):n\in\N$ is an open cover with no finite subcover. Similarly, $(0,1)$ is not compact
    since $(1/n,1-1/n),n=2,3,\ldots$ is an open cover with no finite subcover.
\end{example}
Singleton sets in any metric space are always compact because if $\{G_\alpha\}$ is an open cover, then certainly
some $G_\alpha$ contains the point and is a finite subcover. More generally any finite set is compact.

If $A\subseteq X$ where $X$ is a discrete metric space, then $A$ is compact iff $A$ is finite. If $A$ is infinite, then
the collection of its singleton points is an open cover with no finite subcover.

\subsection{Characterization of Compactness in $\R^n$}
\begin{theorem}
    For $A\subseteq\R^n$, the following are equivalent:
    \begin{enumerate}
        \item $A$ is compact
        \item $A$ is closed and bounded
        \item Every sequence in $A$ has a convergent subsequence with limit in $A$
    \end{enumerate}
\end{theorem}
The equivalence $(1)\Leftrightarrow(3)$ is called the Bolzano-Weierstrass Theorem, and the equivalence $(1)\Leftrightarrow(2)$
is called the Heine-Borel Theorem.
\begin{proposition}
    Compact sets are closed and bounded.
\end{proposition}
\begin{proof}
    Consider the open cover $B\{(x,1),B(x,2),\ldots\}$. Then it has a finite subcover $B(x,n)$ for some $n$ and $K$ is
    bounded.

    To prove closedness, let $x\in K^c$. Consider $U_n=\{y\in X:d(x,y)>1/n\}$. Then $\bigcup_{n=1}^\infty U_n=X\setminus\{x\}\supseteq K$,
    so it has a finite subcover, say $U_{n_1},\ldots,U_{n_L}$ with $n_1<\cdots<n_L$. But then $K\supseteq U_{n_L}$, and
    $B(x,1/n_L)\cap K=\emptyset$.
\end{proof}
\begin{definition}
    A finite set $\{x_1,\ldots,x_n\}\subseteq X$ is called an \textit{$\epsilon-$net for $A\subseteq X$} if each point of
    $A$ has distance $<\epsilon$ for some point $x_j,j=1,\ldots,n$. That is for all $x\in A$, $\exists j\in\{1,\ldots,n\}$
    such that $d(a,x_j)<\epsilon$.
\end{definition}
For example, $\{x_0\}$ is a $2-$net for $X$. Similarly $[0,1]$ has a $1/n-$net given by $x_j=j/n$, $j=0,\ldots,n$.
\begin{definition}
    A set $A$ is totally bounded if $A$ has an $\epsilon-$net for every $\epsilon>0$.
\end{definition}
\subsection{Compactness in Metric Spaces}
In particular, note that an infinite discrete metric space has no $1-$net. Furthermore, totally bounded sets are certainly
bounded: we have a finite covering given by a $1-$net. Any compact set $K$ is totally bounded, by considering $B(x,\epsilon)$
for all $x\in X$: the centers of the balls of a finite subcover is an $\epsilon-$net.
\begin{proposition}
    If $A$ is totally bounded, so is $\overline{A}$.
\end{proposition}
\begin{proof}
    Let $\{x_1,\ldots,x_n\}$ be an $\frac{\epsilon}{2}-$net for $A$. Take $y$ an accomulation point of $A$ for all $x\in A$
    such that $x\in B(y,\epsilon/2)$. Since $x\in A$, for all $y\in\{1,\ldots,n\}$ such that $d(x,x_j)<\epsilon/2$
    so that $d(y,x_j)\leq d(y,x)+d(x,x_j)<\epsilon$ so $\bigcup_{j=1}^n B(x_j,\epsilon)\supseteq\overline{A}$.
\end{proof}
Consider the set $S=(-\sqrt{2},\sqrt{2})\cap\Q$ is a closed set in $\Q$.
\begin{theorem}[Cantor's Intersection]
    If $A_1\supseteq A_2\supseteq A_3\supseteq\cdots$ are non-empty closed sets in a complete metric space $X$ and $\diam A_n\to 0$
    where $\diam A=\sup\{d(x,y):x,y\in A\}$, then $\bigcap_{n=1}^\infty A_n$ is exactly one point.
\end{theorem}
\begin{proof}
    If $n\geq N$, then $x_n\in A_n\subseteq A_N$ so $\{x_n:n\geq N\}\subseteq A_N$. If $m,n\geq N$ then $d(x_m,x_n)\leq\diam A_N\to 0$
    so $(x_n)_{n=1}^\infty$ is a Cauchy sequence. Since $X$ is complete, there exists $x_0$ such that $x_n\to x_0$. Since
    $x_n\in A_N$ for all $n\geq N$, then $x_0\in\overline{A}_N=A_N$ (since $A_N$ is closed). This is true for all $N$, so
    $x_0\in \bigcap A_n$. Furthermore, the intersection cannot contain two points: if $y\in\bigcap A_n$, then $d(x_0,y)\leq\diam A_n$
    for all $n$ and $x_0=y$.
\end{proof}
\begin{definition}
    We say that a collection of sets has the Finite Intersection Property if every finite intersection is non-empty.
\end{definition}
\begin{theorem}
    For any metric space $X$, the following are equivalent:
    \begin{enumerate}
        \item $X$ is compact
        \item Every collection of closed subsets of $X$ with the finite intersection property has non-empty intersection.
        \item Every sequence in $X$ has a convergent subsequence (in $X$).
        \item $X$ is complete and totally bounded.
    \end{enumerate}
\end{theorem}
\begin{proof}
    $(1\Leftrightarrow 3)$ is Bolzano-Weierstrass.

    $(1\Leftrightarrow 4)$ generalizes Heine Borel.
\end{proof}
\begin{corollary}
    In $\R^n$, $E$ is compact iff $E$ is closed and bounded.
\end{corollary}
\begin{proof}
    In $\R^n$, $E$ is closed iff $E$ is complete and $E$ is bounded iff $E$ is totally bounded.
\end{proof}
\begin{proof}
    $(1\Rightarrow 2)$. Suppose $X$ is compact, and we prove the contrapositive (2).
    Suppose $A_\alpha\subseteq X$, closed, and suppose $\bigcap_{\alpha}A_\alpha=\emptyset$. We will
    show this collection does not have the finite intersection property. Consider the open sets $A_\alpha^c$, and their union is
    $X$, so since $X$ is compact, it has a finite subcover $A_\phi^c$. But then $\bigcap_\phi A_\phi=\emptyset$ is a finite
    intersection that is empty.

    $(2\Rightarrow 3)$. Let $(x_n)$ be a sequence in $X$ and define $S_n=\{x_k:k\geq n\}$. Take $\overline{S}_n$ which is closed
    and non-empty. Thus $\{\overline{S}_n\}$ is nested and has the finite intersection property. By (2), there exists
    $x\in\bigcap_{n=1}^\infty\overline{S}_n$. We thus have $x\in\overline{S}_n$ for all $n$. Since $x\in\overline{S}_n$,
    for any $\epsilon>0$, there exists $y\in S_n$ such that $d(x,y)<\epsilon$. We now provide an algorithm to generate the
    subsequence. Start with $\epsilon=1$ and $n=1$. Get $y\in S_1$ such that $d(x,y)<1$, say $y=x_{k_1}$.
    Then repeat with $n=k_1+1$ and $\epsilon=1/n$, and get $y=x_{k_2}\in S_n$ and $k_2>k_1$ and $d(y,x)<1/2$. Repeat to
    get $k_1<k_2<\cdots$ and $y=x_{k_j}$ where $d(x,y)<1/2^{j-1}$ giving our desired convergent subsequence.

    $(3\Rightarrow 4)$. Completeness follows the same proof as in $\R^n$. To prove totally bounded, assume not. Then there exists $\epsilon>0$ so
    that $X$ has no $\epsilon-$net. In particular, $\{x_1\}$ is not an $\epsilon-$net. Thus there exists $x_2\in X\setminus B(x_1,\epsilon)$.
    We can repeat this to get a sequence $\{x_1,x_2,\ldots,\}$ with the property that $d(x_j,x_i)\geq\epsilon$ for all $i=1,\ldots,j-1$.
    Thus the sequence $(x_n)_{n=1}^\infty$ has no convergent subsequence, contradicting (3). Thus $X$ is totally bounded.

    $(4\Rightarrow 1)$. Assume $X$ is complete and totally bounded. Suppose $X$ has an open cover has no finite subcover,
    say $\{U_\alpha\}$. $X$ is totally bounded, so it has a $1/2-$net, say $\{X_1^{(1)},X_2^{(1)},\ldots,X_{N_1}^{(1)}\}$.
    Write $D(x,r)=\{r\in X:d(x,y)\leq r\}$. We have $\bigcup_{j=1}^{N_1} D\left(x_j^{(1)},\frac{1}{2}\right)=X$. Since
    $X$ cannot be covered by finitely many $\{U_{\alpha}\}$, the same is true for at least one of $D(x_j^{(1)},1/2)$,
    say $j=1$. Since we cannot cover $D(x_1^{(1)},1/2)$ with finitely many $U_\alpha$'s. Note that $\diam X_0\leq 1=1/2^0$.
    Subsets of a totally bounded set are totally bounded. We thus get a $1/4-$net for $X_0$, say $\{x_1^{(2)},\ldots,x_{N_2}^{(2)}\}$
    and
    \[\bigcup_{j=1}^{N_2} D(x_j^{(2)},1/4)\cap X_0\supseteq X_0\]
    Again, one of these sets $D(x_1^{(2)},1/4)\cap X_0$ cannot be covered by only finitely many $U_{\alpha}$. Notice $X_1\subseteq X_0$,
    $\diam X_1\leq 1/2^1$, and $X_1$ is closed and non-empty. Continue this process, and by the Cantor intersection theorem,
    $\bigcap_{n=0}^\infty X_n=\{x_0\}$. Since $\{U_\alpha\}$ covers $X$, there exists some index $\alpha_0$ such that
    $x_0\in U_{\alpha_0}$. Hence there exists $\epsilon>0$ such that $B(x_0,\epsilon)\subseteq U_{\alpha_0}$. Choose $N$
    such that $1/2^N<\epsilon$. We know $x_0\in X_N$. If $y\in X_N$, then $d(x,y)\leq\diam X_N\leq1/2^n<\epsilon$ so that
    $y\in B(x_0,\epsilon)\subseteq U_{\alpha_0}$ implying that $X_N\subseteq U_{\alpha_0}$. This contradicts the fact that no $X_j$
    had a finite subcover from $\{U_\alpha\}$.
\end{proof}
\subsection{Properties of Compact Metric Spaces}
Compactness is a super useful property to have!
Here are some particularly useful properties of compact metric spaces.
\begin{theorem}
    Any compact metric space is separable.
\end{theorem}
\begin{proof}
    Consider $\{B(x,1/n):x\in X\}$, an open cover of $X$.
    By compactness of $X$, there is a finite subcover, say $B(x_1^n,1/n),\ldots,B(x_{r_n}^n,1/n)$.
    Let $K_n=\{x_1^n,x_2^n,\ldots,x_{r_n}^n\}$ and set $K=\bigcup\limits_{n=1}^\infty K_n$, a countable set.
    Check that $K$ is dense, so let $x\in X$.
    Since each $K_n$ is an open cover, $x\in K_n$ for all $n$, and in particular is in some $B(x_{i(n)}^n,1/n)$ for all $n$.
    But then by construction $(x_{i(n)}^n)_{n=1}^\infty\to x$ so $K$ is dense.
\end{proof}

\section{Applications}
\subsection{The Cantor Set}\label{cantor}
The Cantor Set is a subset of $[0,1]$ which is compact, has empty interior, uncountable, and perfect (closed and every point
is an accumulation point). To construct the Cantor set, we recursively remove the open middle third of any closed line segment.
We then have $C_0=[0,1]$, $C_1=[0,1/3]\cup[2/3,1]$, $C_2=[0,1/9]\cup[2/9,3/9]\cup[6/9,7/9]\cup[8/9,1]$, and define the cantor
set to be the intersection over all these intervals. The cantor set certainly contains the endpoints of any interval, but
since it is an intersection of closed sets, it also contains their limit points!

The interior is empty: say $(a,b)\subseteq C$ with $b-a>1/3^n$. But then $(a,b)\subseteq C_n$, impossible since $C_n$ is the
union if intervals of length $3^{-n}$ separated by gaps. Furthermore, every point is an accumulation point. Let $x\in C$ and
show for all $\epsilon>0$ there exists $y\in C$, $y\neq x$, such that $d(x,y)<\epsilon$. Pick $N$ such that $1/3^N<\epsilon$.
Then $x\in C_N$, so $x$ is in one interval in the construction of $C_N$ (of length $3^{-n}$). There are two endpoints,
and choose $y$ to be one of them (not equal to $x$), and it certainly satisfies $d(x,y)\leq 3^{-n}<\epsilon$.
\begin{center}
    \begin{tikzpicture}
        \draw (0,0)node[left]{$C_0$} -- (9,0);
        \draw (0,-1)node[left]{$C_1$} -- (3,-1);
        \draw (6,-1) -- (9,-1);
        \draw (0,-2)node[left]{$C_2$} -- (1,-2);
        \draw (2,-2) -- (3,-2);
        \draw (6,-2) -- (7,-2);
        \draw (8,-2) -- (9,-2);
    \end{tikzpicture}
\end{center}
We also see that the Cantor set is uncountable by the below theorem:
\begin{theorem}
    Perfect sets are uncountable.
\end{theorem}
\begin{proof}
    If $S$ is perfect, then $S$ is certainly not finite: given any $x\in S$, we can use increasingly small open
    neighbourhoods about $x$, all of which intersect $S\setminus\{x\}$ and avoid any previous elements of the sequence,
    thus constructing a countably infinite subset. Thus $S$ is either countable or uncountable. Suppose it were countable
    and write
    \[S=\{x_1,x_2,x_3,\ldots\}\]
    and consider the interval $U_1=\{x_1-1,x_1+1\}$. Now we construct inductively a sequence of nested intervals. Let
    $U_1\subset\ldots\subset U_k$ be previous intervals and $x_1,\ldots,x_k$ be previous points. Now choose $x_{k+1}\in U_k$
    and some neighbourhood $U_{k+1}$ so that $x_1,\ldots,x_k\notin U_{k+1}$ (this can be done since we only need to
    avoid finitely many points), and $\overline{U_{k+1}}\subset U_k$. But now we have a sequence $\{U_n\}$ of sets
    and $\{x_n\}$ of points so that
    \begin{enumerate}
        \item $x_k\in U_k$.
        \item $\overline{U_{k+1}}\subset U_k$
        \item $x_j\notin U_k$ for all $0<j<n$
    \end{enumerate}
    But now consider the set
    \[V=\bigcap_{n=1}^\infty \left( \overline{U_n}\cap S \right)\]
    Each set $\overline{U_N}\cap S$ is closed and bounded, hence compact, and $\overline{U_{n+1}}\cap S\subset\overline{U_{n}}\cap S$.
    Then by the nested compact set lemma, $V$ is non-empty and contains some element $v$. But $v\neq x_i$ for all $i$,
    since $v\in U_{i+1}$ but $x_i\notin U_{i+1}$. Thus our enumeration is incomplete, and $S$ is not countable.
\end{proof}
\subsection{The Product Metric}
Suppose $(X,d_X)$ and $(Y,d_Y)$ are metric spaces.
Define a function $d$ by $d( (x_1,y_1),(x_2,y_2))= d_X(x_1,x_2)+d_Y(y_1,y_2)$ for $(x_j,y_j)\in X\times Y$ for $j=1,2$.
$d$ defines a metric on $X\times Y$:
\begin{enumerate}
    \item \textit{Symmetric:} This follows directly from symmetry of $d_X$ and $d_Y$:
        \[d( (x_1,y_1),(x_2,y_2) ) = d_X(x_1,x_2)+d_Y(y_1,y_2) = d_X(x_2,x_1)+d_Y(y_2,y_1)=d( (x_2,y_2),(x_1,y_1) )\]
    \item \textit{Positive Definite:} since $d_X,d_Y\geq 0$, $d_X+d_Y\geq 0$. Furthermore, if $d( (x_1,y_1),(x_2,y_2) )=0$,
        then we must have $d_X(x_1,x_2)+d_Y(y_1,y_2)=0$ so $d_X(x_1,x_2)=0$ and $d_Y(y_1,y_2)=0$
        so $x_i=y_i=0$ for $i=1,2$.
    \item \textit{Triangle Inequality:} We have
        \begin{align*}
            [d( (x_1,y_1),(x_2,y_2) ) &= d_X(x_1,x_2)+d_Y(y_1,y_2) \\
                                      &\leq d_X(x_1,x_3)+d_X(x_3,x_2)+d_Y(y_1,y_3)+d_Y(y_3,y_2)\\
                                      &= d( (x_1,y_1),(x_3,y_3)) + d( (x_3,y_3),(x_2,y_2) )
        \end{align*}
        for arbitrary $(x_1,y_1), (x_2,y_2), (x_3,y_3)$.
\end{enumerate}
\begin{theorem}
    If $X,Y$ are complete, then so is $X\times Y$.
\end{theorem}
\begin{proof}
    Let $( (x_n,y_n) )_{n=1}^\infty$ be Cauchy in $X\times Y$. I claim that $(x_n)$, $(y_n)$ are Cauchy
    in $X$ and $Y$ respectively. Let $\epsilon>0$ be arbitrary and let $N$ be such that
    $d( (x_n,y_n),(x_m,y_m) )<\epsilon$ for all $n,m\geq N$. Then $d_X(x_n,x_m)+d_Y(y_n,y_m)<\epsilon$ so that
    $d_X(x_n,x_m)<\epsilon$ and $d_Y(y_n,y_m)<\epsilon$; thus $(x_n)$ and $(y_n)$ are Cauchy in $X$ and $Y$,
    so there exists a limit $x\in X$ and $y\in Y$. But then $d( (x,y),(x_n,y_n) )=d_X(x,x_n)+d_Y(y,y_n)<\epsilon$
    for $n\geq N$ such that $d_X(x,x_n)<\epsilon/2$ and $d_Y(y,y_n)<\epsilon/2$ by convergence of $(x_n),(y_n)$.

    Thus $\lim_{n\to\infty}( (x_n,y_n) )=(x,y)\in X\times Y$ so that $X\times Y$ is complete.
\end{proof}
\begin{theorem}
    Suppose $A\subseteq X$ and $B\subseteq Y$ are compact.
    Then $A\times B\subseteq X\times Y$ is compact.
\end{theorem}
\begin{proof}
    We use the Bolzano-Weierstrass property, that $S$ is compact if and only if every sequence $X$ in $S$
    has a convergent subsequence.

    Thus it suffices to show that every sequence $\{(a_n,b_n)\}_{n=1}^\infty\subseteq A\times B$
    has a convergent subsequence. Since $(a_n)$ is a sequence in $A$, which is compact, we have some
    subsequence $(a_{n_k})$ which has limit $\lim_{k\to\infty}a_{n_k}=a\in A$. But then
    the sequence $(b_{n_k})$ is also a sequence in $B$, so it has a subsequence $(b_{n_{m_k}})$ which
    converges to some $b\in B$, and since every subsequence converges to the same limit, $(a_{n_{m_k}})\to a$
    as well.

    I now claim that $(\alpha_k,\beta_k):=((a_{n_{m_k}},b_{n_{m_k}}))_{n=1}^\infty$ converges to $(a,b)$.
    Indeed, let $\epsilon>0$ be arbitrary, and choose $N$ such that for all $n\geq N$, $d_X(\alpha_n,a)<\epsilon/2$
    and $d_Y(\beta_n,b)<\epsilon/2$. But then $d( (\alpha_n,\beta_n),(a,b) )= d_X(\alpha_n,a)+d_Y(\beta_n,b)<\epsilon$
    as desired.
\end{proof}

\subsection{Normed Vector Spaces}
We consider vector spaces over $\R$ or $\C$ - they could be finite dimensional or infinite dimensional. Every finite dimensional
vector space is isomorphic to $\R^n$ or $\C^n$ by basis representation.
\begin{definition}
    A \textbf{norm} is a map $\norm{\cdot}:V\to\R$ satisfying
    \begin{enumerate}
        \item $\norm{v}\geq 0$ $\forall v\in V$ and $\norm{v}=0$ iff $v=0$
        \item $\norm{\alpha v}=|\alpha|\norm{v}$ for all $\alpha\in\R, v\in V$.
        \item $\norm{V_1+V_2}\leq\norm{V_1}+\norm{V_2}$ for all $v_1,v_2\in V$.
    \end{enumerate}
\end{definition}
The pair $(V,\norm{\cdot})$ is called a normed vector space.
\begin{example}
    \begin{enumerate}
        \item $\R^n$ with $\norm{x}_2$
        \item $\R^n$ with $\norm{x}_\infty$
        \item $l^\infty$ with $\norm{x}_\infty=\sup_i|x_i|$.
        \item $l^p=\{(x_i)_{i=1}^\infty:\left( \sum\limits_{i=1}^\infty |x_i|^p \right)^{1/p}<\infty$ space with $1\leq p<\infty$.
            Then $\norm{x}_p=\left(\sum\limits_{i=1}^\infty |x_i|^p\right)^{1/p}$.
    \end{enumerate}
\end{example}
\begin{theorem}[Minkowski's Inequality]
    For $1\leq p\leq\infty$, $\norm{x+y}_p\leq\norm{x}_p+\norm{y}_p$ for any $x,y\in l^p$.
\end{theorem}
\begin{proof}
    We prove the case $p<\infty$. It's trivial if $\norm{x}_p=0$ or $\norm{y}_p=0$, so assume otherwise. Let $\alpha=\norm{x}_p$,
    $\beta=\norm{y}_p$. Let $u=\frac{x}{\norm{x}_p}$, $v=\frac{y}{\norm{y}_p}$ which implies $x=\alpha_u$, $y=\beta_v$.
    Choose $\lambda=\frac{\alpha}{\alpha+\beta}$, and notice that $1-\lambda=1-\frac{\alpha}{\alpha+\beta}=\frac{\beta}{\alpha+\beta}$.
    Then
    \begin{align*}
        |x_i+y_i|^p &\leq \left( |x_i|+|y_i| \right)^p = \left( \frac{\alpha|u_i|+\beta|v_i|}{\alpha+\beta} \right)^p(\alpha+\beta)^p\\
        &= \left( \lambda|u_i|+(1-\lambda)|v_i| \right)^p(\alpha+\beta)^p\\
        \intertext{By convexity,}
        &\leq \left( \lambda|u_i|^p+(1-\lambda)|v_i|^p \right)(\alpha+\beta)^p
    \end{align*}
    Taking sums, we have
    \begin{align*}
        \sum\limits_i|x_i+y_i| &\leq (\alpha+\beta)^p\left( \lambda\sum\limits_i|u_i|^p+(1-\lambda)\sum\limits_i|v_i|^p \right)\\
        &= (\alpha+\beta)^p(\lambda\norm{u}_p^p+(1-\lambda)\norm{v}_p^p)\\
        \intertext{Furthermore, $\norm{u}_p=\norm{v}_p=1$ by construction, so}
        &= (\alpha+\beta)^p\\
        &= (\norm{x}_p+\norm{y}_p)^p
    \end{align*}
    and we're done!
\end{proof}
\begin{example}
    \begin{enumerate}
        \item Space of polynomials of degree less than or equal to $n$; $\norm{p}_\infty=\max_{x\in[0,1]}|p(x)|$,
            $\norm{p}_1=\int_0^1|p(x)|$
        \item $C[0,1]=$continuous functions on $[0,1]$. Then $\norm{f}_\infty=\max_{x\in[0,1]}|f(x)|$ so $d(f,g)=\max_{x\in[0,1]}|f(x)-g(x)|$
            $\norm{f}_1=\int_0^1|f|$. We will show that $C[0,1]$ with $d_\infty$ is complete. Polynomials are dense in $C[0,1]$ with
            $d_\infty$.
    \end{enumerate}
\end{example}
\begin{theorem}
    Suppose $V$ is a finite dimensional normed vector space over $\R$ with basis $\{v_1,\ldots,v_k\}$. Then there
    exists $0<A<B<\infty$ such that for all $a_1,\ldots,a_n\in\R$
    \[A\norm{(a_1,\ldots,a_n)}_{\R^n}\leq \norm{\sum\limits_{i=1}^na_iv_i}_V\leq B\norm{(a_1,\ldots,a_n)}_{\R^n}\]
\end{theorem}
\begin{proof}
    We first have
    \begin{align*}
        \norm{\sum\limits_{i=1}^na_iv_i}_V&\leq\sum\limits_{i=1}^n\norm{a_iv_i}_V\\
        &= \sum\limits_{i=1}^n|a_i|\norm{v_i}\\
        &\leq \left( \sum\limits_{i=1}^n|a_i|^2 \right)^{1/2}\left( \sum\limits_{i=1}^n\norm{v_i}^2 \right)^{1/2}\\
        &= \norm{(a_1,\ldots,a_n)}_{\R^n}\cdot B
    \end{align*}
    For the reverse direction, consider the function $F:\R^n\to\R$ by $(a_1,\ldots,a_n)\mapsto\norm{\sum\limits_{i=1}^na_iv_i}_V$.
    We check
    \begin{align*}
        F(x)-F(y) &= \norm{\sum\limits_{i=1}^n x_iv_i}-\norm{\sum\limits_{i=1}^n y_iv_i}\\
        &= \norm{\sum\limits_{i=1}^n(x_i-y_i)v_i+\sum\limits_{i=1}^n}-\norm{\sum\limits_{i=1}^ny_iv_i}\\
        &\leq \norm{\sum\limits_{i=1}^n(x_i-y_i)v_i}+\norm{\sum\limits_{i=1}^n y_iv_i}-\norm{\sum\limits-{i=1}^ny_iv_i}
    \end{align*}
    so that
    \begin{align*}
        |F(y)-F(x)|&\leq\norm{\sum\limits_{i=1}^n(x_i-y_i)v_i}\\
        &\leq \left( \sum\limits_{i=1}^n|x_i-y_i|^2 \right)^{1/2} \left( \sum\limits_{i=1}^n\norm{v_i}^2 \right)^{1/2}\\
        &= B\norm{x-y}_{\R^n}=B_{\R^n}(x,y)
    \end{align*}
    Hence $F$ is continuous. Now let $S=\{x\in\R^n:\norm{x}=1\}$ is the boundary of the sphere in $\R^n$. Then
    $S$ is compact being a closed and bounded set in $\R^n$. Since $\{v_1,\ldots,v_n\}$ are a basis, $F(x)\neq 0$ for
    any $x\in S$. Thus $F>0$ on $S$. By E.V.T, $F$ has a minimum on $S$, say $F(x_0)=\epsilon>0$. So $F(x)\geq\epsilon$
    for all $x\in S$.

    Now let $(a_1,\ldots,a_n)\in\R^n$ with $a\neq 0$ without loss of generality. Then $a/\norm{a}_{\R^n}\in S$,
    and
    \begin{align*}
        F(a)&=\norm{\norm{A}\sum \frac{a_i}{\norm{A}}v_i}_V=\norm{a}\norm{\sum\frac{a_i}{\norm{A}}v_i}_V\\
        &= \norm{a}\norm{F\left( \frac{a}{\norm{a}} \right)}\\
        &\geq\epsilon\norm{(a_1,\ldots,a_n)}_{\R^n}
    \end{align*}
    Thus
    \[\epsilon\norm{(a_1,\ldots,a_n)}\leq F(a)=\norm{\sum a_iv_i}\]
    and the results holds.
\end{proof}
Consider the map $T:\R^n\to V$ given by $(a_1,\ldots,a_n)\mapsto\sum a_iv_i$. We have just proven that
\[A\norm{(a_1,\ldots,a_n)}_{\R^n}\leq\norm{T(a_1,\ldots,a_n)}_V\leq B\norm{(a_1,\ldots,a_n)}_{\R^n}\]
so that $T$ is continuous: $\norm{T(x)-T(y)}\leq B\norm{x-y}_{\R^n}$. In the exact same way, $T^{-1}$ is continuous as
well. Thus $T$ is a homeomorphism between $\R^n$ and $V$. Thus $V\cong\R^n$ are homeomorphic.
\begin{corollary}
    \begin{enumerate}
        \item Subsets of finite dimensional normed vector spaces are compact iff they are closed and bounded
        \item Finite-Dimensional normed vector spaces are complete
        \item Any finite-dimensional subspace of a normed vector space is complete
            \begin{proof}
                Let $V$ be a normed vector space and $W$ finite dimensional subspace, $W$ is a normed space homeomorphic
                to $\R^n$, which is complete.
            \end{proof}
    \end{enumerate}
\end{corollary}
In other words, finite dimensional normed vector spaces are boring! However, infinite dimensional normed vector spaces
can be much more interesting.
\chapter{Function Spaces}
\section{Continuous Functions on Metric Spaces}
\subsection{Continuity}
\begin{definition}
    A function $f$ is continuous at $a\in X$ if for any $\epsilon>0$, there exists a $\delta>0$ such that $D_Y(f(x),f(a))<\epsilon$
    whenever $d_X(x,a)<\delta$.
\end{definition}
We say $f$ is continuous of it is continuous at every $a\in X$.
\begin{proposition}
    $f$ is continuous at $x_0$ iff whenever $(x_n)$ in $X$ converging to $x_0$, then $(f(x_n))$ converges to $f(x_0)$.
\end{proposition}
\begin{example}
    \begin{enumerate}
        \item Consider the metric space $X_1=(\R,\text{usual})$ and $X_2=(\R,\text{discrete})$. Then $f(x)=\Id(x)=x$
            is not continuous. Similarly, if $S$ is any set equipped with the discrete metric, then $f:S\to X$ is
            always continuous for any function $f$ metric space $X$. The choice of metric is important!
        \item Let $f:X\to Y$ and take $a\in X$ that is not an accumulation point. Then $f$ is continuous at $a$ because
            we have $\delta>0$ such that $B_x(a,\delta)=\{a\}$.
    \end{enumerate}
\end{example}
\begin{proof}
    $(\Rightarrow)$ Take a sequence $(x_n)\to x_0$. Check $\forall\epsilon>0$ there exists $N$ such that $d(f(x_n),f(x_0))<\epsilon$
    if $n\geq N$. Since $f$ is continuous at $x_0$, there exists a $\delta>0$ such that $d(f(x),f(x_0))<\epsilon$ if $d(x,x_0)<\delta$.
    Then since $(x_n)\to x_0$, there exists an $N$ such $x_n\in B(x_0,\delta)$ for all $n\geq N$. Then $d(f(x_n),f(x_0))<\epsilon$
    for all $n\geq N$.
    $(\Leftarrow)$ Assume $f$ is not continuous at $x_0$. So there exists $\epsilon>0$ such that ``no $\delta$ works'',
    meaning for all $\delta>0$ there exists $x\in B(x_0,\delta)$, but $f(x)\notin B(f(x_0),\epsilon)$. Do this for
    $\delta=1/n,n\in N$. So for all $n$ get $x_n\in B(x_0,1/n)$ but $f(x_n)\notin B(f(x_0),\epsilon)$. We have $(x_n)\to x_0$
    but $d(f(x_n),f(x_0))\geq\epsilon$ for all $n$, so $(f(x_n))\not\to f(x_0)$.
\end{proof}
\begin{proposition}
    If $f,g:X\to\R$ is continuous, then so are $f\pm g$, $f\cdot g$, $f/g$ for $g\neq 0$.
\end{proposition}
$\forall\epsilon>0$ there exists $\delta>0$ such that $f(B(a,\delta))\subseteq B(f(a),\epsilon)$. Define the preimage
as follows. Let $A\subseteq Y$, then $f^{-1}(A)=\{x\in X:f(x)\in A\}$. We can state continuity equivalently as
$\forall\epsilon,\exists\delta>0\st B(a,\delta)\subseteq f^{-1}(B(f(a)),\epsilon)$. This says that $a\in\int f^{-1}(B(f(a),\epsilon))$.
\begin{theorem}
    The following are equivalent for $f:X\to Y$.
    \begin{enumerate}
        \item $f$ is continuous.
        \item $\forall V$ open in $Y$, $f^{-1}(V)$ is open in $X$.
        \item $\forall W$ closed in $Y$, $f^{-1}(W)$ is closed in $X$.
    \end{enumerate}
\end{theorem}
\begin{proof}
    $(1\Rightarrow 2)$. Take $V$ open in $Y$, and take $x_0\in f^{-1}(V)$. Let $B(f(x_0),\epsilon)\subseteq Y$, and
    by continuity there exists $\delta>0$ such that $f(y)\in B(f(x_0),\epsilon)$ for all $y\in B(x_0,\delta)$. But then
    $f(B(x_0,\delta))\subseteq B(f(x_0),\epsilon)\Rightarrow B(x_0,\delta)\subseteq f^{-1}(B(f(x_0),\epsilon))\subseteq f^{-1}(V)$

    $(2\Rightarrow 3)$. Let $W\subseteq Y$ be closed. $W^c$ is open so $f^{-1}(W^c)$ is open, but $f^{-1}(W^c)=(f^{-1}(W))^c$.
    Since $(f^{-1}(W))^c$ is open, $f^{-1}(W)$ is closed. The proof of $(3\Rightarrow 2)$ is equivalent.

    The proof of $(2\Rightarrow 1)$ is left as an exercise.
\end{proof}
\begin{corollary}
    $f:X\to Y$, $g:Y\to Z$. Then $g\circ f:X\to Z$ is continuous.
\end{corollary}
\begin{theorem}
    Let $X,Y$ be metric spaces and $K\subseteq X$ compact, and $f:K\to Y$ is continuous. Then $f(K)\subseteq Y$ is compact.
\end{theorem}
\begin{proof}
    The idea is that continuous functions take open covers to open covers. Let $\{U_\alpha\}_{\alpha\in A}$ be an open
    cover for $f(K)$. Consider $\{f^{-1}(U_\alpha)\}_{\alpha\in A}$ which are open since $f$ is continuous. This collection
    is an open cover: if $x\in K$, then $f(x)\in U_\alpha$ for some $\alpha$. Hence $x\in f^{-1}(U_\alpha)$ and therefore
    $\{f^{-1}(U_\alpha)\}$ is an open cover of $K$. As $K$ is compact, there is a finite subcover $\{f^{-1}(U_{\alpha_1}),\ldots,f^{-1}(U_{\alpha_n})\}$.
    But then $\{U_{\alpha_1},\ldots,U_{\alpha_n}\}$ covers $f(K)$: say $f(x)\in f(K)$ for $x\in K$, so $x\in f^{-1}(U_{\alpha_i}$
    for some $i$ and thus $f(x)\in U_{\alpha_i}$.
\end{proof}
\begin{corollary}
    Let $K$ be compact and $f:K\to\R$ be continuous. Then $f$ has a maximum and minimum value on $K$.
\end{corollary}
\begin{corollary}
    If $f:K\to\R$ is continuous, $K$ compact, and $f(x)>0$ for all $x\in K$, then there exists $\epsilon>0$ such that
    $f(x)\geq\epsilon$ for all $x\in K$.
\end{corollary}
\begin{proposition}
    Suppose $f:K\subseteq X\to Y$ is continuous and bijective, and $K$ compact. Then $f^{-1}$ is continuous (i.e. $f$
    is actually a homeomorphism).
\end{proposition}
\begin{proof}
    Consider $f^{-1}:Y\to K$. Let $W\subseteq K$ be closed and check that $(f^{-1})^{-1}(W)=f(W)$ is closed. $W$ is a
    closed subset of compact $K$, so $W$ is complete. $W$ is totally bounded since $X$ is totally bounded. Therefore
    $W$ is compact, so $f(W)$ is compact and is therefore closed.
\end{proof}
\begin{example}
    Compactness of $X$ is certainly necessary! Take $f:[0,2\pi)\to\partial S^1$ by $t\mapsto(\cos t,\sin t)$. This function
    is bijective and continuous, but the inverse is not continuous (consider a neighbourhood of $0$) in the subspace
    topology of $\R^2$.
\end{example}
\begin{proposition}
    If $f:X\to f(X)$ is continuous and $X$ is connected, then $f(X)$ is connected.
\end{proposition}
\begin{proof}
    Suppose $f(X)$ is not connected, say $f(X)=A\cup B$ where $A,B$ are open in $f(X)$, disjoint, and non-empty. But
    then $X=f^{-1}(A)\cup f^{-1}(B)$ where $f^{-1}(A),f^{-1}(B)$ are open in $X$, disjoint, and non-empty, contradicting
    connectedness of $X$.
\end{proof}
\subsection{Path Connectedness}
A stronger notion of connected is path-connected.
\begin{definition}
    We say $E$ is $\textbf{path-connected}$ if for any $x,y\in E$, there exists a continuous function defined on an interval
    $[a,b]$ such that $f(a)=x$, $f(b)=y$, and $\Im f\subseteq E$.
\end{definition}
\begin{proposition}
    Path connected implies connected.
\end{proposition}
\begin{proof}
    Say it is not connected, so we have $E=A\cup B$ where $A,B$ are open, disjoint, and non-empty. Then pick
    $x\in A,y\in B$ and let $\gamma:[0,1]\to E$ be a path connecting. Note that $f([a,b])$ is connected ince
    $f$ is continuous and $[a,b]$ is connected. But then $f([a,b])=\Im f=(f([a,b])\cap A)\cup(f([a,b])\cap B)$ are disjoint
    (since $A,B$ are disjoint) and non-empty (since $x,y$ are in the first and second respectively), so
    $A$,$B$ are disconnecting sets for $f([a,b])$, a contradiction.
\end{proof}
\begin{example}
    The ``Topologist Sine Curve'': let $X=\{x,\sin 1/x:x>0\}\cup\{(0,0)\}=E\cup\{(0,0)\}\subseteq\R^2$. Then $X$ is connected
    but not path-connected.
\end{example}
\begin{proof}
    Note that $T_p=\{(x,\sin(1/x):x\in[p,1]\}$ is connected for any $p>0$ since it is the continuous image
    of a connected set. With this in mind, suppose for contradiction that $T$ has disconnecting sets
    $U$ and $V$, and assume without loss of generality that $(0,0)\in U$ and $f(t)=x\in V$ for some $t\neq 0$.
    Since $U$ is open, there exists some $\epsilon-$ball $B_\epsilon\left( (0,0) \right)\subseteq U$.

    Furthermore, the zeros of $\sin(1/x)$ form a sequence which converges to 0, so that there exists some
    $u$ such that $\sin(1/u)=0$ and $u<\epsilon$. Then fix $p=\min\{t,u\}$. We claim that $U$ and $V$ are
    disconnecting sets for $T_p$. $U$ and $V$ are certainly still
    open and satisfy $T_p\subseteq T\subseteq U\cup V$, and $U\cap V$ is empty (equivalent characterization
    given by past assignment). Furthermore, $B_\epsilon\left( (0,0) \right)\cap T_p\neq\emptyset$,
    so $U\cap T_p\neq\emptyset$ by construction, and $V\cap T_p\neq\emptyset$ by choice of $p$. Thus $U$ and
    $V$ disconnect $T_p$, a contradiction.

    However, $T$ is not path connected. To see this, suppose it is. Then there exists a continuous function
    $\gamma$ on $[0,1]$ with $\gamma(0)=(0,0)$ and $\gamma(1)=(1,\sin(1))$. We claim that the image
    of $\gamma$, $\Gamma$, must be equal to $T$. We certainly have $\Gamma\subseteq T$ by definition. Then
    let $(x,f(x))\in\Gamma$. We know that $(0,f(0))$ and $(1,f(1))$ are in $T$, so suppose
    for contradiction again that $(a,f(a))\notin \Gamma$ for some $a\in(0,1)$. But then any point along the
    line $x=a$ is not in $\Gamma$. Note that $\gamma$ is continuous, so it is componentwise continuous. Write
    $\gamma(t) = (x(t),y(t))$. Then $x(0)=0$ and $x(1)=1$, so by the intermediate value theorem we must have
    $x(c)=a$ for some $c$, which is impossible by assumption.

    Therefore, $T=\Gamma$. However, this is a contradiction since $\Gamma$ must be compact and $T$ is not compact.
\end{proof}
\subsection{Uniform Continuity}
\begin{definition}
    Say $f:X\to Y$ is uniformly continuous if for all $\epsilon>0$ there exists a $\delta>0$ such that $d_Y(f(x),f(y))<\epsilon$
    whenever $d_X(x,y)<\delta$.
\end{definition}
\begin{example}
    Here are some functions which are continuous but not uniformly continuous.
    \begin{enumerate}
        \item $f(x)=x^2$ on $\R$.
        \item $f(x)=1/x$ on $(0,1)$.
    \end{enumerate}
\end{example}
\begin{theorem}
    Let $K$ be compact and $f:K\to Y$ continuous. Then $f$ is uniformly continuous.
\end{theorem}
\begin{proof}
    Fix $\epsilon>0$. Since $f$ is continuous, for all $x\in K$, there exists $\delta_x>0$ such that if $d_X(x,y)<\delta$
    then $d_Y(f(x),f(y))<\epsilon$. Consider $B(x,\delta_x/2)$ as an open cover, and get a finite subcover, say $\{B(x_i,\delta_{ x_i}/2)\}$
    for $i=1,\ldots,n$ and choose $\delta=\min\{\delta_{x_i}\}$. Suppose $x,y\in K$ and $d_X(x,y)<\delta$. Then choose $i$
    such that $x\in B(x_i,\delta_{x_i}/2)$. Then
    \[d(y,x_i)\leq d(y,x)+d(x,x_i)<\delta+\frac{\delta_{x_i}}{2}\leq\delta_{x_i}\]
    but this implies $d(x,x_i)<\delta_{x_i}$ so that $d(f(x),f(x_i))<\epsilon$. Furthermore, $d(x_i,y)<\delta_{x_i}$ so
    \[(f(y),f(x_i))<\epsilon\Rightarrow d(f(x),f(y))<2\epsilon\]
    by the triangle inequality.
\end{proof}



\section{Fundamentals of Function Spaces}
\subsection{Basic Notions}
Given metric spaces $X,Y$, let $C(X;Y)$ denote the set of continuous functions $f:X\to Y$.
We can define the uniform distance $d(f,g):=\sup_{x\in X}d_Y(f(x),g(x))$ where $d_Y$ is the metric on $Y$.

Since functions $f:X\to\R$ are very common, let $C(X)$ denote the continuous functions $f:X\to\R$ and $C_b(X)\subseteq C(X)$ denote bounded continuous functions.
If $X$ is compact, then $C(X)=C_b(X)$ by EVT.

Note that $C(X)$ and $C_b(X)$ are (infinite dimensional) normed vector spaces with pointwise addition and scalar multiplication.
We can define the norm $\norm{f}=d(f,0)$.
Furthermore, with the additional operation of pointwise multiplication, we can consider $C(X)$ and $C_b(X)$ as algebras over $\R$.

Let's now define two notions of convergence:
\begin{definition}
    We say $(f_n)$ converges pointwise to $f:X\to Y$ if for all $\epsilon>0$ and all $x\in X$, there exists $N=N(\epsilon,x)$
    such that $d_Y(f_n(x),f(x))<\epsilon$ for all $n\geq N$.
\end{definition}
In other words, for each $x\in X$, the sequence $(f_n(x))$ converges in $Y$.
\begin{definition}
    We say $(f_n)\to f$ uniformly if for all $\epsilon>0$, there exists some $N$ such that $d(f_n,f)<\epsilon$ for all $n\geq N$.
\end{definition}
This is to say that convergence with respect to the uniform metric is, unsurprisingly, uniform convergence.
Note that the choice of $N$ depends only on $\epsilon$ and not on $x$.
Uniform limits are useful:
\begin{theorem}
    Suppose $(f_n)\to f$ uniformly, where $f_n$ is continuous.
    Then $f$ is continuous.
\end{theorem}
\begin{proof}
    Let $\epsilon>0$. Then for all $x\in X$, there exists $\delta$ such that $d(x,y)<\delta$ implies $d(f(x),f(y))<\epsilon$.
    We also have an $N$ such that $d(f_n(t),f(t))<\epsilon/3$ for all $n\geq N$ for all $t\in X$. Now consider $f_N$:
    this is continuous so take $\delta$ such that $d(x,y)<\delta\Rightarrow f(d_N(x),f_N(y))<\epsilon/3$. Suppose
    $d(x,y)<\delta$. Then $d(f(x),f(y))\leq d(f(x),f_N(x))+d(f_N(x),f_N(y))+d(f_N(y),f(y))=\epsilon$. Thus $f$ is continuous
    at $x$.
\end{proof}
Equivalently, we can talk about a sequence being uniformly Cauchy.
This is equivalent to saying that $(f_n)$ is cauchy with respect to the uniform metric.
\begin{definition}
    Say $f_n:X\to Y$ is uniformly Cauchy if for all $\epsilon>0$, there exists $N$ such that $\forall n,m\geq N$,
    $d(f_n(x),f_m(x))<\epsilon$ for all $x\in X$.
\end{definition}
Naturally, we have the standard theorem.
Note that we need to do slightly more work for the forward direction since we only assume completeness of $Y$.
\begin{theorem}
    Let $X,Y$ be metric spaces, and assume $Y$ is complete. Then the sequence of functions $f_n:X\to Y$ is uniformly
    Cauchy iff $(f_n)$ is uniformly convergent.
\end{theorem}
\begin{proof}
    $(\Leftarrow)$.
    This direction does not require completeness.
    Say $f_n\to f$ uniformly.
    Choose $N$ so that $d(f_n(x),f(x))<\epsilon/2$.
    Then if $n,m\geq N$, then $d(f_n(x),f_m(x))\leq d(f_n(x),f(x))+d(f(x),f_m(x))<\epsilon$ so $(f_n)$ is uniformly Cauchy.

    $(\Rightarrow)$.
    We need completeness so that our limiting function exists pointwise.
    Assume $(f_n)$ is uniformly Cauchy.
    Then for all $x\in X$, $(f_n(x))$ is a Cauchy sequence in $Y$.
    Since $Y$ is complete, so for each $x\in X$, $(f_n(x))$ converges, say, to $a_x\in Y$.
    Put $f(x)=a_x$, and we certainly have $f_n\to f$ pointwise.
    We show that this limit is in fact uniform.

    Pointwise convergence means that for all $\epsilon>0$ and for all $x\in X$, there exists $M_x$ so that for all $m\geq M_x$,
    $d(f_m(x),f(x))<\epsilon$. Since this is uniformly Cauchy, for all $\epsilon>0$, there exists $N$ such that
    for all $n,m\geq N$ and for all $x$, $d(f_n(x),f_m(x))<\epsilon$. Take $\epsilon>0$ and get $N$. Temporarily fix
    $x\in X$, and let $n\geq N$. Pick $m\geq\max(N,M_x)$. Then
    \begin{align*}
        f(f_n(x),f(x)) &\leq d(f_n(x),f_m(x))+d(f_m(x),f(x))\\
        <2\epsilon
    \end{align*}
    Since $d(f_n(x),f(x))<2\epsilon$ for all $n\geq N$ and for all $x\in X$, $(f_n)\to f$ uniformly.
\end{proof}
\subsection{Two Conditions for Uniform Convergence}
\begin{theorem}[Weierstrass M-test]
    Let $f_n:X\to Y$ where $Y$ is a complete normed vector space. Suppose there exist constants $M_n\in\R$ such that
    $\norm{f_n(x)}_Y\leq M_n$ for all $x,n$ and $\sum\limits_{n=1}^\infty M_n<\infty$. Then $\sum f_n$ converges
    uniformly.
\end{theorem}
\begin{proof}
    Let $S_n(x)=\sum\limits_{k=1}^n f_n(x)$. Then
    \[d(S_n(x),S_m(x))=\norm{S_n(x)-S_m(x)}=\norm{\sum\limits_{k=m+1}^n f_k(x)}\leq\sum\limits_{k=1}^n\norm{f_k(x)}\leq\sum\limits_{k=n+1}^n M_k<\epsilon\]
    for sufficient $n,m$ since $\sum M_k$ converges independently of $x$.
\end{proof}
\begin{theorem}[Dini's Theorem]
    Suppose $K$ is compact and $f_n:K\to\R$ converges pointwise to $f$ (on $K$). if $f_n,f$ are continuous and $f_{n+1}(x)\leq f_n(x)$
    for all $x\in K$, then $f_n\to f$ uniformly.
\end{theorem}
\begin{proof}
    Let $g_n(x)=f_n(x)-f(x)$.
    We want to show $g_n\to 0$ uniformly.
    We certainly have $g_n\to 0$ pointwise, that $g_n\geq 0$ and $g_n$ are monotonic (from monotonicity of $f_n$).

    Let $\epsilon>0$.
    We want to show that there exists $N$ such that for all $n\geq N$, $0\leq g_n(x)<\epsilon$.
    Since $g_n\to 0$ pointwise, for all $t\in K$ there exists $N_t$ such that $0\leq g_{N_t}(t)<\epsilon/2$ implies $0\leq g_n(t)<\epsilon/2$ for all $n\geq N_t$.
    Furthermore, since each $g_{N_t}$ is continuous, there exists $\delta_t>0$ so that $d(t,y)<\delta_t$ implies $|g_{N_t}(t)-g_{N_t}(y)|<\epsilon/2$.
    Consider $B(t,\delta_t)\subseteq K$ for each $t\in K$.
    These form an open cover of $K$, so take a finite subcover
    \[B(t_1,\delta_{t-1}),\ldots,B(t_r,\delta_{t_r})\]
    Now suppose $x\in B(t_i,\delta_{t_i})$ so that $d(x,t_i)<\delta_{t_i}$
    \[|g_{N_{t_i}}(x)|\leq |g_{N_{t_i}}(t_i)-g_{N_{t_i}}(x)|+|g_{N_{t_i}}(t_i)|<\epsilon\]
    But then fix $N=\max\{N_{t_1},\ldots,N_{t_r}\}$.
    Now suppose $n\geq N$, and let $x\in K$ be arbitrary.
    We have some $i$ so that $x\in B(t_i,\delta_{t_i})$.
    By monotonicity of $g_n$ and the choice of $N$, we have
    \[0\leq g_n(x)\leq g_N(x)\leq g_{N_{t_i}}(x)<\epsilon\]
    and since this holds for every $x\in K$ and $\epsilon>0$, we have $g_n\to 0$ uniformly.
\end{proof}
\begin{example}
    Consider $U=\{f\in C([0,1]):f(x)>0\forall x\in[0,1]\}$. Then $U$ is open.
\end{example}
\begin{proof}
    Take $f\in U$.
    By E.V.T. $f$ has a minimum, say $f(x_1)\leq f(x)$ for all $x\in[0,1]$.
    Set $\epsilon=f(x_1)$.
    But then $B(f,\epsilon)\subseteq U$ since if $g\in B(f,\epsilon)$, then $\norm{f-g}<\epsilon$.
    Thus for all $x$, $g(x)>f(x)-\epsilon=f(x)-f(x_1)\geq 0$ so that $g(x)>0$ for all $x$, and $g\in U$.

    We can also see that $U^c$ is closed, where $U^c=\{f\in C([0,1]):\text{$f(x)\leq 0$ for some $x$}\}$.
    We will see that $U^c$ is closed by checking it contains all its accumulation points.
    Say $f$ is an accumulation point of $U^c$.
    Then $\exists f_n\to f$ uniformly where $f_n\in U^c$.
    We have some $x_n\in[0,1]$ such that $f_n(x_n)\leq 0$.
    Then by Bolzano-Weierstrass, there is a subsequence $x_{n_k}\to x_0$.
    Since $f$ is continuous, notice $f(x_{n_k})\to f(x_0)$ so that $f(x_{n_k})\leq 0$, $f(x_0)\leq 0$.
    Thus $f\in U^c$ so we're done!
\end{proof}
\chapter{The Function Space $C(X)$}
\section{Compactness in $C_b(X)$}
Recall that $E\subseteq C_b(X)$ is compact if and only if it is complete and totally bounded, and that compactness implies closedness and boundedness (but the converse does not hold).
In fact, $E\subseteq C_b(X)$ is bounded if and only if there exists $N$ such that $B(0,N)\supseteq E$.
So in our case, we can take $N=M+\norm{f}$.
If $g\in E$, then $d(g,f)<M$ so $d(g,0)\leq d(g,f)+d(f,0)<M+\norm{f}=N$.
Thus $E\subseteq C_b(X)$ is bounded if and only if there exists $N$ such that $E\subseteq B(0,N)$ if and only of $\norm{f}<N$ for all $f\in E$.
We thus say that ``$E$ is uniformly bounded''.

However, closed and bounded does not imply compact.
For example, consider $E=\{x^n:n=1,2,\ldots\}\subseteq C([0,1])$.
Then $\norm{x^n}\leq 1$ for all $x^n\in E$, so $E$ is bounded.
If $f$ is an accumulation point of $E$, then there exists $n_k$ such that $f_{n_k}\to f$ uniformly.
But
\[f_{n_k}(x)=x^{n_k}\to
    \begin{cases}
        0 &:x\neq 1\\
        1 &:x = 1
    \end{cases}
\]
Since $f_{n_k}\to f$ uniformly, it must also be pointwise.
However, $f_{n_k}\to g$ pointwise so $g=f$, but $g\notin C[0,1]$.
Since $g$ is not continuous, we don't have $f_n\to g$ uniformly, so $f$ cannot exist.
Thus means that $E$ has no accumulation points, so $E$ is closed.

Furthermore, $E$ is not compact because the sequence $(f_n(x)=x^n)$ has no convergent subsequence.
So by Bolzano-Weierstrass, $E$ is not compact.
\begin{example}
    The set
    \[E=\left\{\frac{x^2}{x^2+(1-nx)^2}:n=1,2,3,\ldots\right\}\subseteq C([0,1])\]
    is closed, bounded, and not compact.
\end{example}
\subsection{Equicontinuity}
\begin{definition}
    Let $E\subseteq C_b(X)$.
    We say $E$ is equicontinuous if for all $\epsilon>0$, there exists $\delta>0$ such that for all $f\in E$
    and $x,y\in X$ such that $d(x,y)<\delta$ we have $|f(x)-f(y)|<\epsilon$.
\end{definition}
This is to say that this family is uniformly continuous across the points and between functions as well.
Given an $\epsilon$, the same $\delta$ must show uniform continuity for all $f\in E$.
\begin{example}
    \begin{enumerate}
        \item $E=\{f\}$ where $f$ is uniformly continuous.
        \item $E=\{f_1,f_2,\ldots,f_n\}$ is equicontinuous if and only if each $f_i$ is uniformly continuous for $i=1,\ldots,n$.
            Given $\delta>0$, we take $\delta_i$ from each $f_i$ being uniformly continuous, and take $\delta$ to be their minimum.
        \item $E=\{x^n:n=1,2,3,\ldots\}\subseteq C([0,1])$ is not equicontinuous, even though each function is uniformly continuous.
            Not equicontinuous means that there exists $\epsilon>0$ such that there exists $x,y\in X$ with $d(x,y)<\delta$
            and some $f\in E$ with $f|(x)-f(y)|\geq\epsilon$. Take $\epsilon=1/2$ and $\delta>0$. Take $x=1$, $y=1-\delta/2$,
            and choose $n$ such that $y^n<1/2$. But then $|f_n(x)-f_n(y)|=x^n-y^n>1/2=\epsilon$.
    \end{enumerate}
\end{example}

\subsection{Arzela-Ascoli Theorem}
Our goal is to characterize compactness of $E\subseteq C_b(X)$.
We have the following important theorem:
\begin{theorem}
    Let $X$ be compact.
    Assume the set $\{f_n:n=1,2,3,\ldots\}\subseteq C(X)$ is a pointwise bounded, equicontinuous family.
    Then there exists a subsequence of $(f_n)_{n=1}^\infty$ which converges uniformly in $C(X)$.
\end{theorem}
We will prove the theorem in parts.
\begin{lemma}
    Let $K$ be a countable set and let $\{f_n:K\to\R,n\in\N\}$ be a pointwise bonded family.
    Then there exists a subsequence of $(f_n)_{n=1}^\infty$ that converges pointwise at every $k\in K$.
\end{lemma}
\begin{proof}
    Let $K=\{x_j\}_{j=1}^\infty$.
    Start with $(f_n(x_1))_{n=1}^\infty$.
    This is a bounded sequence of real numbers since $\{f_n\}$ is pointwise bounded.
    Then by Bolzano-Weierstrass, there exists a convergent subsequence $(f_{n_j}(x_1))_{j=1}^\infty$.
    Rename $f_{n_j}$ as $f_j^{(1)}$, so $(f_j^{(1)}(x_1))$ converges.
    Next, look at $(f_j^{(1)}(x_2))$, which as above has a convergent subsequence, say $(f_j^{(2)}(x_2))$.
    Repeat this to get a collection of convergent subsequences $(f_j^{(m)}(x_m))$ for $m\in\N$.
    But then by construction the sequence given by $(f_i^{(i)}(x_i))$ is eventually a subsequence of a convergent subsequence for any $x_i\in K$, so it converges at all $k\in K$.
\end{proof}
We can now prove the theorem!
\begin{proof}
    Since $X$ is compact, it is separable, so let $K\subseteq X$ be a countable dense subset.
    Consider $\{f_n|_K\}$, a pointwise bounded family.
    By Lemma 2, there is a subsequence $(f_{n_k})$ that converges pointwise on $K$.
    Our goal is to show that $(f_{n_k})$ converges uniformly on $X$.
    We will do this by showing that it is uniformly Cauchy on $X$.
    Thus let $\epsilon>0$.
    Since $\{f_n\}$ are equicontinuous, there exists $\delta>0$ such that $d(x,y),\delta$ implies that $|f_n(x)-f_n(y)|<\epsilon/3$ for all $f_n$.
    Now consider $\{B(x,\delta):x\in K\}$.
    Since $K$ is ense, for all $y\in X$, there exists $x\in K$ such that $y\in B(x,\delta)$.
    Thus these balls cover $X$.
    Since $X$ is compact there is a finite subcover, say $B(x_1,\delta),B(x_r,\delta)$.
    We have $(f_{n_k}(x_i))_{k=1}^\infty$ converges for each $i\in 1,\ldots,r$.
    Thus for ech $i$, there are Caucy sequences (in $\R$).
    Hence for each $i$, there exists $N_i$ such that if $n_k,n_l\geq N_i$, then
    \[|f_{n_k}(x_i)-f_{n_l}(x_i)|<\frac{\epsilon}{3}\]
    Let $N=\max\{N_1,\ldots,N_r\}$.
    Then if $n_k,n_l\geq N$, then $|f_{n_k}(x_i)-f_{n_l}(x_i)|<\epsilon/3$.
    Furthermore, if $y\in X$, then $y\in B(x_i,\delta)$ and $d(y,x_i)<\delta$ and equicontinuity implies $|f_{n_k}(y)-f_{n_l}(x_i)|<\epsilon/3$ as well.
    But now with a direct application of the triangle inequality and the above statements, for any $x\in X$, we have
    \[|f_{n_k}(x)-f_{n_l}(x)|\leq |f_{n_k}(x)-f_{n_k}(x_i)|+|f_{n_k}(x_i)-f_{n_l}(x_i)|+|f_{n_l}(x_i)-f_{n_l}(x)|<\epsilon\]
    as desired.
\end{proof}
\begin{corollary}[Arzela-Ascoli]
    Suppose $X$ is compact.
    Then $E\subseteq C(X)$ is compact iff $E$ is pointwise bounded, closed, and equicontinuous.
\end{corollary}
\begin{proof}
    $(\Rightarrow)$ is independent of the theorem.
    Let $E$ be compact.
    Then $E$ is closed and bounded (meaning uniformly bounded and hence pointwise bounded).
    We check that $E$ is equicontinuous.
    Suppose not, then there exists $\epsilon>0$ such that for all $\delta=1/n$, $n\in\N$, there exists $x_n,y_n\in X$ such that $d(x_n,y_n)<1/n$ with $f_n\in E$ with $|f_n(x_n)-f_n(y_n)|\geq\epsilon$.
    Since $E$ is comact by Bolzano-Weierstrass, there exists a convergent subseuence $(f_{n_k})_{k=1}^\infty$.
    By the previous proposition, $\{f_{n_k}:k=1,2,\ldots\}$ is equicontinuous.
    So for this choice of $\epsilon$ there is a $\delta$ that ``works''. Pick $n_k$ large enough that $1/n_k<\delta$. Then
    \[d(x_{n_k},y_{n_k})<\frac{1}{n_k}<\delta\]
    so $|f_{n_k}(x_{n_k})-f_{n_k}(y_{n_k})|<\epsilon$, a contradiction.

    $(\Leftarrow)$ We will verify the Bolzano-Weierstrass characterization of compactness - that every sequence from $E$ has a convergent subsequence with limit in $E$.
    Take a sequence $(f_n)_{n=1}^\infty$ in $E$.
    Since $E$ is pointwise bounded and equicontinuous, so is $\{f_n:n\in\N\}$.
    Thus by the above theorem, there is a convergent subsequence, but since $E$ is closed, this limit must be in $E$.
\end{proof}
\section{Dense Subsets of $C(X)$}
\subsection{Polynomials in $C[0,1]$}
\begin{theorem}[Weierstrass]
    Let $f:[0,1]\to\R$ be continuous and let $\epsilon>0$.
    Then there is a polynomial $p$ such that $\norm{p-f}<\epsilon$. So $\{\text{Poly}\}$ is dense in $C[0,1]$.

    In fact, the Bernstein polynomial
    \[p_n(x)=\sum\limits_{k=0}^n\binom{n}{k}f\left(\frac{k}{n}\right)x^k(1-x)^{n-k}\]
    converge uniformly to $f$.
\end{theorem}
Intuitively, suppose we toss a biased coin with probability $x$ of heads and $1-x$ of tails.
Suppose the payoff is $f(k/n)$ dollars if you get $k$ heads and $n$ tosses.
Then the expected payoff is $\sum\limits_{k=0}^nf\left(\frac{k}{n}\right)\binom{n}{k}x^k(1-x)^{n-k}=p_n(x)$.
In the long run we expected $xn$ heads in $n$ tosses, so the expected payoff in the long run should $f(xn/n)=f(x)$.
With this in mind, let's prove the theorem
\begin{proof}
    First, some technical calculations.
    By the Binomial theorem, we have
    \[(x+y)^n=\sum\limits_{k=0}^n\binom{n}{k}x^ky^{n-k}\]
    which we can view as a function of $x$.
    Differentiate with respect to $x$ twice to get
    \[n(x+y)^{n-1}=\sum\limits_{k=0}^n\binom{n}{k}kx^{k-1}y^{n-k}\]
    \[n(n-1)(x+y)^{n-2}=\sum\limits_{k=0}^n\binom{n}{k}k(k-1)x^{k-2}y^{n-k}\]
    Set $r_k(x)=\binom{n}{k}x^k(1-x)^{n-k}$ and $p_n(x)=\sum\limits_{k=0}^nf(k/n)r_k(x)$.
    At $y=1-x$, we have $nx=\sum\limits_{k=0}^nkr_k(x)$ and $x^2n(n-1)=\sum\limits_{k=0}^nk(k-1)r_k(x)=\sum\limits_{k=0}^nk^2r_k(x)-nx$.
    So
    \begin{align*}
        \sum\limits_{k=0}^n (k-nx)^2r_k(x) &= \sum\limits_{k=0}^n (k^2-2knx+n^2x^2)r_k(x)\\
                                           &= (nx+x^2n(n-1))-2nx\cdot nx+n^2x^2\cdot 1\\
                                           &= nx-x^2n\\
                                           &= nx(1-x)
    \end{align*}
    $f$ is continuous on $[0,1]$ so there exists $M$ such that $|f(x)|\leq M$ for all $x\in[0,1]$.
    Also, $f$ is uniformly continuous so for all $\epsilon>0$, there exists $\delta>0$ such that $|x-y|<\delta$ implies that $|f(x)-f(y)|<\epsilon$.
    We want to prove that for every $\epsilon>0$ there exists $N$ such that $|p_n(x)-f(x)|<\epsilon$ for all $x\in[0,1]$ and $n\geq N$.
    Fix $\epsilon$.
    Take $N$ such that $\frac{2M}{\delta^2N}<\epsilon$.
    Let $n\geq N$ and $x\in[0,1]$.
    Then
    \begin{align*}
        |p_n(x)-f(x)|&=\left\lvert\sum\limits_{k=0}^n f\left(\frac{k}{n}\right)r_k(x)-f(x)\sum\limits_{k=0}^nr_k(x)\right\rvert\\
                     &\leq\left\lvert\sum\limits_{k=0}^n \left(f\left(\frac{k}{n}\right)-f(x)r_k(x)\right)\right\rvert|\\
                     &\leq \sum_{k\in A}\left\lvert f\left(\frac{k}{n}\right)-f(x)\right\rvert||r_k(x)|+\sum_{k\in B}\left\lvert f\left(\frac{k}{n}\right)-f(x)\right\rvert||r_k(x)|\\
    \end{align*}
    Where we break the sum into (i) $|k-nx|<n\delta$ denoted $A$ and (ii) otherwise, denoted $B$
    For $A$, $|k/n-x|<\delta$ implies $f(k/n)-f(x)|<\epsilon$.
    Thus
    \[\sum\limits_{k\in A}\left\lvert f\left(\frac{k}{n}\right)-f(x)\right\rvert||r_k(x)|\leq\sum\limits_{k\in A}\epsilon r_k(x)<\epsilon\]

    For $B$, we have $|k-nx|\geq\delta n$ so that
    \begin{align*}
        \sum\limits_{k\in B}\left\lvert f(x)-f\left(\frac{k}{n}\right)\right\rvert r_k(x) &\leq \sum\limits_{k\in B}\left(|f(x)|+\left\lvert f\left(\frac{k}{n}\right)\right\rvert\right)r_k(x)|\\
                                                                             &\leq 2M\sum\limits_{k\in B}r_k(x)\\
                                                                             &= 2m\sum\limits_{k\in B}\frac{(k-nx)^2}{(k-nx)^2}r_k(x)\\
                                                                             &\leq\frac{2M}{(n\delta)^2}\sum\limits_{k=0}^n(k-nx)^2r_k(x)\\
                                                                             &\leq\frac{2M}{(n\delta)^2}nx(1-x)\\
                                                                             &\leq\frac{2M}{n\delta^2}<\epsilon
    \end{align*}
    by choice of $N$.
\end{proof}
\subsection{The Stone-Weierstrass Theorem}
Recall that an algebra $\mathcal{A}$ is a vector space $V$ over $\F$ equipped with a bilinear form $\cdot:V\times V\to\F$.
The space $C(X)$ is an algebra with pointwise multiplication.
\begin{definition}
    An algebra $\mathcal{A}$ of functions separates points if, for any $x\neq y\in X$, then there exists $f\in\mathcal{A}$ such that $f(x)\neq f(y)$.
\end{definition}
\begin{theorem}[Stone-Weierstrass]
    Let $X$ be compact.
    Let $\mathcal{A}\subseteq C(x)$ be an algebra that separates points and contains the constants.
    Then $\mathcal{A}$ is dense in $C(X)$
\end{theorem}
This will take a bit of time!
\begin{lemma}
    If $\mathcal{A}$ is an algebra that separates points, then so is $\overline{\mathcal{A}}$.
\end{lemma}
\begin{proof}
    If $f,g\in\overline{\mathcal{A}}$, then write $f=\lim f_n$, $g=\lim g_n$, so that for some constant $c$
    \begin{align*}
        cf+g &= c\lim f_n+g\lim g_n\\
             &= \lim (cf_n+g_n)
    \end{align*}
    where the limit exists by the arithmetic properties of limits, and since $\mathcal{A}$ is an algebra, $cf_n+g_n\in\mathcal{A}$ and by closure of $\overline{\mathcal{A}}$, the limit is is in $\overline{\mathcal{A}}$.
    In the exact same way,
    \begin{align*}
        fg &= (\lim f_n)(\lim g_n)\\
           &= \lim f_ng_n
    \end{align*}
    because, when the limits all exist, a product of limits is a limit of products.
    But then in the same way as above, $f_ng_n\in\mathcal{A}$ so by closedness $fg\in\overline{\mathcal{A}}$ as well.
    Since $\overline{\mathcal{A}}\supseteq\mathcal{A}$, it certainly separates points as well.
\end{proof}
\begin{lemma}
    Suppose $B\subseteq\mathcal{A}$ is an algebra that separates points.
    Then if $f,g\in B$, then $\max\{f,g\}$ and $\min\{f,g\}$ are in $B$.
\end{lemma}
\begin{proof}
    Since $\max\{f,g\}=(f+g+|f-g|)/2$ and $\min\{f,g\}=(f+g-|f-g|)/2$, it suffices to show that $f\in B\Rightarrow |f|\in\overline{B}$>
    Note that there exists $(p_n(t))\to\sqrt{t}$ uniformly on $[0,1]$ with constant term $0$.
    Let $c=\norm{f}$ and note that $F_n:=\frac{f^2}{c^2}\in\mathcal{A}$.
    We claim that $F_n\to |f|/c$.
    Let $\epsilon>0$ be arbitrary and let $N$ be such that $|p_n(t)-\sqrt{t}|<\epsilon/2$ for all $n\geq N$ and $t\in[0,1]$.
    Then
    \begin{align*}
        \left\lvert F_n(x)-\frac{|f|}{c}(x)\right\rvert &= \left\lvert p_n\circ (f^2/c^2)(x)-\sqrt{\frac{f^2}{c^2}}(x)\right\rvert\\
                                                        &<\epsilon
    \end{align*}
    since $f^2/c^2(x)\in[0,1]$.
    Thus $|f|/c\in\overline{\mathcal{A}}$ and since $\overline{\mathcal{A}}$ is an algebra, $|f|\in\overline{\mathcal{A}}$.
\end{proof}
\begin{lemma}
    Suppose $x\neq y$ are in $X$ and $a,b\in\R$.
    Then there exists $f\in\mathcal{A}$ so that $f(x)=a$ and $f(y)=b$.
\end{lemma}
\begin{proof}
    Since $\mathcal{A}$ separates points, let $g$ be such that $g(x)\neq g(y)$.
    Then define
    \[f(t)=a+(b-a)\left(\frac{g(t)-g(x)}{g(y)-g(x)}\right)\]
    which has the desired property.
\end{proof}
\begin{lemma}
    If $f\in C(X)$, $x_0\in X$ and $\epsilon>0$, then there exists $g\in\overline{\mathcal{A}}$ such that $g(x_0)=f(x_0)$ and $g(x)\leq f(x)+\epsilon$ for all $x\in X$
\end{lemma}
\begin{proof}
    For each $y\in X$, there exists $h_y\in \mathcal{A}$ so that $h_y(x_0)=f(x_0)$ and $h_y(y)=f(y)$.
    If $y=x_0$, take $h_y=f$, and otherwise use the previous lemma.
    Now, for any $y\in X$, $|h_y-f|=0$ so there exists a $\delta_y$ such that on $B(y,\delta_y)$, $h_y-f<\epsilon$.
    Thus the balls $B(y,\delta_y)$ form an open cover for $X$, so we have a finite subcover $\{B(y_1,\delta_{y_1}),\ldots,B(y_n,\delta_{y_n})\}$.
    Then fix $g=\min\{g_{y_i}\}$ so for any $x\in X$, $x\in B(y_j,\delta_{y_j})$ and $g(x)\leq g_{y_j}(x)<f+\epsilon$, as desired.
\end{proof}
We can finally prove the main theorem!
\begin{proof}
    First, we show for all $f\in C(X)$, $\epsilon>0$, there exists $g\in\overline{\mathcal{A}}$ such that $\norm{g-f}<\epsilon$.
    Once we have this, then we get $h\in\mathcal{A}$ such that $\norm{g-h}<\epsilon$ and then $\norm{f-h}\leq\norm{f-g}+\norm{g-h}<2\epsilon$.
    This proves that $\mathcal{A}$ is dense.

    Now for all $x\in X$, there exists $g_x\in\overline{A}$ such that $g_x(x)=f(x)$ and $g(z)\leq f(z)+\epsilon$ for all $z\in X$.
    The function $f-g_x$ is continuous and $0$ at $x$.
    Get $\delta_x>0$ such that for all $y\in B(x,\delta_x)$, we have $|f(y)-g_x(y)|<\epsilon$.
    Look at $B(x,\delta_x)$ as $x\in X$.
    These cover $X$ so we have a finite subcover $B(x_1,\delta_{x_1}),\ldots,B(x_n,\delta_{x_n})$.
    Put $g=\max(g_{x_1},\ldots,g_{x_n})\in\overline{\mathcal{A}}$ by Lemma 2.
    Let $y\in X$.
    Then there exists $i$ such that $y\in B(x_i,\delta_{x_i})$, s $|f(y)-g_{x_i}(y)|<\epsilon$ and $g_{x_i}(y)>f(y)-\epsilon$.
    Thus $f(y)-\epsilon<g_{x_i}(y)\leq g(y)=g_{x_j}(y)\leq f(y)+\epsilon$ for some $j$ by choice of $g_x$.
    But then $|g(y)-f(y)|\leq\epsilon$ for all $y\in X$, so $\norm{g-f}\leq\epsilon$.
\end{proof}
\begin{definition}
    A function $f:X\to\C$ is continuous at $x\in X$ if whenever $x_n\to x$, then $f(x_n)\to f(x)$.
    We say $z_n\to z$ if $|z_n-z|\to 0$.

    We define $\Re f:X\to\R$, $\Im f:X\to\R$ and $f=\Re f+i\Im f$.
    Then $f$ is continuous if and only if $\Re f$ and $\Im f$ are continuous.

    Given $f$, $\overline{f}(x)=\overline{f(x)}=\Re f(x)-i\Im f(x)$.
\end{definition}
\subsection{An Alternative Proof of Stone-Weierstrass}
\begin{theorem}
    Let $X$ be a compact metric space.
    Suppose $\mathcal{A}$ is a subalgebra of $C(X)$ that separates points.
    If $\overline{\mathcal{A}}\neq C(X)$, prove there is a point $x_0\in X$ such that $\overline{\mathcal{A}}=\{f\in C(X):f(x_0)=0\}$.
\end{theorem}

\begin{lemma}
    If for every $x,y\in X$ there exists $g\in\overline{\mathcal{A}}$ such that $g(x)=f(x)$ and $g(y)=f(y)$, then $f\in\overline{\mathcal{A}}$.
\end{lemma}
\begin{proof}
    Recall the lemma from class that
    \begin{align*}
        f,g\in\mathcal{A}\Rightarrow\min\{f,g\},\max\{f,g\}\in\overline{\mathcal{A}}\tag{$*$}
    \end{align*}
    Since $\overline{\mathcal{A}}$ is an algebra, this holds for $\overline{\mathcal{A}}$ as well.
    Let $\epsilon>0$ be arbitrary.
    Fix $x,y\in X$ and let $g_{xy}$ be such that $g_{xy}(x)=f(x)$ and $g_{xy}(y)=f(y)$.
    Then define
    \[U_{xy}=\{u\in X:f(u)<g_{xy}(u)+\epsilon\}\]
    \[V_{xy}=\{u\in X:f(u)>g_{xy}(u)-\epsilon\}\]
    Note that $x\in U_{xy}$, and that $U,V$ are open by continuity of $f$ and $g$.
    To see the latter statement, let $u\in U_{xy}$.
    Let $\gamma=(g_{xy}(u)+\epsilon)-f(u)>0$ since the inequality is strict.
    By continuity of $f,g_{xy}$ let $\delta>0$ be such that for all $v\in B(u,\delta)$, $|f(u)-f(v)|<\gamma/2$ and $|g_{xy}(u)-g_{xy}(v)|<\gamma/2$.
    But then we must have
    \begin{align*}
        f(v)-g_{xy}(v) &< f(u)+\gamma/2-g_{xy}(u)+\gamma/2\\
                       &= f(u)-g_{xy}(u)+\gamma\\
                       &= f(u)-g_{xy}(u)+g_{xy}(u)-f(u)+\epsilon\\
                       &= \epsilon
    \end{align*}
    so that $v\in U_{xy}$.
    The identical argument works for $V_{xy}$.

    Furthermore, $x\in U_{xy}$ so $\{U_{xy}:x\in X\}$ is an open cover for $X$.
    By compactness of $X$ it has a subcover $\{U_{x_1y},\ldots,U_{x_ny}\}$, and write $g_y=\max\{g_{x_1y},\ldots,g_{x_ny}\}$.
    Now for a given $y$, define $V_y=\bigcap\limits_{i=1}^n V_{x_iy}$.
    $V_y$ are open since it is a finite intersection of open sets, so since $y\in V_y$, $\{V_y:y\in X\}$ is an open cover for $X$.
    Thus we also have a finite subcover $\{V_{y_1},\ldots,V_{y_k}\}$.
    Finally, define $g=\min\{g_{y_1},\ldots,g_{y_k}\}$.
    By ($*$), $g\in\mathcal{\overline{A}}$.

    I claim that $d(g,f)<\epsilon$.
    We have $f<g_y+\epsilon$ on $X$ for all $y$ since on each $U_{x_iy}$, $f<g_{x_iy}+\epsilon\leq g_y+\epsilon$.
    Since this holds for all $y$, it certainly holds for $g$ as well, so $f<g+\epsilon$.
    Similarly, on each $V_{x_iy}$, we have $f>g_{x_iy}-\epsilon$, so $f>g_{x_iy}-\epsilon$ for $i=1,\ldots,n$ on $V_y$.
    Thus $f>g_{y_i}-\epsilon$ on each $V_{y_i}$, and since $g\leq g_{y_i}$ for all $y$ we have $f>g_{y_i}-\epsilon\geq g-\epsilon$.
    Combining these expressions, we have $f-g<\epsilon$ and $g-f<\epsilon$ so $d(g,f)<\epsilon$.

    Furthermore, since $\epsilon$ was arbitrary, for any $f$, we can find a sequence $g_n\to f$ by choosing $d(g_n,f)<1/n$.
    Then since $\overline{\mathcal{A}}$ is closed, we must have $f\in\overline{\mathcal{A}}$ as desired.
\end{proof}
We are now in position to prove the main theorem!
\begin{proof}
    Let $x,y\in X$ with $x\neq y$, and consider the algebra $A_{xy}=\{(f(x),f(y)):f\in \overline{\mathcal{A}}\}\subseteq\R^2$ with component-wise addition, scalar multiplication, and multiplication.
    Thse are algebras since $\mathcal{A}$ is an algebra, and we must have
    \begin{align*}
        A_{xy}\in\{\{(0,x):x\in\R\},\{(x,0):x\in\R\},{\R}^2\}
    \end{align*}
    To see this, suppose $(a,b)\in A_{xy}$.
    If $a\neq b$ and $a,b\neq 0$, then $(a,b)$ and $(a^2,b^2)$ provide a basis for $\R^2$, so $S=\R^2$ by closure under multiplication of $\mathcal{A}$.
    If $a=b$, then $\mathcal{A}$ does not separate points, and if $a=0,b\neq 0$ or the reverse, we have the first two cases above.

    Now suppose that for all $x,y$, $A_{xy}=\R^2$.
    Then for any $f\in C(X)$, by Lemma 1, we must have $f\in\overline{\mathcal{A}}$, so $\overline{\mathcal{A}}=C(X)$, the contents of the Stone-Weierstrass Theorem.

    Otherwise there exists some $x_0,y_0$ so that $A_{x_0y_0}=\{(x,0):x\in\R\}$ without loss of generality.
    Note that such a $y_0$ must satisfy $f(y_0)=0$ for all $f\in\overline{\mathcal{A}}$.
    Furthermore, $y_0$ is unique.
    Suppose not, so we have $y_0,y_1$ such that $f(y_0)=0$ and $f(y_1)=0$ for all $f\in\mathcal{\overline{A}}$ and $\mathcal{\overline{A}}$ would not separate points.
    Finally, for any $x$, we have $A_{xy_0}=\{(x,0):x\in\R\}$ since $f(y_0)=0$ for all $f\in\overline{\mathcal{A}}$.
    We can summarize this by saying
    \begin{align*}
        A_{xy}=\begin{cases}
            {\R}^2 &: x,y\neq y_0\\
            \{(u,0):u\in\R\} &: y=y_0\\
            \{(0,u):u\in\R\} &: x=y_0
        \end{cases}
    \end{align*}
    We certainly have $\overline{\mathcal{A}}\subseteq\{f\in C(X):f(y_0)=0\}$ by the existence of $y_0$ above.
    Now to show the reverse inclusion, suppose $f(y_0)=0$.
    Suppose without loss of generality that $x\neq y_0$.
    Since $A_{xy_0}=\{(x,0):x\in\R\}$, we have $g\in\overline{\mathcal{A}}$ so that $g(y_0)=0=f(y_0)$ and $g(x_0)=f(x_0)$.
    Similarly, for $A_{xy}$ with $y\neq y_0$, $A_{xy}=\R^2$ so there exists $g\in\overline{\mathcal{A}}$ so that $g(x)=f(x)$ and $g(y)=f(y)$.
    Thus by Lemma 1, $f\in\overline{\mathcal{A}}$.
    And we're done!
\end{proof}
\subsection{Applications of Stone-Weierstrass}
\begin{theorem}[Stone-Weierstrass for $\C-$valued functions]
    Let $X$ be compact and suppose $\mathcal{A}$ is an algebra in $C(X,\mathcal{C})=\{f:X\to\C,\text{continuous}\}$ (the scalars are from $\C$).
    that separates points, contains the constants, and is closed under conjugation.
    Then $\mathcal{A}$ is dense in $C(X,\C)$.
\end{theorem}
\begin{proof}
    Let $\mathcal{A}_{\R}$ denote the set of real-valued functions in $\mathcal{A}$.
    If $f\in\mathcal{A}$, then $(f\pm\overline{f})/2\in\mathcal{A}$ so $\Re f$ and $\Im f$ are in $\mathcal{A}$.
    Notice $\mathcal{A}_{\R}$ is an algebra over $\R$ in $C(x)$ and it contains the constants.
    Furthermore, since $\mathcal{A}$ separates points, either the real part or imaginary part separates points, so $\mathcal{A}_{\R}$ separates points as well.
    Thus the result holds by Stone-Weierstrass over $\R$.
\end{proof}
\begin{example}
    Consider the set $X=\{z\in\C:|z|=1\}$, and
    \[\mathcal{A}=\left\{\sum\limits_{n=-N}^Na_nz^n:a_n\in\C\right\}\]
    The collection $\mathcal{A}$ is an algebra, separates points, contains constants, and is closed under conjugation.
\end{example}
\begin{corollary}
    Suppose $X$ is a compact metric space.
    Then $C(X)$ is separable.
\end{corollary}
\begin{proof}
    Since $X$ is compact, it is separable, so let $S\subseteq X$ be countable and dense.
    Then consider the collection of functions $\mathcal{S}=\{f_u(x)=d(x,u):u\in S\}$.
    I claim that $\mathcal{S}$ separates points.
    Thus suppose $x,y\in X$ and fix $d(x,y)=\epsilon$ since $x\neq y$.
    Then since $S$ is dense, choose $x_0$ so that $d(x_0,x)<\epsilon/2$.
    But then $f_{x_0}(x)<\epsilon/2$ and $f_{x_0}(y)>\epsilon/2$ for if not, then $d(x,y)\leq d(x,x_0)+d(x_0,y)=f_{x_0}(x)+f_{x_0}(y)<\epsilon$, a contradiction.
    Thus $f_{x_0}(x)\neq f_{x_0}(y)$ so $\mathcal{S}$ separates points.

    Note that the polynomials of functions in $\mathcal{S}$ (in other words, the smallest algebra containing $S$) is dense in $C(X)$ by Stone-Weierstrass.
    Thus let $\mathcal{H}$ denote the polynomials over $\mathcal{S}$ with rational coefficients.
    It suffices to show that any polynomial in $\mathcal{S}$ can be approximated by functions in $\mathcal{H}$.
    We can write an arbitrary polynomial as $p=\sum c_i \prod f_j^{c_j}$ for $f_i\in\mathcal{S}$ where the sum and all the products are finite.
    Note that each $f_j$ is uniformly continuous on $X$, so $\prod f_j^{c_j}$ is also uniformly continuous.
    Thus let $\left\lvert\prod_{j\in A_i} f_j^{c_j}\right\rvert<M$ for all terms in $p$.
    Furthermore, suppose the summation consists of $N$ terms.
    Let $\epsilon>0$ be arbitrary and let $b_i\in\Q$ be such that $|b_i-c_i|<\epsilon/(NM)$.
    But then define $q=\sum b_i\prod f_j^{c_j}$ so that
    \begin{align*}
        |p(x)-q(x)|&\leq\sum |b_i-a_i|\left\lvert\prod f_j^{c_j}\right\rvert\\
                   &\leq M\sum|b_i-a_i|\\
                   &\leq M\sum \frac{\epsilon}{Mn}\\
                   &<\epsilon
    \end{align*}
    as desired.
\end{proof}
\begin{corollary}
    If $f\in C[0,1]$ and $\int_0^1 f(x)x^n dx = 0$ for all $n=0,1,2,\ldots$, then $f(x)=0$ for all $x\in[0,1]$.
\end{corollary}
\begin{proof}
    $\int_0^1 f(x)x^n=0$ for all $n$ implies $\int_0^1 f(x)p(x)dx=0$ for any polynomial $p$.
    By Stone-Weierstrass, let $(p_n)\to f$ uniformly.
    Then
    \[\int_0^1f(x)p_n(x)dx\to\int_0^1 (f(x))^2dx\]
    Assuming this, it follows that $\int_0^1 f^2=0$ so $f=0$.
    For the claim, consider
    \[|\int_0^1 f(x)(p_n(x)-f(x))dx|\leq\int_0^1|f||p_n-f|dx\leq\norm{f}\int_0^1|p_n-f|\leq\norm{f}\norm{p_n-f}\int_0^1 dx\]
    as desired.
\end{proof}
\section{Two Applications of the Baire Category Theorem to Functions}
\subsection{Sets of Discontinuity}
\begin{theorem}
    No function on $[0,1]$ can be continuous on precisely the rational points.
\end{theorem}
\begin{proof}
    We first see that the set of points at which the function $f:X\to\R$ is discontinuous is a countable union of closed sets.
    Define the set $S_k=\{x\in X:\forall\delta>0,\exists y,z\in B(x,\delta)\text{ s.t. }|f(y)-f(x)|\geq 1/k\}$.
    I will show that (1) the $S_k$ are closed and (2) $D_f=\bigcup_{k=1}^\infty S_k$ where $D_f$ denotes the set of points of discontinuity of $f$.

    To see that $S_k$ is closed, let $x\in S_k^c$.
    Thus we have for some $\delta>0$ for any  $y,z\in B(x,\delta)$ we have $|f(y)-f(x)|<1/k$.
    But then for any $y\in B(x,\delta)$, choose $\delta'$ so $B(y,\delta')\subseteq B(x,\delta)$ and for any $y,z\in B(y,\delta')$ we have $|f(y)-f(x)|<1/k$.
    Therefore $B(x,\delta)\subseteq S_k^c$ so $S_k^c$ is open.

    Furthermore, we certainly have $S_k\subseteq D_f$ by definition, and if $x\in D_f$, then there exists an $\epsilon>0$ so that for all $\delta>0$, there exist $y,z\in B(x,\delta)$ so that $|f(y)-f(x)|\geq\epsilon$.
    Then let $k$ be such that $1/k<\epsilon$ so $x\in S_k$.
    Thus $D_f$ is a countable union of closed sets.

    It now suffices to show that $\R\setminus\Q$ cannot be written as a countable union of closed sets.
    Suppose for contradiction that $\R\setminus\Q=\bigcup\limits_{i=1}^\infty V_i$, and write
    \[\R=\bigcup\limits_{i=1}^\infty V_i\cup\bigcup\limits_{i=1}^\infty\{r_i\}\]
    where $\{r_i\}$ is an enumeration of $\Q$.
    However by BCT, $\R$ is second category, so some $V_k$ must not be nowhere dense.
    But then since the $V_i$ are closed, $V_k=\overline{V_k}$ has non-empty interior, a contradiction since $\R\setminus\Q$ has empty interior.

    However, by (8a), $D_f$ is a countable union of closed sets, a contradiction so $D_f\neq\R\setminus\Q$.
\end{proof}
Exercise: generalize this statement to continuous functions on $C(X)$!
\subsection{Sets of Differentiability}
\begin{theorem}
    The set of functions on $C[0,1]$ which have a derivative at some point on $(0,1)$ is first category.
\end{theorem}
\begin{corollary}
    The set of nowhere differentiable functions is second category.
\end{corollary}
\begin{proof}
    The union of two first category sets is first category, and $C[0,1]$ is second category.
\end{proof}
\begin{proof}
    Let
    \[E_j = \left\{f\in C[0,1]:\exists x\in[0,1-1/j]\text{ s.t. }\forall h\in(0,1/j],\left\lvert\frac{f(x+h)-f(x)}{h}\right\rvert\leq j\right\}\]
    and define
    \[E=\{f\in C[0,1]\text{ $f$ has a derivative somewhere}\}\]
    We now prove the theorem in three parts
    \begin{enumerate}
        \item $E\subseteq\bigcup_{j=1}^\infty E_j$
        \item $E_j$ has empty interior.
        \item $E_j$ is closed.
    \end{enumerate}
    Once this is done, then notice
    \[E=\bigcup\limits_{j=1}^\infty (E\cap E_j)\]
    so $\overline{(E\cap E_j)}\subseteq\overline{E_j}=E_j$.
    Then $\overline{E\cap E_j}$ has empty interior since $E_j$ has empty interior, so $E\cap E_j$ is nowhere dense and $E$ is first category.
    Let's now proceed to prove those three statements!

    (1) Say $f\in E$ and $f$ is differentiable at $x\in(0,1)$.
    Thus there exist some $j_1$ such that for all $j\geq j_1$, $x\in[0,1-1/j]$.
    Since $\lim_{h\to 0}\frac{f(x+h)-f(x)}{h}$ exists, we have $C$ such that for all $h\leq j_2$, we have
    \[\left\lvert\frac{f(x+h)-f(x)}{h}\right\rvert\leq C\leq j_3\]
    Let $j=\max\{j_1,j_2,j_3\}$, then $f\in E_j$.

    (2) To show that $E_j$ has empty interior, we will show that if $f\in E_j$ and $\epsilon>0$, then there exists $g\in B(f,\epsilon)$ such that $g\notin E_j$.
    First, by Stone-Weierstrass get $p$ such that $d(p,f)<\epsilon/2$.
    We then have some $M$ such that $|p'(x)|\leq M$ for all $x\in(0,1]$.
    Let $Q$ be a piecewise linear function of slope $\pm(M+j+1)$ with $0\leq Q(x)\leq\epsilon/2$.
    Set $g=p+Q$, so that $d(f,g)<\epsilon$.
    Now let $x\in(0,1)$ and we have
    \begin{align*}
        \left\lvert\frac{g(x+h)-g(x)}{h}\right\rvert &= \left\lvert\frac{p(x+h)-p(x)+q(x+h)-q(x)}{h}\right\rvert\\
                                                     &\geq -\left\lvert\frac{p(x+h)-p(x)}{h}\right\rvert+\left\lvert\frac{Q(x+h)-Q(x)}{h}\right\rvert\\
                                                     &\geq -\left(M+\frac{1}{2}\right)+M+j+1\\
                                                     &\geq j+\frac{1}{2}
    \end{align*}
    for sufficiently small $h$.

    (3) We finally see that $E_j$ is closed.
    Suppose $f_n\in E_j$ and $f_n\to f$ uniformly.
    We see that $f\in E_j$.
    By definition of $E_j$, for all $n$ there exists $x_n\in[0,1-1/j]$ such that for any $h\in(0,1/j)$
    \[\left\lvert\frac{f_n(x_n+h)-f_n(x_n)}{h}\right\rvert\leq j\]
    By Bolzano-Weierstrass, there exists a subsequence $x_{n_k}\to x_0\in[0,1-1/j]$.
    We will ceck that $x_0$ ``works'' for $f$, so $f\in E_j$.
    Let $\epsilon>0$ and fix $h\in[0,1/j]$.
    We have $M$ such that $d(f_m,f)<\epsilon h/4$ for all $m\geq M$.
    Furthermore, $f$ is uniformly continuous so there exists $\delta>0$ such that $|f(x)-f(y)|\leq\epsilon h/4$ whenever $|x-y|<\delta$.
    We also have $M_2$ such that $|x_m-x_0|<\delta$ if $m\geq M_2$, and fix $M=\max\{M_1,M_2\}$.
    Finally
    \begin{align*}
        \frac{|f(x_0+h)-f(x_0)|}{h} &\leq \frac{|f(x_0+h)-f(x_m+h)|}{h}+\frac{|f(x_m+h)-f_M(x_m+h)|}{h}+\frac{|f_M(x_m+h)-f_M(x_M)|}{h}\\
                                    &\qquad+\frac{|f_M(x_M)-f(x_M)|}{h}+\frac{|f(x_M)-f(x_0)|}{h}\\
                                    &\leq \frac{\frac{\epsilon h}{4}}{h}+\frac{d(f,f_M)}{h}+j+\frac{d(f_m,f)}{h}+\frac{\frac{\epsilon h}{4}}{h}\\
                                    &\leq \epsilon+j
    \end{align*}
    Since $\epsilon>0$ was arbitrary, we must have
    \[\frac{|f(x_0+h)-f(x_0)|}{h}\leq j\]
    for all $h\in(0,1/j)$.
\end{proof}
\begin{example}
    Solve the D.E.
    \begin{align*}
        y' &= 1+x=y\text{ for } x\in\left[-\frac{1}{2},\frac{1}{2}\right]\\
        y(0) &= 1
    \end{align*}

    $y=f(x)$ where $f'(x)=1+x-f(x)$ and $f(0)=1$.
    Convert to integral equation:
    \begin{align*}
        f(x) &= \int_0^x f'(t)\d{t}+f(0)\\
             &= \int_0^x(1+t-f(x))\d{t}+1\\
             &= x+\frac{x^2}{2}+1-\int_0^x f(t)\d{t}
    \end{align*}
    Define $T$ on $C[-1/2,1/2]$ by
    \[Tf(x)=1+x+\frac{x^2}{2}-\int_0^x f(t)\d{t}\]
    so $Tf\in C[-1/2,1/2]$.
    Thus $T:C[-1/2,1/2]\to[-1/2,1/2]$.
    Notice $(Tf)'(x)=1+x-f(x)$.
    If we can find a fixed point of $T$, call it $f$, then $f(x)=Tf(x)\Rightarrow f(0)=Tf(0)=1$ and $f'=(Tf)'=1+x-f(x)$ so $f$ solves the D.E.

    Conversely, if $f$ solves the D.E., then $f'(x)=1+x-f(x)=(Tf)'(x)$.
    But $f(0)=1=Tf(0)$ so $f=Tf$.
    Thus any solution to the DE is a fixed point of $T$.

    Check that $T$ is a contraction.
    Want $r<1$ such that $d(Tf,Tg)\leq rd(f,g)$ for all $f,g\in C[-1/2,1/2]$.
    We have
    \begin{align*}
        |Tf(x)-Tg(x)| &= \left\lvert-\int_0^x f+\int_0^x g\right\rvert\\
                      &= \left\lvert\int_0^x(g-f)\right\rvert\\
                      &\leq \norm{g-f}\left\lvert\int_0^x 1\right\rvert\\
                      &= \norm{g-f}|x|\\
                      &\leq\frac{1}{2}\norm{g-f}
    \end{align*}
    so $T$ is a contraction with $r=\frac{1}{2}$.
    By BCMP, $T$ has a unique fixed point $f$, so our D.E. has a unique solution.
    In fact $f=\lim T^{(n)} f_0$.
    For example, take $f_0=1$.
    Then $T^{(n)}f_0=\sum\limits_{k=0,k\neq 1}^n\frac{(-1)^k}{k!}x^k=e^{-x}+x=f(x)$.
\end{example}
More generally, we have
\begin{theorem}[Global Picard]
    Suppose $\Phi:[a,b]\times\R\to\R$ is continuous and ``Lipschitz in $y-$variable''.
    Then the D.E. $F'(x)=\phi(x,F(x))$, $F(a)=c$ has a unique solution.
\end{theorem}
Lipschitz in $y$ means there exists $L$ such that $\left\lvert\Phi(x,y)-\Phi(x,z)\right\rvert\leq L|y-z|$ for all $x\in[a,b]$, $y,z\in\R$.
\end{document}


