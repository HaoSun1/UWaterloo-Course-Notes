% header -----------------------------------------------------------------------
% Template created by texnew (author: Alex Rutar); info can be found at 'https://github.com/alexrutar/texnew'.
% version (1.13)


% doctype ----------------------------------------------------------------------
\documentclass[11pt, a4paper]{memoir}
\usepackage[utf8]{inputenc}
\usepackage[left=3cm,right=3cm,top=3cm,bottom=4cm]{geometry}
\usepackage[protrusion=true,expansion=true]{microtype}


% packages ---------------------------------------------------------------------
\usepackage{amsmath,amssymb,amsfonts}
\usepackage{graphicx}
\usepackage{etoolbox}

% Set enimitem
\usepackage{enumitem}
\SetEnumitemKey{nl}{nolistsep}
\SetEnumitemKey{r}{label=(\roman*)}

% Set tikz
\usepackage{tikz, pgfplots}
\pgfplotsset{compat=1.15}
\usetikzlibrary{intersections,positioning,cd}
\usetikzlibrary{arrows,arrows.meta}
\tikzcdset{arrow style=tikz,diagrams={>=stealth}}

% Set hyperref
\usepackage[hidelinks]{hyperref}
\usepackage{xcolor}
\newcommand\myshade{85}
\colorlet{mylinkcolor}{violet}
\colorlet{mycitecolor}{orange!50!yellow}
\colorlet{myurlcolor}{green!50!blue}

\hypersetup{
  linkcolor  = mylinkcolor!\myshade!black,
  citecolor  = mycitecolor!\myshade!black,
  urlcolor   = myurlcolor!\myshade!black,
  colorlinks = true,
}


% macros -----------------------------------------------------------------------
\DeclareMathOperator{\N}{{\mathbb{N}}}
\DeclareMathOperator{\Q}{{\mathbb{Q}}}
\DeclareMathOperator{\Z}{{\mathbb{Z}}}
\DeclareMathOperator{\R}{{\mathbb{R}}}
\DeclareMathOperator{\C}{{\mathbb{C}}}
\DeclareMathOperator{\F}{{\mathbb{F}}}

% Boldface includes math
\newcommand{\mbf}[1]{{\boldmath\bfseries #1}}

% proof implications
\newcommand{\imp}[2]{($#1\Rightarrow#2$)\hspace{0.2cm}}
\newcommand{\impe}[2]{($#1\Leftrightarrow#2$)\hspace{0.2cm}}
\newcommand{\impr}{{($\Rightarrow$)\hspace{0.2cm}}}
\newcommand{\impl}{{($\Leftarrow$)\hspace{0.2cm}}}

% align macros
\newcommand{\agspace}{\ensuremath{\phantom{--}}}
\newcommand{\agvdots}{\ensuremath{\hspace{0.16cm}\vdots}}

% convenient brackets
\newcommand{\brac}[1]{\ensuremath{\left\langle #1 \right\rangle}}
\newcommand{\norm}[1]{\ensuremath{\left\lVert#1\right\rVert}}
\newcommand{\abs}[1]{\ensuremath{\left\lvert#1\right\rvert}}

% arrows
\newcommand{\lto}[0]{\ensuremath{\longrightarrow}}
\newcommand{\fto}[1]{\ensuremath{\xrightarrow{\scriptstyle{#1}}}}
\newcommand{\hto}[0]{\ensuremath{\hookrightarrow}}
\newcommand{\mapsfrom}[0]{\mathrel{\reflectbox{\ensuremath{\mapsto}}}}
 
% Divides, Not Divides
\renewcommand{\div}{\bigm|}
\newcommand{\ndiv}{%
    \mathrel{\mkern.5mu % small adjustment
        % superimpose \nmid to \big|
        \ooalign{\hidewidth$\big|$\hidewidth\cr$/$\cr}%
    }%
}

% Convenient overline
\newcommand{\ol}[1]{\ensuremath{\overline{#1}}}

% Big \cdot
\makeatletter
\newcommand*\bigcdot{\mathpalette\bigcdot@{.5}}
\newcommand*\bigcdot@[2]{\mathbin{\vcenter{\hbox{\scalebox{#2}{$\m@th#1\bullet$}}}}}
\makeatother

% Big and small Disjoint union
\makeatletter
\providecommand*{\cupdot}{%
  \mathbin{%
    \mathpalette\@cupdot{}%
  }%
}
\newcommand*{\@cupdot}[2]{%
  \ooalign{%
    $\m@th#1\cup$\cr
    \sbox0{$#1\cup$}%
    \dimen@=\ht0 %
    \sbox0{$\m@th#1\cdot$}%
    \advance\dimen@ by -\ht0 %
    \dimen@=.5\dimen@
    \hidewidth\raise\dimen@\box0\hidewidth
  }%
}

\providecommand*{\bigcupdot}{%
  \mathop{%
    \vphantom{\bigcup}%
    \mathpalette\@bigcupdot{}%
  }%
}
\newcommand*{\@bigcupdot}[2]{%
  \ooalign{%
    $\m@th#1\bigcup$\cr
    \sbox0{$#1\bigcup$}%
    \dimen@=\ht0 %
    \advance\dimen@ by -\dp0 %
    \sbox0{\scalebox{2}{$\m@th#1\cdot$}}%
    \advance\dimen@ by -\ht0 %
    \dimen@=.5\dimen@
    \hidewidth\raise\dimen@\box0\hidewidth
  }%
}
\makeatother


% macros (theorem) -------------------------------------------------------------
\usepackage[thmmarks,amsmath,hyperref]{ntheorem}
\usepackage[capitalise,nameinlink]{cleveref}

% Numbered Statements
\theoremstyle{change}
\theoremindent\parindent
\theorembodyfont{\itshape}
\theoremheaderfont{\bfseries\boldmath}
\newtheorem{theorem}{Theorem.}[section]
\newtheorem{lemma}[theorem]{Lemma.}
\newtheorem{corollary}[theorem]{Corollary.}
\newtheorem{proposition}[theorem]{Proposition.}

% Claim environment
\theoremstyle{plain}
\theorempreskip{0.2cm}
\theorempostskip{0.2cm}
\theoremheaderfont{\scshape}
\newtheorem{claim}{Claim}
\renewcommand\theclaim{\Roman{claim}}
\AtBeginEnvironment{theorem}{\setcounter{claim}{0}}

% Un-numbered Statements
\theorempreskip{0.1cm}
\theorempostskip{0.1cm}
\theoremindent0.0cm
\theoremstyle{nonumberplain}
\theorembodyfont{\upshape}
\theoremheaderfont{\bfseries\itshape}
\newtheorem{definition}{Definition.}
\theoremheaderfont{\itshape}
\newtheorem{example}{Example.}
\newtheorem{remark}{Remark.}

% Proof / solution environments
\theoremseparator{}
\theoremheaderfont{\hspace*{\parindent}\scshape}
\theoremsymbol{$//$}
\newtheorem{solution}{Sol'n}
\theoremsymbol{$\blacksquare$}
\theorempostskip{0.4cm}
\newtheorem{proof}{Proof}
\theoremsymbol{}
\newtheorem{nmproof}{Proof}

% Format references
\crefformat{equation}{(#2#1#3)}
\Crefformat{theorem}{#2Thm. #1#3}
\Crefformat{lemma}{#2Lem. #1#3}
\Crefformat{proposition}{#2Prop. #1#3}
\Crefformat{corollary}{#2Cor. #1#3}
\crefformat{theorem}{#2Theorem #1#3}
\crefformat{lemma}{#2Lemma #1#3}
\crefformat{proposition}{#2Proposition #1#3}
\crefformat{corollary}{#2Corollary #1#3}


% macros (algebra) -------------------------------------------------------------
\DeclareMathOperator{\Ann}{Ann}
\DeclareMathOperator{\Aut}{Aut}
\DeclareMathOperator{\chr}{char}
\DeclareMathOperator{\coker}{coker}
\DeclareMathOperator{\disc}{disc}
\DeclareMathOperator{\End}{End}
\DeclareMathOperator{\Fix}{Fix}
\DeclareMathOperator{\Frac}{Frac}
\DeclareMathOperator{\Gal}{Gal}
\DeclareMathOperator{\GL}{GL}
\DeclareMathOperator{\Hom}{Hom}
\DeclareMathOperator{\id}{id}
\DeclareMathOperator{\im}{im}
\DeclareMathOperator{\lcm}{lcm}
\DeclareMathOperator{\Nil}{Nil}
\DeclareMathOperator{\rank}{rank}
\DeclareMathOperator{\Res}{Res}
\DeclareMathOperator{\Spec}{Spec}
\DeclareMathOperator{\spn}{span}
\DeclareMathOperator{\Stab}{Stab}
\DeclareMathOperator{\Tor}{Tor}

% Lagrange symbol
\newcommand{\lgs}[2]{\ensuremath{\left(\frac{#1}{#2}\right)}}

% Quotient (larger in display mode)
\newcommand{\quot}[2]{\mathchoice{\left.\raisebox{0.14em}{$#1$}\middle/\raisebox{-0.14em}{$#2$}\right.}
                                 {\left.\raisebox{0.08em}{$#1$}\middle/\raisebox{-0.08em}{$#2$}\right.}
                                 {\left.\raisebox{0.03em}{$#1$}\middle/\raisebox{-0.03em}{$#2$}\right.}
                                 {\left.\raisebox{0em}{$#1$}\middle/\raisebox{0em}{$#2$}\right.}}


% macros (analysis) ------------------------------------------------------------
\DeclareMathOperator{\M}{{\mathcal{M}}}
\DeclareMathOperator{\B}{{\mathcal{B}}}
\DeclareMathOperator{\ps}{{\mathcal{P}}}
\DeclareMathOperator{\pr}{{\mathbb{P}}}
\DeclareMathOperator{\E}{{\mathbb{E}}}
\DeclareMathOperator{\supp}{supp}
\DeclareMathOperator{\sgn}{sgn}

\renewcommand{\Re}{\ensuremath{\operatorname{Re}}}
\renewcommand{\Im}{\ensuremath{\operatorname{Im}}}
\renewcommand{\d}[1]{\ensuremath{\operatorname{d}\!{#1}}}


% file-specific preamble -------------------------------------------------------
\DeclareMathOperator{\Free}{Free}
\DeclareMathOperator{\Ind}{Ind}
\DeclareMathOperator{\Tr}{Tr}


% constants --------------------------------------------------------------------
\newcommand{\subject}{Representation Theory of Finite Groups}
\newcommand{\semester}{Fall 2019}


% formatting -------------------------------------------------------------------
% Fonts
\usepackage{kpfonts}
\usepackage{dsfont}

% Adjust numbering
\numberwithin{equation}{section}
\counterwithin{figure}{section}
\counterwithout{section}{chapter}
\counterwithin*{chapter}{part}

% Footnote
\setfootins{0.5cm}{0.5cm} % footer space above
\renewcommand*{\thefootnote}{\fnsymbol{footnote}} % footnote symbol

% Table of Contents
\renewcommand{\thechapter}{\Roman{chapter}}
\renewcommand*{\cftchaptername}{Chapter } % Place 'Chapter' before roman
\setlength\cftchapternumwidth{4em} % Add space before chapter name
\cftpagenumbersoff{chapter} % Turn off page numbers for chapter
\maxtocdepth{section} % table of contents up to section

% Section / Subsection headers
\setsecnumdepth{section} % numbering up to and including "section"
\newcommand*{\shortcenter}[1]{%
    \sethangfrom{\noindent ##1}%
    \Large\boldmath\scshape\bfseries
    \centering
\parbox{5in}{\centering #1}\par}
\setsecheadstyle{\shortcenter}
\setsubsecheadstyle{\large\scshape\boldmath\bfseries\raggedright}

% Chapter Headers
\chapterstyle{verville}

% Page Headers / Footers
\copypagestyle{myruled}{ruled} % Draw formatting from existing 'ruled' style
\makeoddhead{myruled}{}{}{\scshape\subject}
\makeevenfoot{myruled}{}{\thepage}{}
\makeoddfoot{myruled}{}{\thepage}{}
\pagestyle{myruled}
\setfootins{0.5cm}{0.5cm}
\renewcommand*{\thefootnote}{\fnsymbol{footnote}}

% Titlepage
\title{\subject}
\author{Alex Rutar\thanks{\itshape arutar@uwaterloo.ca}\\ University of Waterloo}
\date{\semester\thanks{Last updated: \today}}

\begin{document}
\pagenumbering{gobble}
\hypersetup{pageanchor=false}
\maketitle
\newpage
\frontmatter
\hypersetup{pageanchor=true}
\tableofcontents*
\newpage
\mainmatter


% main document ----------------------------------------------------------------
\chapter{Introduction}
Let $G$ be a finite group of order $n$, and write $G=\{g_1,\ldots,g_n\}$.
Fix $g\in G$; then $gg_i=gg_j$ if and only if $i=j$.
Thus there exists some $\sigma_g\in S_i$ such that $gg_i=g_{\sigma_g(i)}$ for all $i\in\{1,2,\ldots,n\}$.
In particular, $\phi:G\to S_n$ by $\phi(g)=\sigma_g$ is an embedding (injective group homomorphism).
This observation is usually referred to as Cayley's Theorem.

Now let $V$ be an $n-$dimensional complex vector space.
We then denote $\GL(V)$ as the group of invertible linear operators $T:V\to V$.
Now define $\psi:S_n\to\GL_n(V)$ by $\psi(\sigma)=T_\sigma$ where if $\{b_1,\ldots,b_n\}$ is a basis for $V$ and $T_\sigma(b_i)=b_{\sigma(i)}$.
This is an injective group homomorphism, so $\psi\circ\phi:G\to\GL(V)$ is an embedding of $G$ into $\GL(V)$.
\begin{definition}
    Let $G$ be a finite group, and $V$ a finite dimensional $\C-$vector space.
    A \textbf{representation} of $G$ is a group homomorphism $\rho:G\to\GL(V)$.
    We call $\dim(V)$ the \textbf{degree} of the representation.
\end{definition}
In particular, if $V$ is $n-$dimensional, then $\GL(V)\cong\GL_n(\C)$.
\begin{example}
    \begin{enumerate}[nl]
        \item Consider $\rho:G\to\GL(\C)\cong\C^\times$ given by $\rho(g)=1$ for all $g\in G$.
            This is called the \textit{trivial representation}.
        \item Consider $\rho:S_n\to\C^\times$ given by $\rho(\sigma)=\sgn(\sigma)$, which is called the \textit{sign representation}.
        \item The representation fo $G$ afforded by Cayley's theorem is called the \textit{regular representation} of $G$.
            The next example is a good way to understand the regular rep of $G$.
        \item Consider $G$, $X=\{x_1,\ldots,x_n\}$, and $V=\Free(X)$.
            Suppose $G$ acts on $X$.
            Then $\rho:G\to\GL(V)$ given by $\rho(g)(x_i)=gx_i$.
            In particular, if we take $X=G$, then this is the regular representation of $G$
        \item Consider the $4-$gon, with vertices labelled $a,b,c,d$.
            Take $X=\{a,b,c,d\}$ and the regular representation $\rho:D_4\to\GL(V)$.
            This action has a geometric notion.
        \item Let $C_n$ be a cyclic group of order $n$; let us define some $\rho:C_n\to\GL(V)$.
            Say $\rho(x) = T$ where $t\in\GL(V)$; then this is a representation if and only if $T^n=I$.
    \end{enumerate}
\end{example}
\begin{definition}
    We say that two representations $\rho:G\to\GL(V)$ and $\tau:G\to\GL(W)$ are \textbf{isomorphic} if there exists an isomorphism $T:V\to W$ such that for all $g\in G$,
    \begin{equation*}
        T\circ \rho(g) = \tau(g)\circ T
    \end{equation*}
\end{definition}
Suppose $\rho:G\to\GL(V)$ and $T:V\to W$ is an isomorphism.
Then we can define $\tau:G\to\GL(W)$ by $\tau(G)=T\circ\rho(g)\circ T^{-1}$; this $\rho\cong\tau$.
In other words, the representation is unique up to isomorphism under change of basis.
\begin{example}
    Consider $G=\{g_1,\ldots,g_n\}=\{h_1,\ldots,h_n\}$, and fix $g\in G$.
    Let $gg_i=g_{\alpha(i)}$ and $gh_i=h_{\beta(i)}$ where $\alpha,\beta\in S_n$.
    Fix an $n-$dimensional vector space $V$ with basis $\{b_1,\ldots,b_n\}$.
    Then two regular representations are given by
    \begin{align*}
        \rho_1:G\to\GL(V), & \rho(g)(b_i)=b_{\alpha(i)}\\
        \rho_2:G\to\GL(V), & \rho(g)(b_i)=b_{\beta(i)}
    \end{align*}
    Let $\gamma\in S_n$ be such that $h_{\gamma(i)}=g_i$, and define $T:V\to V$ by $T(v_i)=b_{\gamma(i)}$.
    Then
    \begin{equation*}
        gg_i = g_{\alpha(i)}=gh_{\gamma(i)} = h_{\beta\gamma(i)} = g_{\gamma^{-1}\beta\gamma(i)}
    \end{equation*}
    so that $\alpha=\gamma^{-1}\beta\gamma$.
    Thus for each $b_i$,
    \begin{align*}
        T\circ\rho_1(g)\circ T^{-1}(b_i) &= T\circ\rho_1(g)(b_{\gamma^{-1}(i)})\\
                                         &= T(b_{\alpha\gamma^{-1}(i)}) b_{\gamma\alpha\gamma^{-1}(i)}\\
                                         &= b_{\beta(i)}=\rho_2(g)(b_i)\\
    \end{align*}
    so that $T\circ\rho_1(g)\circ T^{-1}=\rho_2(g)$.
\end{example}
Note: conjugate elements have the same cycle type.
\subsection{Subrepresentations}
What should a subrepresentation of $\rho:G\to\GL(V)$ mean?

We would like a subspace $W\leq V$ such that $\tau:G\to\GL(W)$ is a representation given by $\tau(g)(w)=\rho(g)(w)$ for all $w\in W$.
Moreover, to make this well-defined, we need $W$ to b4 $\rho(g)-$invariant for every $g\in G$ ($\rho(g)(W)\subseteq W$).

Suppose $T:V\to V$ is a linear operator, and $W\leq V$ is a $T-$invariant subspace; i.e. $T(W)\subseteq W$.
In particular, the restriction operator $T_W:W\to W$ is well-defined.
\begin{definition}
    Let $\rho:G\to\GL(V)$ be a representation.
    A subspqce $W\subseteq V$ is said to be \mbf{$G-$stable} if $W$ is $\rho(g)-$invariant for all $g\in G$.
    A \textbf{subrepresentation} of $\rho$ is a representation $\rho_W:G\to\GL(W)$ where for all $g\in G$ and $w\in W$, $\rho_W(g)(w)=\rho(g)(w)$ where $W$ is a $G-$stable subspace of $V$.
\end{definition}
\begin{example}
    Suppose $\rho:G\to\GL(V)$ be the regular representation.
    Take $W=\spn\{\sum_{g\in G}v_g\}$, which is clearly $G-$stable, and $\rho_W:G\to\GL(W)$ is isomorphic to the trivial representation.

    Similarly, let $\rho:S_n\to\GL(V)$ be the regular representation, $W=\spn\{\sum_{\sigma\in S_n}\sgn(\sigma)v_\sigma\}$; this is isomorphic to the sign representation.
\end{example}
\begin{theorem}
    Let $\rho:G\to\GL(V)$ be a representation, $W\leq V$ $G-$stable.
    Then there exists a $G-$stable subspace $W'$ such that $V=W\oplus W'$.
\end{theorem}
\begin{proof}
    Take any inner product $\langle x,y\rangle$ on $V$.
    Then for any $x,y\in V$, define
    \begin{equation*}
        \langle x,y\rangle^*=\sum_{g\in G}\langle \rho(g)(x),\rho(g)(y)\rangle
    \end{equation*}
    This is also an inner product.
    Let $x,y\in V$ and let $h\in G$.
    Then
    \begin{align*}
        \langle\rho(h)(x),\rho(h)(y)\rangle^* &= \sum_{g\in G}\langle\rho(g)\rho(h)(x),\rho(g)\rho(h)(y)\rangle\\
                                              &= \sum_{g\in G}\langle\rho(gh)(x),\rho(gh)(y)\rangle\\
                                              &= \sum_{g\in G}\langle\rho(g)(x),\rho(g)(y)\rangle
    \end{align*}
    Thus every $\rho(h)$ is unitary with respect to $\langle\cdot,\cdot\rangle^*$.
    Let $W\leq V$ be $G-$stable, and take $W'=W^\perp$ with respect to $\langle\cdot,\cdot\rangle^*$.
    Then $V=W\oplus W'$.
    Let's see that $W^\perp$ is $G-$stable.
    Let $x\in W^\perp$, $w\in W$, and $g\in G$, so that
    \begin{align*}
        \langle\rho(g)(x),w\rangle^*&=\langle x,\rho(g)^*(w)^*\rangle=\langle x,\rho(g)^{-1}(w)\rangle^*\\
                                    &=\langle x,\underbrace{\rho(g^{-1})(w)}_{\in W}\rangle^*\\
                                    &= 0
    \end{align*}
    and $\rho(g)(W^\perp)\subseteq W^\perp$ as required.
\end{proof}
\begin{definition}
    Let $\rho:G\to\GL(V)$ be a representation, and $V=W_1\oplus W_2\oplus\cdots\oplus W_k$ where each $W_i$ is $G-$stable.
    For each $i$, let $\rho_i=\rho_{w_i}$.
    For each $v=\sum w_i\in V$, we have $\rho(g)(v)=\sum\rho(g)(w_i)=\rho_i(g)(w_i)$.
    In this case, we write
    \begin{equation*}
        \rho=\rho_1\oplus\rho_2\oplus\cdots\oplus\rho_k
    \end{equation*}
    and call $\rho$ a direct sum of the $\rho_i$'s.
\end{definition}
The previous definition is written as an internal direct sum of $V$.
Externally, given vector spaces $W_1,\ldots,W_k$ and representations $\rho_i:G\to\GL(W_i)$, we can define
\begin{equation*}
    (\rho_1\oplus\cdots\oplus\rho_k):G\to\GL(W_1\oplus\cdots\oplus W_k)
\end{equation*}
by $(\rho_1\oplus\cdots\oplus\rho_k)(g)(w_1,\ldots,w_k)=(\rho_1(g)(w_1),\ldots,\rho_k(g)(w_k))$.
If $\rho_i:G\to\GL(W_i)$ is a subrepresentation fo $\rho:G\to\GL(V)$, we often say ``$W_i$ is a subrepresentation of $V$''.
\begin{definition}
    Let $\rho:G\to\GL(V)$ be a representation.
    We say $\rho$ is \textbf{irreducible} if $V\neq\{0\}$ and the only $G-$stable subspaces of $V$ are $\{0\}$ and $V$.
\end{definition}
Clearly,
\begin{theorem}
    Every representation $\rho:G\to\GL(V)$ can be written as a direct sum of irreducible sub-representations.
\end{theorem}
\begin{example}
    Let $\rho:S_3\to\GL(\C^3)$ be the permutation representation with respect to the standard basis $\{e_1,e_2,e_3\}$.
    Consider $W_1=\spn\{e_1+e_2+e_3\}$ and $W_2=\spn\{e_1-e_2,e_2-e_3\}$.
    Is $W_2$ irreducible?
    
    More generally, if $V=W_1\oplus\cdots\oplus W_k$ and $\dim W_i=1$ and $\deg(\rho_i)=1$,
    \begin{equation*}
        \rho(gh)(\sum w_i) = \sum\rho_i(gh)(w_i) = \sum\rho_i(g)\rho_i(h)(w_i)=\sum\rho_i(h)\rho_i(g)(w_i)
    \end{equation*}
    so that $\rho(gh)=\rho(hg)$.
    In the our example, this does not happen, since $\rho(g)\neq I$ when $g\neq 1$ and $S_3$ is not abelian.
\end{example}
\begin{example}
    Let $\rho:S_3\to\GL(V)$ be the regular representation.
    Let $W_1=\spn\{\sum_{\sigma\in S_3}v_\sigma\}$ and $W_2=\spn\{\sum_{\sigma\in S_3}\sgn(\sigma)v_\sigma\}$, and
    % \begin{equation*}
        % W_3 = \left\{\sum\alpha_\sigma v_\sigma | \begin{array}\alpha_\epsilon+\alpha_{(123)}+\alpha_{(1,3,2)}&=0\\\alpha_{(12)}+\alpha_{(13)}+\alpha_{(23)} &= 0\end{array}\right\}
    % \end{equation*}
    Now let's focus on $W_3$.
    A basis for $W_3$ is given by
    \begin{align*}
        e_1&=v_\epsilon-v_{(123)} & e_2 &= v_\epsilon-v_{(123)}\\
        e_3 &= v_{(12)}-v_{(13)} & e_4 &= v_{(12)}-v_{(23)}
    \end{align*}
    Recall that $S_3=\langle(12),(123)\rangle$; suffices to show stability with respect to generators.
    \begin{align*}
        \rho(12) &: e_1\mapsto e_4,e_2\mapsto e_3, e_3\mapsto e_2,e_4\mapsto e_1\\
        \rho(123) & : e_1\mapsto e_2-e_1, e_2\mapsto -e_1,e_3\mapsto e_4-e_3, e_4\mapsto -e_3
    \end{align*}
    Let $U_1=\spn\{e_1-e_4,e_2+e_3-e_1\}$
\end{example}
\section{Tensor Products}
Let $\rho:G\to\GL(V)$ and $\tau:G\to\GL(W)$ be representations.
We define the representation $\rho\otimes\tau:G\to\GL(V\otimes W)$
\begin{equation*}
    (\rho\otimes\tau)(g)(v\otimes w)=\rho(g)(v)\otimes\tau(g)(w)
\end{equation*}
\section{Character Theory}
We define the character of $\rho$ by $\rho:G\to\C$ as $\chi(G)=\Tr(\rho(g))$.
\begin{remark}
    If we choose a basis $\beta$ for $V$, then define $A(g)=[\rho(g)]_\beta$ and $\chi(G)$ is given by the sum of the diagonal entries of $A(g)$.
    Furthermore, if $A,B\in M_n(\C)$, then $\Tr(AB)=\Tr(BA)$.
\end{remark}
The remark implies a number of facts:
\begin{enumerate}[nl,r]
    \item $\rho\cong\tau$, then $\Tr(\rho(g))=\Tr(\tau(g))$.
    \item $\Tr(T)$ is the sum of eigenvalues of $T$
    \item $\chi(1)=\dim(V)$.
\end{enumerate}
\begin{proposition}
    For every $g\in G$ the eigenvalues of $\rho(g)$ have modulus 1.
    In particular, $\chi(g^{-1})=\overline{\chi(g)}$.
\end{proposition}
\begin{proof}
    Set $n=|G|$; then $\rho(g)^n=\rho(g^n)=I$ so that $\lambda^n-1=0$ for any eigenvalue $\lambda$, so $|\lambda|=1$.
    Furthermore,
    \begin{equation*}
        \overline{\chi(g)}=\overline{\sum\lambda_i}=\sum\overline{\lambda_i}=\sum\lambda_i^{-1}=\chi(g^{-1})
    \end{equation*}
    proving the second component.
\end{proof}
\begin{proposition}
    Let $\rho:G\to\GL(V)$ and $\tau:G\to\GL(W)$.
    Then $\chi_{\rho\oplus\tau}=\chi_\rho+\chi_\tau$ and $\chi_{\rho\otimes\tau}=\chi_\rho\cdot\chi_\tau$.
\end{proposition}
\begin{proof}
    Let $\beta_1=\{v_1,\ldots,v_n\}$ be a basis for $V$ and $\beta_2=\{w_1,\ldots,w_m\}$ a basis for $W$.

    Then a basis for $V\oplus W$ is given by $\beta=\{(v_1,0),\ldots,(v_n,0),(0,w_1),\ldots,(0,w_m)\}$.
    In particular,
    \begin{equation*}
        [(\rho\oplus\tau)(g)]_\beta=
        \begin{pmatrix}
            [\rho(g)]_{\beta_1} & \\
                                & [\tau(g)]_{\beta_2}
        \end{pmatrix}
    \end{equation*}
    and the trace result follows.

    A basis for $V\otimes W$ is given by $\gamma=\{v_i\otimes w_j:1\leq i\leq n,1\leq j\leq m\}$ in lexicographic order.
    Fix $g\in G$, and set $A=[\rho(g)]_{\beta_1}$, $B=[\rho(g)]_{\beta_2}$.
    Fix $v_i\otimes w_j\in\gamma$.
    Then
    \begin{align*}
        (\rho\otimes\tau)(g)(v_i\otimes w_j)&=\rho(g)(v_i)\otimes\tau(g)(w_j)\\
                                            &= (a_{1i}v_1+\cdots+a_{ni}v_n)\otimes(b_{1j}w_1+\cdots+b_{mj}v_m)\\
                                            &= \cdots+a_{ii}b_{jj}\cdot(v_i\otimes w_j)+\cdots\\
                                            &= \Tr([\rho\otimes\tau)(g)]_\delta)=\sum_{i,j}a_{ii}b_{jj}=\Tr(A)\Tr()=\chi_\rho(g)\cdot\chi_\tau(g)
    \end{align*}
\end{proof}
\begin{example}
    Suppose $\rho:S_n\to\GL(\C^n)$ is the permutation representation with respect to $\{e_1,\ldots,e_n\}$.
    Then $\chi(\sigma)=|\{e_i:\rho(\sigma)(e_i)=e_i\}|=|\Fix(\sigma)|$, which is the number of indices $i$ fixed by $\sigma$.
    Since $S_n$ acts transitively on $\{1,\ldots,n\}$, there is exactly 1 orbit, so by Burnside's lemma,
    \begin{equation*}
        n!=|S_n| = \sum_{\sigma\in S_n}\chi(\sigma)
    \end{equation*}
\end{example}
\begin{example}
    Let $\rho:G\to\GL(V)$ be the regular representation.
    Note that if $g\neq 1$, then for all $h\in G$, $gh\neq h$.
    In particular, this means that $\chi(g)=0$ if $g\neq 1$, and $\chi(1)=|G|$ (the dimension of $V$).
\end{example}
\begin{example}
    Let $\rho:S_3\to\GL(V)$ be the regular representation.
    Recall that $V=W_1\oplus W_2\oplus U_1\oplus U_2$ where $W_1$ is the trivial representation, $W_2$ is the sign representation, and $U_1,U_2$ are isomorphic.
    Let $S_3=\langle (12),(123)\rangle$; then we have
    \begin{equation*}
        \begin{array}{c|cc}
            x_1 & 1 & 1\\
            \hline
            x_2 & -1 & 1\\
            x_3 & a & b\\
            x_4 & a & b
        \end{array}
    \end{equation*}
    In particular, $\chi(12)=1-1+2a=0$ and $\chi(123)=1+1+2b=0$, so $b=-1$.
\end{example}
\begin{example}
    Let $\rho:G\to\GL(V)$ be a representation.
    In particular, $\rho(ghg^{-1})=\rho(g)\rho(h)\rho(g)$ so that $\Tr\rho(ghg^{-1})=\Tr\rho(h)$ so $\chi(ghg^{-1})=\chi(h)$; in other words, that characters are constant on conjugacy classes.
\end{example}
\begin{lemma}[Schur]
    Let $\rho:G\to\GL(V)$ and $\tau:G\to\GL(W)$ be irreducible representations, and suppose $T:V\to W$ is linear such that for all $g\in G$, $\tau(g)\circ T=T\circ\rho(g)$.
    Then either $T=0$ or $T$ is an isomorphism and $\rho\cong\tau$.
    Moreover, if $V=W$ and $\rho=\tau$, then $T$ is a scalar multiple of the identity.
\end{lemma}
\begin{proof}
    Assume $T\neq 0$.

    Let's first see that $T$ is injective, and let $v\in\ker(T)$.
    Then for any $g\in G$, $T(\rho(g)(v))=\tau(g)(T(v))=0$, so $\rho(g)(v)\in\ker(T)$.
    Thus $\ker(T)$ is $G-$stable (with respect to $\rho$).
    Since $\rho$ is irreducible and $T\neq 0$, $\ker(T)=\{0\}$.

    We also have that $T$ is surjective.
    Let $v\in\Im(T)$ and say $v=T(X)$ with $x\in V$.
    Then for $g\in G$, $\tau(g)(v)=\tau(g)(T(x))=T(\rho(g)(x))\in\Im(T)$ so $\Im(T)$ is $G-$stable, and again by irreducibility of $\tau$, $\Im(T)=W$.
    Thus $T$ is an isomorphism.

    Now let $\lambda\in\C$ be an eigenvalue of $T$ and consider $T'=T-\lambda I$.
    Now, note that for $g\in G$, $\rho(g)T'=T'\rho(g)$, but $T'$ has non-trivial kernel, so in fact $T'=0$.
\end{proof}
\begin{corollary}
    Let $\rho:G\to\GL(V)$ and $\tau:G\to\GL(W)$ be irreducible, and $T:V\to W$ linear.
    Consider
    \begin{equation*}
        T'=\frac{1}{|G|}=\sum_{g\in G}\tau(g)^{-1} T\rho(g)
    \end{equation*}
    Then
    \begin{enumerate}[nl,r]
        \item If $T'\neq 0$, then $\rho\cong\tau$ via $T'$.
        \item If $V=W$, $\rho=\tau$, then $T'=\Tr(T)/\dim(V)\cdot I$.
    \end{enumerate}
\end{corollary}
\begin{proof}
    Clearly $T':V\to W$ is linear, and for any $h\in G$,
    \begin{align*}
        \tau(h)T'&=\tau(h)\frac{1}{|H|}\sum_{g\in G}\tau(g^{-1})T\rho(g)\\
                 &= \frac{1}{|G|}\sum_{g\in G}\tau(hg^{-1})T\rho(g)\\
                 &=\frac{1}{|G|}\sum_{g\in G}\tau(g^{-1})T(\rho(gh))\\
                 &= \frac{1}{|G|}\sum_{g\in G}\tau(g^{-1})T\rho(g)\rho(h)\\
                 &= T'\rho(h)
    \end{align*}
    If $V=W$ and $\rho=T$, then $\Tr(T')=\frac{1}{|G|}\Tr(T)\cdot|G|=\Tr(T)=\alpha\dim(V)$, so $\alpha=\Tr(T)/\dim(V)$.
\end{proof}
Let $\rho:G\to\GL(V)$ and $\tau:G\to\GL(W)$ be irreducible representations, and $T:V\to W$ linear.
Let $\beta$ be a basis for $V$ and $\gamma$ a basis for $W$.
Then for $g\in G$, let $[\rho(g)]_\beta=(a_{ij}(g))$, $[\tau(g)]_{\gamma}=(b_{kl}(g))$, $[T]_\beta^\gamma=(X_{ki})$, and $[T']_{\beta}^\gamma=(x_{ki}')$.

By matrix multiplication, $x'_{ki}=\frac{1}{|G|}\sum_g\sum_{j,l}b_{kl}(g^{-1})x_{lj}a_{ji}$.
If $\rho\ncong\tau$, then $T'=0$, so by viewing the RHS as a polynomial in the $x_{ij}$, we have
\begin{equation*}
    \frac{1}{|G|}\sum_g b_{kl}(g^{-1})a_{ji}(g)=0
\end{equation*}
But now it $\rho=\tau$, then $T'=\lambda I$ where $\lambda=\Tr(T)/\dim(B)$ so that
\begin{equation*}
    \frac{1}{|G|}\sum_g\sum_{j,l}a_{kl}(g^{-1})x_{lj}a_{ji}(g) = \lambda\delta_{ki}=\frac{1}{\dim(V)}\sum_{j,l}\delta_{ki}\delta_{jl}x_{lj}
\end{equation*}
Then by equating coefficients of $x_{lj}$, we have
\begin{equation*}
    \frac{1}{|G|}\sum_g a_{kl}(g^{-1})a_{ji}(g)=\frac{1}{\dim(V)}\delta_{ki}\delta_{jl}
\end{equation*}
\begin{remark}
    If $G$ is a finite group, the consider the vector space of all functions $\phi:G\to\C$.
    For any $\phi,\psi$ in this vector space, $\langle\phi,\psi\rangle=\frac{1}{|G|}\sum_g\phi(g)\overline{\psi(g)}$ defines an inner product.
    Then if $\chi_1$, $\chi_2$ are characters of $G$, then
    \begin{equation*}
        \langle\chi_1,\chi_2\rangle=\frac{1}{|G|}\sum_g\chi_1(g)\chi_2(g^{-1})
    \end{equation*}
    We thus have:
\end{remark}
\begin{theorem}
    If $\chi$ is a character of an irreducible representation, then $\langle{\chi,\chi}=1$, and if $\chi_1$ and $\chi_2$ correspond to non-isomorphic representations, then $\langle\chi_1,\chi_2\rangle=0$.
\end{theorem}
\begin{proof}
    Say $[\rho(g)]_\beta=(a_{ij}(g))$ where $\rho$ is an irreducible representation with character $\chi$.
    Then
    \begin{align*}
        \langle\chi,\chi\rangle &= \frac{1}{|G|}\sum_g\chi(g)\chi(g^{-1}) = \frac{1}{|G|}\sum_g\chi(g^{-1})\chi(g)\\
                                &= \frac{1}{|G|}\sum_g\sum_{i,j}a_{ii}(g^{-1})a_{jj}(g) = \sum_{i,j}\left(\frac{1}{|G|}\sum_g a_{ii}(g^{-1})a_{jj}(g)\right)\\
                                &= \sum_{i,j}\left(\frac{1}{|G|}\sum_g a_{ii}(g^{-1})a_{ii}(g)\right)\\
                                &= \sum_i\frac{1}{\dim(V)} = 1
    \end{align*}
    To see the second part,
    \begin{align*}
        \langle\chi_1,\chi_2\rangle = \frac{1}{|G|}\sum_g\chi_1(g)\chi_2(g^{-1}) = \frac{1}{|G|}\sum_g\sum_{ij} a_{ii}(g)a_{jj}(g^{-1}) = \sum_{i,j}0=0
    \end{align*}
\end{proof}
If $\chi$ is a character corresponding to an irreducible representation, we say $\chi$ is irreducible.
If $\rho$ and $\tau$ are isomorphic representations, we say $\chi_\rho$ and $\chi_\tau$ are isomorphic (in fact $\chi_\rho=\chi_\tau$).
\begin{corollary}
    Let $\rho:G\to\GL(V)$ be a representation with character $\chi$.
    Say $V=W_1\oplus\cdots\oplus W_k$ is an irreducible decomposition of $V$.
    If $\tau:G\to\GL(W)$ is an irreducible representations with character $\phi$, then the number of $W_i$ isomorphic to $W$ (i.e. $\rho_i\cong\tau$) is $\langle\chi,\phi\rangle$.
\end{corollary}
\begin{proof}
    Write $\chi=n_1\chi_1+\cdots+n_l\chi_l$, where the $\chi_i$ are pairwise non-isomorphic.
    Then $\langle\chi,\chi_i\rangle=n_i$.
\end{proof}
Let $\tau:G\to\GL(V)$ be irreducible, and let $\tau$ have character $\varphi$.
Then
\begin{equation*}
    \langle \chi,\varphi\rangle = \sum_{i=1}^k\langle\chi_i,\varphi\rangle
\end{equation*}
Now, $\langle\chi_i,\varphi\rangle=1$ if and only if $\rho_i\cong\tau$, so that $\langle\chi,\varphi\rangle$ counts the number of times in which $\tau$ appears in the irreducible decomposition of $\rho$.

\begin{corollary}
    If two representations of $G$ have the same character, then they are isomorphic.
\end{corollary}
\begin{proof}
    They have the same irreducible decomposition.
\end{proof}
\begin{corollary}
    If $\rho:G\to\GL(V)$ is a representation and $\chi$ is a character, then $\langle\chi,\chi\rangle\in\N$ and $\langle\chi,\chi\rangle=1$ if and only if $\chi$ is ireducible.
\end{corollary}
\begin{proof}
    If $\chi_1,\ldots,\chi_k$ are irreducible, write $\chi=n_1\chi_1+\cdots+n_k\chi_k$ so that $\langle\chi,\chi\rangle=n_1^2+\cdots+n_k^2\in\N$.
\end{proof}
\begin{proposition}
    Every irreducible representation of $G$ occurs as a subgroup fo the regular representation of $G$, with multiplicity equal to its degree.
\end{proposition}
\begin{proof}
    Let $\chi$ be an irreducible character of $G$.
    Then
    \begin{equation*}
        \langle\chi,\chi_{\text{reg}}\rangle=\frac{1}{|G|}\sum_g\chi(g)\overline{\chi_{\text{reg}}(g)}=\frac{1}{|G|}\chi(1)\overline{\chi_{\text{reg}}(1)}=\frac{1}{|G|}\deg(\chi)
    \end{equation*}
\end{proof}
\begin{corollary}
    Let $\chi_1,\ldots,\chi_k$ be the distinct irreducible characters of $G$, with $\deg(\chi_i)=n_i$.
    Then $\sum n_i^2=|G|$ for for $g\neq 1$, $\sum_{i=1}^kn_1\chi_i(g)=0$
\end{corollary}
\begin{proof}
    Recall that $\chi_{\text{reg}}=n_1\chi_1+\cdots+n_k\chi_k$.
    Then $\chi_{\text{reg}}(1)=|G|=n_1^2+\cdots+n_k^2$, and evaluation at $g\neq 1$ gives the desired result.
\end{proof}
\begin{definition}
    Let $G$ be a group.
    A function $f:G\to\C$ is called a class function if $f$ is constant on each conjugacy class, i.e. for all $a,b\in G$, $f(bab^{-1})=f(a)$.
\end{definition}
\begin{proposition}
    Let $\rho:G\to\GL(V)$ be a representation.
    Then
    \begin{equation*}
        \rho_f=\sum_g f(g)\rho(g)
    \end{equation*}
    is a linear operator on $V$.
    If $\rho$ is irreducible of degree $n$, then $\rho_f=\lambda I$, where $\lambda=\frac{|G|}{n}\langle f,\overline{x}\rangle$ where $\chi$ is the character of $\rho$.
\end{proposition}
\begin{proof}
    Note that
    \begin{align*}
        \rho_f\circ\rho(h)&=\sum_g f(g)\rho(g)\rho(h)=\sum_g f(g)\rho(gh)\\
                          &= \sum_g f(hgh^{-1})\rho(hg)\\
                          &= \sum_g f(g)\rho(h)\rho(g)=\rho(h)\circ\rho_f
    \end{align*}
    so by Schur, $\rho_f=\lambda I$ where $\lambda=\Tr(\rho_f)/n$.
    However, $\Tr(\rho_f)=\Tr(\sum_gf(g)\rho(g))=\sum_g f(g)\chi(g)=|G|\langle f,\overline{x}\rangle$.
\end{proof}
Recall that
\begin{itemize}[nl]
    \item $\langle\chi,\chi\rangle=1$ if and only if $\chi$ is irreducible
    \item If $\chi_\rho$ and $\chi_\tau$ are irreducible then $\langle\chi_\rho,\chi_\tau\rangle=0$ if $\rho\ncong\tau$, and $1$ otherwise.
    \item If $\chi'$ is an irreducible subrepresentation of $\chi$, then $\langle\chi,\chi'\rangle$ is the multiplicity of $\chi'$ in $\chi$.
    \item $|G|=n_1^2+\cdots+n_k^2$ where $n_i$ is the multiplicity of $\chi_i$ as an irreducible subrepresentation of the regular representation.
    \item Every irreducible character is a character of some subrepresentation of the regular rep?
    \item ... every irreducible represetation is a subrepresentation of the regular rep?
\end{itemize}
and
\begin{equation*}
    \rho_f=\sum_g f(g)\rho(g)=\lambda I
\end{equation*}
where $\lambda=|G|/\dim(V)\cdot\langle f,\overline{\chi}\rangle$.
\begin{proposition}
    Let $G$ be a group.
    The irreducible characters of $G$ form an orthonormal basis for the vector space of class functions on $G$.
\end{proposition}
\begin{proof}
    Let $\beta=\{\chi_1,\ldots,\chi_k\}$ be the irreducible characters of $G$.
    We know that $\beta$ is orthonormal, and hence linearly independent.
    Let $W=\spn(\beta)$.
    To show $W=V$ where $V$ is the space of class functions, we prove that $W^{\perp}=\{0\}$.
    Let $f\in W^\perp$, and suppose $\rho:G\to\GL(V)$ is irreducible.
    By A2, $\overline{\chi}_1,\ldots,\overline{\chi}_k$ are all irreducible characters of $G$.
    Thus $\rho_f=0$.
    By considering irreducible decompositions, $\rho_f=0$ for all representations $\rho:G\to\GL(V)$.
    In particular, when $\rho$ is the regular representation,
    \begin{equation*}
        0=\rho_f(v_1)=\sum_gf(g)\rho(g)(v_1)=\sum_g f(g)v_g
    \end{equation*}
    so by independence of $\{v_g:g\in G\}$, $f(g)=0$ for all $g\in G$.
\end{proof}
\begin{corollary}
    The number of irreducible characters of $G$ is equal to the number of conjugacy classes of $G$.
\end{corollary}
\begin{proof}
    Let $C_1,\ldots,C_k$ be the conjugacy classes.
    Then a basis for $V_{\text{class}}=\{\phi_1,\ldots,\phi_k\}$ where each $\phi_i$ is the indicator for $C_i$.
    Since bases must have the same size, the result follows.
\end{proof}
\begin{proposition}
    Let $G$ be a group, $g\in G$, and $O_g$ the conjugacy class of $g$.
    Let $\chi_1,\ldots,\chi_k$ be the irreducible characters of $G$.
    Then
    \begin{enumerate}[nl]
        \item $\sum_{i=1}^k|\chi_i(g)|^2=|G|/|O_g|$
        \item If $h$ is not conjugate to $g$, then $\sum_{i=1}^k\chi_i(g)\overline{\chi_i(h)}=0$.
    \end{enumerate}
\end{proposition}
\begin{proof}
    Define $\phi:G\to\C$ where $\phi(x)$ is the indicator function for $O_g$.
    Write $\phi=\sum_{i=1}^k \lambda_i\chi_i$ where
    \begin{equation*}
        \lambda_i=\langle\phi,\chi_i\rangle=\frac{1}{|G|}\sum_x\phi(x)\overline{\chi_i(x)}=\frac{|O_g|\overline{\chi_i(g)}}{|G|}
    \end{equation*}
    Therefore,
    \begin{equation*}
        \phi(x) = \frac{|O_g|}{|G|}\sum_{i=1}^k\overline{\chi_i(g)}\chi_i(x)
    \end{equation*}
    Then the result follows by evaluating $\phi$ at $g$ and $h$.
\end{proof}
\begin{example}
    Let's compute the character table of $S_3$.
    There are 2 degree 1 representations, and 3 irreducible characters since there are three conjugacy classes (cycle types).
    In particular, $|S_3|=6=1^2+1^2+n_3^2$, so $n_3=2$.
    \begin{equation*}
        \begin{array}{c|ccc}
            &\epsilon & (12) & (123)\\
            \text{(triv)}\chi_1 & 1&1&1\\
            \text{(sgn)}\chi_2 & 1&-1&1\\
            \chi_3 & 2 &a & b
        \end{array}
    \end{equation*}
    Note that the columns must be orthogonal, so by the previous proposition, we have $a=0$ and $b=-1$.
\end{example}
Let $\chi_1,\ldots,\chi_k$ be the irreducible characters of $G$.
Then $\sum_{g|\chi_i}^2=|G|$ and $\sum_{i=1}^k|\chi_i(g)|^2=|G|/|O_g|$.

Let $G$ be abelian.
By A1, $G$ has $|G|$ representations of degree 1, and $[G:[G,G]=|G|$.
Since $G$ as $|G|$ conjugacy classes, these are all of the irreducible representations of $G$.
Suppose $G$ is a group whose irreducible representations are all degree one.
Since $n_1^2+\cdots+n_k^2=|G|$, then $k=|G|$.
\begin{proposition}
    Let $H$ be an abelian subgroup of $G$.
    Then any irreducible representation of $G$ has degree at most $[G:H]$.
\end{proposition}
\begin{proof}
    Let $\rho:G\to\GL(V)$ be an irreducible representation of $G$.
    Consider the restriction $\tilde\rho:H\to\GL(V)$.
    Let $W\leq V$ be an irreducible subrepresentation of $\tilde G$.
    Since $H$ is abelian, $\dim W=1$.
    Suppose $W=\spn\{x\}$, and let $W'=\{\rho(g)(x):g\in G\}$ so that $V'$ is $G-$stable, and in fact $V'=V$ since $\rho$ is irreducible.

    Take $g\in G$ and $h\in H$, so $\rho(gh)=\rho(g)\rho(h)(x)=\rho(g)(\alpha x)=\alpha\rho(g)(x)$
    Say $g_1,\ldots,g_m$ are coset representatives of $H$ in $G$.
    Then $V=V'=\spn\{\rho(g_i)(x):1\leq i\leq m\}$, then $\dim(V)\leq m=[G:H]$.
\end{proof}
\begin{example}
    Consider $D_4$.
    Then the number of degree 1 representations is $[D_4:\langle r^2\rangle]=4$.
    Since there are 5 conjugacy classes, we know that there are 5 irreducible representations, so that $n_5^2=8$.
    Let's make the character table:
    \begin{equation*}
        \begin{array}{c|ccccc}
            D_4    & 1 & r &r^2& s & rs\\
            \hline
            \chi_1 & 1 & 1 & 1 & 1 & 1\\
            \hline
            \chi_2 & 1 &-1 & 1 & 1 &-1\\
            \hline
            \chi_3 & 1 & 1 & 1 &-1 &-1\\
            \hline
            \chi_4 & 1 &-1 & 1 &-1 & 1\\
            \hline
            \chi_5 & 2 & a & b & c & d
        \end{array}
    \end{equation*}
    But then by column orthogonality, we have $a=0$, $b=-2$, $c=0$, $d=0$.
\end{example}
\begin{example}
    Consider $S_4$.
    Then $[S_4:A_4]=2$ so there are two degree 1 representations (the trivial and the sign), and the conjugacy classes are given by $1$, $(12)$, $(12)(34)$, $(123)$, $(1234)$, so there are 5 irreducible representations.
    Since $24^2=1^2+1^2+n_3^2+n_4^2+n_5^2$, we have $22=n_3^2+n_4^2+n_5^2$, which forces $n_3=2$ and $n_4=n_5=3$.
    Now we have
    \begin{equation*}
        \begin{array}{c|ccccc}
            D_4    & 1 & (12) &(12)(34)& (123) & (1234)\\
            \hline
            \chi_1 & 1 & 1 & 1 & 1 & 1\\
            \hline
            \chi_2 & 1 &-1 & 1 & 1 &-1\\
            \hline
            \chi_3 & 2 & 1 & 1 &-1 &-1\\
            \hline
            \chi_4 & 3 &-1 & 1 &-1 & 1\\
            \hline
            \chi_5 & 3 & a & b & c & d
        \end{array}
    \end{equation*}
    Note that $K=\{1,(12)(34),(13)(24),(14)(23)\}\trianglelefteq S_4$ and $H=\{1,(12),(13),(123),(132),(23)\}$, so $S_4=KH$.
    Let $\rho$ be an irreducible representation of $H$ of degree 2:
    \begin{equation*}
        \begin{array}{c|cccc}
            S_3 & 1 & (12) & (123)\\
            \hline
            \alpha_1 & 1 & 1 & 1\\
            \alpha_2 & 1 &-1 & 1\\
            \alpha_3 & 2 & 0 & -1
        \end{array}
    \end{equation*}
    Then $\rho:S_4\to\GL(V)$ by $\rho(kh):=\rho(h)$ is an irreducible representation of $S_4$ since $K\trianglelefteq S_4$.
\end{example}
\section{Induced Representations}
Given a subgroup $H\leq G$ and a representation $\rho:H\to\GL(V)$, construct a representation of $G$.
Let $H\leq G$ and $\rho:H\to\GL(V)$ a representation.
Say the cosets of $H$ in $G$ are $g_1H,\ldots,g_mH$.
For each $i$, let $g_iV=\{g_iv:v\in V\}$ be an isomorphic copy of $G$, and let $W=\bigoplus_{i=1}^m g_iV$ so that every $w\in W$ can be uniquely written as $w=g_1v_1+\cdots+g_mv_m$, where $m=[G:H]$.
Fix $g\in G$; then there exists $\pi\in S_m$ such that for every $i$, $gg_i=g_{\pi(i)}h_i$, $h_i\in H$.
We then define $\Ind_H^G(\rho):G\to\GL(W)$ by
\begin{equation*}
    \Ind_H^G(\rho)(g)(\sum g_iw_i0=\sum g_{\pi(i)}\rho(h_i)v_i
\end{equation*}
\begin{example}
    Let $\{1\}\leq G$ and suppose $\rho:\{1\}\to\GL(\C)$ is the trivial representation.
    Then $G=\{g_1,\ldots,g_n\}$.
    Then for $g\in G$, $gg_i1\in G$ and
    \begin{equation*}
        \Ind(\rho)(s)\left(\sum_{i=1}^ng_i\alpha_i\right) = \sum gg_i\rho(1)(\alpha_i)=\sum gg_i\alpha_i
    \end{equation*}
    so that $\Ind(\rho)$ is isomorphic to the regular representation.
\end{example}
\begin{example}
    Consider $\langle r\rangle\leq D_n$, and let $\rho:\langle r\rangle\to\GL(\C)$ be given by $\rho(r)(1)=\zeta_n$.
    Let the coset representatives be given by $\epsilon$ and $s$.
    \begin{enumerate}[nl,r]
        \item Let $r\in D_n$< so $r\epsilon=\epsilon r$ and $rs=sr^{n-1}$.
            Fix $W=\epsilon\C\oplus s\C$.
            Then $\Ind(\rho):D_n\to\GL(W)$ is given by $\Ind(\rho)(r)(\epsilon\alpha_1+s\alpha_2)=\epsilon\zeta_n\alpha+1+s\zeta_n^{n-1}\alpha_2$.
        \item Let $s\in D_n$.
            Then $s\epsilon=s\epsilon$ and $ss=\epsilon\epsilon$.
            Then $\Ind(\rho)(s)(\epsilon\alpha_1+s\alpha_2)=s\rho(\epsilon)(\alpha_1)+\epsilon\rho(\epsilon)(\alpha_2)=s\alpha_1+\epsilon\alpha_2$.
    \end{enumerate}
    Take the basis $\beta=\{\epsilon,s\}$ for $W$, so we have
    \begin{equation*}
        [\Ind(\rho)(r)]_\beta=
        \begin{pmatrix}
            \zeta_n & 0\\
            0 & \zeta_n^{n-1}
        \end{pmatrix}
        \quad
        [\Ind(\rho)(s)]_\beta=
        \begin{pmatrix}
            0 & 1\\
            1 & 0
        \end{pmatrix}
    \end{equation*}
\end{example}
\section{Non-Commutative Module Theory}
Let $R$ be a ring with unity and $(M,+)$ an abelian group.
We can equip $\End(M)$ with a ring structure given by $(f+g)(x)=f(x)+g(x)$ and $fg(x)=f(g(x))$.
\begin{definition}
    A (left) $R-$module is an abelian group $(M,+)$ equipped with a unitary ring homomorphism $\alpha:A\to\End(M)$.
\end{definition}
This map $\alpha$ defines a multiplication between elements of $r$ and $m$ given by $rm=\alpha(r)(m)$.
\begin{example}
    \begin{enumerate}[nl,r]
        \item If $F$ is a field, a $F-$module is a $F-$vector space.
        \item $M$ is a $\Z-$module if and only if $M$ is an abelian group.
        \item $R$ is an $R-$module (left multiplication)
        \item If $I$ is a left ideal of $R$, then $I$ is a left $R-$module.
        \item $R=M_n(F)$, and $V=F^n$.
            Then $V$ is an $R-$module.
        \item Let $R$ be a ring and $I$ a left ideal of $R$.
            Then $R/I=\{a+I:a\in R\}$, so $R/I$ is an $R-$module with $r(a+I)=ra+I$.
    \end{enumerate}
\end{example}
Let $M$ be an $R-$module.
We say a subgroup $(N,+)$ of $(M,+)$ is an $R-$submodule of $M$ if $N$ is $\alpha(r)-$invariant for each $r\in R$.
\begin{definition}
    Let $G$ be a finite group and $F$ a field.
    We define the group algebra $F[G]=\{\alpha_1g_1+\cdots+\alpha_ng_n:\alpha_i\in F\}$ equipped with $G-$pointwise addition and multiplication $ag_i\cdot bg_j=(ab)g_ig_j$, extended by distributivity.
\end{definition}
\begin{example}
    Let $M$ be a $\C[G]-$module.
    Then $M$ is also a $\C-$vector space, and $\rho:G\to\GL(M)$ given by $\rho(g)(m)=gm$ is a representation.
\end{example}
\begin{example}
    If $\rho:G\to\GL(V)$ be a representation, the $\rho$ induces a $\C[G]-$multiplication on $V$, making $V$ a $\C[G]-$module.
    Moreover, if $N\leq M$ is a submodule, then it is $\rho(cg)-$invariant for any $cg\in\C[G]$ if and only if $N$ as a subspace of $M$ is $G-$stable.
\end{example}
To be precise, we have $c g\cdot v=\rho(g)(c v)$.
In fact, there is an isomorphic of categories from representations of $G$ and $\C[G]$-modules.
\begin{definition}
    Let $N,M$ be $R-$modules.
    We say $\psi:N\to M$ is a (module) homomorphism if $\phi$ commutes with the structures on $N$ and $M$.
\end{definition}
If $\phi:N\to M$ is a homomorphism where $N,M$ are $\C[G]-$modules, with multiplication maps $\rho$ and $\tau$.
Then $\phi\circ\rho=\tau\circ\phi$, in other words that it is an intertwining map.
Note that $\rho:G\to\GL(V)$ is faithful if only if the unique zero map on $v$ is $0$.
\begin{definition}
    Let $M$ be an $R-$module.
    The \textbf{annhilator} $\Ann(M)=\{r\in R:rm=0\}$.
    Then $M$ is \textbf{faithful} if $\Ann(M)=(0)$.
\end{definition}
\begin{proposition}
    Let $M$ be an $R-$module.
    Then $\Ann(M)$ is a (2-sided) ideal of $R$.
    Moreover, $M$ is a faithful $R/\Ann(M)-$module.
\end{proposition}
\begin{definition}
    An $R-$module $M$ is \textbf{irreducible} if $M\neq (0)$ and the only submdules of $M$ are $(0)$ and $M$.
\end{definition}
Recall that a division ring is a unital ring such that every non-zero element is invertible.
\begin{theorem}[Schur]
    Let $M$ be an irreducible $R-$module.
    Then $\End_R(M)$ is a division ring.
\end{theorem}
\begin{theorem}
    Let $M$, $N$ be $R-$modules and let $\psi:M\to N$ be a module homomorphism.
    Then $M/\ker\psi\cong\psi(M)\leq N$.
\end{theorem}
\begin{proposition}
    Let $M$ is an irreducible $R-$module, then $M\cong R/I$, where $I$ is a maximal left ideal.
    Conversely, if $I$ is a maximal let ideal, then $R/I$ is irreducible.
\end{proposition}
\begin{proof}
    Let $M$ be an irreducible $R-$module and fix $0\neq m\in M$, and define $\phi:R\to M$ by $\phi(r)=rm$, so $\phi$ is a homomorphism and $R/\ker\phi\cong\phi(R)=M$ by irreducibility.
    But then $I$ is maximal since $R/I\cong M$ is simple.
\end{proof}
\begin{definition}
    Let $R$ be a ring.
    Then the \textbf{Jacobson radical} of $R$ is $J(R)=\bigcap_{\text{irred left }M}\Ann(M)$.
\end{definition}
\begin{definition}
    A left ideal $I$ of $R$ is called \textbf{left quasiregular} if for all $a\in I$, $R(1+a)=R$.
\end{definition}
\begin{theorem}
    If $R$ is a ring, then the following are equivalent:
    \begin{enumerate}[nl,r]
        \item $a\in J(R)$.
        \item $Ra$ is left quasiregular
        \item $a\in\bigcap_{I\leq R\text{ maximal}}I$.
    \end{enumerate}
\end{theorem}
\begin{proof}
    \imp{i}{ii}
    Let $a\in J(R)$ and for contradiction assume for some $x\in R$ $R(1+xa)\neq R$.
    Thus there exists a maximal let ideal $I$ such that $R(1+xa)\subseteq I$, so that $R/I$ is an irreducible $R-$module.
    Thus $a(R/I)=(0)$, so that $a(\overline{1})=\overline{a}=\overline{0}$, so $xa\in I$ and $1\in I$, a contradiction.

    \imp{ii}{iii}
    Assume $Ra$ is left quasiregular.
    Assume there exists some maximal left ideal $I$ with $a\notin I$.
    Since $R/I$ is irreducible, $I+Ra/I\leq R/I$ is a non-zero ideal.
    By irreducibility, $I+Ra/I=R/I$, so there exists $x\in R$ so that $\overline{x}\overline{a}=\overline{-1}$, so $1+xa\in I$ is left-invertible, so $I=R$, a contradiction.

    \imp{iii}{i}
    Let $A=\bigcap_{I\text{ left max}}I$.
    Suppose there exists an irreducible module $M$ so that $AM\neq(0)$.
    Then there exists $0\neq m\in M$ such that $Am\neq(0)$.
    Note that $am$ is a left $R-$submodule of $M$, so there exists $a\in A$ so that $am=-m$.
    Thus $(1+a)m=0$, so if $(1+a)$ is in a maximal left ideal, then $1+a-a$ is as well.
    Thus $(1+a)$ is left-invertible, so $m=0$, a contradiction.
\end{proof}
\begin{remark}
    \begin{equation*}
        J(R)=\bigcap_{M\text{ irreducible}}\Ann(M)=\bigcap_{\text{left max}}I=\sum_{\text{left quasi-reg}}Ra
    \end{equation*}
    Let $a\in J(R)$, $x\in R$, and suppose $R(1+ax)\neq R$, so $R(1+ax)\subseteq I$ where $I$ is left maximal.
    Thus $R/I$ is irreducible, so $a(x+I)=\overline{0}$, so $ax\in I$, so $1\in I$.

    If $a\in J(R)$, then $1+a$ is invertible so get $b\in R$ so that $b(1+a)-a$.
    Then since $a+b+ba=0$, so $b\in J(R)$.
    By the same argumetn, get $c\in J(R)$ with $c(1+b)=-b$.
    But then subtracting, manipulating, we get $cb=ba$ so that $a+b=b+c$ and in fact $a=c$.
    Thus $(1+a)b=b+ab=b+cb=-a$.
    Thus $(1+a)b=-a$, so $(1+a)R=R$.
    Thus $J(R)=\{x:xr\text{ is right quasiregular}\}$.
\end{remark}
\begin{definition}
    A ring is \textbf{semiprimitive} if $J(R)=(0)$.
\end{definition}
Recall that
\begin{equation*}
    J(R)=\bigcap_{\text{left max}}I = \bigcap_{\text{irred left}}\Ann(M) = \bigcap_{\text{left quasi-ref}}\{Ra:\forall x, R(1+xa)=R\}
\end{equation*}
\begin{example}
    \begin{enumerate}
        \item $J(\Z)=\bigcap_{p\text{ prime}}\langle p\rangle$
        \item $J(F[[x]])=\langle x\rangle$
        \item $J(\Z_{12})=\langle 2\rangle\cap\langle 3\rangle=\langle 6\rangle$
    \end{enumerate}
\end{example}
\begin{definition}
    Let $R$ be a ring.
    We say $a\in R$ is \textbf{nilpotent} if there exists $n=n(a)\in\N$ such that $a^n=0$.
    An ideal (left,right,both) is \textbf{nil} if every element is nilpotent.
    An ideal $I$ (left,right,both) is \textbf{nilpotent} if there exists some $n\in\N$ such that $I^n=(0)$.
\end{definition}
\begin{proposition}
    Every nil left ideal of $R$ is contained in $J(R)$.
\end{proposition}
\begin{proof}
    It suffices to show that for every nil element $a$ that $(1+a)$ is invertible.
    Indeed, since $a^n=0$ for some $n$, $(1-a+a^2-\cdots+(-1)^{n-1}a^{n-1})(1+a)=1$.
\end{proof}
\begin{proposition}
    $J\left(\quot{R}{J(R)}\right)=(0)$, in other words, $\quot{R}{J(R)}$ is semiprimitive.
\end{proposition}
\begin{proof}
    \begin{equation*}
        J\left(\quot{R}{J(R)}\right) = \bigcap_{\substack{I\subseteq R\text{ max}\\J(R)\subseteq I}}\quot{I}{J(R)} = \bigcap_{\substack{I\subseteq R\\\text{left max}}}\quot{I}{J(R)}=\quot{J(R)}{J(R)}=(0)
    \end{equation*}
\end{proof}
\begin{definition}
    A ring $R$ is \textbf{(left) Artinian} if whenever $I_1\supseteq I_2\supseteq\cdots$ is a descending chain of left ideals, then there exists $N\in\N$ such that $I_k=I_N$ for all $k\geq N$.
\end{definition}
\begin{example}
    \begin{enumerate}[nl,r]
        \item $\Z$ is not Artinian.
        \item If $R$ Artinian, then $M_n(R)$ is Artinian.
            If $I$ is an ideal of $M_n(R)$, then $I=M_n(I')$ where $I'$ is an ideal of $R$.
        \item Division rings are artinian
        \item Suppose $R$ is an $F-$algebra, where $F$ is a field (isomorphic copy of $F$ contained in the center of $R$).
            If $\dim_FR<\infty$, then $R$ is Artinian
        \item If $F$ is a field and $G$ is a finite group, then $F[G]$ is Artinian since $\dim F[G]=|G|<\infty$
    \end{enumerate}
\end{example}
\begin{proposition}
    If $R$ is Artinian, then $J(R)$ is nilpotent.
\end{proposition}
\begin{proof}
    Consider $J(R)\supseteq J(R)^2\supseteq\cdots$.
    Thus there exists $N$ such that $J(R)^k=J(R)^n$ for all $k\geq N$.
    Let $I=J(R)^N$; let's see that $I=(0)$.
    Suppose $I\neq (0)$.
    Let $A$ be a minimal left ideal fo $R$ such that $IA\neq (0)$.
    Let $a\in A$ so that $Ia\neq(0)$, so $Ia$ is a left ideal and $I(Ia)=I^2a=Ia$.
    Thus by minimality, $A=Ia$ so there is some $x\in I$ such that $a=xa$.
    Thus $(1-x)a=0$ so $a=0$, a contradiction.
\end{proof}
\begin{theorem}[Maschke]
    Let $G$ be a finite group.
    If $F$ is a field such that $\chr(F)=0$ or $\chr(F)=p$ does not divide $|G|$, then $F[G]$ is semiprimitive and Artinian (and hence semisimple, by the assignment).
\end{theorem}
\begin{proof}
    Since $\dim_F F[G]<\infty$, $F[G]$ is Artinian.
    For contradiction, suppose $I$ is a nonzer nil ideal of $R$.
    Take $0\neq x\in I$, so $x=\sum a_gg$ where $a_h\neq 0$ for some $h\in G$.
    By multiplying by $h^{-1}$, we may assume $a_1\neq 0$.
    For each $a\in F[G]$, define $T_a:F[G]\to F[G]$ by $T_a(v)=av$, so $T_a$ is a $F-$linear operator.
    Note that $T_x=\sum a_gT_g$ so that $\Tr(T_x)=\sum a_g\Tr(T_g)$, so $x$ is not nilpotent, a contradiction.
\end{proof}
\subsection{Artin-Wedderburn Theory}
\begin{definition}
    A ring $R$ is \textbf{primitive} if it has a faithful, irreducible module.
\end{definition}
Note that primitive rings are semiprimitive.
\begin{example}
    If $D$ is a division ring, then $M_n(D)$ is primitive.
    In particular, $D^n$ is faithful and irreducible
\end{example}
Let $R$ be primitive and commutative.
Then if $M$ is faithful and irreducible, $M\cong\quot{R}{I}$ where $I$ is a maximal ideal so $R$ is a field.
\begin{definition}
    A ring $R$ is \textbf{simple} if $R\neq(0)$ and $R$ has no proper non-zero two-sided ideals.
\end{definition}
For example, $M_n(D)$ is simple.
If $J\leq M_n(D)$ is an ideal, then $J=M_n(I)$ for some ideal $I$ of $D$.
\begin{remark}
    If $R$ is irreducible, then $R$ is simple.
    However, the converse does not hold since $M_2(\R)$ is simple but $I=\{\begin{bmatrix}a&0\\b&0\end{bmatrix}:a,b\in\R\}$ is a left ideal.
\end{remark}
\begin{proposition}
    Every simple ring is primitive.
\end{proposition}
\begin{proof}
    Let $R$ be simple and $I$ be a maximal left ideal of $R$ so that $M:=R/I$is irreducible.
    Since $\Ann(M)$ is an ideal of $R$ and $\Ann(M)\neq R$, $\Ann(M)=(0)$.
\end{proof}
For the remainder of this section, $R$ is primitive, $M$ is faithful and irreducible, and $D=\End_R(M)$ is a division ring.
We give $M$ the structure of a $D-$module by $\phi\cdot m=\phi(m)$.
\begin{definition}
    We say $R$ \textbf{acts densely} on $M$ if for all $D-$linearly independent $v_1,\ldots,v_n\in M$ and all $w_1,\ldots,w_n\in M$, there exists $r\in R$ such that $rv_i=w_i$ for $i=1,2,\ldots,n$.
\end{definition}
\begin{remark}
    Suppose $\dim_DM<\infty$ and $R$ acts densely on $M$
    If $\{v_1,\ldots,v_n\}$ is a $D-$basis, then for all $w_1,\ldots,w_n$, there exists $r\in R$ so that $rv_i=w_i$.
    Thus $R\cong\{T:M\to M:D-\text{linear}\}\cong M_n(D)$.
\end{remark}
\begin{lemma}
    If for every finite dimensional $D-$subspace $V$ of $M$ and every $m\in M\setminus V$ there exists $r\in R$ such that $rV=(0)$ but $rm\neq 0$, then $R$ acts densely on $M$.
\end{lemma}
\begin{proof}
    Let $v_1,\ldots,v_n$ be $D-$linearly independent in $M$ and suppose $w_1,\ldots,w_n$ are in $M$.
    For each $i$, let $V_i=\spn\{v_1,\ldots,v_{i-1},v_{i+1},\ldots,v_n\}$.
    By assumption, since $v_i\notin V_i$, there exists $t_i\in R$ so that $t_iV_i=(0)$ but $t_iv_i\neq 0$.
    Observe that $Rt_iv_i=M$ since $M$ is irreducible, so get $r_i\in R$ such that $r_it_iV_i=(0))$ and $r_it_iv_i=w_i$.
    Let $t=r_1t_1+\cdots+r_nt_n$, so $tv_i=w_i$.
\end{proof}
\begin{theorem}[Jacobson Density]
    Let $R$ be primitive and $M$ a faithful irreducible $R-$module.
    Then $R$ acts densely on $M$.
\end{theorem}
\begin{proof}
    Let $V$ be a finite dimensional $D-$subspace of $M$, and let $m\in M\setminus V$.
    We proceed by induction on $\dim V$.
    If $\dim V=0$, $V=(0)$, and take $r=1$.
    Proceeding inductively, suppose $\dim V>0$ and $0\neq w\in V$ with $V=V_0\oplus\spn\{w\}$, where $\dim V_0=\dim V-1$.
    Let $A(V_0)=\{x\in R:xV_0=(0)\}$.
    By induction, for every $y\in V_0$, there exists $r\in A(V_0$ such that $ry\neq 0$.
    Note that $A(V_0)$ is a left ideal: since $w\notin V_0$, $A(v_0)w\neq(0)$ so $A(v_0)w=M$ by irreducibility.
    Consider $\tau:M\to M$ given by $\tau(aw)=am$, where $a\in A(v_0)$.
    This is well-defined for if $aw=a'w$, then $(a-a')w=0$ so $(a-a')V=0$ (since $V=V_0\oplus\spn_D\{w\}$).
    For contradiction, assume that if $r\in R$ and $rV=(0)$, then $rm=0$.
    Thus $(a-a')m=0$ so $am=a'm$ and $\tau(a2)=\tau(a'w)$ and $\tau$ is well-defined.
    Notice that $\tau\in\End_R(M)=D$.
    For all $a\in A(v_0)$, $am=\tau(aw)=a\tau(w)$ so $a(m-\tau(w))=0$.
    Thus by the inductive hypothesis, $M-\tau(w)\in V_0$, so $m\in v_0\oplus\spn_D(w)=V$.
\end{proof}
\begin{proposition}
    If $R$ is primitive and (left) Artinian, then $R\cong M_n(D)$ where $D\cong\End_R(M)$.
\end{proposition}

\section{Facts about Non-Commutative Modules}
General structures on modules:
\begin{definition}
    A \mbf{(left) $R-$module} is an abelian group $(M,+)$ equipped with a unitary ring homomorphism $\alpha:A\to\End(M)$.
    If $N,M$ be $R-$modules, then a group homomorphism $\psi:N\to M$ is a \textbf{(module) homomorphism} if $\phi(rm)=r\phi(m)$ for any $r\in R$.
    The kernel and image of $\psi$ are submodules of $N$ and $M$ respectively.
    The \textbf{annhilator} $\Ann(M)=\{r\in R:rm=0\}$.
    Then $M$ is \textbf{faithful} if $\Ann(M)=(0)$.
\end{definition}
Annhilators:
\begin{definition}
    Let $R$ be a ring.
    We say $a\in R$ is \textbf{nilpotent} if there exists $n=n(a)\in\N$ such that $a^n=0$.
    An ideal (left,right,both) is \textbf{nil} if every element is nilpotent.
    An ideal $I$ (left,right,both) is \textbf{nilpotent} if there exists some $n\in\N$ such that $I^n=(0)$.
\end{definition}
Key example:
\begin{definition}
    Let $G$ be a finite group and $F$ a field.
    We define the \textbf{group algebra} $F[G]=\{\alpha_1g_1+\cdots+\alpha_ng_n:\alpha_i\in F\}$ equipped with $G-$pointwise addition and multiplication $ag_i\cdot bg_j=(ab)g_ig_j$, extended by distributivity.
\end{definition}
\begin{example}
    Let $M$ be a $\C[G]-$module.
    Then $M$ is also a $\C-$vector space, and $\rho:G\to\GL(M)$ given by $\rho(g)(m)=gm$ is a representation.
    If $\rho:G\to\GL(V)$ be a representation, the $\rho$ induces a $\C[G]-$multiplication on $V$, making $V$ a $\C[G]-$module.
    Moreover, if $N\leq M$ is a submodule, then it is $\rho(cg)-$invariant for any $cg\in\C[G]$ if and only if $N$ as a subspace of $M$ is $G-$stable.
    To be precise, we have $c g\cdot v=\rho(g)(c v)$.
    In fact, there is an isomorphic of categories from representations of $G$ and $\C[G]$-modules.
\end{example}
Basic results on modules:
\begin{proposition}
    Let $M$ be an $R-$module.
    Then $\Ann(M)$ is a (2-sided) ideal of $R$.
    Moreover, $M$ is a faithful $\quot{R}{\Ann(M)}-$module.
\end{proposition}
\begin{theorem}[First Isomorphism]
    Let $M$, $N$ be $R-$modules and let $\psi:M\to N$ be a module homomorphism.
    Then $M/\ker\psi\cong\psi(M)\leq N$.
\end{theorem}
Types of modules:
\begin{definition}
    Let $M$ be an $R-$module.
    \begin{itemize}[nl]
        \item $M$ is \textbf{irreducible} if $M\neq (0)$ and the only submodules of $M$ are $(0)$ and $M$.
    \end{itemize}
\end{definition}
Types of ideals:
\begin{definition}
    Let $R$ be a ring.
    \begin{itemize}[nl]
        \item A left ideal $I$ of $R$ is called \textbf{left quasiregular} if for all $a\in I$, $R(1+a)=R$.
        \item The \textbf{Jacobson radical} of $R$ is $J(R)=\bigcap_{\text{irred left }M}\Ann(M)$.
    \end{itemize}
\end{definition}
Types of rings:
\begin{definition}
    Let $R$ be a ring.
    \begin{itemize}[nl]
        \item $R$ \textbf{semiprimitive} if $J(R)=(0)$.
        \item $R$ is \textbf{(left) Artinian} if whenever $I_1\supseteq I_2\supseteq\cdots$ is a descending chain of left ideals, then there exists $N\in\N$ such that $I_k=I_N$ for all $k\geq N$.
    \end{itemize}
\end{definition}
Relationships:
\begin{proposition}
    The following hold:
    \begin{itemize}[nl]
        \item Every nil left ideal of $R$ is contained in $J(R)$.
        \item $\quot{R}{J(R)}$ is semiprimitive.
        \item If $R$ is Artinian, then $J(R)$ is nilpotent.
        \item $M$ is an irreducible $R-$module if and only if then $M\cong \quot{R}{I}$ as $R-$modules, where $I$ is a maximal left ideal of $R$.
    \end{itemize}
\end{proposition}
\begin{theorem}[Schur]
    Let $M$ be an irreducible $R-$module.
    Then $\End_R(M)$ is a division ring.
\end{theorem}
\begin{theorem}
    If $R$ is a ring, then the following are equivalent:
    \begin{enumerate}[nl,r]
        \item $a\in J(R)$.
        \item $Ra$ is left quasiregular
        \item $a\in\bigcap_{I\leq R\text{ maximal}}I$.
    \end{enumerate}
\end{theorem}
\end{document}

