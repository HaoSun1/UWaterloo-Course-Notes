% header -----------------------------------------------------------------------
% Template created by texnew (author: Alex Rutar); info can be found at 'https://github.com/alexrutar/texnew'.
% version (1.13)


% doctype ----------------------------------------------------------------------
\documentclass[11pt, a4paper]{memoir}
\usepackage[utf8]{inputenc}
\usepackage[left=3cm,right=3cm,top=3cm,bottom=4cm]{geometry}
\usepackage[protrusion=true,expansion=true]{microtype}


% packages ---------------------------------------------------------------------
\usepackage{amsmath,amssymb,amsfonts}
\usepackage{graphicx}
\usepackage{etoolbox}

% Set enimitem
\usepackage{enumitem}
\SetEnumitemKey{nl}{nolistsep}
\SetEnumitemKey{r}{label=(\roman*)}

% Set tikz
\usepackage{tikz, pgfplots}
\pgfplotsset{compat=1.15}
\usetikzlibrary{intersections,positioning,cd}
\usetikzlibrary{arrows,arrows.meta}
\tikzcdset{arrow style=tikz,diagrams={>=stealth}}

% Set hyperref
\usepackage[hidelinks]{hyperref}
\usepackage{xcolor}
\newcommand\myshade{85}
\colorlet{mylinkcolor}{violet}
\colorlet{mycitecolor}{orange!50!yellow}
\colorlet{myurlcolor}{green!50!blue}

\hypersetup{
  linkcolor  = mylinkcolor!\myshade!black,
  citecolor  = mycitecolor!\myshade!black,
  urlcolor   = myurlcolor!\myshade!black,
  colorlinks = true,
}


% macros -----------------------------------------------------------------------
\DeclareMathOperator{\N}{{\mathbb{N}}}
\DeclareMathOperator{\Q}{{\mathbb{Q}}}
\DeclareMathOperator{\Z}{{\mathbb{Z}}}
\DeclareMathOperator{\R}{{\mathbb{R}}}
\DeclareMathOperator{\C}{{\mathbb{C}}}
\DeclareMathOperator{\F}{{\mathbb{F}}}

% Boldface includes math
\newcommand{\mbf}[1]{{\boldmath\bfseries #1}}

% proof implications
\newcommand{\imp}[2]{($#1\Rightarrow#2$)\hspace{0.2cm}}
\newcommand{\impe}[2]{($#1\Leftrightarrow#2$)\hspace{0.2cm}}
\newcommand{\impr}{{($\Rightarrow$)\hspace{0.2cm}}}
\newcommand{\impl}{{($\Leftarrow$)\hspace{0.2cm}}}

% align macros
\newcommand{\agspace}{\ensuremath{\phantom{--}}}
\newcommand{\agvdots}{\ensuremath{\hspace{0.16cm}\vdots}}

% convenient brackets
\newcommand{\brac}[1]{\ensuremath{\left\langle #1 \right\rangle}}
\newcommand{\norm}[1]{\ensuremath{\left\lVert#1\right\rVert}}
\newcommand{\abs}[1]{\ensuremath{\left\lvert#1\right\rvert}}

% arrows
\newcommand{\lto}[0]{\ensuremath{\longrightarrow}}
\newcommand{\fto}[1]{\ensuremath{\xrightarrow{\scriptstyle{#1}}}}
\newcommand{\hto}[0]{\ensuremath{\hookrightarrow}}
\newcommand{\mapsfrom}[0]{\mathrel{\reflectbox{\ensuremath{\mapsto}}}}
 
% Divides, Not Divides
\renewcommand{\div}{\bigm|}
\newcommand{\ndiv}{%
    \mathrel{\mkern.5mu % small adjustment
        % superimpose \nmid to \big|
        \ooalign{\hidewidth$\big|$\hidewidth\cr$/$\cr}%
    }%
}

% Convenient overline
\newcommand{\ol}[1]{\ensuremath{\overline{#1}}}

% Big \cdot
\makeatletter
\newcommand*\bigcdot{\mathpalette\bigcdot@{.5}}
\newcommand*\bigcdot@[2]{\mathbin{\vcenter{\hbox{\scalebox{#2}{$\m@th#1\bullet$}}}}}
\makeatother

% Big and small Disjoint union
\makeatletter
\providecommand*{\cupdot}{%
  \mathbin{%
    \mathpalette\@cupdot{}%
  }%
}
\newcommand*{\@cupdot}[2]{%
  \ooalign{%
    $\m@th#1\cup$\cr
    \sbox0{$#1\cup$}%
    \dimen@=\ht0 %
    \sbox0{$\m@th#1\cdot$}%
    \advance\dimen@ by -\ht0 %
    \dimen@=.5\dimen@
    \hidewidth\raise\dimen@\box0\hidewidth
  }%
}

\providecommand*{\bigcupdot}{%
  \mathop{%
    \vphantom{\bigcup}%
    \mathpalette\@bigcupdot{}%
  }%
}
\newcommand*{\@bigcupdot}[2]{%
  \ooalign{%
    $\m@th#1\bigcup$\cr
    \sbox0{$#1\bigcup$}%
    \dimen@=\ht0 %
    \advance\dimen@ by -\dp0 %
    \sbox0{\scalebox{2}{$\m@th#1\cdot$}}%
    \advance\dimen@ by -\ht0 %
    \dimen@=.5\dimen@
    \hidewidth\raise\dimen@\box0\hidewidth
  }%
}
\makeatother


% macros (theorem) -------------------------------------------------------------
\usepackage[thmmarks,amsmath,hyperref]{ntheorem}
\usepackage[capitalise,nameinlink]{cleveref}

% Numbered Statements
\theoremstyle{change}
\theoremindent\parindent
\theorembodyfont{\itshape}
\theoremheaderfont{\bfseries\boldmath}
\newtheorem{theorem}{Theorem.}[section]
\newtheorem{lemma}[theorem]{Lemma.}
\newtheorem{corollary}[theorem]{Corollary.}
\newtheorem{proposition}[theorem]{Proposition.}

% Claim environment
\theoremstyle{plain}
\theorempreskip{0.2cm}
\theorempostskip{0.2cm}
\theoremheaderfont{\scshape}
\newtheorem{claim}{Claim}
\renewcommand\theclaim{\Roman{claim}}
\AtBeginEnvironment{theorem}{\setcounter{claim}{0}}

% Un-numbered Statements
\theorempreskip{0.1cm}
\theorempostskip{0.1cm}
\theoremindent0.0cm
\theoremstyle{nonumberplain}
\theorembodyfont{\upshape}
\theoremheaderfont{\bfseries\itshape}
\newtheorem{definition}{Definition.}
\theoremheaderfont{\itshape}
\newtheorem{example}{Example.}
\newtheorem{remark}{Remark.}

% Proof / solution environments
\theoremseparator{}
\theoremheaderfont{\hspace*{\parindent}\scshape}
\theoremsymbol{$//$}
\newtheorem{solution}{Sol'n}
\theoremsymbol{$\blacksquare$}
\theorempostskip{0.4cm}
\newtheorem{proof}{Proof}
\theoremsymbol{}
\newtheorem{nmproof}{Proof}

% Format references
\crefformat{equation}{(#2#1#3)}
\Crefformat{theorem}{#2Thm. #1#3}
\Crefformat{lemma}{#2Lem. #1#3}
\Crefformat{proposition}{#2Prop. #1#3}
\Crefformat{corollary}{#2Cor. #1#3}
\crefformat{theorem}{#2Theorem #1#3}
\crefformat{lemma}{#2Lemma #1#3}
\crefformat{proposition}{#2Proposition #1#3}
\crefformat{corollary}{#2Corollary #1#3}


% macros (algebra) -------------------------------------------------------------
\DeclareMathOperator{\Ann}{Ann}
\DeclareMathOperator{\Aut}{Aut}
\DeclareMathOperator{\chr}{char}
\DeclareMathOperator{\coker}{coker}
\DeclareMathOperator{\disc}{disc}
\DeclareMathOperator{\End}{End}
\DeclareMathOperator{\Fix}{Fix}
\DeclareMathOperator{\Frac}{Frac}
\DeclareMathOperator{\Gal}{Gal}
\DeclareMathOperator{\GL}{GL}
\DeclareMathOperator{\Hom}{Hom}
\DeclareMathOperator{\id}{id}
\DeclareMathOperator{\im}{im}
\DeclareMathOperator{\lcm}{lcm}
\DeclareMathOperator{\Nil}{Nil}
\DeclareMathOperator{\rank}{rank}
\DeclareMathOperator{\Res}{Res}
\DeclareMathOperator{\Spec}{Spec}
\DeclareMathOperator{\spn}{span}
\DeclareMathOperator{\Stab}{Stab}
\DeclareMathOperator{\Tor}{Tor}

% Lagrange symbol
\newcommand{\lgs}[2]{\ensuremath{\left(\frac{#1}{#2}\right)}}

% Quotient (larger in display mode)
\newcommand{\quot}[2]{\mathchoice{\left.\raisebox{0.14em}{$#1$}\middle/\raisebox{-0.14em}{$#2$}\right.}
                                 {\left.\raisebox{0.08em}{$#1$}\middle/\raisebox{-0.08em}{$#2$}\right.}
                                 {\left.\raisebox{0.03em}{$#1$}\middle/\raisebox{-0.03em}{$#2$}\right.}
                                 {\left.\raisebox{0em}{$#1$}\middle/\raisebox{0em}{$#2$}\right.}}


% macros (analysis) ------------------------------------------------------------
\DeclareMathOperator{\M}{{\mathcal{M}}}
\DeclareMathOperator{\B}{{\mathcal{B}}}
\DeclareMathOperator{\ps}{{\mathcal{P}}}
\DeclareMathOperator{\pr}{{\mathbb{P}}}
\DeclareMathOperator{\E}{{\mathbb{E}}}
\DeclareMathOperator{\supp}{supp}
\DeclareMathOperator{\sgn}{sgn}

\renewcommand{\Re}{\ensuremath{\operatorname{Re}}}
\renewcommand{\Im}{\ensuremath{\operatorname{Im}}}
\renewcommand{\d}[1]{\ensuremath{\operatorname{d}\!{#1}}}


% file-specific preamble -------------------------------------------------------
\DeclareMathOperator{\Free}{Free}


% constants --------------------------------------------------------------------
\newcommand{\subject}{Representation Theory of Finite Groups}
\newcommand{\semester}{Fall 2019}


% formatting -------------------------------------------------------------------
% Fonts
\usepackage{kpfonts}
\usepackage{dsfont}

% Adjust numbering
\numberwithin{equation}{section}
\counterwithin{figure}{section}
\counterwithout{section}{chapter}
\counterwithin*{chapter}{part}

% Footnote
\setfootins{0.5cm}{0.5cm} % footer space above
\renewcommand*{\thefootnote}{\fnsymbol{footnote}} % footnote symbol

% Table of Contents
\renewcommand{\thechapter}{\Roman{chapter}}
\renewcommand*{\cftchaptername}{Chapter } % Place 'Chapter' before roman
\setlength\cftchapternumwidth{4em} % Add space before chapter name
\cftpagenumbersoff{chapter} % Turn off page numbers for chapter
\maxtocdepth{section} % table of contents up to section

% Section / Subsection headers
\setsecnumdepth{section} % numbering up to and including "section"
\newcommand*{\shortcenter}[1]{%
    \sethangfrom{\noindent ##1}%
    \Large\boldmath\scshape\bfseries
    \centering
\parbox{5in}{\centering #1}\par}
\setsecheadstyle{\shortcenter}
\setsubsecheadstyle{\large\scshape\boldmath\bfseries\raggedright}

% Chapter Headers
\chapterstyle{verville}

% Page Headers / Footers
\copypagestyle{myruled}{ruled} % Draw formatting from existing 'ruled' style
\makeoddhead{myruled}{}{}{\scshape\subject}
\makeevenfoot{myruled}{}{\thepage}{}
\makeoddfoot{myruled}{}{\thepage}{}
\pagestyle{myruled}
\setfootins{0.5cm}{0.5cm}
\renewcommand*{\thefootnote}{\fnsymbol{footnote}}

% Titlepage
\title{\subject}
\author{Alex Rutar\thanks{\itshape arutar@uwaterloo.ca}\\ University of Waterloo}
\date{\semester\thanks{Last updated: \today}}

\begin{document}
\pagenumbering{gobble}
\hypersetup{pageanchor=false}
\maketitle
\newpage
\frontmatter
\hypersetup{pageanchor=true}
\tableofcontents*
\newpage
\mainmatter


% main document ----------------------------------------------------------------
\chapter{Introduction}
Let $G$ be a finite group of order $n$, and write $G=\{g_1,\ldots,g_n\}$.
Fix $g\in G$; then $gg_i=gg_j$ if and only if $i=j$.
Thus there exists some $\sigma_g\in S_i$ such that $gg_i=g_{\sigma_g(i)}$ for all $i\in\{1,2,\ldots,n\}$.
In particular, $\phi:G\to S_n$ by $\phi(g)=\sigma_g$ is an embedding (injective group homomorphism).
This observation is usually referred to as Cayley's Theorem.

Now let $V$ be an $n-$dimensional complex vector space.
We then denote $\GL(V)$ as the group of invertible linear operators $T:V\to V$.
Now define $\psi:S_n\to\GL_n(V)$ by $\psi(\sigma)=T_\sigma$ where if $\{b_1,\ldots,b_n\}$ is a basis for $V$ and $T_\sigma(b_i)=b_{\sigma(i)}$.
This is an injective group homomorphism, so $\psi\circ\phi:G\to\GL(V)$ is an embedding of $G$ into $\GL(V)$.
\begin{definition}
    Let $G$ be a finite group, and $V$ a finite dimensional $\C-$vector space.
    A \textbf{representation} of $G$ is a group homomorphism $\rho:G\to\GL(V)$.
    We call $\dim(V)$ the \textbf{degree} of the representation.
\end{definition}
In particular, if $V$ is $n-$dimensional, then $\GL(V)\cong\GL_n(\C)$.
\begin{example}
    \begin{enumerate}[nl]
        \item Consider $\rho:G\to\GL(\C)\cong\C^\times$ given by $\rho(g)=1$ for all $g\in G$.
            This is called the \textit{trivial representation}.
        \item Consider $\rho:S_n\to\C^\times$ given by $\rho(\sigma)=\sgn(\sigma)$, which is called the \textit{sign representation}.
        \item The representation fo $G$ afforded by Cayley's theorem is called the \textit{regular representation} of $G$.
            The next example is a good way to understand the regular rep of $G$.
        \item Consider $G$, $X=\{x_1,\ldots,x_n\}$, and $V=\Free(X)$.
            Suppose $G$ acts on $X$.
            Then $\rho:G\to\GL(V)$ given by $\rho(g)(x_i)=gx_i$.
            In particular, if we take $X=G$, then this is the regular representation of $G$
        \item Consider the $4-$gon, with vertices labelled $a,b,c,d$.
            Take $X=\{a,b,c,d\}$ and the regular representation $\rho:D_4\to\GL(V)$.
            This action has a geometric notion.
        \item Let $C_n$ be a cyclic group of order $n$; let us define some $\rho:C_n\to\GL(V)$.
            Say $\rho(x) = T$ where $t\in\GL(V)$; then this is a representation if and only if $T^n=I$.
    \end{enumerate}
\end{example}
\begin{definition}
    We say that two representations $\rho:G\to\GL(V)$ and $\tau:G\to\GL(W)$ are \textbf{isomorphic} if there exists an isomorphism $T:V\to W$ such that for all $g\in G$,
    \begin{equation*}
        T\circ \rho(g) = \tau(g)\circ T
    \end{equation*}
\end{definition}
Suppose $\rho:G\to\GL(V)$ and $T:V\to W$ is an isomorphism.
Then we can define $\tau:G\to\GL(W)$ by $\tau(G)=T\circ\rho(g)\circ T^{-1}$; this $\rho\cong\tau$.
In other words, the representation is unique up to isomorphism under change of basis.
\begin{example}
    Consider $G=\{g_1,\ldots,g_n\}=\{h_1,\ldots,h_n\}$, and fix $g\in G$.
    Let $gg_i=g_{\alpha(i)}$ and $gh_i=h_{\beta(i)}$ where $\alpha,\beta\in S_n$.
    Fix an $n-$dimensional vector space $V$ with basis $\{b_1,\ldots,b_n\}$.
    Then two regular representations are given by
    \begin{align*}
        \rho_1:G\to\GL(V), & \rho(g)(b_i)=b_{\alpha(i)}\\
        \rho_2:G\to\GL(V), & \rho(g)(b_i)=b_{\beta(i)}
    \end{align*}
    Let $\gamma\in S_n$ be such that $h_{\gamma(i)}=g_i$, and define $T:V\to V$ by $T(v_i)=b_{\gamma(i)}$.
    Then
    \begin{equation*}
        gg_i = g_{\alpha(i)}=gh_{\gamma(i)} = h_{\beta\gamma(i)} = g_{\gamma^{-1}\beta\gamma(i)}
    \end{equation*}
    so that $\alpha=\gamma^{-1}\beta\gamma$.
    Thus for each $b_i$,
    \begin{align*}
        T\circ\rho_1(g)\circ T^{-1}(b_i) &= T\circ\rho_1(g)(b_{\gamma^{-1}(i)})\\
                                         &= T(b_{\alpha\gamma^{-1}(i)}) b_{\gamma\alpha\gamma^{-1}(i)}\\
                                         &= b_{\beta(i)}=\rho_2(g)(b_i)\\
    \end{align*}
    so that $T\circ\rho_1(g)\circ T^{-1}=\rho_2(g)$.
\end{example}
Note: conjugate elements have the same cycle type.
\subsection{Subrepresentations}
What should a subrepresentation of $\rho:G\to\GL(V)$ mean?

We would like a subspace $W\leq V$ such that $\tau:G\to\GL(W)$ is a representation given by $\tau(g)(w)=\rho(g)(w)$ for all $w\in W$.
Moreover, to make this well-defined, we need $W$ to b4 $\rho(g)-$invariant for every $g\in G$ ($\rho(g)(W)\subseteq W$).

Suppose $T:V\to V$ is a linear operator, and $W\leq V$ is a $T-$invariant subspace; i.e. $T(W)\subseteq W$.
In particular, the restriction operator $T_W:W\to W$ is well-defined.
\begin{definition}
    Let $\rho:G\to\GL(V)$ be a representation.
    A subspqce $W\subseteq V$ is said to be \mbf{$G-$stable} if $W$ is $\rho(g)-$invariant for all $g\in G$.
    A \textbf{subrepresentation} of $\rho$ is a representation $\rho_W:G\to\GL(W)$ where for all $g\in G$ and $w\in W$, $\rho_W(g)(w)=\rho(g)(w)$ where $W$ is a $G-$stable subspace of $V$.
\end{definition}
\begin{example}
    Suppose $\rho:G\to\GL(V)$ be the regular representation.
    Take $W=\spn\{\sum_{g\in G}v_g\}$, which is clearly $G-$stable, and $\rho_W:G\to\GL(W)$ is isomorphic to the trivial representation.

    Similarly, let $\rho:S_n\to\GL(V)$ be the regular representation, $W=\spn\{\sum_{\sigma\in S_n}\sgn(\sigma)v_\sigma\}$; this is isomorphic to the sign representation.
\end{example}
\begin{theorem}
    Let $\rho:G\to\GL(V)$ be a representation, $W\leq V$ $G-$stable.
    Then there exists a $G-$stable subspace $W'$ such that $V=W\oplus W'$.
\end{theorem}
\begin{proof}
    Take any inner product $\langle x,y\rangle$ on $V$.
    Then for any $x,y\in V$, define
    \begin{equation*}
        \langle x,y\rangle^*=\sum_{g\in G}\langle \rho(g)(x),\rho(g)(y)\rangle
    \end{equation*}
    This is also an inner product.
    Let $x,y\in V$ and let $h\in G$.
    Then
    \begin{align*}
        \langle\rho(h)(x),\rho(h)(y)\rangle^* &= \sum_{g\in G}\langle\rho(g)\rho(h)(x),\rho(g)\rho(h)(y)\rangle\\
                                              &= \sum_{g\in G}\langle\rho(gh)(x),\rho(gh)(y)\rangle\\
                                              &= \sum_{g\in G}\langle\rho(g)(x),\rho(g)(y)\rangle
    \end{align*}
    Thus every $\rho(h)$ is unitary with respect to $\langle\cdot,\cdot\rangle^*$.
    Let $W\leq V$ be $G-$stable, and take $W'=W^\perp$ with respect to $\langle\cdot,\cdot\rangle^*$.
    Then $V=W\oplus W'$.
    Let's see that $W^\perp$ is $G-$stable.
    Let $x\in W^\perp$, $w\in W$, and $g\in G$, so that
    \begin{align*}
        \langle\rho(g)(x),w\rangle^*&=\langle x,\rho(g)^*(w)^*\rangle=\langle x,\rho(g)^{-1}(w)\rangle^*\\
                                    &=\langle x,\underbrace{\rho(g^{-1})(w)}_{\in W}\rangle^*\\
                                    &= 0
    \end{align*}
    and $\rho(g)(W^\perp)\subseteq W^\perp$ as required.
\end{proof}
\begin{definition}
    Let $\rho:G\to\GL(V)$ be a representation, and $V=W_1\oplus W_2\oplus\cdots\oplus W_k$ where each $W_i$ is $G-$stable.
    For each $i$, let $\rho_i=\rho_{w_i}$.
    For each $v=\sum w_i\in V$, we have $\rho(g)(v)=\sum\rho(g)(w_i)=\rho_i(g)(w_i)$.
    In this case, we write
    \begin{equation*}
        \rho=\rho_1\oplus\rho_2\oplus\cdots\oplus\rho_k
    \end{equation*}
    and call $\rho$ a direct sum of the $\rho_i$'s.
\end{definition}
The previous definition is written as an internal direct sum of $V$.
Externally, given vector spaces $W_1,\ldots,W_k$ and representations $\rho_i:G\to\GL(W_i)$, we can define
\begin{equation*}
    (\rho_1\oplus\cdots\oplus\rho_k):G\to\GL(W_1\oplus\cdots\oplus W_k)
\end{equation*}
by $(\rho_1\oplus\cdots\oplus\rho_k)(g)(w_1,\ldots,w_k)=(\rho_1(g)(w_1),\ldots,\rho_k(g)(w_k))$.
If $\rho_i:G\to\GL(W_i)$ is a subrepresentation fo $\rho:G\to\GL(V)$, we often say ``$W_i$ is a subrepresentation of $V$''.
\begin{definition}
    Let $\rho:G\to\GL(V)$ be a representation.
    We say $\rho$ is \textbf{irreducible} if $V\neq\{0\}$ and the only $G-$stable subspaces of $V$ are $\{0\}$ and $V$.
\end{definition}
Clearly,
\begin{theorem}
    Every representation $\rho:G\to\GL(V)$ can be written as a direct sum of irreducible sub-representations.
\end{theorem}
\begin{example}
    Let $\rho:S_3\to\GL(\C^3)$ be the permutation representation with respect to the standard basis $\{e_1,e_2,e_3\}$.
    Consider $W_1=\spn\{e_1+e_2+e_3\}$ and $W_2=\spn\{e_1-e_2,e_2-e_3\}$.
    Is $W_2$ irreducible?
    
    More generally, if $V=W_1\oplus\cdots\oplus W_k$ and $\dim W_i=1$ and $\deg(\rho_i)=1$,
    \begin{equation*}
        \rho(gh)(\sum w_i) = \sum\rho_i(gh)(w_i) = \sum\rho_i(g)\rho_i(h)(w_i)=\sum\rho_i(h)\rho_i(g)(w_i)
    \end{equation*}
    so that $\rho(gh)=\rho(hg)$.
    In the our example, this does not happen, since $\rho(g)\neq I$ when $g\neq 1$ and $S_3$ is not abelian.
\end{example}
\begin{example}
    Let $\rho:S_3\to\GL(V)$ be the regular representation.
    Let $W_1=\spn\{\sum_{\sigma\in S_3}v_\sigma\}$ and $W_2=\spn\{\sum_{\sigma\in S_3}\sgn(\sigma)v_\sigma\}$, and
    \begin{equation*}
        W_3 = \Set{\sum\alpha_\sigma v_\sigma | \begin{array}\alpha_\epsilon+\alpha_{(123)}+\alpha_{(1,3,2)}&=0\\\alpha_{(12)}+\alpha_{(13)}+\alpha_{(23)} &= 0\end{array}}
    \end{equation*}
    Now let's focus on $W_3$.
    A basis for $W_3$ is given by
    \begin{align*}
        e_1&=v_\epsilon-v_{(123)} & e_2 &= v_\epsilon-v_{(123)}\\
        e_3 &= v_{(12)}-v_{(13)} & e_4 &= v_{(12)}-v_{(23)}
    \end{align*}
    Recall that $S_3=\langle(12),(123)\rangle$; suffices to show stability with respect to generators.
    \begin{align*}
        \rho(12) &: e_1\mapsto e_4,e_2\mapsto e_3, e_3\mapsto e_2,e_4\mapsto e_1\\
        \rho(123) & : e_1\mapsto e_2-e_1, e_2\mapsto -e_1,e_3\mapsto e_4-e_3, e_4\mapsto -e_3
    \end{align*}
    Let $U_1=\spn\{e_1-e_4,e_2+e_3-e_1\}$
\end{example}
\section{Tensor Products}
Let $\rho:G\to\GL(V)$ and $\tau:G\to\GL(W)$ be representations.
We define the representation $\rho\otimes\tau:G\to\GL(V\otimes W)$
\begin{equation*}
    (\rho\otimes\tau)(g)(v\otimes w)=\rho(g)(v)\otimes\tau(g)(w)
\end{equation*}
\section{Character Theory}
We define the character of $\rho$ by $\rho:G\to\C$ as $\chi(G)=\Tr(\rho(g))$.
\begin{remark}
    If we choose a basis $\beta$ for $V$, then define $A(g)=[\rho(g)]_\beta$ and $\chi(G)$ is given by the sum of the diagonal entries of $A(g)$.
    Furthermore, if $A,B\in M_n(\C)$, then $\Tr(AB)=\Tr(BA)$.
\end{remark}
The remark implies a number of facts:
\begin{enumerate}[nl,r]
    \item $\rho\cong\tau$, then $\Tr(\rho(g))=\Tr(\tau(g))$.
    \item $\Tr(T)$ is the sum of eigenvalues of $T$
    \item $\chi(1)=\dim(V)$.
\end{enumerate}
\begin{proposition}
    For every $g\in G$ the eigenvalues of $\rho(g)$ have modulus 1.
    In particular, $\chi(g^{-1})=\overline{\chi(g)}$.
\end{proposition}
\begin{proof}
    Set $n=|G|$; then $\rho(g)^n=\rho(g^n)=I$ so that $\lambda^n-1=0$ for any eigenvalue $\lambda$, so $|\lambda|=1$.
    Furthermore,
    \begin{equation*}
        \overline{\chi(g)}=\overline{\sum\lambda_i}=\sum\overline{\lambda_i}=\sum\lambda_i^{-1}=\chi(g^{-1})
    \end{equation*}
    proving the second component.
\end{proof}
\begin{proposition}
    Let $\rho:G\to\GL(V)$ and $\tau:G\to\GL(W)$.
    Then $\chi_{\rho\oplus\tau}=\chi_\rho+\chi_\tau$ and $\chi_{\rho\otimes\tau}=\chi_\rho\cdot\chi_\tau$.
\end{proposition}
\begin{proof}
    Let $\beta_1=\{v_1,\ldots,v_n\}$ be a basis for $V$ and $\beta_2=\{w_1,\ldots,w_m\}$ a basis for $W$.

    Then a basis for $V\oplus W$ is given by $\beta=\{(v_1,0),\ldots,(v_n,0),(0,w_1),\ldots,(0,w_m)\}$.
    In particular,
    \begin{equation*}
        [(\rho\oplus\tau)(g)]_\beta=
        \begin{pmatrix}
            [\rho(g)]_{\beta_1} & \\
                                & [\tau(g)]_{\beta_2}
        \end{pmatrix}
    \end{equation*}
    and the trace result follows.

    A basis for $V\otimes W$ is given by $\gamma=\{v_i\otimes w_j:1\leq i\leq n,1\leq j\leq m\}$ in lexicographic order.
    Fix $g\in G$, and set $A=[\rho(g)]_{\beta_1}$, $B=[\rho(g)]_{\beta_2}$.
    Fix $v_i\otimes w_j\in\gamma$.
    Then
    \begin{align*}
        (\rho\otimes\tau)(g)(v_i\otimes w_j)&=\rho(g)(v_i)\otimes\tau(g)(w_j)\\
                                            &= (a_{1i}v_1+\cdots+a_{ni}v_n)\otimes(b_{1j}w_1+\cdots+b_{mj}v_m)\\
                                            &= \cdots+a_{ii}b_{jj}\cdot(v_i\otimes w_j)+\cdots\\
                                            &= \Tr([\rho\otimes\tau)(g)]_\delta)=\sum_{i,j}a_{ii}b_{jj}=\Tr(A)\Tr()=\chi_\rho(g)\cdot\chi_\tau(g)
    \end{align*}
\end{proof}
\begin{example}
    Suppose $\rho:S_n\to\GL(\C^n)$ is the permutation representation with respect to $\{e_1,\ldots,e_n\}$.
    Then $\chi(\sigma)=|\{e_i:\rho(\sigma)(e_i)=e_i\}|=|\Fix(\sigma)|$, which is the number of indices $i$ fixed by $\sigma$.
    Since $S_n$ acts transitively on $\{1,\ldots,n\}$, there is exactly 1 orbit, so by Burnside's lemma,
    \begin{equation*}
        n!=|S_n| = \sum_{\sigma\in S_n}\chi(\sigma)
    \end{equation*}
\end{example}
\begin{example}
    Let $\rho:G\to\GL(V)$ be the regular representation.
    Note that if $g\neq 1$, then for all $h\in G$, $gh\neq h$.
    In particular, this means that $\chi(g)=0$ if $g\neq 1$, and $\chi(1)=|G|$ (the dimension of $V$).
\end{example}
\begin{example}
    Let $\rho:S_3\to\GL(V)$ be the regular representation.
    Recall that $V=W_1\oplus W_2\oplus U_1\oplus U_2$ where $W_1$ is the trivial representation, $W_2$ is the sign representation, and $U_1,U_2$ are isomorphic.
    Let $S_3=\langle (12),(123)\rangle$; then we have
    \begin{equation*}
        \begin{array}{c|cc}
            x_1 & 1 & 1\\
            \hline
            x_2 & -1 & 1\\
            x_3 & a & b\\
            x_4 & a & b
        \end{array}
    \end{equation*}
    In particular, $\chi(12)=1-1+2a=0$ and $\chi(123)=1+1+2b=0$, so $b=-1$.
\end{example}
\begin{example}
    Let $\rho:G\to\GL(V)$ be a representation.
    In particular, $\rho(ghg^{-1})=\rho(g)\rho(h)\rho(g)$ so that $\Tr\rho(ghg^{-1})=\Tr\rho(h)$ so $\chi(ghg^{-1})=\ghi(h)$; in other words, that characters are constant on conjugacy classes.
\end{example}
\begin{lemma}[Schur]
    Let $\rho:G\to\GL(V)$ and $\tau:G\to\GL(W)$ be irreducible representations, and suppose $T:V\to W$ is linear such that for all $g\in G$, $\tau(g)\circ T=T\circ\rho(g)$.
    Then either $T=0$ or $T$ is an isomorphism and $\rho\cong\tau$.
    Moreover, if $V=W$ and $\rho=\tau$, then $T$ is a scalar multiple of the identity.
\end{lemma}
\begin{proof}
    Assume $T\neq 0$.

    Let's first see that $T$ is injective, and let $v\in\ker(T)$.
    Then for any $g\in G$, $T(\rho(g)(v))=\tau(g)(T(v))=0$, so $\rho(g)(v)\in\ker(T)$.
    Thus $\ker(T)$ is $G-$stable (with respect to $\rho$).
    Since $\rho$ is irreducible and $T\neq 0$, $\ker(T)=\{0\}$.

    We also have that $T$ is surjective.
    Let $v\in\Im(T)$ and say $v=T(X)$ with $x\in V$.
    Then for $g\in G$, $\tau(g)(v)=\tau(g)(T(x))=T(\rho(g)(x))\in\Im(T)$ so $\Im(T)$ is $G-$stable, and again by irreducibility of $\tau$, $\Im(T)=W$.
    Thus $T$ is an isomorphism.

    Now let $\lambda\in\C$ be an eigenvalue of $T$ and consider $T'=T-\lambda I$.
    Now, note that for $g\in G$, $\rho(g)T'=T'\rho(g)$, but $T'$ has non-trivial kernel, so in fact $T'=0$.
\end{proof}
\begin{corollary}
    Let $\rho:G\to\GL(V)$ and $\tau:G\to\GL(W)$ be irreducible, and $T:V\to W$ linear.
    Consider
    \begin{equation*}
        T'=\frac{1}{|G|}=\sum_{g\in G}\tau(g)^{-1} T\rho(g)
    \end{equation*}
    Then
    \begin{enumerate}[nl,r]
        \item If $T'\neq 0$, then $\rho\cong\tau$ via $T'$.
        \item If $V=W$, $\rho=\tau$, then $T'=\Tr(T)/\dim(V)\cdot I$.
\end{corollary}
\begin{proof}
    Clearly $T':V\to W$ is linear, and for any $h\in G$,
    \begin{align*}
        \tau(h)T'&=\tau(h)\frac{1}{|H|}\sum_{g\in G}\tau(g^{-1})T\rho(g)\\
                 &= \frac{1}{|G|}\sum_{g\in G}\tau(hg^{-1})T\rho(g)\\
                 &=\frac{1}{|G|}\sum_{g\in G}\tau(g^{-1})T(\rho(gh))\\
                 &= \frac{1}{|G|}\sum_{g\in G}\tau(g^{-1})T\rho(g)\rho(h)\\
                 &= T'\rho(h)
    \end{align*}
    If $V=W$ and $\rho=T$, then $\Tr(T')=\frac{1}{|G|}\Tr(T)\cdot|G|=\Tr(T)=\alpha\dim(V)$, so $\alpha=\Tr(T)/\dim(V)$.
\end{proof}
Let $\rho:G\to\GL(V)$ and $\tau:G\to\GL(W)$ be irreducible representations, and $T:V\to W$ linear.
Let $\beta$ be a basis for $V$ and $\gamma$ a basis for $W$.
Then for $g\in G$, let $[\rho(g)]_\beta=(a_{ij}(g))$, $[\tau(g)]_{\gamma}=(b_{kl}(g))$, $[T]_\beta^\gamma=(X_{ki})$, and $[T']_{\beta}^\gamma=(x_{ki}')$.

By matrix multiplication, $x'_{ki}=\frac{1}{|G|}\sum_g\sum_{j,l}b_{kl}(g^{-1})x_{lj}a_{ji}$.
If $\rho\ncong\tau$, then $T'=0$, so by viewing the RHS as a polynomial in the $x_{ij}$, we have
\begin{equation*}
    \frac{1}{|G|}\sum_g b_{kl}(g^{-1})a_{ji}(g)=0
\end{equation*}
But now it $\rho=\tau$, then $T'=\lambda I$ where $\lambda=\Tr(T)/\dim(B)$ so that
\begin{equation*}
    \frac{1}{|G|}\sum_g\sum_{j,l}a_{kl}(g^{-1})x_{lj}a_{ji}(g) = \lambda\delta_{ki}=\frac{1}{\dim(V)}\sum_{j,l}\delta_{ki}\delta_{jl}x_{lj}
\end{equation*}
Then by equating coefficients of $x_{lj}$, we have
\begin{equation*}
    \frac{1}{|G|}\sum_g a_{kl}(g^{-1})a_{ji}(g)=\frac{1}{\dim(V)}\delta_{ki}\delta_{jl}
\end{equation*}
\begin{remark}
    If $G$ is a finite group, the consider the vector space of all functions $\phi:G\to\C$.
    For any $\phi,\psi$ in this vector space, $\langle\phi,\psi\rangle=\frac{1}{|G|}\sum_g\phi(g)\overline{\psi(g)}$ defines an inner product.
    Then if $\chi_1$, $\chi_2$ are characters of $G$, then
    \begin{equation*}
        \langle\chi_1,\chi_2\rangle=\frac{1}{|G|}\sum_g\chi_1(g)\chi_2(g^{-1})
    \end{equation*}
    We thus have:
\end{remark}
\begin{theorem}
    If $\chi$ is a character of an irreducible representation, then $\langle{\chi,\chi}=1$, and if $\chi_1$ and $\chi_2$ correspond to non-isomorphic representations, then $\langle\chi_1,\chi_2\rangle=0$.
\end{theorem}
\begin{proof}
    Say $[\rho(g)]_\beta=(a_{ij}(g))$ where $\rho$ is an irreducible representation with character $\chi$.
    Then
    \begin{align*}
        \langle\chi,\chi\rangle &= \frac{1}{|G|}\sum_g\chi(g)\chi(g^{-1}) = \frac{1}{|G|}\sum_g\chi(g^{-1})\chi(g)\\
                                &= \frac{1}{|G|}\sum_g\sum_{i,j}a_{ii}(g^{-1})a_{jj}(g) = \sum_{i,j}\left(\frac{1}{|G|}\sum_g a_{ii}(g^{-1})a_{jj}(g))\\
                                &= \sum_{i,j}\left(\frac{1}{|G|}\sum_g a_{ii}(g^{-1})a_{ii}(g)\\
                                &= \sum_i\frac{1}{\dim(V)} = 1
    \end{align*}
    To see the second part,
    \begin{align*}
        \langle\chi_1,\chi_2\rangle = \frac{1}{|G|}\sum_g\chi_1(g)\chi_2(g^{-1}) = \frac{1}{|G|}\sum_g\sum_{ij} a_{ii}(g)a_{jj}(g^{-1}) = \sum_{i,j}0=0
    \end{align*}
\end{proof}
If $\chi$ is a character corresponding to an irreducible representation, we say $\chi$ is irreducible.
If $\rho$ and $\tau$ are isomorphic representations, we say $\chi_\rho$ and $\chi_\tau$ are isomorphic (in fact $\chi_\rho=\chi_\tau$).
\begin{corollary}
    Let $\rho:G\to\GL(V)$ be a representation with character $\chi$.
    Say $V=W_1\oplus\cdots\oplus W_k$ is an irreducible decomposition of $V$.
    If $\tau:G\to\GL(W)$ is an irreducible representations with character $\phi$, then the number of $W_i$ isomorphic to $W$ (i.e. $\rho_i\cong\tau$) is $\langle\chi,\phi\rangle$.
\end{corollary}
\begin{proof}
    Write $\chi=n_1\chi_1+\cdots+n_l\chi_l$, where the $\chi_i$ are pairwise non-isomorphic.
    Then $\langle\chi,\chi_i\rangle=n_i$.
\end{proof}



\end{document}

