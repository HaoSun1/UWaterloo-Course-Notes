% header -----------------------------------------------------------------------
% Template created by texnew (author: Alex Rutar); info can be found at 'https://github.com/alexrutar/texnew'.
% version (1.13)


% doctype ----------------------------------------------------------------------
\documentclass[11pt, a4paper]{memoir}
\usepackage[utf8]{inputenc}
\usepackage[left=3cm,right=3cm,top=3cm,bottom=4cm]{geometry}
\usepackage[protrusion=true,expansion=true]{microtype}


% packages ---------------------------------------------------------------------
\usepackage{amsmath,amssymb,amsfonts}
\usepackage{graphicx}
\usepackage{etoolbox}
\usepackage{braket}

% Set enimitem
\usepackage{enumitem}
\SetEnumitemKey{nl}{nolistsep}
\SetEnumitemKey{r}{label=(\roman*)}

% Set tikz
\usepackage{tikz, pgfplots}
\pgfplotsset{compat=1.15}
\usetikzlibrary{intersections,positioning,cd}
\usetikzlibrary{arrows,arrows.meta}
\tikzcdset{arrow style=tikz,diagrams={>=stealth}}

% Set hyperref
\usepackage[hidelinks]{hyperref}
\usepackage{xcolor}
\newcommand\myshade{85}
\colorlet{mylinkcolor}{violet}
\colorlet{mycitecolor}{orange!50!yellow}
\colorlet{myurlcolor}{green!50!blue}

\hypersetup{
  linkcolor  = mylinkcolor!\myshade!black,
  citecolor  = mycitecolor!\myshade!black,
  urlcolor   = myurlcolor!\myshade!black,
  colorlinks = true,
}


% macros -----------------------------------------------------------------------
\DeclareMathOperator{\N}{{\mathbb{N}}}
\DeclareMathOperator{\Q}{{\mathbb{Q}}}
\DeclareMathOperator{\Z}{{\mathbb{Z}}}
\DeclareMathOperator{\R}{{\mathbb{R}}}
\DeclareMathOperator{\C}{{\mathbb{C}}}
\DeclareMathOperator{\F}{{\mathbb{F}}}

% Boldface includes math
\newcommand{\mbf}[1]{{\boldmath\bfseries #1}}

% proof implications
\newcommand{\imp}[2]{($#1\Rightarrow#2$)\hspace{0.2cm}}
\newcommand{\impe}[2]{($#1\Leftrightarrow#2$)\hspace{0.2cm}}
\newcommand{\impr}{{($\Rightarrow$)\hspace{0.2cm}}}
\newcommand{\impl}{{($\Leftarrow$)\hspace{0.2cm}}}

% align macros
\newcommand{\agspace}{\ensuremath{\phantom{--}}}
\newcommand{\agvdots}{\ensuremath{\hspace{0.16cm}\vdots}}

% convenient brackets
\newcommand{\brac}[1]{\ensuremath{\left\langle #1 \right\rangle}}
\newcommand{\norm}[1]{\ensuremath{\left\lVert#1\right\rVert}}
\newcommand{\abs}[1]{\ensuremath{\left\lvert#1\right\rvert}}

% arrows
\newcommand{\lto}[0]{\ensuremath{\longrightarrow}}
\newcommand{\fto}[1]{\ensuremath{\xrightarrow{\scriptstyle{#1}}}}
\newcommand{\hto}[0]{\ensuremath{\hookrightarrow}}
\newcommand{\mapsfrom}[0]{\mathrel{\reflectbox{\ensuremath{\mapsto}}}}
 
% Divides, Not Divides
\renewcommand{\div}{\bigm|}
\newcommand{\ndiv}{%
    \mathrel{\mkern.5mu % small adjustment
        % superimpose \nmid to \big|
        \ooalign{\hidewidth$\big|$\hidewidth\cr$/$\cr}%
    }%
}

% Convenient overline
\newcommand{\ol}[1]{\ensuremath{\overline{#1}}}

% Big \cdot
\makeatletter
\newcommand*\bigcdot{\mathpalette\bigcdot@{.5}}
\newcommand*\bigcdot@[2]{\mathbin{\vcenter{\hbox{\scalebox{#2}{$\m@th#1\bullet$}}}}}
\makeatother

% Big and small Disjoint union
\makeatletter
\providecommand*{\cupdot}{%
  \mathbin{%
    \mathpalette\@cupdot{}%
  }%
}
\newcommand*{\@cupdot}[2]{%
  \ooalign{%
    $\m@th#1\cup$\cr
    \sbox0{$#1\cup$}%
    \dimen@=\ht0 %
    \sbox0{$\m@th#1\cdot$}%
    \advance\dimen@ by -\ht0 %
    \dimen@=.5\dimen@
    \hidewidth\raise\dimen@\box0\hidewidth
  }%
}

\providecommand*{\bigcupdot}{%
  \mathop{%
    \vphantom{\bigcup}%
    \mathpalette\@bigcupdot{}%
  }%
}
\newcommand*{\@bigcupdot}[2]{%
  \ooalign{%
    $\m@th#1\bigcup$\cr
    \sbox0{$#1\bigcup$}%
    \dimen@=\ht0 %
    \advance\dimen@ by -\dp0 %
    \sbox0{\scalebox{2}{$\m@th#1\cdot$}}%
    \advance\dimen@ by -\ht0 %
    \dimen@=.5\dimen@
    \hidewidth\raise\dimen@\box0\hidewidth
  }%
}
\makeatother


% macros (theorem) -------------------------------------------------------------
\usepackage[thmmarks,amsmath,hyperref]{ntheorem}
\usepackage[capitalise,nameinlink]{cleveref}

% Numbered Statements
\theoremstyle{change}
\theoremindent\parindent
\theorembodyfont{\itshape}
\theoremheaderfont{\bfseries\boldmath}
\newtheorem{theorem}{Theorem.}[section]
\newtheorem{lemma}[theorem]{Lemma.}
\newtheorem{corollary}[theorem]{Corollary.}
\newtheorem{proposition}[theorem]{Proposition.}

% Claim environment
\theoremstyle{plain}
\theorempreskip{0.2cm}
\theorempostskip{0.2cm}
\theoremheaderfont{\scshape}
\newtheorem{claim}{Claim}
\renewcommand\theclaim{\Roman{claim}}
\AtBeginEnvironment{theorem}{\setcounter{claim}{0}}

% Un-numbered Statements
\theorempreskip{0.1cm}
\theorempostskip{0.1cm}
\theoremindent0.0cm
\theoremstyle{nonumberplain}
\theorembodyfont{\upshape}
\theoremheaderfont{\bfseries\itshape}
\newtheorem{definition}{Definition.}
\theoremheaderfont{\itshape}
\newtheorem{example}{Example.}
\newtheorem{remark}{Remark.}

% Proof / solution environments
\theoremseparator{}
\theoremheaderfont{\hspace*{\parindent}\scshape}
\theoremsymbol{$//$}
\newtheorem{solution}{Sol'n}
\theoremsymbol{$\blacksquare$}
\theorempostskip{0.4cm}
\newtheorem{proof}{Proof}
\theoremsymbol{}
\newtheorem{nmproof}{Proof}

% Format references
\crefformat{equation}{(#2#1#3)}
\Crefformat{theorem}{#2Thm. #1#3}
\Crefformat{lemma}{#2Lem. #1#3}
\Crefformat{proposition}{#2Prop. #1#3}
\Crefformat{corollary}{#2Cor. #1#3}
\crefformat{theorem}{#2Theorem #1#3}
\crefformat{lemma}{#2Lemma #1#3}
\crefformat{proposition}{#2Proposition #1#3}
\crefformat{corollary}{#2Corollary #1#3}


% macros (algebra) -------------------------------------------------------------
\DeclareMathOperator{\Ann}{Ann}
\DeclareMathOperator{\Aut}{Aut}
\DeclareMathOperator{\chr}{char}
\DeclareMathOperator{\coker}{coker}
\DeclareMathOperator{\disc}{disc}
\DeclareMathOperator{\End}{End}
\DeclareMathOperator{\Fix}{Fix}
\DeclareMathOperator{\Frac}{Frac}
\DeclareMathOperator{\Gal}{Gal}
\DeclareMathOperator{\GL}{GL}
\DeclareMathOperator{\Hom}{Hom}
\DeclareMathOperator{\id}{id}
\DeclareMathOperator{\im}{im}
\DeclareMathOperator{\lcm}{lcm}
\DeclareMathOperator{\Nil}{Nil}
\DeclareMathOperator{\rank}{rank}
\DeclareMathOperator{\Res}{Res}
\DeclareMathOperator{\Spec}{Spec}
\DeclareMathOperator{\spn}{span}
\DeclareMathOperator{\Stab}{Stab}
\DeclareMathOperator{\Tor}{Tor}

% Lagrange symbol
\newcommand{\lgs}[2]{\ensuremath{\left(\frac{#1}{#2}\right)}}

% Quotient (larger in display mode)
\newcommand{\quot}[2]{\mathchoice{\left.\raisebox{0.14em}{$#1$}\middle/\raisebox{-0.14em}{$#2$}\right.}
                                 {\left.\raisebox{0.08em}{$#1$}\middle/\raisebox{-0.08em}{$#2$}\right.}
                                 {\left.\raisebox{0.03em}{$#1$}\middle/\raisebox{-0.03em}{$#2$}\right.}
                                 {\left.\raisebox{0em}{$#1$}\middle/\raisebox{0em}{$#2$}\right.}}


% macros (analysis) ------------------------------------------------------------
\DeclareMathOperator{\M}{{\mathcal{M}}}
\DeclareMathOperator{\B}{{\mathcal{B}}}
\DeclareMathOperator{\ps}{{\mathcal{P}}}
\DeclareMathOperator{\pr}{{\mathbb{P}}}
\DeclareMathOperator{\E}{{\mathbb{E}}}
\DeclareMathOperator{\supp}{supp}
\DeclareMathOperator{\sgn}{sgn}

\renewcommand{\Re}{\ensuremath{\operatorname{Re}}}
\renewcommand{\Im}{\ensuremath{\operatorname{Im}}}
\renewcommand{\d}[1]{\ensuremath{\operatorname{d}\!{#1}}}


% file-specific preamble -------------------------------------------------------
\DeclareMathOperator*{\esssup}{ess\,sup}
\DeclareMathOperator{\Lip}{Lip}


% constants --------------------------------------------------------------------
\newcommand{\subject}{Functional Analysis}
\newcommand{\semester}{Fall 2019}


% formatting -------------------------------------------------------------------
% Fonts
\usepackage{kpfonts}
\usepackage{dsfont}

% Adjust numbering
\numberwithin{equation}{section}
\counterwithin{figure}{section}
\counterwithout{section}{chapter}
\counterwithin*{chapter}{part}

% Footnote
\setfootins{0.5cm}{0.5cm} % footer space above
\renewcommand*{\thefootnote}{\fnsymbol{footnote}} % footnote symbol

% Table of Contents
\renewcommand{\thechapter}{\Roman{chapter}}
\renewcommand*{\cftchaptername}{Chapter } % Place 'Chapter' before roman
\setlength\cftchapternumwidth{4em} % Add space before chapter name
\cftpagenumbersoff{chapter} % Turn off page numbers for chapter
\maxtocdepth{section} % table of contents up to section

% Section / Subsection headers
\setsecnumdepth{section} % numbering up to and including "section"
\newcommand*{\shortcenter}[1]{%
    \sethangfrom{\noindent ##1}%
    \Large\boldmath\scshape\bfseries
    \centering
\parbox{5in}{\centering #1}\par}
\setsecheadstyle{\shortcenter}
\setsubsecheadstyle{\large\scshape\boldmath\bfseries\raggedright}

% Chapter Headers
\chapterstyle{verville}

% Page Headers / Footers
\copypagestyle{myruled}{ruled} % Draw formatting from existing 'ruled' style
\makeoddhead{myruled}{}{}{\scshape\subject}
\makeevenfoot{myruled}{}{\thepage}{}
\makeoddfoot{myruled}{}{\thepage}{}
\pagestyle{myruled}
\setfootins{0.5cm}{0.5cm}
\renewcommand*{\thefootnote}{\fnsymbol{footnote}}

% Titlepage
\title{\subject}
\author{Alex Rutar\thanks{\itshape arutar@uwaterloo.ca}\\ University of Waterloo}
\date{\semester\thanks{Last updated: \today}}

\begin{document}
\pagenumbering{gobble}
\hypersetup{pageanchor=false}
\maketitle
\newpage
\frontmatter
\hypersetup{pageanchor=true}
\tableofcontents*
\newpage
\mainmatter


% main document ----------------------------------------------------------------
\chapter{Fundamentals of Functional Analysis}
\section{Basic Elements of Functional Analysis}
Throughout, we denote by $\F$ either the field $\R$ or the field $\C$.
\subsection{Banach Spaces}
\begin{definition}
    Let $X$ be a vector space over $\F$.
    A \textbf{norm} is a functional $\norm{\cdot}:X\to\R$ such that it is
    \begin{itemize}[nl]
        \item \textit{(non-negative)} $\norm{x}\geq 0$ for any $x\in X$
        \item \textit{(non-degenerate)} $\norm{x}=0$ if and only if $x=0$
        \item \textit{(subadditivity)} $\norm{x+y}\leq\norm{x}+\norm{y}$ for $x,y\in X$
        \item \textit{($\abs{\cdot}-$homogeneity)} $\norm{\alpha x}=\abs{\alpha}\norm{x}$ for $\alpha\in\F$, $x\in X$.
    \end{itemize}
    We call the pair $(X,\norm{\cdot})$ a \textbf{normed vector space}.
    Furthermore, we say that $(X,\norm{\cdot})$ is a \textbf{Banach space}  provided that $X$ is complete with respect to the metric $\rho(x,y)=\norm{x-y}$.
\end{definition}
\begin{example}
    \begin{enumerate}[r]
        \item $(\F,\abs{\cdot})$ is a Banach space.
        \item $(\F^b,\norm{\cdot}_p)$, $x=(x_j)_{j=1}^n$,
            \begin{equation*}
                \norm{x}_p=\begin{cases}
                    \left(\sum_{i=1}^n\abs{x_j}^p\right)^{1/p} & 1\leq p<\infty\\
                    \max_{j=1,\ldots,n}|x_j| & p=\infty
                \end{cases}
            \end{equation*}
        \item Consider the space
            \begin{equation*}
                \quot{L_p^{\F}=\Set{f:[0,1]\to\F | f\text{ is Lebesgue measurable}, \left(\int_0^1\abs{f}^p\right)^{1/p}<\infty}}{\sim_{\text{a.e.}}}
            \end{equation*}
            where $1\leq p<\infty$.
        \item $\displaystyle L_\infty^{\F}[0,1]$, $\norm{f}_\infty=\esssup_{t\in[0,1]}|f(t)|$.
        \item Let $(X,d)$ be a metric space.
            Then
            \begin{equation*}
                C_b^{\F}(x)=\Set{f:X\to\F | f\text{ is continuous and bounded}},\quad\norm{f}_\infty=\sup_{x\in X}|f(x)|
            \end{equation*}
            is a Banach space.
    \end{enumerate}
\end{example}
Here is a more interesting example:
\begin{example}
    Let $(X,d)$ be a metric space.
    We define the space of Lipschitz functions
    \begin{equation*}
        \Lip^{\F}(X,d)=\Set{f:X\to\F | f\text{ is bounded}, L(f)=\sup_{\substack{x,y\in X\\x\neq y}}\frac{|f(x)-f(y)|}{d(x,y)}<\infty}
    \end{equation*}
    We note that for $f:X\to\F$ that
    \begin{equation}\label{e:lip}
        f\in\Lip^{\F}(X,d)\Leftrightarrow\text{there is }L\geq 0\text{ s.t. }|f(x)-f(x)|\leq Ld(x,y)\text{ for all }x,y\in X
    \end{equation}
    It is easy to verify that $L(f)=\min\{L\geq 0:\text{\cref{e:lip} holds for }f\}$.
    It is an easy exercise to see that $\Lip^{\F}$ is a vector space, and that $L:\Lip^{F}(X,d)\to\R$ is a \textbf{semi-norm} (non-negative, subadditive, $\abs{\cdot}-$homogeneous).
    However, we do not have non-degeneracy (for example, constants are taken to $0$).
    We define the Lipschitz norm
    \begin{equation*}
        \norm{f}_{\Lip}=\norm{f}_\infty+L(f)
    \end{equation*}
\end{example}
\begin{proposition}
    $(\Lip^{\F}(X,d),\norm{\cdot}_{\Lip})$ is a Banach space.
\end{proposition}
\begin{proof}
    Let $(f_n)_{n=1}^\infty$ be a Cauchy sequence in $(\Lip^{\F}(X,d),\norm{\cdot}_{\Lip})$.
    Since $\norm{\cdot}_\infty\leq\norm{\cdot}_{\Lip}$ on $\Lip^{F}(X,d)$, we see that $(f_n)_{n=1}^\infty$ is uniformly Cauchy (and bounded), and hence there is $f=\lim_{n\to\infty} f_n$ in $C_b^{\F}(X)$, where the limit is taken with respect to $\norm{\cdot}_\infty$, since $(C_b^{\F}(X),\norm{\cdot}_\infty)$ is a Banach space.
    If $x,y\in X$, then
    \begin{align*}
        |f(x)-f(x)| &= \lim_{n\to\infty}|f_n(x)-f_n(y)|\leq \sup_{n\in\N}|f_n(x)-f_n(y)|\\
                    &\leq \sup_{n\in\N} L(f_n)d(x,y)\leq\sup_{n\in\N}\norm{f_n}_{\Lip} d(x,y)
    \end{align*}
    Since Cauchy sequences are bounded, we see that $|f(x)-f(y)|\leq Ld(x,y)$, where $L=\sup_{n\in\N}\norm{f_n}_{\Lip}<\infty$.
    Thus by \cref{e:lip}, $f\in\Lip^{\F}(X,d)$.
    Exercise: one may verify that $\norm{f-f_n}_{\Lip}\to 0$.
\end{proof}
Another collection of basic examples are given by the sequence spaces.
We can define
\begin{equation*}
    \ell_1^{\F}=\Set{x=(x_j)_{j=1}^\infty\in\F^{\N} | \norm{x}_1=\sum_{j=1}^\infty|x_j|<\infty}
\end{equation*}
It is easy to see that $(\ell_1,\norm{\cdot}_1)$ is a normed vector space.

For $1<p<\infty$, and write
\begin{equation*}
    \ell_p^{\F}=\Set{x\in\F^{\N} |\norm{x}_p=\left(\sum_{j=1}^\infty|x_j|^p\right)^{1/p}<\infty}
\end{equation*}
Note that $0\in\ell_p$, $\alpha\in\F$, $\alpha x\in\ell_p$ if $x\in\ell_p$.
Let $q=p/(p-1)$ so that $1/p+1/q=1$.
Then $q$ is called the \textbf{conjugate index}.
We have
\begin{proposition}[Young's Inequality]
    If $a,b\geq 0$ in $\R$, then $ab\leq a^p/p+b^q/q$, with equality only if $a^p=b^q$.
\end{proposition}
and
\begin{proposition}[H\"older's Inequality]
    If $x\in\ell_p$ and $y\in\ell_q$, then $xy=(x_iy_i)_{i=1}^\infty\in\ell_1$, with
    \begin{equation*}
        \abs{\sum_{i=1}^\infty x_iy_i}\leq\norm{x}_p\norm{y}_q
    \end{equation*}
    with equality exactly when $\sgn(x_iy_i)=\sgn(x_ky_k)$ for all $j,k\in\N$ where $x_iy_i\neq 0\neq x_ky_k$, and $|x|^p=(|x_j|^p)_{j=1}^\infty$ and $|y|^q$ are linearly dependent in $\ell_1$.
\end{proposition}
and finally
\begin{proposition}[Minkowski's Inequality]
    If $x,y\in\ell_p$, then $\norm{x+y}_p\leq\norm{x}_p+\norm{y}_p$ with equality exactly when one of $x$ or $y$ is a non-negative scalar combination of the other.
\end{proposition}

\subsection{Review of Topology}
Let $X$ denote a non-empty set, and $\mathcal{P}(X)$ denote the power set of $X$.
\begin{definition}
    A \textbf{topology} on a set $X$ is a set $\tau$ of subsets of $X$ such that
    \begin{enumerate}[nl,r]
        \item $\emptyset,X\in\tau$
        \item If $U_\alpha\in\tau$ for all $\alpha\in A$, then $\bigcup_{\alpha\in A}U_\alpha\in\tau$.
        \item If $n\in\N$ and $U_i\in\tau$ for each $1\leq i\leq n$, then $\bigcap_{i=1}^n U_i\in\tau$.
    \end{enumerate}
    The sets $U\in\tau$ are called the \textbf{open sets} in $X$, and sets of the form $X\setminus U$ for some open set $U$ are caled the \textbf{closed sets} in $X$.
    The pair $(X,\tau)$ is called a \textbf{topological space}.
\end{definition}
The metric topology on a metric space $(X,d)$ is the topology
\begin{equation*}
    \tau_d = \Set{U\subseteq X |\text{ for each }x_0\in U\text{, there is }\delta=\delta(x_0)\text{ s.t. }\\B_\delta(x_0)\subseteq U}
\end{equation*}
\begin{example}
    \begin{enumerate}[r]
        \item Given two metrics $d,\rho$ on $X$, we say that $d\sim\rho$ if and only if there are $c,C>0$ such that
            \begin{equation*}
                cd(x,y)\leq\rho(x,y)\leq Cd(x,y)\text{ for any }x,y\in X
            \end{equation*}
            Note that $d\sim\rho$ implies that $\tau_d=\tau_\rho$, but the reverse implication is not true.
            An example of this are the metrics on $X=\R$ given by $d(x,y)$ and $\rho(x,y)=\frac{|x-y|}{1+|x-y|}$.
            Then $d\not\sim\rho$ but $\tau_d=\tau_\rho$.
        \item ``Sorgenfry line''
            Set $X=\R$, and consider
            \begin{equation*}
                \sigma=\Set{V\subseteq\R | \text{ for any }s\in V,\text{ there is }\delta=\delta(s)>0\text{ s.t. }[s,s+\delta)\subseteq V}
            \end{equation*}
            It is an exercise to verify that $\tau_{\abs{\cdot}}\subsetneq\sigma$.
            We say that $\sigma$ is \textbf{finer} than $\tau_{\abs{\cdot}}$.
        \item Relative topology: let $(X,\tau)$ be a topological space, and $\emptyset\neq A\subseteq X$.
            Then we can define a topology $\tau|_A=\{U\cap A:U\in\tau\}$.
    \end{enumerate}
\end{example}
\begin{definition}
    Let $(X,\tau)$ and $(Y,\sigma)$ be topological spaces, and $f:X\to Y$.
    We say that $f$ is \textbf{$(\tau-\sigma-)$continuous} at $x_0$ in $X$ if,
    \begin{itemize}
        \item given $V\in\sigma$ such that $f(x_0)\in V$, then there exists $U\in\tau$ such that $x_0\in U$ and $f(U)\subseteq V$.
    \end{itemize}
    We say that $f$ is $(\tau-\sigma-)$continuous if it is continuous at each $x_0$ in $X$.
\end{definition}
\subsection{Space of bounded continuous functions into a normed space}
Let $(Y,\norm{\cdot})$ denote a normed space.
We let $\tau_{\norm{\cdot}}$ denote the topology given by the metric $\rho(x,y)=\norm{x-y}$.
Let $(X,\tau)$ denote any topological space.
Then we write
\begin{equation*}
    C_b^Y(X) = \Set{f:X\to Y | f\text{ is bounded  and }\tau-\tau_{\norm{\cdot}}-\text{continuous}}
\end{equation*}
With pointwise operations, we see that $C_b^Y(X)$ is a vector space.
We also define for $f\in C_b^Y(X)$, $\norm{f}_\infty=\sup\{\norm{f(x)}:x\in X\}$, making $(C_b^Y(X),\norm{\cdot}_\infty)$ a normed vector space.
\begin{theorem}
    If $(Y,\norm{\cdot})$ is a Banach space, then $(C_b^Y(X),\norm{\cdot}_\infty)$ is a Banach space.
\end{theorem}
\begin{proof}
    Let $(f_n)_{n=1}^\infty$ be a Cauchy sequence in $(C_b^Y(X),\norm{\cdot}_\infty)$.
    Then for any $x\in X$, we have that $(f_n(x))_{n=1}^\infty$ is Cauchy in $(Y,\norm{\cdot})$ since $\norm{f_n(x)-f_m(x)}\leq\norm{f_n-f_m}_\infty$, and hence admis a limit $f(x)$.
    In particular, $x\mapsto f(x)$ defines a function from $X$ to $Y$.
    We shall fix $x_0\in X$ and show that $f$ is continuous at $x_0$.
    Given $\epsilon>0$, we let
    \begin{itemize}[nl]
        \item $n_1$ be so $n,m\geq n_1$ so that $\norm{f_n-f_m}_\infty<\epsilon/4$.
        \item $n_2$ be so $n\geq n_2$ so that $\norm{f_n(x_0)-f(x_0)}<\epsilon/4$.
        \item $N=\max\{n_1,n_2\}$.
        \item $U\in \tau$, $x_0\in U$ such that $f_N(U)\subseteq B_{\epsilon/4}(f(x_0))\subset Y$.
    \end{itemize}
    Then for $x\in U$, we let $n_x$ be so $n_x\geq n_1$ and $n\geq n_x$, so that $\norm{f_n(x)-f(x)}<\epsilon/4$.
    We then have
    \begin{align*}
        \norm{f(x)-f(x_0)} &\leq \norm{f(x)-f_{n_x}(x)} + \norm{f_{n_x}(x)-f_N(x)} + \norm{f_N(x)-f_N(x_0)} + \norm{f_N(x_0)-f(x_0)}\\
                           &<\frac{\epsilon}{4}+\norm{f_{n_x}-f_N}_\infty+\frac{\epsilon}{4}+\frac{\epsilon}{4}<\epsilon
    \end{align*}
    in other words that $f(U)\subseteq B_\epsilon(f(x_0))$.

    Now let us check that $\norm{f}_\infty<\infty$.
    Since $|\norm{f_n}_\infty-\norm{f_m}_\infty|\leq\norm{f_n-f_m}_\infty$, so $(\norm{f_n}_\infty)_{n=1}^\infty\subseteq\R$ is Cauchy, hence bounded.
    If $x\in X$, then
    \begin{equation*}
        \norm{f(x)} = \lim_{n\to\infty}\norm{f_n(x)}\leq\sup_{n\in\N}\norm{f_n(x)}\leq\sup_{n\in\N}\norm{f_n}_\infty<\infty
    \end{equation*}
    so $\norm{f}_\infty=\sup_{x\in X}\norm{f(x)}<\infty$.

    Notice that if $\epsilon$, $n_1$ are as above, and further $x_0,N$ are as above, we have for $n\geq n_1$
    \begin{equation*}
        \norm{f_n(x_0)-f(x_0)} \leq \norm{f_n(x_0)-f_N(x_0)}+\norm{f_N(x_0)-f(x_0)}<\frac{\epsilon}{2}
    \end{equation*}
    so $\norm{f_n-f}_\infty=\sup_{x_0\in X}\norm{f_n(x_0)-f(x_0)}\leq\epsilon/2<\epsilon$.
    This is uniform since $n_1$ is chosen uniformly in $X$.
\end{proof}
\begin{corollary}
    $(C_b^{\F}(X),\norm{\cdot}_\infty)$ is a Banach space.
\end{corollary}
\end{document}


