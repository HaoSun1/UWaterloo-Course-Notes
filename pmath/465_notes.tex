% header -----------------------------------------------------------------------
% Template created by texnew (author: Alex Rutar); info can be found at 'https://github.com/alexrutar/texnew'.
% version (1.13)


% doctype ----------------------------------------------------------------------
\documentclass[11pt, a4paper]{memoir}
\usepackage[utf8]{inputenc}
\usepackage[left=3cm,right=3cm,top=3cm,bottom=4cm]{geometry}
\usepackage[protrusion=true,expansion=true]{microtype}


% packages ---------------------------------------------------------------------
\usepackage{amsmath,amssymb,amsfonts}
\usepackage{graphicx}
\usepackage{etoolbox}

% Set enimitem
\usepackage{enumitem}
\SetEnumitemKey{nl}{nolistsep}
\SetEnumitemKey{r}{label=(\roman*)}

% Set tikz
\usepackage{tikz, pgfplots}
\pgfplotsset{compat=1.15}
\usetikzlibrary{intersections,positioning,cd}
\usetikzlibrary{arrows,arrows.meta}
\tikzcdset{arrow style=tikz,diagrams={>=stealth}}

% Set hyperref
\usepackage[hidelinks]{hyperref}
\usepackage{xcolor}
\newcommand\myshade{85}
\colorlet{mylinkcolor}{violet}
\colorlet{mycitecolor}{orange!50!yellow}
\colorlet{myurlcolor}{green!50!blue}

\hypersetup{
  linkcolor  = mylinkcolor!\myshade!black,
  citecolor  = mycitecolor!\myshade!black,
  urlcolor   = myurlcolor!\myshade!black,
  colorlinks = true,
}


% macros -----------------------------------------------------------------------
\DeclareMathOperator{\N}{{\mathbb{N}}}
\DeclareMathOperator{\Q}{{\mathbb{Q}}}
\DeclareMathOperator{\Z}{{\mathbb{Z}}}
\DeclareMathOperator{\R}{{\mathbb{R}}}
\DeclareMathOperator{\C}{{\mathbb{C}}}
\DeclareMathOperator{\F}{{\mathbb{F}}}

% Boldface includes math
\newcommand{\mbf}[1]{{\boldmath\bfseries #1}}

% proof implications
\newcommand{\imp}[2]{($#1\Rightarrow#2$)\hspace{0.2cm}}
\newcommand{\impe}[2]{($#1\Leftrightarrow#2$)\hspace{0.2cm}}
\newcommand{\impr}{{($\Rightarrow$)\hspace{0.2cm}}}
\newcommand{\impl}{{($\Leftarrow$)\hspace{0.2cm}}}

% align macros
\newcommand{\agspace}{\ensuremath{\phantom{--}}}
\newcommand{\agvdots}{\ensuremath{\hspace{0.16cm}\vdots}}

% convenient brackets
\newcommand{\brac}[1]{\ensuremath{\left\langle #1 \right\rangle}}
\newcommand{\norm}[1]{\ensuremath{\left\lVert#1\right\rVert}}
\newcommand{\abs}[1]{\ensuremath{\left\lvert#1\right\rvert}}

% arrows
\newcommand{\lto}[0]{\ensuremath{\longrightarrow}}
\newcommand{\fto}[1]{\ensuremath{\xrightarrow{\scriptstyle{#1}}}}
\newcommand{\hto}[0]{\ensuremath{\hookrightarrow}}
\newcommand{\mapsfrom}[0]{\mathrel{\reflectbox{\ensuremath{\mapsto}}}}
 
% Divides, Not Divides
\renewcommand{\div}{\bigm|}
\newcommand{\ndiv}{%
    \mathrel{\mkern.5mu % small adjustment
        % superimpose \nmid to \big|
        \ooalign{\hidewidth$\big|$\hidewidth\cr$/$\cr}%
    }%
}

% Convenient overline
\newcommand{\ol}[1]{\ensuremath{\overline{#1}}}

% Big \cdot
\makeatletter
\newcommand*\bigcdot{\mathpalette\bigcdot@{.5}}
\newcommand*\bigcdot@[2]{\mathbin{\vcenter{\hbox{\scalebox{#2}{$\m@th#1\bullet$}}}}}
\makeatother

% Big and small Disjoint union
\makeatletter
\providecommand*{\cupdot}{%
  \mathbin{%
    \mathpalette\@cupdot{}%
  }%
}
\newcommand*{\@cupdot}[2]{%
  \ooalign{%
    $\m@th#1\cup$\cr
    \sbox0{$#1\cup$}%
    \dimen@=\ht0 %
    \sbox0{$\m@th#1\cdot$}%
    \advance\dimen@ by -\ht0 %
    \dimen@=.5\dimen@
    \hidewidth\raise\dimen@\box0\hidewidth
  }%
}

\providecommand*{\bigcupdot}{%
  \mathop{%
    \vphantom{\bigcup}%
    \mathpalette\@bigcupdot{}%
  }%
}
\newcommand*{\@bigcupdot}[2]{%
  \ooalign{%
    $\m@th#1\bigcup$\cr
    \sbox0{$#1\bigcup$}%
    \dimen@=\ht0 %
    \advance\dimen@ by -\dp0 %
    \sbox0{\scalebox{2}{$\m@th#1\cdot$}}%
    \advance\dimen@ by -\ht0 %
    \dimen@=.5\dimen@
    \hidewidth\raise\dimen@\box0\hidewidth
  }%
}
\makeatother


% macros (theorem) -------------------------------------------------------------
\usepackage[thmmarks,amsmath,hyperref]{ntheorem}
\usepackage[capitalise,nameinlink]{cleveref}

% Numbered Statements
\theoremstyle{change}
\theoremindent\parindent
\theorembodyfont{\itshape}
\theoremheaderfont{\bfseries\boldmath}
\newtheorem{theorem}{Theorem.}[section]
\newtheorem{lemma}[theorem]{Lemma.}
\newtheorem{corollary}[theorem]{Corollary.}
\newtheorem{proposition}[theorem]{Proposition.}

% Claim environment
\theoremstyle{plain}
\theorempreskip{0.2cm}
\theorempostskip{0.2cm}
\theoremheaderfont{\scshape}
\newtheorem{claim}{Claim}
\renewcommand\theclaim{\Roman{claim}}
\AtBeginEnvironment{theorem}{\setcounter{claim}{0}}

% Un-numbered Statements
\theorempreskip{0.1cm}
\theorempostskip{0.1cm}
\theoremindent0.0cm
\theoremstyle{nonumberplain}
\theorembodyfont{\upshape}
\theoremheaderfont{\bfseries\itshape}
\newtheorem{definition}{Definition.}
\theoremheaderfont{\itshape}
\newtheorem{example}{Example.}
\newtheorem{remark}{Remark.}

% Proof / solution environments
\theoremseparator{}
\theoremheaderfont{\hspace*{\parindent}\scshape}
\theoremsymbol{$//$}
\newtheorem{solution}{Sol'n}
\theoremsymbol{$\blacksquare$}
\theorempostskip{0.4cm}
\newtheorem{proof}{Proof}
\theoremsymbol{}
\newtheorem{nmproof}{Proof}

% Format references
\crefformat{equation}{(#2#1#3)}
\Crefformat{theorem}{#2Thm. #1#3}
\Crefformat{lemma}{#2Lem. #1#3}
\Crefformat{proposition}{#2Prop. #1#3}
\Crefformat{corollary}{#2Cor. #1#3}
\crefformat{theorem}{#2Theorem #1#3}
\crefformat{lemma}{#2Lemma #1#3}
\crefformat{proposition}{#2Proposition #1#3}
\crefformat{corollary}{#2Corollary #1#3}


% macros (algebra) -------------------------------------------------------------
\DeclareMathOperator{\Ann}{Ann}
\DeclareMathOperator{\Aut}{Aut}
\DeclareMathOperator{\chr}{char}
\DeclareMathOperator{\coker}{coker}
\DeclareMathOperator{\disc}{disc}
\DeclareMathOperator{\End}{End}
\DeclareMathOperator{\Fix}{Fix}
\DeclareMathOperator{\Frac}{Frac}
\DeclareMathOperator{\Gal}{Gal}
\DeclareMathOperator{\GL}{GL}
\DeclareMathOperator{\Hom}{Hom}
\DeclareMathOperator{\id}{id}
\DeclareMathOperator{\im}{im}
\DeclareMathOperator{\lcm}{lcm}
\DeclareMathOperator{\Nil}{Nil}
\DeclareMathOperator{\rank}{rank}
\DeclareMathOperator{\Res}{Res}
\DeclareMathOperator{\Spec}{Spec}
\DeclareMathOperator{\spn}{span}
\DeclareMathOperator{\Stab}{Stab}
\DeclareMathOperator{\Tor}{Tor}

% Lagrange symbol
\newcommand{\lgs}[2]{\ensuremath{\left(\frac{#1}{#2}\right)}}

% Quotient (larger in display mode)
\newcommand{\quot}[2]{\mathchoice{\left.\raisebox{0.14em}{$#1$}\middle/\raisebox{-0.14em}{$#2$}\right.}
                                 {\left.\raisebox{0.08em}{$#1$}\middle/\raisebox{-0.08em}{$#2$}\right.}
                                 {\left.\raisebox{0.03em}{$#1$}\middle/\raisebox{-0.03em}{$#2$}\right.}
                                 {\left.\raisebox{0em}{$#1$}\middle/\raisebox{0em}{$#2$}\right.}}


% macros (analysis) ------------------------------------------------------------
\DeclareMathOperator{\M}{{\mathcal{M}}}
\DeclareMathOperator{\B}{{\mathcal{B}}}
\DeclareMathOperator{\ps}{{\mathcal{P}}}
\DeclareMathOperator{\pr}{{\mathbb{P}}}
\DeclareMathOperator{\E}{{\mathbb{E}}}
\DeclareMathOperator{\supp}{supp}
\DeclareMathOperator{\sgn}{sgn}

\renewcommand{\Re}{\ensuremath{\operatorname{Re}}}
\renewcommand{\Im}{\ensuremath{\operatorname{Im}}}
\renewcommand{\d}[1]{\ensuremath{\operatorname{d}\!{#1}}}


% file-specific preamble -------------------------------------------------------
\usepackage{braket}


% constants --------------------------------------------------------------------
\newcommand{\subject}{PMATH 465}
\newcommand{\semester}{Fall 2019}


% formatting -------------------------------------------------------------------
% Fonts
\usepackage{kpfonts}
\usepackage{dsfont}

% Adjust numbering
\numberwithin{equation}{section}
\counterwithin{figure}{section}
\counterwithout{section}{chapter}
\counterwithin*{chapter}{part}

% Footnote
\setfootins{0.5cm}{0.5cm} % footer space above
\renewcommand*{\thefootnote}{\fnsymbol{footnote}} % footnote symbol

% Table of Contents
\renewcommand{\thechapter}{\Roman{chapter}}
\renewcommand*{\cftchaptername}{Chapter } % Place 'Chapter' before roman
\setlength\cftchapternumwidth{4em} % Add space before chapter name
\cftpagenumbersoff{chapter} % Turn off page numbers for chapter
\maxtocdepth{section} % table of contents up to section

% Section / Subsection headers
\setsecnumdepth{section} % numbering up to and including "section"
\newcommand*{\shortcenter}[1]{%
    \sethangfrom{\noindent ##1}%
    \Large\boldmath\scshape\bfseries
    \centering
\parbox{5in}{\centering #1}\par}
\setsecheadstyle{\shortcenter}
\setsubsecheadstyle{\large\scshape\boldmath\bfseries\raggedright}

% Chapter Headers
\chapterstyle{verville}

% Page Headers / Footers
\copypagestyle{myruled}{ruled} % Draw formatting from existing 'ruled' style
\makeoddhead{myruled}{}{}{\scshape\subject}
\makeevenfoot{myruled}{}{\thepage}{}
\makeoddfoot{myruled}{}{\thepage}{}
\pagestyle{myruled}
\setfootins{0.5cm}{0.5cm}
\renewcommand*{\thefootnote}{\fnsymbol{footnote}}

% Titlepage
\title{\subject}
\author{Alex Rutar\thanks{\itshape arutar@uwaterloo.ca}\\ University of Waterloo}
\date{\semester\thanks{Last updated: \today}}

\begin{document}
\pagenumbering{gobble}
\hypersetup{pageanchor=false}
\maketitle
\newpage
\frontmatter
\hypersetup{pageanchor=true}
\tableofcontents*
\newpage
\mainmatter


% main document ----------------------------------------------------------------
\chapter{Fundamentals of Manifolds}
\section{Introduction to Topology}
\begin{definition}
    A \textbf{topology} on a set $X$ is a set $\tau$ of subsets of $X$ such that
    \begin{enumerate}[nl,r]
        \item $\emptyset\in\tau$ and $X\in\tau$
        \item If $U_\alpha\in\tau$ for all $\alpha\in A$, then $\bigcup_{\alpha\in A}U_\alpha\in\tau$.
        \item If $n\in\N$ and $U_i\in\tau$ for each $1\leq i\leq n$, then $\bigcap_{i=1}^n U_i\in\tau$.
    \end{enumerate}
    The sets $U\in\tau$ are called the \textbf{open sets} in $X$, and sets of the form $X\setminus U$ for some open set $U$ are caled the \textbf{closed sets} in $X$.
\end{definition}
\begin{definition}
    When $X$ is a topological space and $A\subseteq X$, the \textbf{interior} of $A$ (denoted $A^\circ$) is the union of all open sets contained in $A$.
    Similarly, we define the \textbf{closure} of $A$ (denoted $\overline{A}$) as the intersction of all closed sets containing $A$.
    Then the \textbf{boundary} of $A$, denoted by $\partial A$, is the set $\partial A=\overline{A}\setminus A^\circ$.
\end{definition}
\begin{example}
    Let $X$ be any set.
    The \textbf{discrete topology} on $X$ is the topology $\tau=\mathcal{P}(X)$, and the \textbf{trivial topology} on $X$ is the topology $\tau=\{\emptyset,X\}$.
\end{example}
\begin{definition}
    A \textbf{basis} for a topology on a set $X$ is a set $\mathcal{V}$ of subsets of $X$
    \begin{enumerate}[nl,r]
        \item $\bigcup_{B\in \mathcal{B}}b=X$
        \item for all $a\in X$ and $U,V\in\mathcal{B}$ such that $a\in U\cap V$, then there exists $W\in\mathcal{B}$ with $a\in W\subseteq U\cap V$.
    \end{enumerate}
    When $\mathcal{B}$ is a basis for a topology on $X$, the topology on $X$ \textbf{generated} by $\mathcal{B}$ is the set $\tau$ of subsets of $X$ such that for $W\subseteq X$, $W\in\tau$ if and only if for all $a\in W$, there exists $U\in \mathcal{B}$ such that $a\in U\subseteq W$.
\end{definition}
Note that $\tau$, as above, is a topology on $X$ since
\begin{enumerate}[nl,r]
    \item $\emptyset\in\tau$ vacuously and $X\in\tau$ obviously.
    \item If $A_k\in\tau$ for all $k\in K$ (where $K$ is any set of indices), then given $a\in\bigcup_{x\in K}A_k$, we can choose $\ell\in K$ so that $a\in A_\ell$.
        Then since $A_\ell\in\tau$, we can choose $U_\ell\in\mathcal{B}$ so that $a\in U_\ell\subseteq A_\ell$.
        Thus $a\in U_\ell\subseteq A_\ell\subseteq\bigcup_{k\in K}A_k$.
    \item By induction, it suffices to prove that if $A,B\in\tau$, then $A\cap B\in\tau$.
        Suppose $A,B\in\tau$, and let $a\in A\cap B$.
        Since $A\in\tau$, we cna choose $U\in\mathcal{B}$ so that $a\in U\subseteq A$.
        Since $B\in\tau$, we can choose $V\in\mathcal{B}$ so that $a\in V\subseteq B$.
        Then we have $a\in U\cap V$.
        Since $\mathcal{B}$ is a basis, we can chose $W\in\mathcal{B}$ with $a\in W\subseteq U\cap V$, so $a\in W\subseteq U\cap V\subseteq A\cap B$.
\end{enumerate}
Note that when $\tau$ is the topology on $X$ generated by the basis $\mathcal{B}$, for $A\subseteq X$, $A\in\tau$ if and only if there exists some $S\subseteq\mathcal{B}$ such that $A=\bigcup_{s\in S}s$.
In this sense, the topology $\tau$ on $X$ generated by the basis $\mathcal{B}$ is the coarsest topology which contains $\mathcal{B}$.
\begin{definition}[Subspace Topology]
    When $Y$ is a topological space and $X\subseteq Y$ is a subset of $Y$, we define the \textbf{subspace topology} on $X$ to be the topology for which as set $U\subseteq X$ is open if and only if $U=X\cap V$ for some open set $V$.
\end{definition}
If $\mathcal{C}$ is a basis for the topology on $Y$, then $\mathcal{B}=\Set{X\cap V | V\in\mathcal{C}}$ is a basis for the subspace topology on $X$.
\begin{definition}[Disjoint Union Topology]
    If $X$ and $Y$ are topological spaces with $X\cap Y=\emptyset$, then the \textbf{disjoint union topology} on $X\cup Y$ is the topology in which a subset $U\subseteq X\cup Y$ is open in $X\cup Y$ if and only if $U\cap X$ is open in $X$ and $U\cap Y$ is open in $Y$.
\end{definition}
\begin{definition}[Product Topology]
    If $X$ and $Y$ are topological spaces, the \textbf{product topology} on $X\times Y$ is the topology generted by the basis
    \begin{equation*}
        \mathcal{B}=\Set{U\times V | U\in\mathcal{C},V\in\mathcal{D}}
    \end{equation*}
    where $\mathcal{C}$ and $\mathcal{D}$ are bases for the topologies on $X,Y$ respectively.
\end{definition}
\begin{definition}[Infinite Product Topology]
    We define the infinite product to be
    \begin{equation*}
        \prod_{k\in K}\Set{f:K\to \bigcup_{k\in K}X_k | f(k)\in X_k\text{ for all }k\in K}
    \end{equation*}
    There are two standard topologies on $X$.
    The first is the \textbf{box topology},
    \begin{equation*}
        \mathcal{B}=\Set{\prod_{k\in K}U_k | U_k\text{ is open in }X_k}
    \end{equation*}
    and the \textbf{product topology}
    \begin{equation*}
        \mathcal{B}=\Set{\prod_{k\in K}U_k | \begin{array}{l}U_k\text{ is open in }X_k\\U_k=X_k\text{ for all but finitely many indices }k\end{array}}
    \end{equation*}
\end{definition}
\begin{example}[Metric Topology]
    $\R^n$ has a standard \textbf{inner product}, and for $u,v\in\R^n$, $\inner{u}{v}=u\cdot v=V^Tu=\sum_{i=1}^n u_iv_i$.
    This gives the standard norm on $\R^n$ for $u\in\R^n$, $\norm{u}=\sqrt{\inner{u}{v}}$.
    This gives the standard metric on $\R^n$: for $a,\in\R^n$, $d(a,b)=\norm{b-a}$.

    Given a metric on a set $Y$, we obtain (by restriction) an induced metric on any subset $X\subseteq Y$.
    Given a metric space $X$, we define the \textbf{metric topology} on $X$ to be the topology which is generated by the set of open balls
    \begin{equation*}
        B(a,r) = \Set{x\in X | d(a,x)<r}
    \end{equation*}
    where $x\in X$, $r>0$.
\end{example}
\begin{definition}
    When $X$ and $Y$ are topological spaces and $f:X\to Y$, we say that $f$ is \textbf{continuous} when it has the property that $f^{-1}(V)$ is open in $X$ for every open set $V$ in $Y$.
    We say that $f:X\to Y$ is a \textbf{homeomorphism} when $f$ is bijective and both $f$ and $f^{-1}$ are continuous.
    Then $X,Y$ are \textbf{homeomorphic} if there exists a homeomorphism $f:X\to Y$.
\end{definition}
\begin{theorem}[Glueing Lemma]
    Let $X$ and $Y$ be topological spaces, and let $f:X\to Y$ be a function.
    Suppose either
    \begin{enumerate}[nl,r]
        \item $X=\bigcup_{k\in K}A_k$ where each $A_k$ is open in $X$, or
        \item $X=\bigcup_{k=1}^nA_k$ where each $A_k$ is closed in $X$
    \end{enumerate}
    and each restriction map $f_k:A_k\to Y$ is continuous, then $f$ is continuous.
\end{theorem}
\begin{proof}
    Exercise.
\end{proof}
\begin{definition}
    A topological space $X$ is \textbf{compact} when it has the property that for every set $\mathcal{S}$ of open subsets of $X$ with $X=\bigcup_{U\in S}U$, there exists a finite subset $\mathcal{F}\subseteq \mathcal{S}$ such that $X=\bigcup_{F\in\mathcal{F}}F$.
\end{definition}
Note that when $X\subseteq Y$ is a subspace, $X$ is compact if and only if $X$ has the property that for every set $\mathcal{T}$ with $X\subseteq\bigcup_{T\in\mathcal{T}}T$, there exists a finite subset $\mathcal{G}\subseteq\mathcal{T}$ uch that $X\subseteq\bigcup_{G\in\mathcal{G}}G$.
\begin{definition}
    A topological space $X$ is \textbf{connected} when there do not exist non-empty disjoint open sets $U,V\in X$ such that $X=U\cup V$.
\end{definition}
Note that if $Y$ is a metric space and $X\subseteq Y$ is a subspsace, then $X$ if connected if and only if there do not exist open sets $U,V\in Y$ such that
\begin{equation*}
    X\cap U\neq\emptyset, X\cap V\neq\emptyset, U\cap V=\emptyset,\text{ and }X\subseteq U\cap V
\end{equation*}
\begin{definition}
    A topological space $X$ is called \textbf{path connected} when it has the property that for all $a,b\in X$, there exists a continuous map $\alpha:[0,1]\to X$ with $\alpha(0)=a$ and $\alpha(1)=b$.
\end{definition}
It is easy to see that if $X$ is path connected, then $X$ is connected.
\begin{definition}
    Let $X$ be a topological space.
    If we define a relation $\sim$ on $C$ by $a\sim b$ if and only if there exists a connected subspsace $A\subseteq X$ with $a\in A$ and $b\in B$.
\end{definition}

\end{document}



