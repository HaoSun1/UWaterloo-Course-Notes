% header -----------------------------------------------------------------------
% Template created by texnew (author: Alex Rutar); info can be found at 'https://github.com/alexrutar/texnew'.
% version (1.12)


% doctype ----------------------------------------------------------------------
\documentclass[11pt, a4paper]{memoir}
\usepackage[ascii]{inputenc}
\usepackage[left=3cm,right=3cm,top=3cm,bottom=4cm]{geometry}
\usepackage[protrusion=true,expansion=true]{microtype}


% packages ---------------------------------------------------------------------
\usepackage{amsmath,amssymb,amsfonts}
\usepackage{graphicx}
\usepackage{etoolbox}

% Set enimitem
\usepackage{enumitem}
\SetEnumitemKey{nl}{nolistsep}
\SetEnumitemKey{r}{label=(\roman*)}

% Set tikz
\usepackage{tikz, pgfplots}
\pgfplotsset{compat=1.15}
\usetikzlibrary{intersections,positioning,cd,fit,backgrounds}
\usetikzlibrary{arrows,arrows.meta}
\tikzcdset{arrow style=tikz,diagrams={>=stealth}}
\tikzset{>=stealth}


% macros -----------------------------------------------------------------------
\DeclareMathOperator{\N}{{\mathbb{N}}}
\DeclareMathOperator{\Q}{{\mathbb{Q}}}
\DeclareMathOperator{\Z}{{\mathbb{Z}}}
\DeclareMathOperator{\R}{{\mathbb{R}}}
\DeclareMathOperator{\C}{{\mathbb{C}}}
\DeclareMathOperator{\F}{{\mathbb{F}}}

% Boldface includes math
\newcommand{\mbf}[1]{{\boldmath\bfseries #1}}

% proof implications
\newcommand{\imp}[2]{($#1\Rightarrow#2$)\hspace{0.2cm}}
\newcommand{\impe}[2]{($#1\Leftrightarrow#2$)\hspace{0.2cm}}
\newcommand{\impr}{{($\Rightarrow$)\hspace{0.2cm}}}
\newcommand{\impl}{{($\Leftarrow$)\hspace{0.2cm}}}

% align macros
\newcommand{\agspace}{\ensuremath{\phantom{--}}}
\newcommand{\agvdots}{\ensuremath{\hspace{0.16cm}\vdots}}

% convenient brackets
\newcommand{\brac}[1]{\ensuremath{\left\langle #1 \right\rangle}}
\newcommand{\norm}[1]{\ensuremath{\left\lVert#1\right\rVert}}
\newcommand{\abs}[1]{\ensuremath{\left\lvert#1\right\rvert}}

% arrows
\newcommand{\lto}[0]{\ensuremath{\longrightarrow}}
\newcommand{\fto}[1]{\ensuremath{\xrightarrow{\scriptstyle{#1}}}}
\newcommand{\hto}[0]{\ensuremath{\hookrightarrow}}
\newcommand{\mapsfrom}[0]{\mathrel{\reflectbox{\ensuremath{\mapsto}}}}
 
% Divides, Not Divides
\renewcommand{\div}{\bigm|}
\newcommand{\ndiv}{%
    \mathrel{\mkern.5mu % small adjustment
        % superimpose \nmid to \big|
        \ooalign{\hidewidth$\big|$\hidewidth\cr$/$\cr}%
    }%
}

% Convenient overline
\newcommand{\ol}[1]{\ensuremath{\overline{#1}}}

% Big \cdot
\makeatletter
\newcommand*\bigcdot{\mathpalette\bigcdot@{.5}}
\newcommand*\bigcdot@[2]{\mathbin{\vcenter{\hbox{\scalebox{#2}{$\m@th#1\bullet$}}}}}
\makeatother

% Big and small Disjoint union
\makeatletter
\providecommand*{\cupdot}{%
  \mathbin{%
    \mathpalette\@cupdot{}%
  }%
}
\newcommand*{\@cupdot}[2]{%
  \ooalign{%
    $\m@th#1\cup$\cr
    \sbox0{$#1\cup$}%
    \dimen@=\ht0 %
    \sbox0{$\m@th#1\cdot$}%
    \advance\dimen@ by -\ht0 %
    \dimen@=.5\dimen@
    \hidewidth\raise\dimen@\box0\hidewidth
  }%
}

\providecommand*{\bigcupdot}{%
  \mathop{%
    \vphantom{\bigcup}%
    \mathpalette\@bigcupdot{}%
  }%
}
\newcommand*{\@bigcupdot}[2]{%
  \ooalign{%
    $\m@th#1\bigcup$\cr
    \sbox0{$#1\bigcup$}%
    \dimen@=\ht0 %
    \advance\dimen@ by -\dp0 %
    \sbox0{\scalebox{2}{$\m@th#1\cdot$}}%
    \advance\dimen@ by -\ht0 %
    \dimen@=.5\dimen@
    \hidewidth\raise\dimen@\box0\hidewidth
  }%
}
\makeatother


% macros (theorem) -------------------------------------------------------------
\usepackage[hidelinks]{hyperref}
\usepackage[thmmarks,amsmath,hyperref]{ntheorem}
\usepackage[capitalise,nameinlink]{cleveref}

% Numbered Statements
\theoremstyle{change}
\theoremindent\parindent
\theorembodyfont{\itshape}
\theoremheaderfont{\bfseries\boldmath}
\newtheorem{theorem}{Theorem.}[section]
\newtheorem{lemma}[theorem]{Lemma.}
\newtheorem{corollary}[theorem]{Corollary.}
\newtheorem{proposition}[theorem]{Proposition.}

% Claim environment
\theoremstyle{plain}
\theorempreskip{0.2cm}
\theorempostskip{0.2cm}
\theoremnumbering{roman}
\theoremheaderfont{\scshape}
\newtheorem{claim}{Claim}
\AtBeginEnvironment{theorem}{\setcounter{claim}{0}}

% Un-numbered Statements
\theorempreskip{0.1cm}
\theorempostskip{0.1cm}
\theoremindent0.0cm
\theoremstyle{nonumberplain}
\theorembodyfont{\upshape}
\theoremheaderfont{\bfseries\itshape}
\newtheorem{definition}{Definition.}
\theoremheaderfont{\itshape}
\newtheorem{example}{Example.}
\newtheorem{remark}{Remark.}

% Proof / solution environments
\theoremseparator{}
\theoremheaderfont{\hspace*{\parindent}\scshape}
\theoremsymbol{$//$}
\newtheorem{solution}{Sol'n}
\theoremsymbol{$\blacksquare$}
\theorempostskip{0.4cm}
\newtheorem{proof}{Proof}
\theoremsymbol{}
\newtheorem{nmproof}{Proof}


% macros (algebra) -------------------------------------------------------------
\DeclareMathOperator{\Ann}{Ann}
\DeclareMathOperator{\Aut}{Aut}
\DeclareMathOperator{\chr}{char}
\DeclareMathOperator{\coker}{coker}
\DeclareMathOperator{\disc}{disc}
\DeclareMathOperator{\End}{End}
\DeclareMathOperator{\Fix}{Fix}
\DeclareMathOperator{\Frac}{Frac}
\DeclareMathOperator{\Gal}{Gal}
\DeclareMathOperator{\GL}{GL}
\DeclareMathOperator{\Hom}{Hom}
\DeclareMathOperator{\id}{id}
\DeclareMathOperator{\im}{im}
\DeclareMathOperator{\lcm}{lcm}
\DeclareMathOperator{\Nil}{Nil}
\DeclareMathOperator{\rank}{rank}
\DeclareMathOperator{\Res}{Res}
\DeclareMathOperator{\Spec}{Spec}
\DeclareMathOperator{\spn}{span}
\DeclareMathOperator{\Stab}{Stab}
\DeclareMathOperator{\Tor}{Tor}

% Lagrange symbol
\newcommand{\lgs}[2]{\ensuremath{\left(\frac{#1}{#2}\right)}}

% Quotient (larger in display mode)
\newcommand{\quot}[2]{\mathchoice{\left.\raisebox{0.14em}{$#1$}\middle/\raisebox{-0.14em}{$#2$}\right.}
                                 {\left.\raisebox{0.08em}{$#1$}\middle/\raisebox{-0.08em}{$#2$}\right.}
                                 {\left.\raisebox{0.03em}{$#1$}\middle/\raisebox{-0.03em}{$#2$}\right.}
                                 {\left.\raisebox{0em}{$#1$}\middle/\raisebox{0em}{$#2$}\right.}}


% macros (analysis) ------------------------------------------------------------
\DeclareMathOperator{\M}{{\mathcal{M}}}
\DeclareMathOperator{\B}{{\mathcal{B}}}
\DeclareMathOperator{\ps}{{\mathcal{P}}}
\DeclareMathOperator{\pr}{{\mathbb{P}}}
\DeclareMathOperator{\E}{{\mathbb{E}}}
\DeclareMathOperator{\supp}{supp}
\DeclareMathOperator{\sgn}{sgn}

\renewcommand{\Re}{\ensuremath{\operatorname{Re}}}
\renewcommand{\Im}{\ensuremath{\operatorname{Im}}}
\renewcommand{\d}[1]{\ensuremath{\operatorname{d}\!{#1}}}
\newcommand{\trleq}{\trianglelefteq}
\newcommand{\trgeq}{\trianglerighteq}


% file-specific preamble -------------------------------------------------------
\DeclareMathOperator{\conj}{conj}
\usepackage{showkeys}
\makeatletter
\DeclareRobustCommand{\rvdots}{%
    \vbox{
        \baselineskip4\p@\lineskiplimit\z@
        \kern-\p@
        \hbox{.}\hbox{.}\hbox{.}
}}
\makeatother


% constants --------------------------------------------------------------------
\newcommand{\subject}{Introduction to Galois Theory}
\newcommand{\semester}{Winter 2019}
\renewcommand{\textbf}[1]{{\boldmath\bfseries#1}}


% formatting -------------------------------------------------------------------
% Fonts
\usepackage{kpfonts}
\usepackage{dsfont}

% Equation numbering
\numberwithin{equation}{section}

% Footnote
\setfootins{0.5cm}{0.5cm} % footer space above
\renewcommand*{\thefootnote}{\fnsymbol{footnote}} % footnote symbol

% Table of Contents
\renewcommand{\thechapter}{\Roman{chapter}}
\counterwithout{section}{chapter}
\counterwithin*{chapter}{part}
\renewcommand*{\cftchaptername}{Chapter } % Place 'Chapter' before roman
\setlength\cftchapternumwidth{4em} % Add space before chapter name
\cftpagenumbersoff{chapter} % Turn off page numbers for chapter

% Section / Subsection headers
\newcommand*{\shortcenter}[1]{%
    \sethangfrom{\noindent ##1}%
    \Large\boldmath\scshape\bfseries
    \centering
\parbox{5in}{\centering #1}\par}
\setsecheadstyle{\shortcenter}
\setsubsecheadstyle{\large\scshape\boldmath\bfseries\raggedright}

% Chapter Headers
\chapterstyle{verville}

% Page Headers / Footers
\setsecnumdepth{section}
\copypagestyle{myruled}{ruled} % Draw formatting from existing 'ruled' style
\makeoddhead{myruled}{}{}{\scshape\subject}
\makeevenfoot{myruled}{}{\thepage}{}
\makeoddfoot{myruled}{}{\thepage}{}
\pagestyle{myruled}

% Titlepage
\title{\subject}
\author{Alex Rutar\thanks{\itshape arutar@uwaterloo.ca}\\ University of Waterloo}
\date{\semester\thanks{Last updated: \today}}

\begin{document}
\pagenumbering{gobble}
\hypersetup{pageanchor=false}
\maketitle
\newpage
\frontmatter
\hypersetup{pageanchor=true}
\tableofcontents*
\newpage
\mainmatter


% main document ----------------------------------------------------------------
\chapter{Structure of Finite Groups}
\section{Group Quotients}
\subsection{Universal Property of Quotients}
Let $H\trianglelefteq G$ be a normal subgroup of $G$, and let $\pi:G\to G/H$ be the natural projection map.
This map has the following universal property:
\begin{theorem}[Universal Property of Quotients]
    Let $\phi:G\to G'$ be a homomorphism.
    If $H\subset\ker(\phi)$, there is a unique homomorphism $\overline{\phi}:G/H\to G'$ so that $\phi=\overline{\phi}\circ\pi$.

    In particular, $\ker(\overline{\phi})=\ker(\phi)/H$ and $\im(\overline{\phi})=\im(\phi)$.
\end{theorem}
One can rephrase this universal property as follows.
Suppose $\phi:G\to G'$ is a homomorphism of groups and $H\trianglelefteq G$ is a normal subgroup.
If $H\leq\ker(\phi)$, then $\phi$ induces a homomorphism $\overline{\phi}:G/H\to G'$ given by $xH\mapsto\phi(x)$ such that $\ker(\overline{\phi})=\ker(\phi)/H$, $\im(\overline{\phi})=\im(\phi)$.
\begin{proof}
    Define $\overline{\phi}(xH)=\phi(x)$.
    Then $\overline{\phi}\circ\pi(g)=\overline{\phi}(gH)=\phi(g)$, so $\overline{\phi}\circ\pi=\phi$.
    This map is well-defined: suppose $xH=yH$.
    Then $y^{-1}x\in H$, so $\phi(y^{-1}x)=0$ since $H\leq\ker(\phi)$.
    Thus
    \begin{equation*}
        \overline{\phi}(xH)=\phi(x)=\phi(yy^{-1}x)=\phi(y)\phi(y^{-1}x)=\phi(y)=\overline{\phi}(yH)
    \end{equation*}
    so $\overline{\phi}$ is well-defined.

    To see that $\overline{\phi}$ is unique, let $\psi$ satisfy the universal property as well, so $\psi\circ\pi=\phi$.
    In particular, $\phi(h)=\psi\circ\pi(g)=\psi(gN)$, so $\psi(gN)=\overline{\phi}(gN)$ so $\overline{\phi}$ is unique.

    $\overline{\phi}$ is a homomorphism since $\phi$ is:
    \begin{equation*}
        \overline{\phi}((aH)(bH))=\overline{\phi}((ab)H)=\phi(ab)=\phi(a)\phi(b)=\overline{\phi}(aH)\overline{\phi}(bH)
    \end{equation*}

    Finally,
    \begin{equation*}
        xH\in\ker(\overline{\phi})\iff\overline{\phi}(xH)=0\iff\phi(x)=0\iff x\in\ker(\phi)
    \end{equation*}
\end{proof}
\begin{corollary}[First Isomorphism]
    Suppose $\phi:G\to H$ is a surjective homomorphism.
    Then $G/\ker(\phi)\cong H$.
\end{corollary}
\begin{proof}
    Take $H=\ker(\phi)$, so $\overline{\phi}:G/\ker(\phi)\to H$ is surjective since $\im(\overline{\phi})=\im(\phi)=H$ and injective since $\ker(\overline{\phi})=\ker(\phi)/\ker(\phi)=\{1\}$.
\end{proof}
\subsection{Correspondence Theorem}
\begin{theorem}
    Let $\phi:G\to G'$ be a homomorphism of groups.
    $\phi$ induces two maps on the set of subgroups $\Gamma$ and $\Gamma'$ of $G$ and $G'$ respectively:
    \begin{align*}
        \phi_*:\Gamma\to\Gamma'\text{ given by }&\phi_*(H)=\phi(H)\\
        \phi^*:\Gamma'\to\Gamma\text{ given by }&\phi^*(H')=\phi^{-1}(H')
    \end{align*}
    Then $\phi_*\circ\phi^*(H')=H'\cap\im(\phi)$ and $\phi^*\circ\phi_*(H)=\langle H,\ker(\phi)\rangle$.
\end{theorem}
Recall that $H'\cap\im(\phi)$ is the largest subgroup of $H'$ contained in $\im(\phi)$, and $\langle H,\ker(\phi)\rangle$ is the smallest group containing $H$ and $\ker(\phi)$.
\begin{corollary}
    Let $G$ be a group and $N\trianglelefteq G$.
    Then the quotient map $\pi:G\to G/N$ is a bijection from the set of subgroups of $G$ containing $N$ to the set of subgroups of $G/N$.
\end{corollary}
\begin{proof}
    Recall that $\pi$ is a group homomorphism, and $\ker(\phi)=N$ and $\im(\phi)=G/N$.
    Then $\pi_*\circ\pi^*(H')=H'\cap\im(\pi)=H'$ and $\pi^*\circ\pi_*(H)=\langle H,\ker(\pi)\rangle=H$ so $\pi$ is a bijection.
\end{proof}
\section{Group Actions}
\begin{definition}
    We say that a group \mbf{$G$ acts on a set $X$} if there is a map $G\times X\to X$ satisfying $g(hx)=(gh)x$ and $1x=x$.
\end{definition}
Equivalently, an action of $G$ on $X$ is a map $g\mapsto\pi_g$, which assigns to each $g\in G$ a permutation $\pi_G\in S_X$ which respects the operation of $G$; that is to say, if $g,h\in G$, then $\pi_{gh}=\pi_g\circ\pi_h$.
In other words, an action of $G$ on $X$ is a homomorphism $\pi:G\to S_X$.

The action is often written in multiplicative form: we say $\pi_g(a)=b$ and can write $g\cdot a=b$, with $a,b\in X$ and $g\in G$.
\begin{example}
    The most classic example of a group action is the action of $G$ on itself by conjugation.
    For each $g\in G$, define the map $\phi_g:G\to G$ given by $\phi_g(x)=gxg^{-1}$.
    Since $\phi_g$ is an automorphism, it is certainly a permutation, and for any $g,h\in G$,
    \begin{equation*}
        \phi_{gh}(x)=(gh)x(gh)^{-1}=g(hgh^{-1})g^{-1}=\phi_g\circ\phi_h(x)
    \end{equation*}
\end{example}
\begin{definition}
    Let $\pi$ be an action of $G$ on $X$.
    \begin{enumerate}[nolistsep]
        \item The \mbf{kernel} of the action is the kernel of $\pi$ as a homomorphism $G\to S_X$; in other words, the set $\{g\in G:g\cdot a=a\text{ for all }a\in X\}$.
        \item The action is \mbf{faithful} if the kernel is $\{1\}$ (equivalently, if $\pi$ is injective).
        \item Given $a\in X$, the \mbf{orbit} of $a$ is the set $G\cdot a=\{g\cdot a:g\in G\}$
    \end{enumerate}
\end{definition}
If $G$ acts faithfully on $X$, then $G$ is isomorphic to a subgroup of $S_X$ with isomorphism given by $\pi$.
\begin{proposition}
    Let $G$ act on $X$.
    The orbits of the action partition $X$.
\end{proposition}
\begin{proof}
    The orbits clearly cover $X$ since $a\in G\cdot x$ for any $a\in X$.
    Suppose $G\cdot a$ and $G\cdot b$ are orbits.
    Either they or disjoint, or $x\in G\cdot a\cap G\cdot b$.
    Thus get $g,h$ so that $x=g\cdot a=h\cdot b$.
    But
    \begin{equation*}
        (g^{-1}h)\cdot b=g^{-1}\cdot(h\cdot b)=g^{-1}\cdot(g\cdot a)=(g^{-1}g)\cdot a=1\cdot 1=a
    \end{equation*}
    so $a\in G\cdot b$.
    Thus $G\cdot a\subseteq G\cdot b$; the reverse inclusion follows identically, so $G\cdot a=G\cdot b$.
\end{proof}
\begin{definition}
    An action of $G$ on $X$ is \mbf{transitive} if it has only one orbit, $X$.
\end{definition}
\begin{definition}
    Let $\pi$ be an action of $G$ on $X$.
    Given $a\in X$, the \mbf{stabilizer} of $a$ is the set $G_a=\{g\in G:g\cdot a=a\}$.
\end{definition}
\begin{proposition}[Orbit-Stabilizer]
    Suppose $G$ acts on $X$.
    For every $a\in X$,
    \begin{enumerate}[nolistsep,label=(\roman*)]
        \item $G_a\leq G$
        \item $|G\cdot a|=[G:G_a]$
    \end{enumerate}
    Hence if $G$ is finite, then every orbit has size dividing $G$.
\end{proposition}
\begin{proof}
    \begin{enumerate}
        \item It suffices to show that $G_a$ is closed under multiplication and inverses.
            Let $g,h\in G_a$.
            Then $(gh)\cdot a=g\cdot(h\cdot a)=g\cdot a=a$, so $gh\in G_a$.
            Similarly, $g^{-1}\cdot a=g^{-1}\cdot(g\cdot a)=(g^{-1}g)\cdot a=1\cdot a=1$.
        \item Let $g,h$ be arbitrary.
            Then
            \begin{align*}
                g\cdot a=h\cdot a &\iff h^{-1}\cdot(g\cdot a)=h^{-1}\cdot(h\cdot a)\\
                                  &\iff (h^{-1}g)\cdot a=a\\
                                  &\iff h^{-1}g\in G_a\\
                                  &\iff hG_a=gG_a
            \end{align*}
            so that $g\cdot a$ depends only on $gG_a$.
            Thus the number of distinct values of $g\cdot a$ equals the number of left cosets of $G_a$.
    \end{enumerate}
\end{proof}
\subsection{Conjugation and the Class Equation}
Recall the action of $G$ on itself by conjugation: the maps $\phi_g$ are given by $\phi_g(x)=gxg^{-1}$.
\begin{definition}
    The \mbf{conjugacy class} of an element $a\in A$ is the set $G\cdot a=\{gag^{-1}:g\in G\}:=\conj(a)$.
\end{definition}
By general properties of group actions, $G$ is partitioned by its conjugacy classes, and $|\conj(g)|=[G:G_a]$.
In particular, when $G$ is finite, $|\conj(a)|\div|G|$ for any $g\in G$.
Furthermore, the stabilizer $G_a$ satisfies
\begin{equation*}
    G_a = \{g\in G:g\cdot a=a\}=\{g\in G:gag^{-1}=g\}=\{g\in G:ga=ag\}=C_G(a)
\end{equation*}
which is the centralizer of $a$ in $G$.
We thus have that $|\conj(g)|=[G:C_G(g)]$.

What happens when $\conj(g)=\{g\}$?
In this case, we say that $g$ is \mbf{central} (and otherwise call the conjugacy classes \mbf{non-central}).
In this special case,
\begin{align*}
    |\conj(g)|=1 &\iff [G:C_G(g)]=1\\
                 &\iff G=C_G(g)\\
                 &\iff ga=ag\forall a\in G\\
                 &\iff g\in Z(G)
\end{align*}
Thus $G$ is the disjoint union of $Z(G)$ and its non-central conjugacy classes.
In particular, if $a_1,\ldots,a_m$ are representatives of the non-central conjugacy classes, we have
\begin{equation*}
    |G|=|Z(G)|+\sum\limits_{i=1}^m|\conj(a_i)|=|Z(G)+\sum\limits_{i=1}^m[G:C_G(a_i)]
\end{equation*}
\subsection{Conjugation Action on Subgroups}
Let $G$ be a group, $P,Q\leq G$ be subgroups.
Let $\mathcal{K}$ denote the set of conjugates of $P$ in $G$.
\begin{proposition}
    For any $A\in\mathcal{K}$, $A\leq G$.
    If $A,B\in\mathcal{K}$, then $|A|=|B|$.
\end{proposition}
In other words, $\mathcal{K}$ is composed of subgroups of $G$ conjugate to $P$, all of which have the same size as $P$.
\begin{proof}
    If $a,b\in hPh^{-1}$, then $a=hp_1h^{-1}$, $b=hp_2h^{-1}$ so $ab=h(p_1p_2)h^{-1}\in hPh^{-1}$.
    Similarly, $a^{-1}=(hp_1h^{-1})^{-1}=hp_1^{-1}h^{-1}\in hPh^{-1}$ as well.
    
    To see that $|A|=|B|$, since $A,B$ are conjugate, get $x$ so $B=xAx^{-1}$.
    The map $\alpha:A\to B$ given by $a\mapsto xax^{-1}$ is a bijection.
    It is injective, since if $xa_1x^{-1}=xa_2x^{-1}$ then $a_1=a_2$; and it is surjective, since if $b\in B$, get $a\in A$ so $xax^{-1}=b$.
\end{proof}
Given this setup, $Q$ acts on $\mathcal{K}$ by conjugation: for $g\in Q$ and $hPh^{-1}\in\mathcal{K}$, we define $g\cdot hPh^{-1}=g(hPh^{-1})g^{-1}=(gh)P(gh)^{-1}\in\mathcal{K}$.

The orbits are equivalence classes of conjugates of $P$, where $h_1Ph_1^{-1}\sim h_2Ph_2^{-1}$ if they are conjugate by some element of $Q$.

Recall that $N_G(H)=\{g\in G:gHg^{-1}=H\}$; note that $N_G(H)$ is the largest subgroup of $G$ containg $H$ as a normal subgroup.
Then the stabilizers are given by $Q_{P_i}=\{q\in Q:qP_iq^{-1}=P_i\}=N_G(P_i)\cap Q$.
\section{Structure of Finitely Generated Abelian Groups}
\section{Sylow Theorems}
Lagrange's theorem, that says that the order of any subgroup of a group $G$ must divide its order.
From the previous section, for finite abelian $G$, if $m\div|G|$ is any factor, then $G$ has a subgroup of order $m$.
This does not necessarily hold for groups which are not abelian.
\begin{proposition}
    There exists a group $G$ and $m\div|G|$ so there is no subgroup of $G$ with order $m$.
\end{proposition}
\begin{proof}
    Take $G=A_4$, so $|G|=12$.
    I claim that $H$ has no group of order $6$.
    For contradiction, suppose $H\leq G$ and $|H|=6$.
    Let $a\in G$ such that $|a|=3$; there are 8 such elements.
    Consider the cosets $H$, $aH$, $a^2H$.
    Since $[G:H]=2$, there are 3 cases:
    \begin{itemize}[nolistsep]
        \item $aH=H$, so $a\in H$
        \item $aH=a^2H$, so $H=aH$ and $a\in H$
        \item $a^2H=H$ so $H=aH$ and $a\in H$, since $a^3=1$.
    \end{itemize}
    Thus all 8 elements of order 3 are in $H$, contradiction.
\end{proof}
While in general these subgroups do not exist, a partial converse is given by the First Sylow Theorem.
\subsection{Sylow $p-$groups}
\begin{definition}
    Let $p$ be a prime.
    We say that a group $G$ is a \mbf{p-group} if $|G|=p^k$, $k\in\N$.
    If $H\leq G$ is a p-group, we say that $H$ is a \mbf{p-subgroup}.
    If $|H|=p^k||G|$ with $k$ maximal, then we say that $G$ is a \mbf{Sylow p-subgroup of $G$}.
\end{definition}
Before we prove the First Sylow Theorem, let's recall Cauchy's Theorem.
Some standard proofs resort to the class equation; here, I will present a different alternative approach.
\begin{theorem}[Cauchy]
    Let $G$ be a finite group and let $p\div|G|$ be prime.
    If $r$ is the number of solutions to the equation $x^p=1$, then $p|r$.
\end{theorem}
\begin{proof}
    Let $|G|=n$, $p|n$ prime, and define
    \begin{equation*}
        S=\{(a_1,a_2,\ldots,a_p):a_i\in G,a_1a_2\cdots a_p=1\}
    \end{equation*}
    and note that $|S|=n^{p-1}$.
    Define $\sim$ on $S$ by $a\sim b$ if $a$ and $b$ are cyclic permutations of each other.

    If all components of a $p-$tuple are equal, then its equivalence class has 1 member.
    Otherwise, its equivalence class has $p$ members.

    If $r$ denotes the number of solutions to $x^p=1$, then $r$ is equal to the number of equivalence classes with exactly 1 member.
    Let $s$ denote the number of equivalence classes with $p$ members; then, $r+ps=n^{p-1}$ and since $p|n$, $p|r$ as well.
\end{proof}
\begin{corollary}
    If $p\div |G|$ is prime, then there exists $H\leq G$ with $|H|=p$.
\end{corollary}
\begin{proof}
    By Cauchy's Theorem, there is at least one non-trivial solution to the equation $x^p=1$.
    Let $g$ be such an element; then $H=\langle g\rangle\leq G$ has order $p$.
\end{proof}
In a sense, Cauchy's Theorem provides a partial converse to Lagrange's Theorem.
However, the First Sylow Theorem is a strengthening of this claim.
In particular, Cauchy's Theorem follows as an easy corollary.
\begin{theorem}[First Sylow]
    Let $G$ be a finite group and let $p$ be a prime dividing its order.
    Then $G$ contains a Sylow $p-$subgroup.
\end{theorem}
\begin{proof}
    The proof follows by induction on $|G|$.
    If $|G|=2$, then $G$ is its own Sylow 2-subgroup.
    If $|G|\geq 2$ is finite, let $p\div|G|$, and say $|G|=p^nm$ where $p\ndiv m$.

    Case 1: $p\div|Z(G)|$.
    By Cauchy, there exists $a\in Z(G)$ so that $o(a)=p$.
    Since $\langle a\rangle\subseteq Z(G)$, $\langle a\rangle\trianglelefteq G$.
    If $n=1$, we are done; otherwise, by induction, $G/\langle a\rangle$ has a Sylow $p-$subgroup $\overline{H}$.
    By correspondence, $\overline{H}=H/\langle a\rangle$ for some $H\leq G$.
    Thus, $p^{n-1}=|H|/p$, so $|H|=p^n$ and $H$ is a Sylow $p-$subgroup of $G$.

    Case 2: $p\ndiv|Z(G)|$.
    By the Class equation, there is some $a_i$ so that $p\nmid[G:C_G(a_i)]=|G|/|C_G(a_i)|$.
    Thus $p^n\div|C_G(a_i)|$ where $a_i$ is non-central.
    Since $a_i\notin Z(G)$, $|C_G(a_i)|<|G|$.
    By induction, $C_G(a_i)$ has a Sylow $p-$subgroup, which is also a Sylow $p-$subgroup of $G$.
\end{proof}
\subsection{Structure of Sylow $p-$subgroups}
Let $G$ be a group and suppose $H\leq G$.
\begin{lemma}
    Suppose $p\div|G|$, $P$ is a Sylow $p-$subgroup of $G$, and $Q$ is a $p-$subgroup of $G$.
    Then $Q\cap N_G(P)=Q\cap P$.
\end{lemma}
\begin{proof}
    Since $P\subseteq N_G(P)$, $P\cap Q\subseteq N_G(P)\cap Q$.
    For notation, set $N=N_G(P)$ and $H=N_G(P)\cap Q$.
    It remains to show $H\subseteq P\cap Q$.

    Write $|P|=p^n$ and $|H|=p^m$.
    Since $P\trianglelefteq N$, $HP\leq N$.
    Thus
    \begin{equation*}|HP|=\frac{|H|\cdot|P|}{|H\cap P|}=p^k,k\leq n\end{equation*}
    As well, $P\subseteq HP$ so $n\leq k$, and $P=HP$.
    Thus $H\subseteq HP=P$.
\end{proof}
\begin{lemma}
    Let $G,p,P,Q$ be as in the previous lemma, and let $\mathcal{K}$ denote the set of conjugates of $P$ in $G$.
    Let $Q$ act on $\mathcal{K}$ by conjugation, so the orbits have representatives $P=P_1,P_2,\ldots,P_r$.
    Then, $|\mathcal{K}|=\sum_{i=1}^r[Q:Q\cap P_i]$.
\end{lemma}
\begin{proof}
    By the Orbit-Stabilizer lemma,
    \begin{align*}
        |\mathcal{K}| &= \sum_{i=1}^r|Q\cdot P_i| = \sum\limits_{i=1}^r[Q:Q_{P_i}]\\
                      &= \sum\limits_{i=1}^r[Q:N_G(P_i)\cap Q]\\
                      &= \sum\limits_{i=1}^r[Q:P_i\cap Q]
    \end{align*}
    where the last line follows from the previous lemma.
\end{proof}
\begin{theorem}[Second Sylow]
    If $P$ and $Q$ are Sylow $p-$subgroups of $G$, then there exists $g\in G$ so that $P=gQg^{-1}$.
\end{theorem}
Since the conjugation action preserves the order of groups, the Sylow $p-$subgroups of $G$ are precisely the equivalence class of any Sylow $p-$subgroup of $G$.
\begin{proof}
    Let $\mathcal{K}$ be the set of conjugates of $P$ in $G$, and let $P$ act on $\mathcal{K}$ by conjugation.
    Recall that for $P_i,P_j\in\mathcal{K}$, $|P_i|=|P_j|$.

    Let $P=P_1,P_2,\ldots,P_r$ be orbit represntatives.
    Then by the Lemma above,
    \begin{equation*}
        |\mathcal{K}|=\sum_{i=1}^r[P:P\cap P_i]=1+\sum\limits_{i=2}^r[P:P_i\cap P]\equiv 1\pmod{p}
    \end{equation*}
    since $p\div[P:P_i\cap P]$: this follows since $P_i\cap P\lneq P$ and $|P|=p^n$.

    Now let $Q$ act on $K$ by conjugation.
    Reindexing if necessary, let the orbits have representatives $P=P_1,P_2,\ldots,P_s$.
    If $Q\neq P_i$ for $i=1,2,\ldots,s$, then by the same argument as above, $|\mathcal{K}|=\sum_{i=1}^s[Q:P_i\cap Q]\equiv 0\pmod{p}$, a contradiction.
    Thus $Q=P_i$ and so $Q$ is a conjugate of $P$.
\end{proof}
Now Sylow's third theorem follows easily:
\begin{theorem}[Third Sylow]
    Let $p\div|G|$ be prime, $|G|=p^nm$ with $\gcd(p,m)=1$, and $n_p$ denote the number of Sylow $p-$subgroups of $G$.
    Then if $P$ is any Sylow $p-$subgroup of $G$,
    \begin{enumerate}[nolistsep]
        \item $n_p\equiv 1\pmod{p}$
        \item $n_p=[G:N_G(P)]$
    \end{enumerate}
    In particular, $n_p|m$, and $n_p=1$ if and only if $N_G(P)=G$; in other words, that $P$ is a normal subgroup of $G$.
\end{theorem}
\begin{proof}
    Let $P$ be a Sylow $p-$subgroup of $G$ and let $\mathcal{K}$ be the set of conjugates of $P$ in $G$.
    From the proof of Sylow's second theorem, $n_p=|\mathcal{K}|\equiv 1\pmod{p}$.

    Now let $G$ act on $K$ by conjugation so $\mathcal{K}=G\cdot P$.
    By the Orbit-Stabilizer theorem, $|G|=|G_P|\cdot|G\cdot P|$.
    Since $G_P=N_G(P)\cap G=N_G(P)$, $p^nm=|N_G(P)|\cdot n_p$.
    Thus $n_p|p^nm$, and since $n_p\nequiv 0\pmod{p}$, $n_p|m$.
\end{proof}
\begin{remark}
    $\disc f(x)$ is not a square in $F$ iff $\Gal f(x)\nsubseteq A_2$ iff $\Gal f(x)=S_2$ iff $f(x)$ is irreducible.
\end{remark}
\begin{example}
    Prove that there is no simple group of order 56.

    Note that $56=2^3\cdot 7$.
    Since $n_7\equiv 1\pmod{7}$ and $n_7|8$, we have $n_7\in\{1,8\}$.
    If $n_7=1$, then $G$ has a normal Sylow $7-$subgroup.
    By Lagrange, distinct Sylow 7-subgroups intersect trivially.
    Thus there are $8\cdot 6=48$ elements of order $7$ in $G$.
    This forces $n_2=1$.
    In either case, $G$ is not simple.
\end{example}
\begin{remark}
    If $p\neq q$ are prime, $p,q\div|G|$.
    Then if $H_p,H_q$ are $p-$ and $q-$subgroups, then $H_p\cap H_q=\{1\}$.
    Similarly, if $|G|=pm$ and $H,K$ are Sylow $p-$subgroups, then $H=K$ or $H\cap K=\{1\}$.
\end{remark}
\begin{example}
    If $|G|=pq$, where $p,q$ prime, $p<q$, $p\ndiv q-1$.
    Then $G$ is cyclic.

    Since $n_p\equiv 1\pmod{p}$ and $n_p\div|q$.
    We cannot have $n_p=q$, so $G$ has a normal Sylow $p-$subgroup $H_p$.
    Since $p<q$, $q\ndiv p-1$, so $n_q=1$ and $G$ has a normal Sylow $q-$subgroup $H_q$, say $H_q$.
    Since $H_p\cap H_q=\{1\}$, $G\cong H_p\times H_q\cong\Z_{pq}$ since $p,q$ are coprime.
\end{example}
\begin{example}
    If $|G|=30$, then $G$ has a subgroup isomorphic to $\Z_{15}$.
    Since $n_5\equiv 1\pmod{5}$ and $n_5|6$, $n_5\in\{1,6\}$.
    Similarly, $n_3\equiv 1\pmod{3}$, and $n_3|10$, so $n_3\in\{1,10\}$.
    By counting elements, at least one must be normal.
    Let $H_3,H_5$ be Sylow subgroups.
    Since $3\ndiv 5-1$, $Z_{15}\cong H_3H_5\leq G$ by the previous example.
\end{example}
\begin{example}
    If $|G|=60$, $n_5>1$, then $G$ is simple.
    Since $|G|=60$, $n_5\equiv 1\pmod{5}$ and $n_5|12$, we must have $n_5=6$ (accounting for 25 elements).
    Suppose $N\trianglelefteq G$.

    Case 1: $5\div|H|$.
    Then $H$ contains a Sylow $5-$subgroup of $G$.
    Since $H$ is normal, $H$ contains all conjugate other Sylow 5-subgroups, so $|H|\geq 25$ and $|H|=30$.
    By the previous example, $n_5=1$ since $\Z_{15}$ has only 1 Sylow 5-subgroup.

    Case 2: $|H|\in\{2,3,4,6,12\}$.
    If $|H|=12$, $H$ has a normal Sylow 2- or 3-subgroup, which is normal in $G$.
    Call it $K$.
    If $|H|=6$, then $H$ has a normal Sylow 3-subgroup which is normal in $G$.
    Call it $K$.
    By replacing $H$ with $K$ if necessary, we may assume $|H|\in\{2,3,4\}$.
    Consider $\overline{G}=G/H$.
    Then $|\overline{G}|=\{15,20,30\}$.
    In any case, $\overline{G}$ has a normal Sylow 5-subgroup; call it $\overline{P}$.
    By correspondence, $\overline{P}=P/H$.
    $P$ is a normal subgroup of $G$, so $P$ is a proper, non-trivial normal subgroup of $G$.
    As well, $|P|=|\overline{P}|\cdot|H|=5$, so $5\div|H|$ and $5\div|P|$.
    This contradicts Case 1.
\end{example}
\begin{example}
    $A_5$ is simple since $|A_5|=60$ and $\langle(12345)\rangle$, $\langle(13245)\rangle$ are distinct Sylow 5-subgroups.
\end{example}
\chapter{Fields}
\section{Irreducible Polynomials}
\begin{definition}
    Let $R$ be an integral domain.
    We say $f(x)\in R[x]$ is \mbf{irreducible} over $R$ if $f$ is a non-unit, non-irreducible, and whenever $f(x)=g(x)h(x)$, then either $g$ is a unit or $h$ is a unit.
    Otherwise, $f$ is \mbf{reducible}.
\end{definition}
\begin{remark}
    A canonical way to construct new fields as follows.
    Suppose $F$ be a field and $I$ an ideal of $F[x]$.
    Since $F[x]$ is a PID ($F[x]$ has a division algorithm), then $I=\langle p(x)\rangle$, $p(x)\in F[x]$.
    Moreover, $I$ is maximal if and only if $p(x)$ is irreducible.
    Thus $F[x]/I$ is a field if and only if $p(x)$ is irreducible.
\end{remark}
\begin{proposition}
    Let $F$ be a field.
    If $f(x)\in F[x]$, $\deg f(x)>1$ and $f(x)$ has a root in $F$, then $f(x)$ is reducible over $F$.
    In particular, if $\deg f(x)\in\{2,3\}$, then $f(x)$ is irreducible over $F$ if and only if $f$ has no roots in $F$.
\end{proposition}
\begin{proof}
    By the division algorithm, $f(x)=(x-a)q(x)+r(x)$ where $\deg r(x)\leq 1$.
    Then $f(x)=0+r=r$, so $f(x)=(x-a)q(x)+f(a)$, so $(x-a)\div f(x)$ if and only if $f(a)=0$.
    From this, the first claim follows immediately.

    For the second claim, if $g(x)|f(x)$, then either $\deg g=\deg f$, $\deg g=2$, or $\deg g=1$.
    If every divisor has the same degree as $f$, then $f$ is irreducible; otherwise, $f$ has a factor of degree 1 and the claim follows by the initial observation.
\end{proof}
\begin{lemma}[Gauss' Lemma]
    Let $R$ be a UFD with field of fractions $F$.
    Let $p(x)\in R[x]$.
    If $p(x)=A(x)B(x)$ with $A(x),B(x)$ non-constant in $F[x]$, then there exists $r\in F^\times$ such that $a(x)=rA(x),b(x)=r^{-1}B(x)\in R[x]$.
\end{lemma}
\begin{proof}
    PMATH 347.
\end{proof}
\begin{remark}
    Gauss' Lemma states that if $p(x)\in R[x]$ is reducible over $F$, then $p(x)$ is reducible over $R$.
    In particular, if $p(x)$ is irreducible over $\Z$, then $p(x)$ is irreducible over $\Q$ as well.
\end{remark}
Let $R$ be an integral domain and $I$ a proper ideal.
If $p(x)\in R[x]$ with coefficients $a_i$, then $\overline{p}(x)\in (R/I)[x]$ with coefficients $a_i+I$.
The map $p(x)\mapsto\overline{p}(x)$ is a ring homomorphism.
\begin{proposition}
    Let $I$ be a proper ideal of an integral domain $R$, and $p(x)\in R[x]$ non-constant and monic.
    If $\overline{p}(x)$ cannot be factored in $(R/I)[x]$ into polynomials of lesser degree, then $p(x)$ is irreducible in $\Frac(R)[x]$.
\end{proposition}
\begin{proof}
    Suppose $p(x)$ is reducible over $\Frac(R)$; by Gauss' Lemma, write $p(x)=f(x)g(x)$ is a non-trivial factorization over $R[x]$ with $\deg f,\deg g<\deg p$.
    Without loss of generality, $f(x)$ and $g(x)$ are also monic.
    Thus, in $(R/I)[x]$, $\overline{p}(x)=\overline{f}(x)=\overline{g}(x)$.
    Since $I\subsetneq R$, $1\notin I$, so $\deg\overline{f}=\deg f$, $\deg\overline{g}=\deg g$, $\deg\overline{p}=\deg p$ and $\overline{f}=\overline{g}\overline{h}$ is a non-trivial factorization.
\end{proof}
\begin{corollary}
    Let $f(x)\in\Z[x]$, $\deg f(x)\geq 1$.
    Let $p\in\Z$ be a prime.
    If $\overline{f}(x)\in\Z_p[x]$ such that $\deg f(x)=\deg\overline{f}(x)$ and $\overline{f}(x)$ is irreducible over $\Z_p$, then $f(x)$ is irreducible over $\Q$.
\end{corollary}
\begin{proof}
    Take $R=\Z$, $I=(p)$ in the previous lemma.
\end{proof}
\begin{proposition}[Eisenstein's Criterion]
    Let $R$ be an integral domain and $P$ a prime ideal of $R$.
    Let $f(x)=x^n+a_{n-1}x^{n-1}+\cdots+a_1x+a_0$.
    If $a_i\in P$ and $a_0\notin P^2$, then $f(x)$ is irreducible over $R$.
\end{proposition}
\begin{proof}
    Suppose $f(x)$ is reducible over $R$.
    Since $f(x)$ is monic, $f(x)=g(x)h(x)$, where $g(x),h(x)\in R[x]$ with $\deg g(x),\deg h(x)<\deg f(x)$.
    Therefore,
    \begin{align*}
        \overline{f}(x) &=\overline{g}(x)\overline{h}(x)\\
                        &= x^n\in (R/P)[x]
    \end{align*}
    Since $P$ is prime, $R/P$ is an integral domain.
    Thus $\overline{g}(0)=\overline{h}(0)=0$ and $g(0),h(0)\in P$, so $a_0=g(0)h(0)\in P^2$.
\end{proof}
\begin{example}
    \begin{enumerate}
        \item $f(x,y)=x^2+y^2-1\in \Q[x,y]$ is irreducible.
            Let $g(y)=y^2+(x^2-1)$, and take $P=\langle x+1\rangle$.
            Since $x+1$ is irreducible, $P$ is a prime ideal of $\Q[x]$.
            Moreover, $x^2-1\in P$ but $(x+1)^2\notin P^2$, so by Eisenstein, $f(x,y)$ is irreducible.
        \item Suppose $f(x)=x^n-d$, where $d$ is not a perfect square.
            Then $f$ is irreducible over $\Q$ by Eisenstein.
        \item $f(x)=x^3+2x+16$.
            Consider modulo 3, $\overline{f}(x)=x^3+2x+1$, which is irreducible by checking $0,1,2$ as roots.
        \item $f(x)=x^4+5x^3+6x^2-1$.
            Then $\overline{f}=x^4+x^3+1\in\Z_2[x]$ is irreducible by checking roots and the unique irreducible quadriatic $x^2+x+1$.
        \item Let $p$ be a prime, and $f(x)=x^{p-1}+x^{p-2}+\cdots+x+1=(x^p-1)/(x-1)$, so
            \begin{equation*}
                f(x+1)=\frac{(x+1)^p-1}{x}=x^{p-1}+\binom{p}{p-1}x^{p-2}+\cdots+\binom{p}{2}x+\binom{p}{1}
            \end{equation*}
            Since $f(x)$ is irreducible if and only if $f(x+a)$ is irreducible, $f(x)$ is irreducible by Eisenstein.
\end{enumerate}
\end{example}

\section{Field Extensions}
\begin{proposition}
    The polynomial ring $F[x]$ has a division algorithm (i.e. it is a Euclidean domain).
    Thus $F[x]$ is a PID.
\end{proposition}
\begin{proof}
    PMATH 347.
\end{proof}
\begin{definition}
    Let $K$ be a field.
    $F\subseteq K$ is a \mbf{subfield} of $K$ if $F$ is a field under the same operations.
    A \mbf{field extension} of $F$ is a field $K$ which contains an isomorphic copy of $F$ as a subfield.
    In this case, we write $K/F$.
    We say  $F_1/F_2/\cdots/F_n$ is a \mbf{tower of fields} if each $F_i/F_{i+1}$ is a field extension.
\end{definition}
\begin{remark}
    Suppose $f(x)\in F[x]$ is irreducible.
    Then $K=\quot{F[x]}{\langle f(x)\rangle}$ contains $F$ in the following natural way: define $\phi:F\to K$ by $\phi(x)=x+\langle f(x)\rangle$.
    It follows that $\phi$ is injective: if $\phi(x)=\phi(y)$, then $x-y\in\langle f(x)\rangle$.
    Since $x-y\in F$ but $\langle f(x)\rangle\neq F[x]$, we must have $x-y=0$ so $x=y$.

    If $\operatorname{char}(F)=p>0$, then there is a natural injection $\Z_p\to F$: consider the map $\phi:\Z\to F$ given by $n\mapsto n\cdot 1_F$; apply the first isomorphism theorem.
\end{remark}
\begin{definition}
    Let $\alpha_1,\ldots,\alpha_n\in K$.
    The \mbf{field extension of $F$ generated by $\alpha_1,\ldots,\alpha_n$} is
    \begin{equation*}
        F(\alpha_1,\ldots,\alpha_n)=\left\{\frac{f(\alpha_1,\ldots,\alpha_n)}{g(\alpha_1,\ldots,\alpha_n)}:f,g\in F[x_1,\ldots,x_n], g(\alpha_1,\ldots,\alpha_n)\neq 0\right\}
    \end{equation*}
\end{definition}
\begin{remark}
    Note that $K/F(\alpha_1,\ldots,\alpha_n)/F$.
\end{remark}
\begin{proposition}\label{prop:f-ext}
    Suppose $K/F$, $\alpha\in K$.
    If $\alpha$ is a root of some non-zero $f(x)\in F[x]$, which is irreducible over $F$, then $F(\alpha)\cong \quot{F[x]}{\langle f(x)\rangle}$.
    Moreover, if $\deg f(x)=n$, then $F(\alpha)=\spn_F\{1,\alpha,\ldots,\alpha^{n-1}\}$.
\end{proposition}
\begin{proof}
    Let $\alpha\in K$ be a root of $f(x)\in F[x]$ with $\deg f(x)=n$.
    Consider the map
    \begin{equation*}
        \phi:F[x]\to F(\alpha),\qquad\phi(g(x))=g(\alpha)
    \end{equation*}
    One can verify that this is a ring homomorphism.
    Set $I=\ker(\phi)$: since $F[x]$ is a PID, $I=\langle g(x)\rangle$; since $f(x)\in I$, $f(x)=g(x)h(x)$ for some $h(x)\in F[x]$.
    Since $I$ is a proper ideal, $g$ is not a unit, so by irreducibility of $f$, $h$ is a unit and $\langle g(x)\rangle=\langle f(x)\rangle$.
    Thus by the first isomorphism theorem, $\quot{F[x]}{\langle f(x)\rangle}\cong \phi(F[x])$ via $h(x)+\langle f(x)\rangle\mapsto h(\alpha)$.

    By definition, $\phi(F[x])\subseteq F(\alpha)$.
    Since $\phi(F[x])$ is a field (up to isomorphism) which contains $\alpha=\phi(x)$ and $F$, $F(\alpha)\subseteq\phi(F[x])$, so equality holds.

    Finally, by the division algorithm,
    \begin{equation*}
        \quot{F[x]}{\brac{f(x)}}=\bigl\{c_{n-1}x^{n-1}+c_{n-2}x^{n-2}+\cdots+c_0+\langle f(x)\rangle,c_i\in F\bigr\}
    \end{equation*}
    Thus $F(\alpha)=\{c_{n-1}\alpha^{n-1}+\cdots+c_a\alpha+c_0:c_i\in F\}=\spn_F\{1,\alpha,\ldots,\alpha^{n-1}\}$.
\end{proof}
\begin{remark}
    Suppose $g\in F[x]$ such that $g(\alpha)=0$.
    Since $F[x]$ is an integral domain, $g$ must have an irreducible factor $f$ with $f(\alpha)=0$.
    In particular,
    \begin{enumerate}[nolistsep]
        \item If $h(x)\in F[x]$, $h(\alpha)=0$ then $h(x)\in\langle f(x)\rangle$ and $f(x)\div h(x)$.
        \item $\langle f(x)\rangle$ contains a unique, monic, irreducible polynomial.
            If $g(x)\in\langle f(x)\rangle$ is irreducible, then $g(x)=uf(x)$.
    \end{enumerate}
\end{remark}
\begin{definition}
    Let $K/F$ be an extension and $\alpha\in K$ a root of a nonzero polynomial in $F[x]$.
    Then, there exists a unique monic irreducible $f(x)\in F[x]$ such that $f(\alpha)=0$.
    We call $f(x)$ the \mbf{minimal polynomial} of $\alpha$ over $F$.
    If $\deg f(x)=n$, then $n$ is the \mbf{degree of $\alpha$ over $F$}.
\end{definition}
\begin{proposition}
    Let $K/F$ and $\alpha\in K$ with minimal polynomial $f(x)\in F[x]$, with $\deg_F(\alpha)=n$.
    Then $\{1,\alpha,\ldots,\alpha^{n-1}\}$ is a basis for $K/F$.
\end{proposition}
\begin{proof}
    That it spans follows from the previous proposition (\cref{prop:f-ext}).
    If the set is linearly dependent, then the coefficients in the dependence relation would give a polynomial $g$ with $g(\alpha)=0$ and $\deg g\leq n-1$, a contradiction.
\end{proof}
\begin{corollary}
    Let $\alpha,\beta\in K$ have the same minimal polynomial $f(x)\in F[x]$.
    Then $F(\alpha)\cong F(\beta)$.
\end{corollary}
\begin{proof}
    This is immediate since $F(\alpha)\cong\quot{F[x]}{\langle f(x)\rangle}\cong F(\beta)$.
\end{proof}
\subsection{Finite Extensions}
\begin{definition}
    We say that $K/F$ is a \mbf{finite extension} if $K$ is a finite dimensional $F-$vector space.
    We call $\dim_F K$ the \mbf{degree} of $K/F$ and denote this dimension by $[K:F]$.
\end{definition}
\begin{theorem}
    If $K/E$ and $E/F$ are extensions, then $[K:F]=[K:E][E:F]$.
\end{theorem}
\begin{proof}
    Let $\{v_1,\ldots,v_n\}$ be a basis for $K/E$ and $\{w_1,\ldots,w_m\}$ a basis for $E/F$.
    Let's show $\{w_iv_j:i\in[n],j\in[m]\}$ is a basis for $K/F$.
    Suppose $\sum_{i,j}c_{ij}v_iw_j=0$.
    Then $\sum_i\left(\sum_j c_{ij}w_j\right)v_i=0$; since the $v_i$ are linearly independent, for each $i$, $\sum_j c_{ij}w_j=0$ is linearly independent.
    It is clear that this sets spans, so it is indeed a basis.
\end{proof}
\begin{definition}
    Let $K/F$ be an extension.
    We say $\alpha\in K$ is \mbf{algebraic over $F$} if it is the root of a non-zero polynomial.
    Otherwise, we say $\alpha$ is \mbf{transcendental over $F$}.
    We say $K/F$ is algebraic if every $\alpha\in K$ is algebraic over $F$.
    Otherwise, we say $K/F$ is transcendental.
\end{definition}
\begin{remark}
    If $\alpha\in K$ is algebraic over $F$, then $\alpha$ has a minimal polynomial in $F[x]$.
\end{remark}
\begin{theorem}
    If $K/F$ is finite, then $K/F$ is algebraic.
\end{theorem}
\begin{proof}
    Suppose $[K:F]=n<\infty$, and let $\alpha\in K$.
    Consider $\alpha,\alpha^2,\ldots,\alpha^{n+1}$.
    If $\alpha^i=\alpha^j$ for some $i\neq j$ then $\alpha$ is a root of $f(x)=x^j-x^i$.
    Otherwise, since $\{\alpha,\alpha^2,\ldots,\alpha^{n+1}\}$ is linearly dependent over $F$, there is some dependence relation and $\alpha$ is a root of $f(x)=c_{n+1}x^{n+1}+\cdots+c_1x\neq 0$.
\end{proof}
\begin{definition}
    We say that $K$ is a \mbf{finitely generated} extension of $F$ if there exists $\alpha_1,\ldots,\alpha_n\in K$ such that $K=F(\alpha_1,\ldots,\alpha_n)$.
\end{definition}
\begin{proposition}
    If $K$ is a finitely generated and algebraic extension of $F$, then $K/F$ is finite.
\end{proposition}
\begin{proof}
    Suppose $K/F$ is algebraic, where $K=F(\alpha_1,\ldots,\alpha_n)$, $\alpha_i\in K$.
    If $n=1$, then $[F(\alpha_1):F]=\deg_F(\alpha_1)<\infty$.

    Assume the result for $n$ and consider $K=F(\alpha_1,\ldots,\alpha_n,\alpha_{n+1})$.
    Then
    \begin{align*}
        [F(\alpha_1,\ldots,\alpha_n,\alpha_{n+1})]=[F(\alpha_1,\ldots,\alpha_n)(\alpha_{n+1}):F(\alpha_1,\ldots,\alpha_n)]\cdot[F(\alpha_1,\ldots,\alpha_n):F]<\infty
    \end{align*}
    by the tower theorem.
\end{proof}
\begin{proposition}
    If $K/E$ and $E/F$ are both algebraic, then $K/F$ is algebraic.
\end{proposition}
\begin{proof}
    Let $\alpha\in K$.
    Since $K/E$ is algebraic, $\alpha$ has a minimal polynomial in $E$:
    \begin{equation*}
        p(x)=x^n+c_{n-1}x^{n-1}+\cdots+c_1x+c_0\in E[x]
    \end{equation*}
    Thus $\alpha$ is algebraic over $F(c_0,c_1,\ldots,c_{n-1})$.
    Note that $[F(c_{n-1},\ldots,c_1,c_0)(\alpha):F(c_{n-1},\ldots,c_1,c_0)]<\infty$.
    Since $F(c_{n-1},\ldots,c_1,c_0)\subseteq E$, $F(c_{n-1},\ldots,c_1,c_0)/F$ is algebraic and finitely generated, so $[F(c_{n-1},\ldots,c_1,c_0):F]<\infty$.
    By the tower theorem, $[F(c_{n-1},\ldots,c_1,c_0,\alpha):F]<\infty$, so $\alpha$ is algebraic over $F$.
\end{proof}
\begin{proposition}
    Let $K/F$ be a extension.
    The set of elements of $K$ which are algebraic over $F$ form a subfield of $K$.
\end{proposition}
\begin{proof}
    Let $L$ denote the elements algebraic over $F$.
    If $\alpha,\beta\in L$, then $\alpha,\beta,\alpha-\beta,\alpha\beta,\beta^{-1}\in F(\alpha,\beta)$ and $[F(\alpha,\beta):F]<\infty$ and since finite implies algebraic, these elements are all algebraic.
\end{proof}
\subsection{Splitting Fields}
\begin{definition}
    Let $f(x)\in F[x]$ be non-constant.
    We say $f(x)$ \mbf{splits} in an extension $K$ of $F$ if it factors completely into linear factors over $K$.
\end{definition}
\begin{theorem}[Kronecker]
    Let $f(x)\in F[x]$ be non-constant.
    Then there exists an extension $K$ of $F$ such that $f(x)$ has a root in $K$.
\end{theorem}
\begin{proof}
    Let $f(x)\in F[x]$ be non-constant; since $F[x]$ is a UFD, let $p|f$ where $p$ is irreducible.
    Let $K=F[t]/(p(t))$, so $t+(p(t))$ is a root of $p(x)$, which is also a root of $f(x)$.
\end{proof}
\begin{corollary}
    Let $f(x)\in F[x]$ be non-constant.
    There exists an extension $K$ of $F$ such that $f(x)$ splits over $K$.
\end{corollary}
\begin{proof}
    Repeated application of Kronecker.
\end{proof}
\begin{definition}
    Let $f(x)\in F[x]$ be non-constant.
    A minimal extension $K$ of $F$ with the property that $f$ splits over $K$ is called a \mbf{splitting field} for $f$.
\end{definition}
If $f(x)\in F[x]$, there is an extension $K/F$ such that $f(x)$ splits over $K$.
But then a splitting field for $f(x)$ over $F$ is $F(\alpha_1,\ldots,\alpha_n)$ where the $\alpha_i$ are the roots of $f$.
\begin{example}
    Find a splitting field for $f(x)=x^4+x^2-6$ over $\Q$.
    Over $\C$, $f(x)=(x+\sqrt{3}i)(x-\sqrt{3}i)(x-\sqrt{2})(x+\sqrt{2})$.
    Thus a splitting field for $f(x)$ over $\Q$ is $\Q(\sqrt{2},\sqrt{3}i)$.
\end{example}
\begin{lemma}
    Let $F,F'$ be fields.
    If $\phi:F\to F'$ is an isomorphism, then the natural map $\tilde\phi:F[x]\to F'[x]$ is an isomorphism.
\end{lemma}
\begin{proof}
    It's long but easy.
\end{proof}
We'll just write $\overset{\sim}{\varphi}\equiv\varphi$.
\begin{lemma}[Isomorphism Extension]\label{lem:iso-ext}
    Let $F,F'$ be fields, $\phi:F\to F'$ be an isomorphism.
    Let $f(x)\in F[x]$ be irreducible, $\alpha$ a root of $f(x)$ in an extension of $F$.
    $\beta$ is a root of $\phi(f(x))$ in some extension of $F'$.
    Then there exists an isomorphism $\psi:F(\alpha)\to F'(\beta)$ such that $\psi|_F=\phi$ and $\psi(\alpha)=\beta$.
\end{lemma}
\begin{proof}
    The following diagram commutes:
    \begin{center}
        \begin{tikzcd}
            F(\alpha)\arrow[dd,rightarrow,"\rho_1"',"\rotatebox{90}{$\sim$}"]\arrow[rrr,dashed,"\psi"]&   &   & F'(\beta)\arrow[dd,leftarrow,"\rho_2"',"\rotatebox{90}{$\sim$}"]\\
                                                     & F\arrow[ul,hook']\rar["\phi"]& F'\arrow[ur,hook]&\\
            \quot{F[x]}{\brac{f(x)}}\arrow[rrr,rightarrow,"\sim","\sigma:\overline{g(x)}\mapsto\overline{\phi(g(x))}"']&   &   & \quot{F'[x]}{\brac{\phi(f(x))}}
            % F(\alpha)\rar["\sim"]&\quot{F[x]}{\langle f(x)\rangle}\rar["\sigma"]&\quot{F'[x]}{\langle\phi(f(x))\rangle}\rar["\sim"]&F(\beta)
        \end{tikzcd}
    \end{center}
    where $\psi$ exists by composing maps.
    If $a\in F$, then
    \begin{equation*}
        \psi(a)=\rho_2\circ\sigma\circ\rho_1(a)=\rho_2\circ\sigma(\overline{a})=\rho_2(\overline{\phi(a)})=\phi(a)=a
    \end{equation*}
    As well, we verify that
    \begin{equation*}
        \psi(\alpha)=\rho_2\circ\sigma\circ\rho_1(\alpha)=\rho_2\circ\sigma(\overline{x})=\rho_2(\overline{\phi(x)})=\rho_2(\overline{x})=\beta
    \end{equation*}
\end{proof}
\begin{corollary}
    Let $F$ be a field, $f(x)\in F[x]$ non-constant.
    Let $K$ be a splitting field for $f(x)$ over $F$.
    If $F'$ is a field and $\phi:F\to F'$ is an isomorphism, then for any $K'$ splitting field for $\phi(f(x))$ over $F'$, there is an isomorphism $\psi:K\to K'$ such that $\psi|_F=\phi$.
\end{corollary}
\begin{proof}
    Repeatedly apply the isomorphism extension lemma (\cref{lem:iso-ext}) to the roots of $f$.
\end{proof}
\begin{corollary}
    Let $f(x)\in F[x]$ be non-constant.
    If $K$ and $K'$ are splitting fields for $f(x)$ over $F$, then $K\cong K'$.
\end{corollary}
\begin{proof}
    Take $\phi=\id$ in the previous corollary.
\end{proof}
\subsection{Algebraic Closure}
\begin{definition}
    A field $\overline{F}$ is an \mbf{algebraic closure} of a field $F$ if
    \begin{itemize}[nl]
        \item $\overline{F}/F$ is algebraic
        \item Every non-constant polynomial in $F[x]$ splits over $\overline{F}$.
    \end{itemize}
    A field $F$ is \textbf{algebraically closed} if every non-constant polynomial $f(x)\in F[x]$ has a root in $F$.
\end{definition}
\begin{example}
    $\C$ is an algebraic closure for $\R$, but not for $\Q$.
\end{example}
\begin{proposition}
    If $\overline{F}$ is an algebraic closure for $F$, then $\overline{F}$ is algebraically closed.
\end{proposition}
\begin{proof}
    Let $\overline{F}$ be an algebraic closer for $F$.
    Let $f(x)\in\overline{F}(x)$ be non-constant; by Kronecker, $f(x)$ has a root $\alpha$ in some extension of $\overline{F}$.
    Since $\overline{F}(\alpha)/\overline{F}$ is algebraic and $\overline{F}/F$ is algebraic, $\overline{F}(\alpha)/F$ is algebraic.
    Thus $\alpha$ is the root of some non-zero polynomial $p(x)\in F[x]$.
    Now, $p(x)$ splits over $\overline{F}$ so $\alpha\in\overline{F}$ and $\overline{F}$ is algebraically closed.
\end{proof}
\begin{theorem}
    For every field $F$, there exists an algebraically closed field containing $F$.
\end{theorem}
\begin{proof}
    Exercise.
\end{proof}
\begin{theorem}
    Let $K$ be an algebraically closed field which contains $F$.
    The collection of elements in $K$ which are algebraic over $F$ is an algebraic closure.
\end{theorem}
\begin{proof}
    Let $L=\{\alpha\in K:\alpha\text{ is algebraic over}F\}$.
    We claim that $L$ is an algebraic closure for $F$.
    By construction, $L/F$ is algebraic.
    Let $f(x)\in F[x]$, $\deg f(x)\geq 1$.
    Since $f(x)$ splits over $K$, $f(x)=u(x-\alpha_1)\cdots(x-\alpha_n)$.
    Since $u\in F$, $\alpha_i\in K$.
    But, $f(\alpha_i)=0$ for $i=1,\ldots,n$ and so $\alpha_i\in L$ and $f(x)$ splits over $L$.
\end{proof}
\section{Examples of Field Extensions}
\subsection{Cyclotomic Extensions}
What is the splitting field of $f(x)=x^n-1$?
\begin{definition}
    We call the roots of $x^n-1$ (in $\C$) the \mbf{$n$\textsuperscript{th} roots of unity}.
\end{definition}
If $\zeta_n=e^{2\pi i/n}$, they are $1,\zeta_n,\zeta_n^2,\cdots,\zeta_n^{n-1}$.
Thus, the splitting field over $\Q$ is $\Q(\zeta_n)$.
What is $[\Q(\zeta_n):\Q]$?
When $n=p$ is prime, $x^p-1=(x-1)(1+x+x^2+\cdots+x^{p-1})$.
Since $\Phi_p(x)=x^{p-1}+\cdots+x+1$ is irreducible over $\Q$ (from before), so $[\Q(\zeta_p):\Q]=p-1$.
\begin{example}
    Since $\zeta_5=\frac{1}{2}+i\frac{\sqrt{3}}{2}$, $\Q(\zeta_6)=\Q(i\sqrt{3})$ so $\deg(x^2+3)=2$.
\end{example}
Note that the $n^\text{th}$ roots of unity form a finite cyclic subgroup of $\C$; in fact, they are the only finite cyclic subgroups of $\C$.
A generator of this group is called a \mbf{primitive $n^\text{th}$ root of unity}, which happens precisely for $\zeta_n^k$ where $\gcd(k,n)=1$.
Thus there are $\phi(n)$ primitive $n^\text{th}$ roots of unity.
\begin{definition}
    The \mbf{$n^\text{th}$ cyclotomic polynomial} is
    \begin{equation*}
        \Phi_n(x)=\prod\limits_{k\in(\Z_n)^\times}(x-\zeta_n^k)
    \end{equation*}
\end{definition}
% \begin{remark}
    % By Gauss' Lemma, $\Phi_n(x)\in\Z[x]$.
% \end{remark}
% \begin{lemma}
    % $x^n-1=\prod_{d|n}\Phi_d(x)$.
% \end{lemma}
% \begin{proof}
    % This is straightforward.
% \end{proof}
% \begin{proposition}
    % For every $n\geq 1$, $\Phi_n(x)\in\Z[x]$.
% \end{proposition}
% \begin{proof}
    % We proceed by induction on $n$.
    % If $n=1$, $\Phi_1(x)=x-1\in\Z[x]$.
    % Now, $x^n-1=f(x)\Phi_n(x)$, where $f(x)=\prod_{d|n,d<n}\Phi_d(x)$.
    % By induction and Gauss' Lemma, $f(x)\in\Z[x]$.
    % Let $F=\Q(\zeta_n)$ so that $\Phi_n(x)\in F[x]$.
    % By the division algorithm, there exist unique $q,r\in F[x]$ such that $x^n-1=f(x)q(x)+r(x)$.
    % Similarly, there exist unique $\tilde q(x)$, $\tilde r(x)$ in $\Q[x]$ so that $x^n-1=f(x)\tilde q(x)+\tilde r(x)$.
    % By uniqueness, $\Phi_n(x)=q(x)=\tilde q(x)\in\Q[x]$, so by Gauss' Lemma, $\Phi_n(x)\in\Z[x]$.
% \end{proof}
\begin{theorem}
    $\Phi_n(x)$ is the minimal polynmial for $\zeta_n$, and $[\Q(\zeta_n):\Q]=\phi(n)$.
\end{theorem}
\begin{proof}
    Note that $\zeta_n$ is a root of $x^n-1$, so $\zeta_n$ is algebraic over $\Q$.
    By Gauss' lemma, let $f(x)\in\Z[x]$ be the minimal polynomial of $\zeta_n$ over $\Q$ so that $f(x)\div(x^n-1)$ over $\Z[x]$.
    Recall that
    \begin{equation*}
        x^n-1=\prod\limits_{j\in\Z_n}(x-\zeta_n^j)
    \end{equation*}
    If $j\notin(\Z_n)^\times$, then $\zeta_n^j$ satisfies $x^{\frac{n}{\gcd(n,j)}}-1$ but $\zeta_n$ does not, so $\zeta$ and $\zeta_n^j$ are not conjugates.
    Thus the only possible conjugates for $\zeta_n$ are the $\zeta_n^j$ where $j\in(\Z_n)^\times$; it suffices to show that these are precisely the conjugates.
    In particular, let's show that if $\theta=\zeta_n^t$ and $p$ is prime with $p\nmid n$, then $\theta^p$ is conjugate to $\theta$.
    With this, the result follows: if $j$ is coprime to $n$, write $j=p_1^{e_1}\cdots p_m^{e_m}$ with $p_i\nmid n$ and repeatedly apply the above result to $\zeta_n$ for each $p_i$, $e_i$ times.

    Thus let's prove the claim.
    Write $x^n-1=f(x)g(x)$ with $f,g\in\Z[x]$; since $\theta^p$ is a root of $x^n-1$, either it is a root of $f(x)$ - in which case we're done - or it is a root of $g(x)$.
    Suppose $g(\theta^p)=0$, so $\theta$ is a root of $g(x^p)\in\Z[x]$ so $f(x)\div g(x^p)$ over $\Z[x]$.
    Modulo $p$, $\ol{f}(x)\div\ol{g}(x^p)=\ol{g}(x)^p$ in $\Z_p[x]$.
    Since $\Z_p[x]$ is a UFD, let $s(x)$ be an irreducible factor of $f(x)$ so that $s|\ol{f}$ and thus $s|\ol{g}$.
    But then $x^{n}-\ol{1}=\ol{f}\ol{g}$, so $s^2\div(x^n-1)$ and $s\div\ol{n}x^{n-1}$.
    Since $n$ is coprime to $p$, this implies $s=cx$ for some $c\in\Z_p$.
    But then $cx\div x^n-\ol{1}$, a contradiction.
\end{proof}
% \begin{corollary}
    % $\Phi_n(x)$ is the minimal polynomial for $\zeta_n$ over $\Q$ and $[\Q(\zeta_n):\Q]=\phi(n)$.
% \end{corollary}
% \begin{proof}
    % Since $g(x)|\Phi_n(x)$, $\Phi_n(x)=g(x)h(x)$, where $h\in\Z$.
    % Since $\Phi_n(\zeta^p)=0$, $g(\zeta^p)\neq 0$.
    % Let $f(x)=h(x^p)\in\Z[x]$, we have $f(\zeta)=0$.
    % Furthermore, $g(x)|f(x)$, so $f(x)=g(x)k(x)$, $k\in\Z[x]$.
    % Say $h(x)=\sum b_jx^j$, so $f(x)\sum b_jx^{pj}$.
    % In $\Z_p[x]$,
    % \begin{align*}
        % \overline{f}(x) &= \sum\overline{b_j}x^{pj}=\sum\overline{b_j}x^{pj}\\
                        % &=\left(\sum\overline{b}_jx^j\right)^p=(\overline{h}(x))^p
    % \end{align*}
    % Thus, $\overline{h}(x)^p=\overline{g}(x)\overline{k}(x)\in\Z_p[x]$.
% \end{proof}
\subsection{Finite Fields}
\begin{definition}
    Let $F$ be a field of characteristic $p$.
    Then the map $\phi:F\to F$ given by $x\mapsto x^p$ is called the \mbf{Frobenius map}.
\end{definition}
\begin{proposition}
    The Frobenius map is an injective ring homomorphism.
\end{proposition}
\begin{proof}
    We have that $\phi(xy)=x^py^p=(xy)^p$, and
    \begin{equation*}
        \phi(x+y)=(x+y)^p=\sum_{i=0}^px^iy^{p-i}\binom{p}{i}=x^p+y^p
    \end{equation*}
    since $p\div\binom{p}{i}$ for all $1\leq i\leq p-1$.
    Injectivity is immediate since $\phi(1)=1$ and the only ideals of $F$ are $\{0\}$ and $\{F\}$, forcing $\ker(\phi)=\{0\}$.
\end{proof}
\begin{corollary}
    If $F$ is a finite field, the Frobenius map is an automorphism.
\end{corollary}
\begin{proposition}
    Suppose $F$ is finite.
    Then
    \begin{enumerate}[nolistsep]
        \item $F^\times=\langle \alpha\rangle$ is a cyclic group.
        \item $|F|=p^n$.
        \item $|F|=p^n$ if and only if $F$ is the splitting field for $x^{p^n}-x$ over $\Z_p$.
        \item Finite fields of a fixed size are unique up to isomorphism.
    \end{enumerate}
\end{proposition}
\begin{proof}
    \begin{enumerate}[nl]
        \item Write $F^\times\cong C_{n_1}\times\cdots\times C_{n_k}$ where $n_1|n_2|\cdots|n_k$.
            Then each $C_{n_i}$ has a subgroup $D_i\cong C_{n_k}$; but then every $x\in D_1\times\cdots\times D_k$ satisfies $x^{n_k}=1$.
            Since there are ${n_k}^k$ such elements and $x^{n_k}=1$ has at most $n_k$ roots, this forces $k=1$ and $F^\times$ is cyclic.
        \item Recall that $F/\Z_p$ where $p=\chr F$.
            Thus $[F:\Z_p]=n<\infty$ so that $F=\Z_p(\alpha)$ and $|F|=p^n$.
        \item Suppose $|F|=p^n$; by Lagrange, every $a\in F^\times$ satisfies $x^{p^n-1}-1$ so that every $a\in F$ satisfies $x^{p^n}-x$, so $x^{p^n}-x$ splits over $F$.
            Take $f(x)=x^{p^n}-x$, so that $f'(x)=-1$ and $f$ is separable.
            Thus, any splitting field $F$ must have at least $p^n$ elemenets, so $|F|$ is minimal and $F$ is a splitting field of $x^{p^n}-x$.

            Conversely, suppose $F$ is the splitting field of $x^{p^n}-x$ over $\Z_p$.
            Consider $K=\{\alpha\in F:f(\alpha)=0\}$, so that $K\leq F$.
            In particular, $F$ splits in $K$, forcing $K=F$.
            Thus, $|F|=|K|\leq p^n$ since $f$ can have at most $p^n$ roots.
            However, as above, $f(x)$ is separable, so $|F|=|K|=p^n$.
        \item Splitting fields are unique up to isomorphism.
    \end{enumerate}
\end{proof}
Since the splitting field is unique, for any prime $p$ and $n\in\N$, there exists a unique field of order $p^n$ (up to isomorphism).
We denote the field $\F_{p^n}$.
\begin{theorem}
    If $E$ is a subfield of $\F_{p^n}$, then $E\cong \F_{p^r}$, where $r|n$.
    Moreover, if $r|n$, then $\F_{p^n}$ has a unique subfield of order $p^r$.
\end{theorem}
\begin{proof}
    Let $E$ be a subfield of $\F_{p^n}$, so $n=[\F_{p^n}:\F_p]=[\F_{p^n}:E][E:\F_p]$.
    Set $r=[E:\F_p]$, $r|n$, and $|E|=p^r$.

    Conversely, suppose $r|n$, and consider $\F_{p^n}=\{\alpha\in\overline{\F_p}:\alpha^{p^n}-\alpha=0\}$.
    Since $r|n$, write $p^n-1=(p^r-1)(p^{n-r}+p^{n-2r}+\cdots+p^r+1)$.
    From before,
    \begin{align*}
        E &= \{\alpha\in\overline{\F_p}:\alpha^{p^r}-\alpha=0\}\\
          &= \{\alpha\in\overline{\F_p}:\alpha^{p^r-1}-1=0\}\cup\{0\}\\
          &\subseteq\F_{p^n}
    \end{align*}
    Moreover, $|E|=p^r$.
    If $K$ is any other subfield and $|K|=p^r$, then for any $0\neq\alpha\in K$, $\alpha^{p^r-1}=1$ since $K^\times$ is cyclic, and $K\subseteq E$.
\end{proof}
\chapter{Galois Theory}
TODO
\begin{itemize}[nl]
    \item talk about maps $\sigma:K\hookrightarrow k^a$ (algebraic closure of $k$).
    \item full proof of algebraic closure
    \item isomorphism extension lemma in terms of emebeddings
    \item use lower case $k$ for base field to distinguish.
    \item Use universal property of simple field extensions
\end{itemize}
\section{Galois Groups}
Let $f(x)\in F[x]$ be non-constant, and $\alpha_1,\ldots,\alpha_n$ be the roots of $f(x)$ in its splitting field.
Our goal is to study these roots by permuting them using automorphisms of $K$.
\begin{definition}
    Let $K/F$.
    Recall that $\Aut(K)$ is the group of automorphisms of $K$.
    We define $\Gal(K/F)=\{\phi\in\Aut(K):\phi|_F=\id\}\leq\Aut(K)$.
\end{definition}
\begin{lemma}
    Let $K/F$.
    If $\alpha\in K$ is a root of $f(x)\in F[x]$ and $\phi\in\Gal(K/F)$, then $\phi(\alpha)$ is also a root of $f(x)$.
\end{lemma}
\begin{proof}
    Note that $0=\phi(f(\alpha))=f(\phi(\alpha))$ since $\phi$ fixes the coefficients of $f$.
\end{proof}
\begin{corollary}
    If $\alpha\in K$ is algebraic over $F$ and $\phi\in\Gal(K/F)$, then $\phi(\alpha)$ is algebraic over $F$ and has the same minimal polynomial in $F[x]$.
\end{corollary}
\begin{example}
    Compute $\Gal(\Q(\sqrt{2},\sqrt{3})/\Q)$.
    If $\phi\in\Gal(\Q(\sqrt{2},\sqrt{3})/\Q)$, then $\phi(\sqrt{2})=\pm\sqrt{2}$ and $\phi(\sqrt{3})=\pm\sqrt{3}$.
    Thus the automorphisms are given by.
    \begin{align*}
        \phi_1=&\begin{cases}\sqrt{2}\mapsto\sqrt{2}\\\sqrt{3}\mapsto\sqrt{3}\end{cases} & \phi_2=&\begin{cases}\sqrt{2}\mapsto-\sqrt{2}\\\sqrt{3}\mapsto\sqrt{3}\end{cases}\\
        \phi_3=&\begin{cases}\sqrt{2}\mapsto\sqrt{2}\\\sqrt{3}\mapsto-\sqrt{3}\end{cases} & \phi_4=&\begin{cases}\sqrt{2}\mapsto-\sqrt{2}\\\sqrt{3}\mapsto-\sqrt{3}\end{cases}
    \end{align*}
    and $G=\{\phi_1,\phi_2,\phi_3,\phi_4\}$.
    Since $|\phi_i|=2$ for all $i$, $G$ is abelian, so $G\cong\Z_2\times\Z_2$.
\end{example}
\begin{example}
    Consider $G=\Gal(\Q(\sqrt[3]{2})/\Q)$.
    If $\phi\in G$, then $\phi(\sqrt[3]{2})\in\{\sqrt[3]{2},\sqrt[3]{2}\zeta_3,\sqrt[3]{2}\zeta_3^2\}$, so $\phi(\sqrt[3]{2})=\sqrt[3]{2}$.
    Thus $\phi=\id$ and $G=\{\id\}$.
\end{example}
Let $F$ be a field, $f(x)\in F[x]$, $\deg f(x)=n\geq 1$.
Let $K$ be the splitting field for $f(x)$ over $F$, so the roots of $f(x)$ are $\alpha_1,\alpha_2,\ldots,\alpha_n$.
Let $G=\Gal(K/F)$, so for any $\phi\in G$, $\phi(\alpha_i)=\alpha_j$.
In particular, for any $\phi\in\Gal(K/F)$, $\phi(\alpha_i)=\alpha_{\pi(i)}$ for some $\pi\in S_n$.
Thus the map $\Gal(K/F)\to S_n$ given by $\phi\mapsto\pi$ is injective.
\begin{remark}
    If $f(x)\in F[x]$, $K$ the splitting field for $f(x)$, then we write $\Gal(K/F)=\Gal(f(x))$.
\end{remark}
\begin{example}
    Consider $f(x)=(x^2-2)(x^2-3)\in\Q[x]$.
    Then $\Gal(f(x))\cong\Z_2\times\Z_2$.
    Let $\alpha_1=\sqrt{2}$, $\alpha_2=-\sqrt{2}$, $\alpha_3=\sqrt{3}$, $\alpha_4=-\sqrt{3}$, so $\Gal(f(x))=\{\epsilon,(34),(12),(12)(34)\}$.
\end{example}
\begin{example}
    $\Gal(x^2+1)\cong\Z_2$ over $\Q[x]$, but $\Gal(x^2+1)=\{1\}$ over $\Z_2[x]$.
\end{example}
\begin{corollary}
    Let $F$ be a field, $f(x)\in F[x]$ irreducible, $K$ the splitting field for $f(x)$ over $F$.
    Then for any roots $\alpha,\beta\in K$ of $f(x)$, there exists $\phi\in\Gal(K/F)$ such that $\phi(\alpha)=\beta$.
\end{corollary}
\begin{proof}
    By the isomorphism extension lemma (\cref{lem:iso-ext}), $\id:F\to F$ extents to an automorphism $\phi:F(\alpha)\to F(\beta)$ such that $\alpha\mapsto\beta$, which extends to an isomorphism $K\to K$.
\end{proof}
\begin{definition}
    A subgroup $H$ of $S_n$ is \mbf{transitive} if for all $i,j\in\{1,2,\ldots,n\}$, there exists $\pi\in H$ such that $\pi(i)=j$.
\end{definition}
\begin{corollary}
    Let $f(x)\in F[x]$, $\deg f(x)=n\geq 1$, $f(x)$ separable and irreducible.
    Then $\Gal(f(x))$ is isomorphic to a transitive subgroup of $S_n$.
\end{corollary}
\begin{example}
    Compute $G=\Gal(x^3-2)$ over $\Q[x]$.
    Since $f(x)=x^3-2$ is irreducible, $f(x)$ is also separable.
    Then $G$ is isomorphic to a transitive subgroup of $S_3$.
    Let $\alpha_1,\alpha_2,\alpha_3$ be the roots of $f(x)$, and $x=\{\alpha_1,\alpha_2,\alpha_3\}$.
    Then $G$ acts on $X$ via $\phi\cdot\alpha_i=\phi(\alpha_i)$.
    By Orbit-Stabilizer, $|G|=|G\cdot\alpha|\cdot|\Stab(\alpha_1)|$.
    By transitivity, $|G\cdot\alpha|=3$, so $3\div|G|$ and $G\cong A_3$ or $S_3$.

    Consider $G$ as a subgroup of $S_3$ relative to the order $\alpha_1=\sqrt[3]{2}$, $\alpha_2=\alpha_1\zeta_3$, $\alpha_3=\alpha_1\zeta_3^2$.
    Note that $x^3-2$ is irreducible over $\Q(\zeta_3)$ since $x^3-2$ has no roots in $\Q(\zeta_3)$.
    Thus by the isomorphism extension lemma, there exists $\phi\in G$ such that the following diagram commutes:
    \begin{center}
        \begin{tikzcd}[column sep=large]
            \Q(\zeta_3,\alpha_1) \arrow[r,"\phi:\phi(\alpha_1)=\alpha_1"] & \Q(\zeta_3,\alpha_1)\\
            \Q(\zeta_3)\arrow[r,"\zeta_3\mapsto\zeta_3^2"]\arrow[u,hook]&\arrow[u,hook]\Q(\zeta_3)\\
            \Q\arrow[r,"\id"]\arrow[u,hook]&\arrow[u,hook]\Q
        \end{tikzcd}
    \end{center}
    Thus $\phi(\alpha_1)=\alpha_1$, $\phi(\alpha_2)=\alpha_3$ and $\phi(\alpha_3)=\alpha_2$.
    Hence $\phi\sim(23)\in G$ is an element of order 2, so $G\cong S_3$.
\end{example}
\begin{remark}
    When computing $G=\Gal(K/F)$, it is useful to know $|G|$.
\end{remark}
\begin{definition}
    Suppose $K/F$ and $E/F$ are field extensions.
    Any homomorphism $\phi:K\to E$ which fixes $F$ is called an \mbf{$F-$map} from $K$ to $E$.
\end{definition}
\begin{remark}
    If $\phi:K\to E$ is a $F-$map, since $K$ is a field, $\phi$ is automatically injective.
    Furthermore, for any $\alpha\in F$, $v\in K$, $\phi(\alpha v)=\alpha\phi(v)$, so $\phi$ is $F-$linear.

    If $\phi:K\to K$ and $[K:F]<\infty$, then $\phi$ is surjective and $\phi:K\to K$ is an $F-$map if and only if $\phi\in\Gal(K/F)$.
\end{remark}
\begin{lemma}
    Let $K/F$, $E/F$, $[K:E]<\infty$.
    The number of distinct $F-$maps $\phi:K\to E$ is at most $[K:F]$.
\end{lemma}
\begin{proof}
    We proceed inductively on the number of generators of $K/F$.
    If $K=F(\alpha_1)$ and $\phi:K\to E$ is an $F-$map, then $\alpha_1$ and $\phi(\alpha_1)$ have the same minimal polynomial over $F$.
    Thus there are at most $[F(\alpha_1):F]=[K:F]$ options $\phi(\alpha_1)$, so there are at most $[K:F]$ many such $F-$maps.

    Now assume $K=F(\alpha_1,\ldots,\alpha_n)$, and let $L=F(\alpha_1,\ldots,\alpha_{n-1})$.
    Let $\phi:K\to E$ be an $F-$map, so $\phi|_L:L\to E$ is an $F-$map.
    By induction, the number of possible $\phi|_L$ is at most $[L:F]$.
    Since $\phi$ is completely determined by $\phi|_L$ and $\phi(\alpha_n)$, there are at most $[L:F][L(\alpha_n):L]=[K:F]$ possibilities for $\phi$.
\end{proof}
\begin{remark}
    How can it happen that $|\Gal(K/F)|<[K:F]$?
    It could be that the extension is not normal; i.e. the extension has conjugates not contained in the extension.

    It can also happen that there are repeated roots: consider $G=\Gal(\Z_2(t)/\Z_2(t^2))$, so $[\Z_2(t):\Z_2(t^2)]=2$.
    Then $t\mapsto x^2-t^2\in\Z(t^2)[x]$, so $(x-t)^2\in\Z(t)[x]$.
    Thus if $\phi\in G$, then $\phi(t)=t$, so $\phi=\id$ and $G=\{1\}$.
\end{remark}
\section{Separable and Normal Extensions}
\begin{definition}
    We say $\alpha\in K$ is \mbf{separable} if $\alpha$ is algebraic over $F$ and its minimal polynomial is separable (over $F$).
    We say $K/F$ is \mbf{separable} if $K/F$ is algebraic and all elements of $K$ are separable over $F$.
    A field $F$ is \mbf{perfect} if every algebraic extension of $F$ is separable.
\end{definition}
\begin{remark}
    Suppose $f(x)\in F[x]$ is irreducible.
    Then $f(x)$ is separable if and only if $f'(x)\neq 0$.
\end{remark}
\begin{proposition}
    Let $f(x)\in F[x]$ be irreducible.
    \begin{enumerate}[nolistsep]
        \item If $\chr(F)=0$, then $f(x)$ is separable.
        \item If $\chr(F)=p>0$ then $f(x)$ is not separable if and only if $f(x)=g(x^p)$ for some $g(x)\in F[x]$.
    \end{enumerate}
\end{proposition}
\begin{proof}
    Immediate from the preceding remark.
\end{proof}
\begin{corollary}
    \begin{enumerate}[nl]
        \item If $\chr(F)=0$, then $F$ is perfect.
        \item If $\chr(F)=p$, then $F$ is perfect if and only if $\phi(x)=x^p$ is an automorphism.
    \end{enumerate}
\end{corollary}
\begin{proof}
    (1) is clear, so we prove (2).
    In characteristic $p$, $\phi$ is always injective.

    First suppose $\phi(x)=x^p$ is also surjective.
    Suppose there exists $f(x)\in F[x]$ irreducible but not separable.
    Thus $f(x)=g(x^p)$, and write
    \begin{align*}
        f(x) &=a_nx^{pm_n}+\cdots+a_1x^{pm_1}+a_0\\
             &=b_n^px^{pm_n}+\cdots+b_1^px^{pm_1}+b_0^p\\
             &= (b_nx^{m_n}+\cdots+b_xx^{m_1}+b_0)^p
    \end{align*}
    Conversely, suppose $x^p$ is not an automorphism; in particular, $x^p$ is not surjective.
    Let $\alpha\notin\im(\phi)$.
    But then $f(x)=x^p-\alpha$ is irreducible, but if $K$ is the splitting field for $F$, then $r$ is a root so $r^p=\alpha$ and $(x-r)^p=x^p-\alpha$ and $f$ is not separable.
\end{proof}
\begin{remark}
    Since the Frobenius map is an isomorphism when $F$ is a finite field, every finite field is perfect.
\end{remark}
\begin{theorem}\label{thm:gal-size}
    Let $f(x)\in F[x]$ be non-constant and separable, and $K$ the splitting field for $f(x)$ over $F$.
    Then $|\Gal(K/F)|=[K:F]$.
\end{theorem}
\begin{proof}
    We proceed by induction on $[K:F]$.
    If $[K:F]=1$, this is obvious.

    Otherwise, let $[K:F]=n>1$.
    Let $p(x)\in F[x]$ be an irreducible factor of $f(x)$, so $p(x)$ is also separable over $F$.
    Say the roots of $p(x)$ are $\alpha_1,\alpha_2,\ldots,\alpha_m$ where $m=\deg p(x)$; suppose $\alpha_1\notin F$ and let $E=F(\alpha_1)$.
    Then $K/E/F$ is a tower of fields with $[K:E]=\frac{n}{m}<n$.
    Furthermore, $K$ is the splitting field for $f(x)$ over $E$, so by induction, $|\Gal(K/E)|=[K:E]=\frac{n}{m}$.

    Since $p(x)\in F[x]$ is irreducible, for all $j$, get $\phi_j\in\Gal(K/F)$ such that $\phi_j(\alpha_1)=\alpha_j$; note that $\phi_1,\ldots,\phi_m$ are distinct in $\Gal(K/F)$.
    % For $i\neq j$, $\alpha_i\neq\alpha_j$ if and only if $\phi_i(\alpha_1)\neq\phi_j(\alpha_1)$, so $\phi_1,\ldots,\phi_n$ are distinct in $\Gal(K/F)$.
    Moreover, $\phi_j^{-1}\circ\phi_i(\alpha_1)\neq\alpha_1\in E$.
    Thus $\phi_j^{-1}\circ\phi_i\notin\Gal(K/E)$, so $\phi_i\Gal(K/E)\neq \phi_j\Gal(K/E)$.
    Thus $|\Gal(K/F)/\Gal(K/E)|\geq m$.
    Thus $|\Gal(K/F)|\geq m|\Gal(K/E)|=n$, and we're done.
\end{proof}
\begin{definition}
    We say an extension $K/F$ is \mbf{simple} if there exists $\alpha\in K$ such that $K=F(\alpha)$.
    We say $\alpha$ is a \mbf{primitive element} for $K/F$.
\end{definition}
\begin{theorem}[Primitive Element]\label{thm:prim-el}
    If $K/F$ is finite and separable, then $K/F$ is simple.
\end{theorem}
\begin{proof}
    Suppose $K/F$ is finite and separable.

    First suppose $F$ is finite, so that $K$ is also finite and $K^\times=\langle\alpha\rangle$ for some $\alpha\in K$.
    Thus, $K=F(\alpha)$.

    Otherwise, $F$ is infinite, and write $K=F(\pi_1,\ldots,\pi_n)$ for some $\pi_i\in K$.
    It suffices to prove the result for $n=2$; say, $K=F(\alpha,\beta)$.
    Let $p,q$ be the minimal polynomial of $\alpha$ and $\beta$ respectively.
    Let $L$ be the splitting field for $p(x)q(x)$ over $K$, and let $\alpha=\alpha_1,\ldots,\alpha_n$ and $\beta=\beta_1,\ldots,\beta_k$ the distinct conjugates in $L$ of $\alpha$ and $\beta$ (since $K/F$ is separable).
    Let
    \begin{equation*}
        S=\left\{\frac{\alpha_i-\alpha_1}{\beta_1-\beta_j}:1<i\leq n,1<j\leq m\right\}
    \end{equation*}
    Since $S$ is finite and $F$ is infinite, get $u\in S\setminus F$ so that $\gamma:=\alpha+u\beta\neq\alpha_i+u\beta_j$ for any $i,j\neq 1$.
    Certainly $F(\gamma)\subseteq F(\alpha,\beta)$.
    Let $h(x)$ be the minimal polynomial for $\beta$ over $F(\gamma)$.
    Since $q(x)\in F(\gamma)[x]$ and $q(\beta)=0$, $h(x)|q(x)$.
    As well, $h(x)|p(\gamma-ux)$ since $p(\gamma-u\beta)=0$;  but the only shared root is $\beta$ by choice of $u$, $\deg h=1$ and $\beta\in F(\gamma)$.
\end{proof}
\begin{corollary}
    If $F$ is perfect and $[K:F]<\infty$, then $K/F$ is simple.
\end{corollary}
TODO: move def'n of conjugates somewhere more logical.
\begin{definition}
    Let $[K:F]<\infty$.
    We say $K/F$ is \mbf{normal} if $K$ is the splitting field of some non-constant $f(x)\in F[x]$ over $F$.
    Suppose $\alpha\in K$ has minimal polynomial $p(x)\in F[x]$.
    The roots of $p(x)$ in its splitting field are called the \mbf{$F-$conjugates} (or just \mbf{conjugates} when the base field is clear) of $\alpha$.
\end{definition}
\begin{remark}
    If $\phi:K\to E$ is an $F-$map and $\alpha$ has minimal polynomial $p(x)\in F[x]$, then $p(\phi(\alpha))=\phi(p(\alpha))=\phi(0)=0$, so that $\phi(\alpha)$ is also a conjugate of $p(x)$ in a splitting field $L/F$.
\end{remark}
\begin{theorem}[Characterization of Normal Extensions]\label{thm:char-norm}
    Let $[K:F]<\infty$.
    The following are equivalent:
    \begin{enumerate}[nolistsep]
        \item $K/F$ is normal.
        \item For every $L/K$, if $\phi$ is an $F-$map from $L$ to $L$, then $\phi|_K\in\Gal(K/F)$.
        \item If $\alpha\in K$, then all of the $F-$conjugates of $\alpha$ are in $K$.
        \item If $\alpha\in K$, then its minimal polynomial splits over $K$.
    \end{enumerate}
\end{theorem}
\begin{proof}
    \imp{1}{2}
    If $K/F$ is normal, then $K$ is the splitting field of some $f(x)\in F[x]$.
    Let $\phi:L\to L$ be an $F-$map.
    Write $K=F(\alpha_1,\ldots,\alpha_n)$ where $\alpha_i$ are the roots of $f(x)$ in $K$.
    It suffices to show that $\phi|_K(K)\subseteq K$.
    For each $i$, there exists $j$ such that $\phi|_K(\alpha_i)=\phi(\alpha_i)=\alpha_j\in K$.
    Since each $x\in K$ is a $F-$linear combination of the $\alpha_i$, it follows that $\phi(x)\in K$, and the result follows.

    \imp{2}{3}
    Let $\alpha\in K$ with minimal polynomial $f(x)\in F[x]$.
    Since $[K:F]<\infty$, $K=F(\alpha_1,\ldots,\alpha_n)$ with $\alpha_i\in K$.
    For each $i$, let $h_i$ be the minimal polynomial for $\alpha_i$ over $F$.
    Let $p(x)=f(x)h_1(x)h_2(x)\cdots h_n(x)$ and $L$ be the splitting field of $p(x)$ over $F$.
    Such a choice is necessary to ensure $L/K/F$.
    Let $\beta\in L$ be a root of $f(x)$, and get $\phi\in\Gal(L/F)$ such that $\phi(\alpha)=\beta$.
    By assumption, $\phi|_K\in\Gal(K/F)$, so $\beta=\phi(\alpha)\in K$, as required.

    \imp{3}{4}
    Immediate.

    \imp{4}{1}
    Since $[K:F]<\infty$, $K=F(\alpha_1,\ldots,\alpha_n)$ for $\alpha_i\in K$.
    Let $h_i(x)$ be the minimal polynomial for $\alpha_i$ over $F$, and set $f(x)=h_1(x)\cdots h_n(x)$.
    Then the splitting field for $f(x)$ over $F$ is $F(\alpha_1,\ldots,\alpha_n)=K$.
\end{proof}
\begin{example}
    $\Q(\sqrt[3]{2})/\Q$ is not normal.
    $\F_{p^n}/\F_p$ is normal, since it is the splitting field of $x^{p^n}-x$.
    $\Q(\zeta_n)/\Q$ is normal with $\Phi_n(x)$.
    $\Z_p(t)/\Z_p(t^n)$ is normal with $x^p-t^p$.
\end{example}
\section{Galois Extensions and the Fundamental Theorem}
\begin{definition}
    We say that $K/F$ is \mbf{Galois} if $K/F$ is normal and separable.
\end{definition}
\begin{remark}
    If $F$ is perfect and $K/F$ is finite, then $K/F$ is Galois if and only if $K/F$ is normal.
\end{remark}
\begin{definition}
    Let $K$ be a field and $G\leq\Aut(K)$.
    Then the \mbf{fixed field} of $G$ is
    \begin{equation*}
        \Fix(G)=\bigl\{a\in K:\phi(a)=a\text{ for all }\phi\in G\bigr\}
    \end{equation*}
\end{definition}
\begin{remark}
    Certainly $\Fix(\Gal(K/F))\supseteq F$ by definition.
\end{remark}
\begin{theorem}[Characterization of Galois Extensions]\label{thm:char-gal}
    The following are equivalent:
    \begin{enumerate}[nl]
        \item $K$ is the splitting field of a non-constant separable $f(x)\in F[x]$ over $F$.
        \item $|\Gal(K/F)|=[K:F]$
        \item $\Fix(\Gal(K/F))=F$
        \item $K/F$ is Galois
    \end{enumerate}
\end{theorem}
\begin{proof}
    \imp{1}{2}
    This is \cref{thm:gal-size}.
     
    \imp{2}{3}
    Assume $|\Gal(K/F)|=[K:F]$ and set $E=\Fix(\Gal(K/F))$ so that $K/E/F$ is a tower of fields.
    Moreover, $\Gal(K/E)\leq\Gal(K/F)$ is a subgroup so $[K:F]=|\Gal(K/F)|\geq|\Gal(K/E)|$.
    Let $a\in E$ and $\phi\in\Gal(K/F)$.
    Then $\phi(a)=a$ by the definition of $E$, so $\Gal(K/E)=\Gal(K/F)$.
    Thus
    \begin{equation*}
        [K:F]=|\Gal(K/F)|=|\Gal(K/E)|\leq[K:E]\leq[K:F]
    \end{equation*}
    so equality holds and $[E:F]=1$ by the tower theorem.

    \imp{3}{4}
    Assume $\Fix(\Gal(K/F))=F$.
    Let $\alpha\in K$ with minimal polynomial $p(x)\in F[x]$; we must show $p(x)$ that splits over $K$ with no repeated roots.
    Let $G=\Gal(K/F)$ and $\{\alpha_1,\ldots,\alpha_n\}=\{\phi(\alpha):\phi\in G\}\subseteq K$.
    Without loss of generality, $\alpha=\alpha_1$, and consider $h(x)=(x-\alpha_1)\cdots(x-\alpha_n)\in K[x]$.
    Then if $\phi\in G$, $\phi(h(x))=h(x)\in (\Fix G)[x]=F[x]$ since $\phi$ acts by permutation on the $\alpha_i$.
    Thus $h(x)$ splits over $K$ with no repeated roots, and in fact $h(x)=p(x)$ since every root of $h(x)$ is a $F-$conjugate of $\alpha$, and thus a root of $p(x)$.

    \imp{4}{1}
    Since $K/F$ is finite, $K=F(\alpha_1,\ldots,\alpha_n)$, $\alpha_i\in K$.
    For each $i$, let $q_i(x)\in F[x]$ be its minimal polynomial.
    Say $p_1(x),\ldots,p_m(x)$ is a list of distinct $q_i(x)$.
    Then $f(x)=p_1(x)\cdots p_m(x)$, and since $K/F$ is normal, its splitting field over $F$ is $K$, and by A6, $f(x)$ is separable.
\end{proof}
\begin{example}
    Consider $\alpha=\sqrt{2+\sqrt{3}}\in\C$, with minimal polyomial $x^4-4x^2+1$.
    Since $\Q$ is perfect, we only need to check normality, and $f(x)$ has roots $\pm\sqrt{2\pm\sqrt{3}}$.
    The $\Q-$conjugates of $\alpha$ are $\pm\alpha,\pm\beta$ where $\beta=\sqrt{2-\sqrt{3}}$.
    Since $\alpha\beta=1$, $\beta=\alpha^{-1}$.
    Thus $\pm\alpha$, $\pm\beta\in\Q(\alpha)$ and $\Q(\alpha)/\Q$ is normal.
    \begin{equation*}
        \begin{array}{c|cccc|c}
            & \alpha & -\alpha & \beta & -\beta & S_4\\
            \hline
            \phi_1 & \alpha & -\alpha & \beta & -\beta & \epsilon\\
            \phi_2 & -\alpha & \alpha & -\beta & \beta & (12)(34)\\
            \phi_3 & \beta & -\beta & \alpha & -\alpha & (13)(24)\\
            \phi_3 & -\beta & \beta & -\alpha & \alpha & (14)(23)
        \end{array}
    \end{equation*}
    so $G\cong\Z_2\times\Z_2$.
\end{example}
\begin{theorem}[Artin]\label{thm:artin}
    Let $K$ be a field, $H$ a finite subgroup of $\Aut(K)$.
    Let $F=\Fix H$.
    Then
    \begin{enumerate}[nolistsep]
        \item $K/F$ is Galois
        \item $\Gal(K/F)=H$
        \item $|H|=[K:F]$
    \end{enumerate}
\end{theorem}
\begin{proof}
    Let $\alpha\in K$ and $\sigma_1,\ldots,\sigma_r\in H$ with $r$ maximal such that the $\sigma_i(\alpha)$ are distinct.
    If $\tau\in G$ is arbitrary, then $(\tau\circ\sigma_i(\alpha))$ differs from $(\sigma_i(\alpha))$ only by a permutation: by maximality of $r$, $\tau\circ\sigma_i(\alpha)=\sigma_j(\alpha)$ for every $i$ and some $j$.
    Injectivity of $\tau$ shows that it is indeed a permutation.
    Thus taking $\tau=\sigma_1^{-1}$ if necessary, we may assume that $\sigma_1(\alpha)=\alpha$ and $\alpha$ is a root of the polynomial
    \begin{equation*}
        f(x)=\prod_{i=1}^r(x-\sigma_i(\alpha))
    \end{equation*}
    and for any $\tau\in G$, $\tau(f)=f$.
    Thus $f(x)\in (\Fix H)[x]=F[x]$.
    Since the $\sigma_i(\alpha)$ are distinct, $f$ is separable.

    Since $\alpha\in K$ was arbitrary and $r\leq|H|$, we see that every $\alpha\in K$ is the root of a separable polynomial with degree at most $|H|$ and coefficients in $F$, and the polynomial splits in $K$.
    Thus $K/F$ and since the minimal polynomial of each $\alpha\in F$ splits completely in $K$, $K/F$ is normal by \cref{thm:char-norm}.
    In particular, by the primitive element theorem (\cref{thm:prim-el}), $K=F(\alpha)$ where the degree of $\alpha$ is at most $|H|$, so that $[K:F]\leq |H|$.

    Note that $H\subseteq\Gal(K/F)$ and $|H|\leq|\Gal(K/F)|\leq[K:F]$; we have shown that $[K:F]\leq |H|$, so we're done.
\end{proof}
\subsection{The Fundamental Theorem of Galois Theory}
We adopt the following notation for the rest of this section.
Suppose $K/F$: then $\mathcal{E}=\{E:F\subseteq E\subseteq K\}$ is the set of intermediate subfields of $K/F$, and $\mathcal{H}$ is the set of subgroups of $\Gal(K/F)$.
We then define the \mbf{Galois correspondence} by
\begin{center}
    \begin{tikzcd}[row sep=tiny]
        \mathcal{E}\rar[leftrightarrow]&\mathcal{H}\\
        E\rar[maps to]&\Gal(K/E)\\
        \Fix H & H\lar[maps to]
    \end{tikzcd}
\end{center}
Note that if $E_1\subseteq E_2$ in $\mathcal{E}$, then $\Gal(K/E_1)\supseteq\Gal(K/E_2)$.
Similarly, if $H_1\subseteq H_2$ in $\mathcal{H}$, then $\Fix H_1\supseteq\Fix H_2$.
Thus the Galois correspondence is inclusion reversing.
\begin{theorem}[Fundamental Theorem of Galois Theory]\label{thm:ftgt}
    Let $K/F$ be a finite Galois extension.
    The Galois correspondences give an inclusion-reversing bijection (antitone Galois connection) between $\mathcal{E}$ and $\mathcal{H}$:
    \begin{enumerate}[nolistsep]
        \item If $E\in\mathcal{E}$, then $\Fix(\Gal(K/E))=E$.
            In particular, $K/E$ is Galois.
        \item If $H\in\mathcal{H}$, then $\Gal(K/\Fix(H))=H$.
    \end{enumerate}
\end{theorem}
\begin{proof}
    \begin{enumerate}[nl]
        \item $K/F$ is normal and separable, so $K/E$ is also normal and separable so that $K/E$ is Galois.
            Thus the result follows by \cref{thm:char-gal}.
        \item This is a direct application of \cref{thm:artin}.
    \end{enumerate}
\end{proof}
\begin{corollary}
    Suppose $K/F$ is finite Galois.
    If $H_1\subseteq H_2$ in $\mathcal{H}$, then $[H_2:H_1]=[\Fix H_1:\Fix H_2]$.
\end{corollary}
\begin{proof}
    We have
    \begin{align*}
        [\Fix H_1:\Fix H_2] &= \frac{[K:\Fix H_2]}{[K:\Fix H_1]}\\
                            &= \frac{|\Gal(K/\Fix H_2)|}{|\Gal(K/\Fix H_1)|}\\
                            &= \frac{|H_2|}{|H_1|}=[H_2:H_1]
    \end{align*}
\end{proof}
To summarize the previous results, perhaps the easiest way to visualize it is with a digram.
On the left, we have the subgroup lattice of $G=\Gal(K/F)$, and on the right, we have the intermediate fields of $K/F$.
\begin{center}
    \begin{tikzpicture}
        \begin{scope}[xshift=-4.2cm,yshift=2.25cm]
            \node (G)                  {$G$};
            \node (H2) [below=of G]  {$H_2$};
            \node (H1) [left=of H2]  {$H_1$};
            \node (H3) [right=of H2] {$H_3$};

            \node (D2) [below=0.7cm of H2] {$\rvdots$};
            \node (D1) [below=0.7cm of H1]  {$\rvdots$};
            \node (D3) [below=0.7cm of H3] {$\rvdots$};

            \node (Z) [below=1cm of D2] {$\{1\}$};

            \draw[shorten >=5pt] (Z) -- (D1.south);
            \draw[shorten >=5pt] (Z) -- (D2.south);
            \draw[shorten >=5pt] (Z) -- (D3.south);

            \draw[shorten >=2pt] (D1) -- (H1);
            \draw (D2) -- (H1);
            \draw[shorten >=2pt] (D2) -- (H2);
            \draw (D2) -- (H3);
            \draw[shorten >=2pt] (D3) -- (H3);

            \draw (G) --node[above left]{$\scriptstyle n_1$}(H1);
            \draw (G) -- node[fill=white]{$\scriptstyle n_2$}(H2);
            \draw (G) -- node[above right]{$\scriptstyle n_3$}(H3);
        \end{scope}
        \draw[->,thick](-2,0.5) -- node[above]{$H_i\mapsto \Fix(H_i)=E_i$}(2,0.5);
        \draw[->,thick](2,-0.5) -- node[below]{$H_i=\Gal(K/E_i)\mapsfrom E_i$}(-2,-0.5);
        \begin{scope}[xshift=4.2cm,yshift=-2.25cm]
            \node (Gp)                  {$F$};
            \node (H2p) [above=of Gp]  {$E_2$};
            \node (H1p) [left=of H2p]  {$E_1$};
            \node (H3p) [right=of H2p] {$E_3$};

            \node (D2p) [above=0.7cm of H2p] {$\rvdots$};
            \node (D1p) [above=0.7cm of H1p]  {$\rvdots$};
            \node (D3p) [above=0.7cm of H3p] {$\rvdots$};

            \node (Zp) [above=1cm of D2p] {$K$};

            \draw[shorten >=5pt] (Zp) -- (D1p.north);
            \draw[shorten >=5pt] (Zp) -- (D2p.north);
            \draw[shorten >=5pt] (Zp) -- (D3p.north);

            \draw[shorten >=2pt](D1p) -- (H1p);
            \draw (D2p) -- (H1p);
            \draw[shorten >=2pt] (D2p) -- (H2p);
            \draw (D2p) -- (H3p);
            \draw[shorten >=2pt] (D3p) -- (H3p);

            \draw (Gp) --node[below left]{$\scriptstyle n_1$}(H1p);
            \draw (Gp) -- node[fill=white]{$\scriptstyle n_2$}(H2p);
            \draw (Gp) -- node[below right]{$\scriptstyle n_3$}(H3p);
        \end{scope}
    \end{tikzpicture}
\end{center}
\begin{example}
    Consider $G=\Gal(x^3-2)$ and set $\alpha=\sqrt[3]{2}$.
    Since $\Q$ is perfect and $x^3-2$ is irreducible, then $x^3-2$ is separable, so $\Q(\alpha,\zeta_3)$ is the splitting field for $x^3-2$ over $\Q$.
    Then $|G|=[\Q(\alpha,\zeta_3):\Q]=6$ and since $G\leq S_3$, $G\cong S_3$.
\end{example}
\begin{proposition}\label{prop:int-conj}
    Let $E$ be an intermediate subfield of $K/F$.
    For any $\phi\in\Gal(K/F)$, $\phi\Gal(K/E)\phi^{-1}=\Gal(K/\phi(E))$.
\end{proposition}
\begin{proof}
    For any $\psi\in\Aut(K)$,
    \begin{align*}
        \psi\in\Gal(K/E) &\iff\psi(\alpha)=\alpha\text{ for all }\alpha\in E\\
                         &\iff\psi\circ\phi^{-1}\circ\phi(\alpha)=\phi^{-1}\circ\phi(\alpha)\text{ for all }\alpha\in E\\
                         &\iff \psi\circ\phi^{-1}(\beta)=\phi^{-1}(\beta)\text{ for all } \beta\in\phi(E)\\
                         &\iff\phi\circ\psi\circ\phi^{-1}(\beta)=\beta\text{ for all } \beta\in\phi(E)\\
                         &\iff\phi\circ\psi\circ\phi^{-1}\in\Gal(K/\phi(E))
    \end{align*}
\end{proof}
\begin{definition}
    Let $K/E/F$ and $H\leq\Gal(K/F)$.
    We say $E$ is \mbf{invariant} under $H$ if $\phi(E)=E$ for all $\phi\in H$.
\end{definition}
\begin{proposition}
    Suppose $K/F$ is finite and Galois.
    If $E$ is an intermediate subfield of $K/F$, then the following are equivalent:
    \begin{enumerate}[nolistsep]
        \item $E/F$ is Galois
        \item $E$ is $\Gal(K/F)-$invariant
        \item $\Gal(K/E)\trianglelefteq\Gal(K/F)$
    \end{enumerate}
\end{proposition}
\begin{proof}
    \impe{2}{3}
    This is straightfoward in light of \cref{prop:int-conj}.

    \imp{1}{2}
    Suppose $E/F$ is Galois and take $\phi\in\Gal(K/F)$.
    Since $E/F$ is Galois, $\phi|_E\in\Gal(E/F)$; thus, $\phi|_E(E)=\phi(E)=E$.

    \imp{2}{1}
    Suppose $E$ is $G-$invariant where $G=\Gal(K/F)$.
    By A7, $E/F$ is separable.
    To show normality, we show that $E$ is closed under conjugation.
    Let $\alpha\in E$ with minimal polynomial $f(x)\in F[x]$.
    Since $K/F$ is normal, $f(x)$ splits over $K$.
    Let $\beta\in K$ be a $F-$conjugate of $\alpha$.
    Since $f(x)\in F[x]$ is irreducible, there exists $\phi\in G$ such that $\phi(\alpha)=\beta$ so that $\beta=\phi(\alpha)\in\phi(E)=E$.
\end{proof}
\begin{proposition}
    Let $K/E/F$, $K/F$ finite and Galois.
    If $E/F$ is Galois, then $\Gal(E/F)\cong\quot{\Gal(K/F)}{\Gal(K/E)}$.
\end{proposition}
\begin{proof}
    Consider the map $\psi:\Gal(K/F)\to\Gal(E/F)$ given by $\psi(\phi)=\phi|_E$.
    Then $\ker\psi=\Gal(K/E)$ and the result follows by the first isomorphism theorem.
\end{proof}
\section{Galois Group Computations}
\begin{example}[Cyclotomic Galois Group]
    Let's compute $\Gal(\Q(\zeta_n)/\Q)$.
    Note that $\Q(\zeta_n)$ is the splitting field for the separable polynomial $\Phi_n(x)$ over $\Q$ so that $\Q(\zeta_n)/\Q$ is Galois.
    To see that $\Gal(\Q(\zeta_n)/\Q)\cong\Z_n^\times$, one can realize that the map $\psi:\Z_n^\times\to G$ by $\psi(k)=\{\zeta_n\mapsto\zeta_n^k\}$ is an isomorphism.
\end{example}
\begin{example}[Finite Field Galois Group]
    We can also compute $\Gal(\F_{p^n}/\F_p)$.
    Since $\F_{p^n}$ is the splitting field of $x^{p^n}-x$ over $\F_p$, $\F_{p^n}/\F_p$ is Galois with index $n$.
    Consider the Frobenius map $\phi:\F_{p^n}\to \F_{p^n}$ such that $\phi(a)=a^p$; by Fermat, $\phi\in\Gal(\F_{p^n}/\F_p)$.
    Let $j=|\phi|$, so $j\leq n$.
    Furthermore, since $\phi$ is an automorphism, every element of $\F_{p^n}$ is a root of $x^{p^j}-x$, which is only possible if $j\geq n$.
    Thus equiality holds and $G=\langle\phi\rangle$.
\end{example}
We now turn towards computing the Galois groups of arbitrary splitting fields of cubic and quadratic polynomials.
To do this, we need to introduce some new machinery.
\begin{definition}
    Let $f(x)\in F[x]$ be non-constant with splitting field $K$.
    Say $f(x)=u(x-\alpha_1)\cdots(x-\alpha_n)\in K[x]$.
    We say
    \begin{equation*}
        \disc f(x)=\prod_{i<j}(\alpha_i-\alpha_j)^2
    \end{equation*}
    is the \mbf{discriminant} of $f(x)$.
\end{definition}
\begin{remark}
    \begin{enumerate}[nl,r]
        \item $\disc(f(x))\neq 0$ if and only if $f(x)$ is separable.
        \item If $f(x)=x^2+bx+c$, then $\disc f(x)=b^2-4c$.
    \end{enumerate}
\end{remark}
\begin{lemma}
    Suppose $f(x)\in F[x]$ is non-constant.
    Then $\disc f(x)\in F$.
\end{lemma}
\begin{proof}
    If $f(x)$ is not separable, this is obvious, so suppose $f(x)$ is separable.
    For all $\phi\in\Gal(f(x))$, $\phi(\disc f(x))=\disc f(x)$, so $\disc f(x)\in\Fix(\Gal(f(x)))=F$.
\end{proof}
\begin{proposition}\label{prop:disc-an}
    Suppose $\chr F\neq 2$, $f(x)$ separable with degree $n\geq 2$.
    Set $G=\Gal f(x)$ and $d=\prod_{i<j}(\alpha_i-\alpha_j)$.

    If $\phi\in G\subseteq S_n$, then $\phi(d)=\pm d$.
    Moreover, $\phi(d)=d$ if and only if $\phi\in A_n$.
    In particular, $\Gal(K/F(d))=G\cap A_n$ and $G\subseteq A_n$ if and only if $d\in\Fix(G)=F$.
\end{proposition}
\begin{proof}
    Let $\phi\in G$, so $d,\phi(d)$ are roots of $x^2-d^2\in F[x]$; thus, $\phi(d)=\pm d$.
    Observe that $S_n$ acts on $X=\{d,-d\}$ by
    \begin{equation*}
        \sigma\cdot\prod_{i<j}(\alpha_i-\alpha_j)=\prod_{i<j}(\alpha_{\sigma(i)}-\alpha_{\sigma(j)})
    \end{equation*}
    Moreover, $\epsilon\cdot d=d$ and $((n)(n-1))\cdot d=-d$, so the action is transitive.
    By Orbit-Stabilizer, $n!=|S_n|=|\Stab(d)|\cdot|S_n\cdot d|=|\Stab(d)|\cdot 2$, so $\Stab(d)=A_n$ since $A_n$ is the only index 2 subgroup of $S_n$.
\end{proof}
For the remainder of this section, we will assume that $\chr F\neq 2,3$.
\subsection{Galois Groups from Cubic Splitting Fields}
We first treat the case where $f(x)$ is cubic.
If $f(x)\in F[x]$ is irreducible and separable, then $\Gal f(x)\cong S_3$ or $A_3$.
Suppose $g(x)=x^3+\alpha x^2+\beta x+\gamma\in F[x]$ irreducible and separable and consider $f(x)=g(x-\alpha/3)=x^3+bx+c\in F[x]$.
Note that $f(x)$ is still irreducible and separable; in particular, $\Gal f(x)=\Gal g(x)$.
Such a cubic is called a \mbf{depressed cubic}.
One can compute $\disc f(x)=-4b^3-27c^2$.
Then by applying \cref{prop:disc-an}, we see that
\begin{equation*}
    \Gal f(x)=
    \begin{cases}
        A_3&:\disc f(x)=d^2, d\in F\\
        S_3&:\text{otherwise}
    \end{cases}
\end{equation*}
\subsection{Galois Groups from Quartic Splitting Fields}
Suppose $f(x)=x^4+\alpha x^3+\beta x^2+\gamma x+\delta\in F[x]$; as before, we take $g(x)=f(x-\alpha/4)=x^4+bx^2+cx+d$, and $\Gal(f(x))=\Gal(g(x))$.
If $G=\Gal f(x)$, then $G$ is a transitive subgroup of $S_4$ with $4\div|G|$.
Thus, the possible options are $S_4$, $A_4$, $D_4$, $V$, $C_4$, where $V=\{\epsilon,(12)(34),(13)(24),(14)(23)\}$.

Let the roots of $f(x)$ be given by $\alpha_1,\ldots,\alpha_4$.
Let $K=F(\alpha_1,\alpha_2,\alpha_3,\alpha_4)$ and set
\begin{align*}
    u&=\alpha_1\alpha_2+\alpha_3\alpha_4\\
    v&=\alpha_1\alpha_3+\alpha_2\alpha_4\\
    w&=\alpha_1\alpha_4+\alpha_2\alpha_3
\end{align*}
We define the \mbf{resolvent cubic} of $f(x)$
\begin{align*}
    \Res f(x) = (x-u)(x-v)(x-w)=x^3-bx^2-4dx+4bd-c^2\in F[x]
\end{align*}
where the coefficients may be evaluated by the reader.

Let $L=F(u,v,w)$, so that $K/L/F$.
Since $K/F$ is Galois, $K/L$ is Galois, and $\Gal(\Res f(x))=\Gal(L/F)$.
Since $\Gal(K/L)=G\cap V$ and $L/F$ is Galois, $\Gal(K/L)\trianglelefteq\Gal(K/F)$, and $\Gal(L/F)=G/G\cap V$.
Let $m=|\Gal(\Res f(x))|$.
\begin{equation*}
    \begin{array}{c|ccccc}
        G & S_4 & A_4 & D_4 & V & C_4\\
        \hline
        G\cap V &V&V&V&V&C_2\\
        G/(G\cap V) & S_3 & C_3 & C_2 & \{1\} & C_2\\
        m & 6 & 3 & 2 & 1 & 2
    \end{array}
\end{equation*}
Note that $G$ is uniquely determined when $m\in\{1,3,6\}$, so let's examine the case $m=2$.
Since $\deg(\Res f(x))=3$ and $m=2$, exactly one of $u$, $v$, or $w$ is in $F$.
Without loss of generality, assume $u\in F$.
Either option for $G$ has a 4-cycle which fixes $u$, so $\sigma=(1324)\in G$ and $\sigma^2=(12)(34)\in G$.
Consider
\begin{align*}
    (x-\alpha_1\alpha_2)(x-\alpha_3\alpha_4) &=x^2-ux+d\\
    (x-(\alpha_1+\alpha_2))(x-(\alpha_3+\alpha_4)) &=x^2+(b-u)
\end{align*}
Let's see that $G=\langle\sigma\rangle\cong C_4$ if and only if both of these polynomials split over $L$.

\impr
Suppose $G=\langle\sigma\rangle$.
Then $\Gal(K/L)=G\cap V=\langle\sigma^2\rangle$, so $\alpha_1\alpha_2$, $\alpha_3\alpha_4$,$\alpha_1+\alpha_2$,$\alpha_3+\alpha_4\in\Fix\langle\sigma^2\rangle=L$.

\impl
Conversely, suppose $\alpha_1\alpha_2$, $\alpha_3\alpha_4$,$\alpha_1+\alpha_2$,$\alpha_3+\alpha_4\in L$.
Then $\alpha_1\alpha_2\in L(\alpha_1)$ that $\alpha_1,\alpha_2\in L(\alpha_1)$.
Then since $v-w=(\alpha_1-\alpha_2)(\alpha_3-\alpha_4)\in L$, so $\alpha_3-\alpha_4\in L(\alpha_1)$ as well, so that $\alpha_3,\alpha_4\in L(\alpha_1)$.

Now, $K=F(\alpha_1,\ldots,\alpha_4)=L(\alpha_1)$, and $[K:L]=[L(\alpha_1):L]=|\Gal(K/L)|$.
The polynomial $p(x)=x^2-(\alpha_1+\alpha_2)x+\alpha_1\alpha_2\in L[x]$ has $p(\alpha_1)=0$ so that $[K:L]\leq 2$.
Thus $[K:F]\leq 4$, which forces $G=C_4$.
TODO: why is $[L:F]\leq 2$?
\begin{example}
    Consider $f(x)=x^4-2x-2$.
    Then $\Res f(x)=x^3+8x-4$ has no rational roots, and is irreducible.
    Now, $\disc(\Res f(x))=-4\cdot(8^3)-27\cdot 4^2<0$ is not a square in $\Q$, so $\Gal(\Res f(x))=S_3$.
    Thus $\Gal f(x)\cong S_4$.
\end{example}
\begin{example}
    Consider $g(x)=x^4+5x+5$, irreducible by Eisenstein, so $\Res g(x)=x^3-20x-25=(x-5)(x^2+5x+5)$.
    Thus $\Gal\Res g(x)=\Z_2$, and $m=2$.
    We let $u=5\in\Q$.
    Consider $x^2-5x-5$ and $x^2-5$.
    The roots of $x^2+5x+5$ are $\frac{-5\pm\sqrt{5}}{2}$, so $L=\Q(\sqrt{5})$.
    The roots of $x^2-5$ are also in $L$.
    Thus $\Gal f(x)=\Z_4$.
\end{example}

\section{Solvability and Radical Extensions}
Throughout this section, we assume that $\chr F=0$.
\begin{definition}
    A group $G$ is \mbf{solvable} if there exists a chain of subgroups $G=G_0\trgeq G_1\trgeq G_2\trgeq\cdots\trgeq G_n=\{1\}$ such that $\quot{G_i}{G_{i+1}}$ is abelian.
\end{definition}
\begin{example}
    Any abelian solvable is abelian.
    We have $S_4\supseteq A_4\supseteq V\supseteq\{1\}$, so $S_4$ is solvable.
    If $G$ is simple, then $G$ is solvable if and only if $G$ is abelian.
    For example, $A_5$ is simple and non-abelian, and thus not solvable.
\end{example}
\begin{proposition}
    If $G$ is solvable and $N\leq G$, then $N$ is solvable; if $N\trianglelefteq G$, then $G/N$ is solvable.
\end{proposition}
\begin{proof}
    Since $G$ is solvable, get $G=G_0\trgeq G_1\trgeq\cdots\trgeq G_n=\{1\}$.
    Then
    \begin{itemize}[nl]
        \item Consider the sequence $N=G_0\cap N\trgeq G_1\cap N\trgeq\cdots\trgeq G_n\cap N=\{1\}$, since normality is preserved under intersection.
            Furthermore,
            \begin{equation*}
                \quot{N\cap G_i}{N\cap G_{i+1}}\cong \quot{(N\cap G_i)G_{i+1}}{G_{i+1}}\subseteq\quot{G_i}{G_{i+1}}
            \end{equation*}
            is abelian.
        \item Consider the sequence $\quot{G}{N}=\quot{G_0}{N}\trgeq\quot{G_1}{N}\trgeq\cdots\trgeq\quot{G_n}{N}=\{1\}$ and use the third isomorphism theorem.
            TODO: finish this, something is weird: $N$ is not a normal subgroup of $G_i$, use correspondence theorem for normal subgroups.
    \end{itemize}
\end{proof}
\begin{proposition}
    Let $N\trleq G$; then $N$ is solvable if and only if $N$ and $\quot{G}{N}$ are solvable.
\end{proposition}
\begin{proof}
    The forward direction is done; conversely, suppose $N$ and $G/N$ are solvable.
    Let
    \begin{align*}
        N=N_0\supseteq N_1\supseteq\cdots\supseteq N_m=\{1\}\\
        G/N=G_0/N\supseteq G_1/N\supseteq\cdots\supseteq G_l/N=\{N\}
    \end{align*}
    By the third isomorphism theorem, $\quot{G_i/N}{G_{i+1}/N}\cong \quot{G_i}{G_{i+1}}$, so $G=G_0\supseteq G_1\supseteq\cdots\supseteq N$.
    TODO: fix this.
\end{proof}
\begin{remark}
    Let $G$ be finite, solvable.
    By refining the chain as much as possible, we may assume $G=G_0\supseteq G_1\supseteq\cdots\supseteq G_n=\{1\}$ with $G_i/G_{i+1}$, and no $H_i\leq G$ with $G_i\supsetneq H_i\supseteq G_{i+1}$ normal.
    That is to say, $G_i/G_{i+1}$ is abelian and simple, so $\left\lvert\quot{G_i}{G_{i+1}}\right\rvert$ prime.
\end{remark}
\begin{definition}
    We say $K/F$ is a \mbf{simple radical extension} if $K=F(\alpha)$ for some $\alpha\in K$ such that $\alpha^n\in F$ for some $n\in\N$.
    A \mbf{radical tower} over $F$ is a tower $K_m/K_{m-1}/\cdots/K_1/F$ such that $K_1/F$ and $K_{i+1}/K_i$ are each simple radical extensions.
    We say $K/F$ is \mbf{radical} if there exists a radical tower over $F$ starting at $K$.
    We say $f(x)\in F[x]$ is \mbf{solvable by radicals} over $F$ if its splitting field is contained in a radical extension of $F$.
\end{definition}
\begin{example}
    Consider $f(x)=x^4-4x^2+2$.
    Then $\Q(\sqrt{2+\sqrt{2}})\supseteq\Q(\sqrt{2})\supseteq\Q$ is solvable by radicals over $\Q$.
\end{example}
\begin{definition}
    We say an extension $K/F$ is \mbf{cyclic} if $K/F$ is finite and Galois, and $\Gal(K/F)$ is cyclic.
\end{definition}
\begin{proposition}\label{prop:prim-cy}
    If $F$ contains a primitive $n^\text{th}$ root of unity and $K=F(\alpha)$ with $\alpha^n\in F$, then $K/F$ is cyclic.
\end{proposition}
\begin{proof}
    Consider $f(x)=x^n-\alpha^n\in F[x]$.
    Let $\zeta\in F$ be a primitive $n^\th$ root of unity.
    The roots of $f(x)$ in $K$ are $\alpha\zeta^i$ for $i\in\{0,1,\ldots,n-1\}$.
    Thus $K$ is the splitting field for $f(x)$ over $F$, so $K/F$ is Galois.
    For each $\phi\in\Gal(K/F)$, there exists a unique $0\leq i\leq n-1$ such that $\phi(\alpha)=\alpha\zeta^i$.
    Write $i=\Gamma(\phi)$, and it is straightforward to verify that $\Gamma:\Gal(K/F)\to\Z_n$ is an injective homomorphis.
    Thus $\Gal(K/F)$ is isomorphic to a cyclic subgroup of $Z_n$, and thus cyclic.
\end{proof}
TODO: finish all the proofs in this section.
\begin{definition}
    We say $\{\sigma_1,\ldots,\sigma_n\}\subseteq\Aut K$ is \mbf{linearly dependent} over $K$ if there exists $a_i\in L$, not all zero, such that $a_1\sigma_1(\alpha)+\cdots+a_n\sigma_n(\alpha)=0$ for all $\alpha\in K$.
    Otherwise, we say $\{\sigma_1,\ldots,\sigma_n\}$ is \mbf{linearly independent}.
\end{definition}
\begin{lemma}
    Let $[K:F]<\infty$.
    Then any finite subset of $\Gal(K/F)$ is linearly independent over $K$.
\end{lemma}
\begin{proof}
    Suppose not; it suffices to prove the result for $\Gal(K/F)$.
    Let $\{\sigma_1,\ldots,\sigma_r\}$ be a minimal linearly dependent subset of $\Gal(K/F)$ and let
    \begin{equation*}
        a_1\sigma_1+\cdots+a_r\sigma_r=0
    \end{equation*}
    be a non-trivial dependence relation; note that each $a_i\in K^\times$ by minimality.
    Certainly, $r>1$.

    Let $\beta\in K$ be such that $\sigma_1(\beta)\neq\sigma_2(\beta)$.
    We then have for any $\alpha\in K$ that
    \begin{align}
        a_1\sigma_1(\alpha)\sigma_1(\beta)+a_2\sigma_2(\alpha)\sigma_2(\beta)+\cdots+a_r\sigma_r(\alpha)\sigma_r(\beta)&=0\label{eq:1}\\
        a_1\sigma_1(\alpha)\sigma_1(\beta)+a_2\sigma_2(\alpha)\sigma_1(\beta)+\cdots+a_r\sigma_r(\alpha)\sigma_1(\beta)&=0\label{eq:2}
    \end{align}
    where \cref{eq:1} follows since $\sigma_i(\alpha\beta)=\sigma_i(\alpha)\sigma_i(\beta)$.
    Subtracting \cref{eq:1} and \cref{eq:2}, we get
    \begin{equation*}
        a_2\sigma_2(\alpha)[\sigma_2(\beta)-\sigma_1(\beta)]+\cdots+a_r\sigma_r(\alpha)[\sigma_r(\beta)-\sigma_1(\beta)]=0
    \end{equation*}
    which is a dependence relation on $\{\sigma_2,\ldots,\sigma_r\}$, contradicting minimality.
\end{proof}
We now provide a converse to \cref{prop:prim-cy}.
TODO: maybe merge the theorems?
\begin{proposition}
    Let $F$ be a field which contains a primitive $n^\text{th}$ root of unity.
    If $K/F$ is cyclic with $[K:F]=n$, then $K/F$ is simple radical.
\end{proposition}
\begin{proof}
    Suppose $\zeta\in F$ is a primitive $n^\text{th}$ root of unity and $K/F$ is cyclic of degree $n$.
    Let $G=\Gal(K/F)=\langle\sigma\rangle$, $|G|=n$ for some $\sigma\in G$.
    For $\alpha\in K$, define
    \begin{equation*}
        g(\alpha):=\alpha+\zeta\sigma(\alpha)+\zeta^2\sigma^2(\alpha)+\cdots+\zeta^{n-1}\sigma^{n-1}(\alpha)
    \end{equation*}
    Note that $\zeta\sigma(g(\alpha))=g(\alpha)$ so that $\sigma(g(\alpha))=\zeta^{-1}g(\alpha)$.
    In particular,
    \begin{equation*}
        \sigma(g(\alpha)^n)=\sigma(g(\alpha))^n=\left(\zeta^{-1}g(\alpha)\right)^n=g(\alpha)^n
    \end{equation*}
    Thus for all $\alpha\in K$, since $G=\brac{\sigma}$, $g(\alpha)^n\in\Fix G=F$.
    Moreover, since $G$ is linearly independent over $K$, there exists $\alpha\in K$ such that $g(\alpha)\neq 0$.
    Furthermore, $\sigma^i(g(\alpha))=\zeta^{-i}g(\alpha)\neq g(\alpha)$ for any $1\leq i\leq n-1$; thus $g(\alpha)\notin\Fix H$ for any $\{1\}\neq H\leq G$.
    Thus by the fundamental theorem of galois theory (\cref{thm:ftgt}), $g(\alpha)\notin E$ for any $F\subseteq E\subsetneq K$, so $F(g(\alpha))=K$.
\end{proof}
\begin{proposition}
    Let $K/E/F$, $E/F$ Galois, $K/E$ radical.
    Then there exists $L/K$ such that $L/F$ is Galois and $L/E$ is radical such that $\Gal(L/E)$ is solvable.
\end{proposition}
\begin{proof}
    We prove the result when $K/E$ is simple radical; the more general case follows by induction.
    Suppose $K=E(\alpha)$ where $\alpha^n=\beta\in E$.
    Also suppose $G=\Gal(E/F)=\{\sigma_1,\ldots,\sigma_r\}$.
    Consider
    \begin{equation*}
        f(x)=\Phi_n\prod_{i=1}^r(x^n-\sigma_i(\beta))\in(\Fix G)[x]=F[x]
    \end{equation*}
    and let $L$ be the splitting field for $f(x)$ over $K$; let's show that $L$ has the desired properties.
    \begin{itemize}
        \item $L/F$ is Galois.
            First note that $L$ is the splitting field for $f(x)$ over $E$.
            Since $E/F$ is Galois, $E$ is the splitting field of some separable polynomial $h(x)\in F[x]$.
            Then $L$ is the splitting field for $h(x)f(x)$, and since $\chr F=0$ so that $F$ is perfect, $L/F$ is Galois.
        \item $L/E$ is radical.
            Let $\zeta$ be a root of $\Phi_n(x)$ in $L$.
            We extend each $\sigma_i\in G$ to a $\sigma_i^*\in\Gal(L/F)$.
            Thus, the roots of $f(x)$ are of the form $\zeta^i\sigma_i^*(\alpha)$, so $L=E(\zeta,\sigma_1^*(\alpha),\ldots,\sigma_r^*(\alpha))$.

            Let $E_0=E(\zeta)$ and for $1\leq i\leq r$, $E_i=E(\zeta,\sigma_1^*(\alpha),\ldots,\sigma_i^*(\alpha))$ so $E_r=L$.
            Note that $\zeta^n=1\in E$ and $\sigma_i^*(\alpha)^n=\sigma_i^*(\alpha^n)=\sigma_i^*(\beta)=\sigma_i(\beta)\in E$.
            Thus,
            \begin{equation*}
                E\subseteq E_0\subseteq E_1\subseteq\cdots\subseteq E_r=L
            \end{equation*}
            is a radical tower, so that $L/E$ is radical.
        \item $\Gal(L/E)$ is solvable.
            Let $G_i=\Gal(L/E_i)$, so by the fundamental theorem of galois theory,
            \begin{equation*}
                \{1\}=G_r\leq G_{r-1}\leq\cdots\leq G_2\leq G_1\leq G_0\leq G'
            \end{equation*}
            where $G_0=\Gal(L/E(\zeta))$.
            Moreover, $G_0\leq G':=\Gal(L/E)$.
            First, $G_0=\Gal(L/E(\zeta))\trleq\Gal(L/E)$ since $E(\zeta)/E$ is Galois (splitting field of $\Phi_n(x)$ over $E$).
            Furthermore, $\quot{G'}{G_0}\cong\Gal(E(\zeta)/E)$ is abelian in the same way that $\Q(\zeta)/\Q$ is abelian.

            Now, $\Gal(L/E_{i+1})\trianglelefteq\Gal(L/E_i)$ since $E_{i+1}/E_i$ is Galois ($E_{i+1}/E_i$ is simple radical with $\zeta\in E_i$ and $\sigma_{i+1}^*(\alpha)^n\in E_i$.
            By the proposition, $E_{i+1}/E_i$ is cyclc.
            Also, $G_i/G_{i+1}\cong\Gal(E_{i+1}/E_i)$ is cyclic (correspondence between simple radical and cyclic).
    \end{itemize}
\end{proof}
\begin{corollary}
    Take $E=F$.
    If $K/F$ is radical, then there exists $L/K$ such that $L/F$ is radical and Galois with $\Gal(L/F)$ is solvable.
\end{corollary}
\begin{theorem}[Galois]
    Let $f(x)\in F[x]$.
    Then $f(x)$ is solvable over $F$ if and only if $\Gal f(x)$ is solvable.
\end{theorem}
\begin{proof}
    \impr Reading

    \impl Suppose $f(x)$ is solvable by radicals over $F$.
    Say $f(x)=p_1(x)^{i_1}\cdots p_l(x)^{i_l}$ where the $p_i$ are distinct and irreducible.
    By replacing $f(x)$ with $p_1(x)\cdots p_l(x)$, we may assume $f(x)$ is separable.
    Let $E$ be the splitting field of $f(x)$ over $F$.
    Then $E/F$ is Galois.
    Moreover, $E\subseteq K$, $K/F$ is radical.
    Then by the proposition, there exists $L/K$ such that $L/F$ is Galois and radical.
    Since $E/F$ is Galois, $\Gal(L/E)\trianglelefteq\Gal(L/F$.
    Thsn $\Gal(E/F)\cong\quot{\Gal(L/F)}{\Gal(L/E)}$.
\end{proof}
\begin{example}
    If $1\leq\deg (x)<5$, then $f(x)$ is solvable by raicals.
    Let $g(x)$ be the product of distinct factors of $f(x)$.
    Then $\Gal(g(x))\leq S_4$ since $g(x)$ is separable, and $S_4$ is solvable.
\end{example}
\begin{remark}
    Note that $S_n=\langle(12),(123\cdots n)\rangle$.
    If $p$ is prime, then $S_p=\langle\tau,\sigma\rangle$ where $\tau$ is any transposition and $\sigma$ is any $p-$cycle.
\end{remark}
\begin{lemma}
    Let $f(x)\in\Q[x]$ be irreducible with prime degree $p$.
    If $f(x)$ has exactly 2 non-real roots, then $\Gal f(x)=S_p$.
\end{lemma}
\begin{proof}
    Let $\alpha$ be a root of $f(x)$, then $[\Q(\alpha):\Q]=\deg f(x)=p$.
    Thus $p\div[K:\Q]$ where $k$ is the splitting field of $f(x)$ over $\Q$.
    Thus there exists $\sigma\in\Gal f(x)$, $|\sigma|=p$.
    Without loss of generality, $\sigma=(123\cdots p)$.
    Moreover, $\phi:\C\to\C$ by $\phi(z)=\overline{z}$ is a $\Q-$map.
    By the normality theorem, $\phi|_K\in\Gal f(x)$.
    Since $f(x)$ has only 2 non-real roots, $\phi|_K=(ij)$.
    Thus $\Gal f(x)=S_p$.
\end{proof}
\begin{example}
    Consider $f(x)=x^5+2x^3-24x-2$, irreducible by Eisenstein.
    By IVT, $f(x)$ has at least 3 real roots.
    Computing the sum of squares of roots as $\sum \alpha_i^2=\left(\sum\alpha_i\right)^2-2\sum_{i<j}\alpha_i\alpha_j=-4$, one sees that not all rots of $f(x)$ are real.
    Since non-real roots of $f(x)$ appear in conjugate pairs, $f(x)$ has exactly 2 non-real roots.
    By the lemma, $\Gal f(x)=S_5$, $S_5$ is not solvable, so $f(x)$ is not solvable by radicals.
\end{example}
Exam questions!
\begin{enumerate}
    \item Minimal polynomials / field extensions
    \item show $K/F$ Galois, compute $\Gal(K/F)$
    \item Answer questions about $\Gal(f(x))$ (probably quartic)
    \item questions similar to assignment questions, times 3
    \item 2 proofs from lecture, from the second half (post midterm)
    \item new proof, and an assignment proof
    \item solvability by radicals
    \item give example / DNE (10 parts)
\end{enumerate}
\end{document}

